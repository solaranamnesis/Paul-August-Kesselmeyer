\documentclass[a4paper, 8pt, oneside, polutonikogreek, german]{article}
\usepackage{gfsbaskerville}
% Load encoding definitions (after font package)
\usepackage[LGR,T1]{fontenc}
\usepackage{textalpha}
\usepackage{tabularx}
\usepackage{pdflscape}
\usepackage{listings}
\lstset{basicstyle=\ttfamily}
\usepackage{longtable}

% Babel package:
\usepackage{babel}

% With XeTeX$\$LuaTeX, load fontspec after babel to use Unicode
% fonts for Latin script and LGR for Greek:
\ifdefined\luatexversion \usepackage{fontspec}\fi
\ifdefined\XeTeXrevision \usepackage{fontspec}\fi

% "Lipsiakos" italic font `cbleipzig`:
\newcommand*{\lishape}{\fontencoding{LGR}\fontfamily{cmr}%
		       \fontshape{li}\selectfont}
\DeclareTextFontCommand{\textli}{\lishape}

\usepackage{booktabs}
\setlength{\emergencystretch}{15pt}
\usepackage{fancyhdr}
\usepackage{microtype}
\begin{document}
\begin{titlepage} % Suppresses headers and footers on the title page
	\centering % Centre everything on the title page
	%\scshape % Use small caps for all text on the title page

	%------------------------------------------------
	%	Title
	%------------------------------------------------
	
	\rule{\textwidth}{1.6pt}\vspace*{-\baselineskip}\vspace*{2pt} % Thick horizontal rule
	\rule{\textwidth}{0.4pt} % Thin horizontal rule
	
	\vspace{1\baselineskip} % Whitespace above the title
	
	{\scshape\LARGE Über den Ursprung der Meteorsteine.}
	
	\vspace{1\baselineskip} % Whitespace above the title

	\rule{\textwidth}{0.4pt}\vspace*{-\baselineskip}\vspace{3.2pt} % Thin horizontal rule
	\rule{\textwidth}{1.6pt} % Thick horizontal rule
	
	\vspace{1\baselineskip} % Whitespace after the title block
	
	%------------------------------------------------
	%	Subtitle
	%------------------------------------------------
	
	{\scshape von P. A. Kesselmeyer.} % Subtitle or further description
	
	\vspace*{1\baselineskip} % Whitespace under the subtitle
	% Subtitle or further description
    
	%------------------------------------------------
	%	Editor(s)
	%------------------------------------------------
    \vspace*{\fill}

	\vspace{1\baselineskip}

	{\small\scshape Frankfurt a. M. 1860.}
	
	{\small\scshape{Druck und Verlag von Heinrich Ludwig Brönner.}}
	
	\vspace{0.5\baselineskip} % Whitespace after the title block

    \scshape Internet Archive Online Edition  % Publication year
	
	{\scshape\small Namensnennung Nicht-kommerziell Weitergabe unter gleichen Bedingungen 4.0 International} % Publisher
\end{titlepage}
\setlength{\parskip}{1mm plus1mm minus1mm}
\clearpage
\tableofcontents
\clearpage
\section*{}
\hspace*{6mm}A. bedeutet: Arago, Astronomie populaire; Paris u. Leipzig 1857.

B. bedeutet: Buchner, die Feuermeteore, insbesondere die Meteoriten; Gießen 1859.

CR. bedeutet: Comptes rendus de l’academie des sciences a Paris.

G. bedeutet: Gilberts Annalen.

K. bedeutet: Kämtz, Lehrbuch der Metereologie; Halle 1836.

P. bedeutet: Poggendorffs Annalen.

RPG. bedeutet: Greg, an Essay on Meteorites, 1855.

S. bedeutet: Shepard, Catalogue of the Meteoric Collection of Charles Upham Shepard; New-Haven 1860.

SJ. bedeutet: Sillimans American Journal,

W. bedeutet: Haidinger, die Meteoriten des k. k. Hof-Naturalien-Kabinetts am 30. Mai 1860.

WA. bedeutet: Sitzungsberichte der mathematisch-naturwissenschaftlichen Klasse der k. Akademie in Wien.
\paragraph{}
Die Frage, woher wohl jene eigentümlichen mineralogischen Gebilde stammen mögen, die von Zeit zu Zeit teils als völlig gediegene Eisenmassen, teils unter der Form von Basalt- und Dolerit-ähnlichen Gesteinen, stets aber unter den auffallendsten Naturerscheinungen auf unsere Erde herabzufallen pflegen, musste mit Notwendigkeit von jeder die Geister beschäftigen. Jene mittelalterliche Ansicht, dass solche Donnerkeile — wie man sie nannte — als Zeichen göttlichen Zornes mit unseren gewöhnlichen Blitzschlagen vom Himmel kamen, konnte sich natürlich nur so lange halten, als man, in Folge eines wenig erleichterten Verkehres, die meisten dieser Tatsachen nur vom Hörensagen oder aus alten Chroniken kannte. Als aber mit der Zeit die Zahl wirklich beobachteter Meteorsteinfälle sich stets mehrte; als alle Nachrichten und zwar aus den verschiedensten Ländern Europas, darin übereinstimmten, dass sie meistenteils gerade bei völlig heiterem und wolkenlosem Himmel sich ereigneten: da konnte eine solche Ansicht nicht länger mehr bestehen. Ähnlich musste es einer anderen Erklärungsweise ergehen, wonach namentlich die Gediegen-Eisenmassen nichts Anderes sein sollen, als vom Blitz getroffene und eben dadurch innerlich wie äußerlich veränderte gewöhnliche Eisengänge* unserer Erde. Auch sie musste zerfallen, nachdem man das Herabkummen glühender Eisenmassen nicht allein wirklich beobachtet, sondern auch bemerkt hatte, dass fast alle für meteorisch zu haltenden gediegenen Eisenmassen gerade vorzugsweise in solchen Gegenden sich vorfinden, wo weit und breit keine sonstigen Eisenlager vorhanden sind. Darum konnte denn auch nach allen diesen Tatsachen über den wirklich überirdischen Ursprung dieser rätselhaften Gesteine kein Zweifel mehr obwalten. Aber wie und woher kommen sie in jene luftigen Höhen, aus denen sie, begleitet von so ungewöhnlichen Erscheinungen, auf unsere Erde herabfallen? Diese Frage einmal angeregt, konnte der zunächst liegende Gedanke wohl kaum ein anderer sein, als sie für Felsbruchstücke zu halten, welche durch die Gewalt irischer Vulkane in die Höhe geschlendert, nun in Folge ihrer Schwere wiederum in anderen Gegenden herabfallen. Allein die große Entfernung der Niederfälle von den zunächst liegenden, noch jetzt tätigen Feuerbergen, so wie das ungeheure Gewicht einzelner dieser Steine, mussten sofort gegen eine solche Annahme sprechen. Auch die Vergleichung der Steine selbst mit denen, wie sie in der Nähe unserer Vulkane wirklich sich vorfinden, erschien einer solche Annahme nicht günstig.

*) G. 14. 1803. Fol. 55.

Auf der Erde also — so schien es nach allem Diesem — war ihr Ursprung nicht zu suchen. Von Himmel schienen sie in der Tat zu kommen. Was war daher wohl wahrscheinlicher, als sic von nun an für fremde Eindringlinge, für die handgreiflichen, tast- und fühlbaren Boten einer uns unbekannten und unzugänglichen Welt zu halten? Aber wo in dem weiten Weltenall sollte man ihre wirkliche Heimat suchen? Bei diesen Gedanken einmal angelangt, lag nichts näher, als die Blicke nach dem Monde zu lenken, dem uns bekanntesten und nächsten aller Himmelskörper. Nach den Beobachtungen der Astronomen schien es nicht zu bezweifeln, dass tätige Vulkane auf seiner Oberfläche sich befinden. Auch hielt man es nach angestellten Berechnungen nicht für unmöglich, dass dieselben im Stande sein dürften, Felsenmassen bis in eine solche Entfernung in die Höhe zu schleudern, dass sie — die Grenze der Anziehung ihres eigenen Himmelskörpers überschreitend und derjenigen unserer Erde nun verfallend — in immer rascherem Falle endlich auf diese Letztere selbst herabzustürzen gezwungen seien. Die bedeutendsten Naturforscher, wie Laplace, Olbers, Berzelius* und Andere, huldigten dieser Ansicht. Der verschiedenartige Charakter der einzelnen Meteorsteine erklärte sich hiernach einfach und natürlich durch die geognostische Verschiedenheit der einzelnen Mondgebirge. Die Feuererscheinung, das Erglühen der ganzen Masse kurz vor dem Niederfall, war eine Folge der Reibung, welche der Eindringling durch die in Folge seines Falles gewaltsam zusammengepresste Luft erlitt. Selbst die Beobachtung, dass alle diese fallenden Körper trotz ihrer weiten Herkunft am Ende doch nur mit der gewöhnlichen Fallgeschwindigkeit auf unserer Erde anlangten, schien in dieser gewaltsamen Zusammenpressung der Luft und in dem durch sie hervorgerufenen Widerstande ihre natürliche Erklärung zu finden.

*) P. 33. 1834. Fol. 1 u. 113. P. 36. 1835. Fol. 161.

Allein ungeachtet aller dieser Gründe vermochte diese Ansicht doch nicht, nach allen Seiten hin vollständig zu genügen. Die ungeheure Gewalt der Mondvulkane, wie sie zu einer solchen Annahme nötig war, erschien Vielen nicht minder rätselhaft als die ganze Erscheinung selbst, welche durch sie ihre Erklärung finden sollte. Daher versuchte denn Chladni eine neue Bahn, und trat allen bisherigen Ansichten mit der Theorie von dem kosmischen Ursprung* aller meteorischen Gesteine gegenüber. Alle vom Himmel fallenden Körper, alle Meteorsteine, alle Sternschnuppen, Feuerkugeln u. s. w. stammten nach ihm aus dem weiten Weltenraume, wo sie, entweder schon geballt als feste planetarische Körper, oder noch ungeballt als planetarische Dunst- und Nebelmassen, ihre uns unbekannten Bahnen beschreiben. Gelangt - so nahm er an - einer dieser „Weltspäne“ in die Nähe eines größeren Himmelskörpers, so wird er von diesem aus seiner Bahn herausgezogen, bis er, dieser übermächtigen Anziehung immer mehr folgend, endlich nach denselben Gesetzen wie jene Auswürflinge des Mondes in immer unwiderstehlicherem Fluge auf den anziehenden Himmelskörper selbst herabstürzt, um nie und nimmermehr in seine frühere Bahn zurückzukehren. Das namentlich bei Feuerkugeln öfters beobachtete sogenannte Rikoschettieren, dies sprungweise sich Auf- und Ab-bewegen galt ihm als ein unverkennbares Zeichen des wirklichen Eindringens von außen in die dichteren Schichten unseres irdischen Dunstkreises: es war das von unserer Erde aus betrachtete Abprallen der eindringenden Masse von der im Vergleich zum Weltäther weit dichteren, elastisch-flüssigen Oberfläche unserer Atmosphäre. Das plötzliche Erglühen erkannte er ebenfalls als eine Folge der durch Reibung und Kompression der Luft erzeugten Warme, und das häufig wahrgenommene Anschwellen der feurigen Kugel für ein durch eben diese Hitze erzeugtes blasenähnliches Aufschwellen der eingedrungenen Masse, dessen endliche Folge das Zerplatzen und das Herabfallen der in ihr enthaltenen oder gebildeten Steine sein musste.

*) G. 13. 1803. Fol. 350. G. 57. 1817. Fol. 121. G. 68. 1821. Fol. 369. P. 36. 1835. Fol. 176.

Diese Ansicht Chladnis gewann sich bald viele und sehr bedeutende Anhänger. Die angesehensten Naturforscher traten ihr bei, und auch noch jetzt ist sie die am Meisten verbreitete. Allein nichtsdestoweniger erhoben sich auch gegen sie schon frühzeitig gar manche und gewiss nicht zu missachtende Bedenken. Die Vermutung, dass trotz der scheinbaren Unmöglichkeit unsere irdische Atmosphäre vielleicht dennoch die Grundstoffe sollte liefern können, aus denen diese „Luftsteine“ gewoben, war schon frühe hier und dort geäußert worden. Als feste Massen können sie sich freilich nicht in derselben aufhalten. Ob dieses aber nicht im dunst- oder gasförmigen Zustand möglich wäre? Diese Frage war, wenn gleich Anfangs erfolglos, doch schon ziemlich frühe aufgestellt worden. So hielt Musschenbroek* die Meteorsteine für schwefelhaltige Dämpfe aus unseren irdischen Vulkanen, und Dominicus Tata* äußerte sich bei Gelegenheit des Steinfalles von Siena dahin, dass derselbe kiesigen Materialien seinen Ursprung zu verdanken haben dürfte, welche sich in Dampfgestalt von unserer Erde erhoben, und innerhalb unserer Atmosphäre durch elektrische und andere Kräfte in den festen Zustand gebracht worden seien. Auch Patrin* erklärte die Bildung der Meteorsteine geradezu für identisch mit der Bildung derjeniger Massen, die auch unsere irdischen Vulkane auswerfen, d. h. für chemische Verbindungen verschiedener, durch vulkanische Hitze in Gasgestalt übergeführter Substanzen. Später waren es namentlich Wrede, Egen und von Hof, welche sich in ähnlicher Weise gegen den kosmischen Ursprung erklärten. Wrede* wies darauf hin, wie unrecht man getan, Sternschnuppen, Steinfällen, Feuermeteoren, Sand- und Staubregen, - allen den gleichen kosmischen Ursprung zuzuschreiben. Letztere, die Sand- und Staubregen, so wie die bloß leuchtenden Feuerkugeln erklärte er für Erscheinungen, die entschieden unserer irdischen Atmosphäre angehörten. Aber auch für die Meteorsteine erkannte er wenigstens die Möglichkeit eines irdischen Ursprungs an, und es erschien ihm hierbei als völlige unerklärlich, wie die nemlichen wägbaren Stoffe, die nach der kosmischen Lehre innerhalb unserer irdischen Atmosphäre nicht sollten vorhanden sein können, dennoch in dem den freien Weltraum erfüllenden Äther, also in einem noch unendlich feineren Medium, sollten anzutreffen sein. Daher war denn auch Egen* vornehmlich bemüht, durch statistische Berechnungen nachzuweisen, welche ungeheure Mengen fester Stoffe alljährlich in unseren Hüttenwerken sich verflüchtigen, und somit wirklich in Gasgestalt in unsere Atmosphäre übergehen. Ebenso wies er darauf hin, dass Pflanzen, die in destilliertem, mithin von fremden Stoffen völlig freiem Wasser leben, nichtsdestoweniger Erd- und Eisenteile in ihrem Inneren enthalten: ein Beweis, dass diese Stoffe in der die Pflanzen umgebenden Luft, aus welcher sie sie allein aufzunehmen im Stande waren, auch notwendig enthalten sein müssen. Von Hof suchte endlich vorzugsweise die Ansicht zu bekämpfen, dass die meteorischen Gesteine von außen her als bereits feste Massen in unsere Atmosphäre eindrängen. Denn - so hob er nicht ohne Grund hervor - wäre das beobachtete Erglühen wirklich eine Folge jener ungeheuren Reibung des eindringenden festen Körpers an den einzelnen Luftteilchen unserer Atmosphäre: dann müsste dieses Erglühen auch notwendig immer stärker werden, je mehr der fallende Körper der Oberfläche unserer Erde sich nähert. Denn mit der größeren Nähe an unserer Erde wächst nicht allein die Geschwindigkeit des Falles, sondern auch die Dichtigkeit der Luft, mithin die Reibung selbst und ihre erhitzende Wirkung auf den im Fall begriffenen Körper. Dem ist aber nicht so. Nicht bei seiner Ankunft auf der Erde zeigt sich der Stein in seiner höchsten Gluth, sondern im Gegenteil vorher, und zwar gerade in den höchsten und dünnsten Schichten unserer Atmosphäre. Ebenso wies er darauf hin, dass, wenn auch durch gewaltsame Zusammenpressung von Luft, wie z. B. in dem pneumatischen Feuerzeuge, eine große Hitze erzeugt werde, dies letztere Beispiel mit dem vorliegenden Fall doch in keiner Weise verwechselt werden dürfe. Im pneumatischen Feuerzeug sei die Luft von allen Seiten fest eingeschlossen; in freier Atmosphäre dagegen - ein Punkt, auf den auch Scherer* schon aufmerksam gemacht hatte - vermöchten die einzelnen Teilchen bei ihrer großen Beweglichkeit sofort vor dem fallenden Körper nach allen Seiten hinzuentweichen. Aber auch die Ansicht einer Bildung der Gesteine einzig und allein aus Stoffen unserer Atmosphäre schien ihn nicht zu befriedigen. Daher neigte er denn auch mehr zu der schon von Chladni geäußerten Ansicht von den kosmischen Urnebeln hin, so wie zu der Möglichkeit eines gegenseitigen Austausches der Stoffe zwischen dem freien Weltraum und unserer irdischen Atmosphäre. So viel aber - fügt er endlich hinzu* - gehe aus Allem hervor, dass in demselben Augenblick, wo in unserer Atmosphäre die Lichtentwicklung und die Explosion stattfindet, eine tatsächliche chemisch-physische Operation vor sich gehe, kraft welcher aus dem erglühten Urstoff ein neuer Körper sich bilde, und dieser neue Körper sei der herabfallende Meteorstein. Inmitten unserer Atmosphäre sei er jedenfalls gebildet: von außen könne er fertig nicht gekommen sein.

*) G. 14. 1803. Fol. 55.

*) G. 6. 1800. Fol. 156.

*) G. 33. 1809. Fol. 189.

*) G. 14. 1803. Fol. 55.

*) G. 72. 1822. Fol. 375.

*) P. 36. 1835. Fol. 161.

*) G. 31. 1809. Fol. 1.

So sehen wir, wie die verschiedenartigsten Ansichten sich äußerten, sich bekämpften, und gegenseitig zur Geltung zu gelangen suchten. Man ist von den Massen geballter und ungeballter Materien im Weltraum, über Nebelflecke und durch Sternschnuppenschwärme, über große und über kleine Planeten herabgestiegen bis zu den Meteorsteinen und Feuerkugeln, ja herunter bis zu unseren Blut- und Staubregen, einzig und allein um für die Meteorsteine einen kosmischen Ursprung zu begründen. Für die Blut- und Staubregen aber ist eine solche außerirdische Herkunft gewiss mehr als zu bezweifeln. Eine wirkliche Identität zwischen Feuerkugeln und Sternschnuppen ist ebenfalls noch keineswegs erwiesen. Denn wenn es gleich hier und dort vorgekommen, dass bei sehr lebhaften Sternschnuppenschwärmen gleichzeitig auch Feuerkugeln beobachtet worden sind: so lehrt doch die Erfahrung, dass Feuerkugeln im Allgemeinen unbegleitet von Sternschnuppen, und auch nicht, wie diese, an bestimmte Perioden gebunden am Himmelszelt erscheinen.* Berücksichtigen wir überdies auch noch die nach angestellten Beobachtungen langsame Bewegung der Feuerkugeln im Vergleich zu der der Sternschnuppen, so wie die nach aller Wahrscheinlichkeit weit größere Entfernung dieser letzteren von der Oberfläche unserer Erde: so darf ein gemeinschaftlicher Ursprung der Feuerkugeln - namentlich derer, die in der Luft zergehen, ohne Steine zu uns herabzusenden - und der zu bestimmten Perioden unsere Erdbahn durchkreuzenden Sternschnuppen gewiss für jetzt noch sehr bezweifelt werden. Allein auch für solche Feuerkugeln, die wirklich in Steine sich auflösen, haben wir gesehen, dass nicht unerhebliche Gründe gegen die Annahme eines außerirdischen Ursprunges vorhanden sind. Zu diesen Gründen ist vorzugsweise der schon oben erwähnte Umstand zu rechnen, dass das sofortige Erglühen der Steine - wenn diese wirklich in einem bereits festen Zustand von außen her in unsere Atmosphäre eindrängen - gerade in den obersten und darum auch noch allerdünnsten Schichten unseres Dunstkreises wohl kaum nach den uns bekannten natürlichen Gesetzen eine befriedigende Lösung finden kann. Denn wollte man auch annehmen, dass jene meteorischen Massen zwar wohl im festen Zustand, aber nicht als fest zusammenhängende Körper, sondern nur im Zustände feinster Verteilung, gleichsam als ein kosmischer Staub oder als ein kosmisches Pulver, im Weltraum sich befänden, und auch in solcher Weise nun in die obersten Schichten unserer Atmosphäre gelangten: so ließe sich hierdurch die große Entzündlichkeit solcher pulverförmigen Massen beim Eintritt in die sauerstoffreichere Atmosphäre unserer Erde zwar befriedigender erklären; allein andere Schwierigkeiten würden dafür auftauchen. Für das wirkliche Vorhandensein fester und dabei doch außerordentlich kleiner Weltkörper innerhalb unserer Sonnensysteme sprechen unsere kleinen Planeten. Auch die Sternschnuppenschwärme scheinen darauf hinzudeuten. Wir kennen in gleicher Weise kosmische Dünste und Nebelflecken, die zum Theil, selbst bei den stärksten Vergrößerungen, noch in keine bestimmten Sternhaufen aufgelöst werden konnten. Aber von solchen kosmischen Staub- und Pulvermassen, wie sie zur Erklärung jener lebhaften Entzündbarkeit gerade in den obersten und dünnsten Gebieten unserer Atmosphäre notwendig sein würden, gewahren wir nirgends die allergeringste Andeutung. Zudem muss es aber auch weiterhin sehr rätselhaft bleiben, wie durch die bloße Anziehung unserer Erde planetarische Körper, die gleich unserem eigenen Erdkörper mit planetarischer Geschwindigkeit um die Sonne sich bewegen, von jenem sollten gänzlich zu sich herabgezogen werden; während doch sonst die Himmelskörper selbst in ihrer größten Nähe sich höchstens nur in ihrer gegenseitigen Geschwindigkeit ein wenig aufhalten, oder in ihrem Laufe nur unbedeutend aus ihren gewöhnlichen Bahnen sich abzulenken vermögen. Wollte man aber annehmen, ein solches Herabstürzen des kleineren Weltkörpers auf den größeren sei in Bezug auf unsere Meteorsteine deshalb doch wohl denkbar, weil diese ungewöhnlich kleinen Miniaturweltkörperchen wohl auch in einer weit größeren Nähe bei unserer Erde ihre Bahnen beschreiben: so würde eine solche Annahme doch jedenfalls nur allein für die spezifisch leichteren unter unseren Meteorsteinen eine Geltung haben können. Denn nach einem bekannten Naturgesetze befinden sich die dichteren und spezifisch schwereren Planeten auch in größerer Nähe bei der Sonne als die spezifisch leichteren. Die mittlere Dichtigkeit des Merkurs gleicht der des Goldes oder des Platins; die der Venus derjenigen des Glases; der Erde des Flussspates u. s. w.* Die metallischen dichten Eisenmassen, welche von Zeit zu Zeit ebenfalls auf unsere Erde herabstürzen, mussten demnach notwendig in einer so bedeutenden Entfernung von unserer Erde ihre Bahnen beschreiben, dass für sie eine solche übermächtige Anziehung unserer Erde wohl kaum für wahrscheinlich zu halten sein dürfte. Sollten sie durch Anziehung wirklich auf einen anderen Planeten hinabzustürzen gezwungen werden, so müsste für sie der anziehende Himmelskörper gewiss weit eher der ihnen nicht allein nähere, sondern auch dichtere Merkur sein, als die von ihnen entferntere Erde. Neigt man sich dagegen aber zu der Ansicht einer Entstehung aus bloßem Urnebel hin, so bleiben nicht allein die Rätsel wegen des Herausreißens aus der ursprünglichen Umlaufsbahn dieselben; sondern es hält auch außerdem schwer, den Grund dafür zu finden, weshalb diese Nebelmassen, die selbst in dem nach angestellten Berechnungen weit über 100$^\circ$ kalten Weltraum noch nicht erstarrt sind, nun mit einem Male in den festen Zustand übergehen, sobald sie in unserer Atmosphäre, also in einem Mittel anlangen, das wohl kaum noch kälter sein dürfte als dasjenige, aus welchem sie stammen, - ja wo sie in Folge der durch ihre Reibung angeblich erzeugt werden sollenden Hitze sofort in eine solche Gluth versetzt werden, dass eine jede Idee an eine auf solchem Wege zu bewirkende Verdichtung gasförmiger Stoffe - wie es scheint - von vornherein ausgeschlossen werden muss. Aber auch gegen die Annahme, als drängen unsere Meteorsteine in bereits festem Zustand aus dem freien Weltraum in den Dunstkreis unserer Erde ein, erhebt sich aus astronomischen Rücksichten eine weitere, bisher zwar noch wenig beachtete, aber doch, wie es scheint, nicht ganz unwesentliche Schwierigkeit. Beschreiben nemlich unsere Meteorsteine als bereits feste planetarische Massen innerhalb unseres Sonnensystems ihre uns unbekannten Bahnen um die Sonne: dann müssen sie notwendig auch alle dieselbe Richtung von West nach Ost einhalten, der alle übrigen Planeten folgen, und die Ebenen ihrer Bahnen müssen gleich denjenigen aller übrigen Planeten mit der ungefähren Richtung des Tierkreises übereinstimmen. Außerdem haben wir alsdann - wie oben bereits angedeutet, - allen Grund, anzunehmen, dass die spezifisch schwereren Gesteinsmassen, also namentlich die meteorischen Eisenmassen, näher bei der Sonne, die spezifisch leichteren dagegen weiter von der Sonne als unsere Erde ihre Bahnen beschreiben. Die der Sonne näheren Himmelskörper, mögen sie nun groß oder klein sein, beschreiben aber bekanntlich mit größerer Schnelligkeit ihren Lauf um die Sonne, als die von der Sonne entfernteren. Wenn daher unsere Erde mit irgendeinem dieser Miniaturweltkörper in solche Nähe kommen soll, dass sie im Stande sei, ihn vermöge ihrer Anziehung zu sich herabzuziehen: dann müsste sie es sein, welche alle langsamer sich bewegenden, d. h. mit anderen Worten alle spezifisch leichteren Massen in ihrem Laufe einholt, unterdes sie von allen sich schneller bewegenden, d. h, spezifisch schwereren, eingeholt wird. Daraus würde nun aber mit Notwendigkeit auch folgen, dass, während alle spezifisch schwereren Meteorsteine und also namentlich alle meteorischen Gediegen-Eisenmassen stets von Westen her auf unserer Erde anlangen würden, im Gegenteil alle spezifisch leichteren, weil von unserer schneller sich bewegenden Erde in ihrem Laufe überholt, dem äußeren Anscheine nach von Osten her zu uns gelangen müssten. Die Erfahrung bestätigt dieses aber keineswegs. Im Gegenteil finden wir, dass die Meteorsteine so ziemlich aus allen Himmelsgegenden bei uns anlangen. Ja selbst in Bezug auf die Gediegen-Eisenmassen ersehen wir aus den uns erhaltenen Aufzeichnungen, dass auch sie nicht einmal die gleiche und beständige Richtung einhalten: der Meteor-Eisenfall von Hraschina (1751) kam aus Nordwesten*; der von Braunau (1847) dagegen aus Nordosten.* Bei dunst- und gasförmigen Massen mögen wir uns nun zwar wohl denken, dass sie - innerhalb unserer Atmosphäre von Winden und Luftströmungen hin- und hergetragen - leicht und häufig die ursprüngliche Richtung ihres Laufs verlassen, und darum auch so ziemlich aus allen möglichen Wind- und Himmelsgegenden nach eingetretener Verdichtung zu uns herabzugelangen im Stande sind. Bei festen Massen dagegen, die mit einer schon an und für sich planetarischen Geschwindigkeit in unseren Dunstkreis eindringen, und deren Geschwindigkeit überdies auch noch in Folge ihres Falles, ungeachtet des Widerstandes der nach allen Seiten hin frei entweichenden Luft, eine fortwährend sich beschleunigende sein muss, dürfte die Annahme einer ähnlichen Einwirkung von irdischen Wind- und Luftströmungen gewiss von vornherein als unstatthaft sich erweisen. Die Gewalt auch der heftigsten Orkane muss als verschwindend erscheinen, gegenüber der ungeheuren Heftigkeit und Schnelligkeit des Falles, womit aus dem freien Weltraum stammende feste planetarische Körper in unseren Dunstkreis eindringen. An ein Herausreißen aus ihrer natürlichen Richtung durch lokale irdische Verhältnisse darf daher bei ihnen gewiss auch nicht im Entferntesten gedacht werden.

*) P. 36. 1835. Fol. 176.

*) A. v. Humboldt. Kosmos 3. Fol. 609 u. 610. RPG Fol. 10 u. 16.

*) Littrow. Wunder des Himmels. 3. Fol. 68.

*) WA. 35. 1859. Fol. 17 u. 18.

*) P. 72. 1847. Fol. 170.

Sollte es nun, nach all diesen Zweifeln und Ungewissheiten, nicht zweckmäßig und erlaubt erscheinen, auch wieder einmal den umgekehrten Weg wie zeither zu versuchen? d. h. anstatt von den uns entferntesten und allerfremdesten Gegenständen, von den Planeten und ihren Urmaterien auszugehen, vielmehr mit den uns bekanntesten und nächsten meteorologischen Tatsachen, wie sie fortwährend hier auf Erden uns umgeben, zu beginnen, und von ihnen aus uns allmählich zu jenen uns noch unbekannteren Naturerscheinungen zu erheben, mit deren Erklärung wir uns eben jetzt beschäftigen?

Die nächste Brücke, um von der Oberfläche unserer Erde in jene luftigen Räume zu gelangen, in welchen jene eigentümlichen Ereignisse stattfinden, bilden wohl jedenfalls die wässerigen Dünste unserer Atmosphäre.* Sie sind die ersten und uns zunächst liegenden Beweise einer ununterbrochenen Wechselwirkung zwischen Stoffen unserer Erde und der diese umlagernden Dunsthülle. In unsichtbarer Gasgestalt erheben sie sich von unserer Erde, werden durch Winde und Luftströmungen in weite Fernen getragen, durch Kälte in den höheren Regionen unserer Atmosphäre wiederum verdichtet, um endlich in Gestalt von Regen, Schnee und Hagel wieder zu uns herabzugelangen. Zwar finden diese Übergänge ohne jene eigentümlichen Verbrennungs- und Feuererscheinungen statt, wie wir solche stets bei der Bildung der Meteorsteine gewahren. Allein die innere Natur der diesen beiden Erscheinungen zu Grunde liegenden Stoffe scheint hinreichend zu sein für die Erklärung dieser Verschiedenheit. Und will man einwenden, dass Regen und Hagel nur in verhältnismäßig kleineren Tropfen und Körnern zur Erde kämen, die meteorischen Gesteine dagegen meistenteils in großen und selbst ungeheuren Massen: so wird eine nähere Prüfung des Tatbestandes uns zeigen, dass auch in dieser Beziehung zwischen beiden Naturerscheinungen kein so großer Unterschied herrscht, als es in dem ersten Augenblick wohl den Anschein hat. Als Regen kommt das atmosphärische Wasser freilich nur tropfenweise zur Erde. Aber selbst diese Tropfen sind oft sehr verschieden an Größe; und richten wir unsere Blicke auf das auf unsere Erde herabkommende meteorische Eisen - die einzigen Massen, welche, wenn auch nicht völlig flüssig, so doch in mehr oder minder weichem Zustande bei uns eintreffen -: so finden wir auch hier tatsächlich dieselbe Tropfenbildung wieder. Das Eisen von Hraschina* ist, wie die Berichte ausdrücklich melden, in Gestalt „feuriger Ketten,“ d. h. in nicht zusammenhängender, sondern in zerrissener, tropfenähnlicher Weise auf unserer Erde angelangt. Aus der Bezeichnung „feurige Ketten“ geht hervor, dass diese Tropfen jedenfalls weit grösser gewesen sei müssen, als unsere gewöhnlichen Regentropfen: ein Umstand, der bei dem nicht völlig flüssigen, sondern nur halbweichen Zustande der fallenden Masse nicht zu verwundern ist. Das zerrissene, unzusammenhängende Herabkommen, also das, was den Tropfen charakterisiert, sehen wir jedenfalls entschieden ausgeprägt. Noch grösser aber wird die Ähnlichkeit zwischen den wässerigen Niederschlägen unserer Atmosphäre und den Naturerscheinungen, welche uns beschäftigen, wenn wir auf den Hagel unsere Blicke lenken. Die Meteorsteinchen im Gewicht von mitunter nur 2 Quäntchen, welche 1803 in ungeheurer Menge zu l'Aigle* herabgefallen sind, werden in Bezug auf Größe und Umfang den Vergleich mit unseren gewöhnlichen Hagelkörnern sehr wohl aushalten. Aber kennen wir nicht auch Schlossen von weit bedeutenderer Größe? 1767 fielen am Comer See* Hagelkörner bis zur Größe von Hühnereiern, und 1819 zu Mayenne bis zu 15'' Umfang. Und trotz dieser Größe wird gerade bei diesen letzteren von Delcross* berichtet, dass es häufig nur Bruchstücke noch größerer, durch irgendeine innere Explosion schon während des Niederfalls gewaltsam zerrissener Eismassen von Kugelgestalt gewesen seien: - ein Umstand, der stark an das so häufig beobachtete Bersten der Meteorsteine in verschiedene kleinere Bruchstücke erinnert, bevor sie noch auf unserer Erde angelangt sind. Indessen sind die eben beschriebenen Hagelkörner noch bei weitem nicht die größten. Am 28. Mai 1802 fiel bei Puztemischel in Ungarn* während eines Hagelwetters ein Eisklumpen zur Erde, der 3 Fuß Länge, 3 Fuß Breite und 2 Fuß Dicke hatte; er ward auf 11 Zentner geschätzt. Ein zweiter hatte die Größe eines Reisekoffers. Doch die größte vom Himmel gefallene Eismasse, die an Umfang und Gewicht wohl nur wenigen Meteorsteinen nachstehen dürfte, ist diejenige, deren L. von Buch* Erwähnung tut, indem er aus Heynes Tracts historical und statistical on India als eine wohlbeglaubigte Tatsache berichtet, dass sie zur Zeit des Tippoo Saheb nahe bei Seringapatam in Indien zur Erde gefallen sei. Sie war von der Größe „eines Elephanten,“ und es vergingen trotz der Hitze des Landes 2 Tage, bis sie vollständig geschmolzen war. Zwar sind bei Hagel Massen von solcher Ausdehnung allerdings nur Seltenheiten. Dieser Umstand findet aber, im Vergleich mit den Meteorsteinen, sicherlich in der Verschiedenheit der zu Grunde liegenden Stoffe und vor Allem in der Ungleichheit ihrer inneren Dichte und der daraus hervorgehenden Verschiedenheit in der gegenseitigen Anziehung der einzelnen Massenteilchen seine hinlängliche Begründung. - Haben wir nun aber einmal mit Regen und Hagel begonnen: so ist der Schritt zu den ihnen sichtbarlich verwandten Blutregen* nur ein kleiner. Hier haben wir schon einen metallischen Stoff, das Kobalt, und zwar in der Form von Chlorkobalt vor uns. Er muss zu der Zeit, wo der Regen sich bildet, und zwar ebenfalls in Dunstform, in unserer Atmosphäre notwendig in Wirklichkeit vorhanden sein. Einen weiteren Beweis, dass derartige metallische Stoffe wirklich bald mehr bald weniger in Gasgestalt in unserer Atmosphäre sich befinden, liefern die Hagelfälle mit festen Metall- oder Steinkernen.* Hier wurden offenbar die durch eintretende Kälte sich verdichtenden Metalldünste die anziehenden Mittelpunkte, um welche die ebenfalls aus der Luft sich ausscheidenden Wasserteilchen sich ansammelten, und auf diese Weise nun eine äußere Eishülle um dieselben bildeten.

*) Shepard, Report on American Meteorites Fol. 52.

*) G. 50. 1815. Fol. 263. WA. 35. 1859. Fol. 364-373.

*) 15. 1803. Fol. 74 u. G. 16. 1804. Fol. 44.

*) P. 13, 1828. Fol. 344.

*) G. 68. 1821. Fol. 323.

*) G. 16. 1804. Fol. 75.

*) 65. 76. 1824, Fol, 342.

Nun wäre aber die wichtigste Frage, wie solche metallische Dünste wohl von unserer Erde aus in unsere Atmosphäre zu gelangen vermögen, und es zeigen sich uns hierfür vornehmlich zwei Wege: einmal durch allmähliche, unserer unmittelbaren Wahrnehmung meist sich entziehende langsame Verdunstung, ähnlich derjenigen unseres Wassers, - und zum Andern durch ein zeitweises massenhafteres Ausströmen aus unseren irdischen, tätigen Vulkanen, namentlich zur Zeit heftiger Ausbrüche; so dass wir vorzugsweise diese Letzteren wohl nicht ohne Grund als die Hauptquellen aller jener mannigfachen mineralischen Grundstoffe zu betrachten hätten, die wir, bald unter der Form von Blut- und Staubregen, bald unter der Form von Meteorsteinen und von Gediegen-Eisenmassen auf unsere Erde herabgelangen sehen. Gehen wir daher, zur näheren Begründung dieser Ansicht, nun in Kürze zu denjenigen Erscheinungen über, wie sie an den in Tätigkeit begriffenen Vulkanen in Wirklichkeit wahrgenommen werden. Von dem Ausbruch des Vesuvs von 1794 besitzen wir von Hamilton* eine besonders ausführliche Beschreibung. Erdbeben und Auswürfe glühender Dämpfe waren seine Begleiter. Eine Riesenwolke von Pinus-ähnlicher Gestalt und voll Feuers lagerte über dem Gipfel des Berges, und durch sie hindurch brach die senkrecht aufsteigende, von schwarzen Wolken und Qualm begleitete Feuersäule sich ihre Bahn. Außer den Blitzen, die nach allen Seiten zuckten, entstiegen der erwähnten Riesenwolke Feuerkugeln von zum Theil beträchtlicher Größe. Diese den Gipfel des Berges überlagernde Wolke findet sich übrigens bei den meisten vulkanischen Ausbrüchen wieder. Ihr verdanken die sogenannten vulkanischen Bomben oder Vesuvstränen* ihren Ursprung: feste Steine von der Größe eines Sperlingseies bis zu der einer Kokosnuss, ja bisweilen bis zu einer Schwere von 40 und 60 Pfd. Ihre Oberfläche ist rau und porös, und ihre äußere Gestalt birnförmige: ein Beweis, dass sie nicht als feste Körper von den Vulkanen ausgeworfen, sondern als wirkliche Erzeugnisse entweder jener vulkanischen Wolke selbst und der in ihr enthaltenen dunstförmigen Stoffe, oder des noch in flüssigem Zustande befindlichen Innern des Vulkanes zu betrachten sind. Die Übereinstimmung mit den wirklichen Meteorsteinen, bei denen ebenfalls in vielen Fällen einer solchen birn-, keil- oder pyramidenförmigen Gestalt Erwähnung geschieht,* ist wohl kaum zu verkennen. Aber die auffallendste und für die gegenwärtige Untersuchung vielleicht lehrreichste Erscheinung berichtet Abbe Tata. Er sah bei dem erwähnten Ausbruch des Vesuvs dem Krater eine Feuerkugel entsteigen,* welche von gewaltiger Größe war. Sie fuhr in großer Höhe über ihm daher, und zerplatzte mit Geräusch zwischen Torre del Greco, Bosco und Torre dell' Annunziata. An derselben Stelle, wo dies geschah, gewahrte er einen großen, senkrechten Streifen, wie ein dichtes Hagelwetter, und er hörte ein Geräusch, wie wenn Steine zur Erde fielen. Und in der Tat erfuhr er bald nachher, dass in jener Gegend damals viele Steine gefallen seien. Hier haben wir also ein merkwürdiges, von einem glaubwürdigen Augenzeugen beobachtetes Beispiel, dass eine einem irdischen Vulkan entstiegene Feuerkugel wirklich in einen wahren Steinregen sich auflöste, und zwar ganz unter denselben Erscheinungen, wie sie uns auch sonst bei Meteorsteinen beschrieben werden. Man hat zwar die Vermutung ausgesprochen, dass eben diese von Abbe Tata erwähnte Feuerkugel weniger eine Zusammenballung glühender Dunst- als glühender flüssiger Massen gewesen sein dürfte, welche gleich den Materialien zu den sogenannten Vesuvstränen aus dem Inneren des Vulkans gewaltsam in die Höhe geschleudert worden seien. Allein wenn dieses auch in Wirklichkeit der Fall ist, so dürfte es eher für, als gegen die Annahme einer näheren Verwandtschaft jener Erscheinung mit den eigentlichen Meteorsteinen sprechen. Denn es würde sich daraus auf natürliche Weise erklären, weshalb diese Feuerkugel schon verhältnismäßig so nahe bei ihrem ursprünglichen Ausgangspunkte in wirkliche Steine sich auflöste, unterdes dieses bei den eigentlichen, den vulkanischen Dünsten entstammenden Meteorsteinen erst in weit größeren Fernen der Fall ist. Denn dass vulkanische Ausbrüche stets auch von Ausströmungen wirklich gasförmiger Massen begleitet sind, kann auf keine Weise in Zweifel gezogen werden. Aus den ausströmenden Laven entwickeln sich Dämpfe und Gase, und während ihres Erkaltens hört man nicht selten laute Explosionen und heftiges Krachen. Die Bewohner jener Gegenden versichern, dass man oft aus diesen Laven Dämpfe aufsteigen sähe, die sich in der Luft entzündeten, und dann gleich Sternschnuppen wiederum herabfielen.

*) G. 64. 1820. Fol. 335.

*) G. 72. 1822. Fol. 436. G. 31. 1809. 307. u. P. 28. 1833. Fol. 570.

*) G. 5 1800. Fol. 408. G. 6. 1800. Fol. 21.

*) G. 63. 1819. Fol. 55.

*) P. 94. 1854. Fol. 169. P. 60. 1843. Fol. 157. P. 72 Suppl. Fol. 376. G. 23. 1806. Fol. 93. G. 24. 1806. Fol. 261. G. 41. 1812. Fol. 96. WA. 40. 1860. Fol. SJ. 49. 1845. Fol. 339.

*) G. 6. 1800. Fol. 168.

Aber nicht allein in Bezug auf diese äußeren Verhältnisse, auch in Hinsicht ihrer inneren Zusammensetzung zeigen sich, trotz mannigfacher Verschiedenheiten, große Ähnlichkeiten zwischen unseren Meteorsteinen und den Produkten unserer Vulkane. Die durch Vulkane ausgeworfenen Aschen werden als sandig und eisenhaltig beschrieben. Die Laven des Vesuvs enthalten nach Bergmann* Kieselerde, Tonerde, Kalkerde, Eisen und Kupfer, also lauter Stoffe, die uns auch von den Meteorsteinen her wohl bekannt sind. Viele Laven sollen sogar stark magnetisch sein, und diese Eigenschaft kommt - wie der Stein von Nord-Carolina* von 1820 dartut, der deutliche Nord- und Südpolarität zeigte - hin und wieder auch bei Meteorsteinen vor. Selbst Olivin und stärke Spuren von reduziertem Eisen hat Hermann in Moskau* in den Laven des Vesuvs nachgewiesen; und auf die große Ähnlichkeit der Steine von Invinas und Stannern mit den Doleriten vom Meissner in Hessen hat nach Rammelsberg schon Mohs, so wie auf deren Ähnlichkeit mit den Basalten vom Rautenberge in Mähren noch neuerlich v. Reichenbach* aufmerksam gemacht. Rummelsberg wies Augit und Labrador, beides Bestandteile unserer irdischen plutonischen Gebilde, in den Meteorsteinen nach; und Nickel, dieses Hauptmerkmal eines meteorischen Ursprungs, fand Stromeyer* in den Olivinen des Vogelsberges. Bittererde ist nach Breislack* in allen vulkanischen Materien vorhanden. Dass endlich auch der ungeachtet seiner leichten Verbrennlichkeit in allen Meteorsteinen nie gänzlich fehlende Schwefel eines der hauptsächlichsten Produkte unserer Vulkane ist, ist bekannt. Diese Übereinstimmung in den Grundstoffen ist so auffallend, dass sie in der Tat nicht wenig für einen gemeinsamen Ursprung beider Naturerzeugnisse zu sprechen scheint. Jedenfalls sehen wir, dass wir das sämtliche Material zum Aufbau unserer Meteorsteine so vollständig hier bei uns auf Erden vorfinden,* dass wir noch nicht genötigt sind, dasselbe erst vom Monde oder aus dem fernen Weltenraum herbeizuholen, um deren Ursprung zu erklären. Zwar ist es nicht zu leugnen, dass bei all diesen Ähnlichkeiten, bei all dieser auffallenden Übereinstimmung in den Grundstoffen, dennoch auch manche und nicht unbedeutende Verschiedenheiten obwalten; namentlich in Bezug auf die innere Struktur der Gesteine. Man hat in der Nähe der Vulkane noch durchaus keine Steine angetroffen, die mit den in entfernteren Gegenden aus der Luft gefallenen Meteorsteinen in Allem völlig übereinstimmten. Allein berücksichtigen wir die große Verschiedenheit in den Verhältnissen, unter denen die Steine endlich ihre letzte Ausbildung erlangt haben und in die feste Aggregatform übergegangen sind: so darf uns jene Verschiedenheit im inneren Bau, selbst bei sonst gemeinschaftlichem Ursprung, wohl nicht so sehr wundern. Die Laven bilden wahrscheinlich nicht den eigentlichen flüssigen Kern unserer Vulkane, sondern nur die dem feurig-flüssigen Metallkerne aufschwimmenden schlackenähnlichen Massen. Nicht in gasförmigem Zustand, sondern nur in feurig-flüssiger Gluth entquellen sie aus einer wahrscheinlich verhältnismäßig nur geringeren Tiefe dem Inneren des Vulkans; unterdessen die metallischen Gase und Dämpfe, die zu unseren meteorischen Gebilden die erste und eigentliche Grundlage bilden dürften, gewiss einer weit bedeutenderen Tiefe ihren Ursprung zu verdanken haben. Durch die Kraft der vulkanischen Gewalten in ungewöhnliche Höhen geschleudert, und hier durch Luftströmungen in weit entlegene Gegenden fortgeführt, muss ihr Übergang aus dem gasförmigen Zustand in den festen notwendig unter ganz anderen äußeren Umständen und Verhältnissen vor sich gehen, als dieses auf der unmittelbaren Oberfläche unserer Erde bei den Vulkanen in flüssigem und vielleicht selbst in nur erst weichem Zustand entströmenden und darnach langsam und ruhig erkaltenden Laven der Fall ist. Eben so wenig kann aber auch der Umstand, dass die aus dem Inneren unserer Vulkane aufsteigenden Dämpfe häufig schon an den inneren Wänden der Krater sublimieren, und dass in diesen Sublimationen noch niemals weder gediegenes Eisen noch Nickel gefunden worden, einen Beweis gegen die Möglichkeit der bisherigen Annahme bieten. Denn diejenigen Sublimationen, welche bei Besuchen von Kratern, also zur Zeit ihrer Untätigkeit, an ihren inneren Wänden gefunden werden, haben sich sicherlich auch nur während der Zeiten der Ruhe hier angesetzt. Nur in diesem Falle ist es möglich, dass die steinigen Kraterwände einen so niedrigen eigenen Wärmegrad besitzen, dass an ein Niederschlagen gasförmiger Stoffe an ihrer Oberfläche kann gedacht werden. Dass aber solche Ausbauchungen, wie sie wohl jederzeit bald mehr bald weniger stark bei allen noch tätigen Feuerbergen vorkommen, gerade während der Zeiten größerer Ruhe keine oder nur sehr wenige metallische Dämpfe mit sich führen, sondern nur aus leichter zu verflüchtigenden Stoffen bestehen können: dieses bedarf wohl kann der Erwähnung. Eisen und Nickel verlangen gleich allen übrigen Metallen die allerhöchsten Wärmegrade, um in den gasförmigen Zustand übergeführt zu werden. Nur zur Zeit der höchsten Aufregung und während der größten Tätigkeit der Vulkane ist aber solch ein übermäßiger Wärmegrad vorhanden, und wenn dieses der Fall ist, alsdann erstreckt er sich auch gewiss nicht einzig und allein auf das in Aufregung begriffene tiefste Innere der Feuerberge, sondern auch ihre Krater müssen in gleicher Weise mit Notwendigkeit davon ergriffen werden. Wie kenn aber unter solchen Umständen auch nur noch im Entferntesten an ein Niederschlagen von metallischen oder sonstigen Dämpfen an den inneren Wänden eines Kraters zu denken sein? Und lehrt uns nicht auch überdies noch die Erfahrung, dass, wie sich im Innern der Vulkane Niederschläge vorfinden, die keine Spur von Eisen und Nickel aufzuweisen haben, es ganz ebenso auch wirkliche Meteorsteine gibt, die als völlig eisen- und nickelfrei sich darstellen? Schon in den Steinen, welche 1819 zu Jonzac und Barbézieux,* Depart. de la Charente et de la Charente-Inferieure, fielen, ist das Eisen mit bloßem Auge nicht mehr sichtbar: nur auf künstlichem Wege ist es zu entdecken. Auch die Steine vom Bokkeveld* am Cap der guten Hoffnung (1838), die von Alais und Valence* in Südfrankreich (1806), welche Letztere nur ein spez. Gew. von 1,94 bis 1,70 besitzen, sowie diejenigen von Lontalax* in Finnland (1813) enthalten nur überaus schwache Spuren von Eisen. Die Steine von Stannern* in Mähren dagegen (1808), bekannt wegen ihres überaus lockeren und sandsteinartigen Gefüges, zeigen auch nicht mehr die geringste Menge von Eisenteilchen, welche durch den Magneten künstlich sich herausziehen ließen. Und ebenso werden auch die Steine von Langres,* Depart. de la Haute-Marne (1815), als völlig frei von metallischem Eisen und Nickel beschrieben. Man sieht aus diesen Beispielen, wie wenig aus dem oben angedeuteten Einwurf, sobald man der Sache näher auf den Grund geht, ein Anhaltspunkt gegen den vulkanischen Ursprung der Meteorsteine sich ergeben dürfte. Im Gegenteil, da eine weitere und gewiss nicht unwesentliche Ähnlichkeit zwischen den Erzeugnissen unserer irdischen Vulkane und den zahlreichen wirklich vom Himmel gefallenen Steinen aus dem angestellten Vergleiche unzweifelhaft hervorgeht: so dürfen wir in den eben angeführten Tatsachen wohl eher noch einen Grund mehr für als gegen die aufgestellte Ansicht erblicken. Eben so wenig dürfte aber auch die zum Teil ungeheure Größe mancher Meteorsteine und namentlich der oft mehrere Hunderte von Zentnern schweren Eisenmassen gegen die Möglichkeit eines solchen vulkanischen Ursprunges sprechen. Man ist zwar zu der Annahme geneigt, dass schon um des ungeheuren Umfanges willen, den solche namhafte Massen in Gasgestalt notwendig einnehmen müssen, unsere Atmosphäre nicht im Stande sei, sie in luftförmigem Zustande in ihrem Innern zu beherbergen. Allein auch diese Vermutung dürfte sich als ungegründet erweisen, sobald wir die folgende Tatsache berücksichtigen. Nach dem oben erwähnten Ausbruch des Vesuvs fand man auf den Laven eine bedeutende Menge eines Salzes als Sublimation niedergeschlagen. Es wird berichtet, dass viele 100 Zentner* dieses Salzes durch die Bauern in die Stadt gebracht worden seien, sowie das außerdem noch eine weit größere Menge desselben in die Luft davongegangen sein müsse. Ist nun auch das Letztere bloß eine Vermutung, so bleibt doch jedenfalls die vorherige Gasform der wirklich zur Stadt gebrachten vielen 100 Zentner eine Tatsache, und wir können daraus abnehmen, welche ungeheure Quantitäten von Stoffen unsere Atmosphäre selbst innerhalb eines verhältnismäßig kleinen Raumes in Gasform in sich aufzunehmen und - sei es nun längere oder kürzere Zeit - auch in sich zu beherbergen im Stande ist. Und sollte nun Dasjenige, was hiernach bei gasförmigen Salzen offenbar ganz ebenso möglich ist wie bei den wässerigen Bestandteilen unserer Atmosphäre, nicht auch bei gasförmigem Eisen für ebenso möglich zu halten sein?

*) G. 5. 1800. Fol. 408.

*) G. 41. 1812. Fol. 449.

*) P. 28. 1833. Fol. 574.

*) P. 60. 1843. Fol. 130. P. 106. 1859. Fol. 476.

*) P. 28. 1833. Fol. 575.

*) G. 6. 1800. Fol. 33.

*) B. Fol. 155-157.

*) G. 68. 1821. Fol. 335.

*) P. 47. 1839. Fol. 384.

*) G. 24. 1806. Fol. 189.

*) P. 33. 1834. Fol. 30.

*) G. 29. 1808. Fol. 226.

*) G. 58. 1818. Fol. 171.

*) G. 6. 1800. Fol. 32.

Auch das bekannte Gesetz von der Diffusion der Gase, nach welchem alle gasförmigen Stoffe, ohne Unterschied ihrer inneren stofflichen Natur, gegenseitig völlig gleichförmig sich durchdringen und gleichmäßig über gegebene Räume sich verbreiten, - auch dieses Gesetz, aus welchem gewiss eines der ersten und begründetsten Bedenken gegen die Richtigkeit der dargelegten Ansicht sich ableiten ließe, dürfte gar leicht in dem weiten Gesamtbereiche unserer Atmosphäre den verschiedenartigsten Modifikationen unterworfen sein. Diese gegenseitige Vermischung verschiedener Gasarten kann jedenfalls nur allmählich vor sich gehen, und es kann daher auch keinem Zweifel unterworfen sein, dass namentlich in solchen Fällen, wo massenhafte Ausströmungen von Gasen und Dämpfen stattfinden, wie bei unseren vulkanischen Ausbrüchen, diese allgemeine Verteilung der einzelnen Gasteilchen unter die übrigen Luftteile unserer Atmosphäre umso langsamer von Statten gehen muss, je bedeutender diese aufsteigenden Gasmassen an und für sich sind, und je grösser zugleich die anziehende Kraft ist, mit welcher nach ihrer eigenen stofflichen Natur ihre einzelnen Teilchen auf einander einzuwirken im Stande sind. Das obige Beispiel scheint hierfür zu sprechen. Und kommt es nicht schon in Bezug auf die wässerigen Bestandteile unserer Atmosphäre vor, dass dieselben selbst in ihrem gasförmigen Zustand zu ein und derselben Zeit in der einen Gegend reichlicher sich vorfinden als in einer anderen? Sollten wir da nicht annehmen dürfen, dass namentlich metallische Dünste und Dämpfe, sobald sie schon von Anfang an in größeren und kompakteren Massen aus den Schlünden unserer Vulkane sich erheben, auch eine weit längere Zeit unverteilt und unvermischt mit den übrigen Luftarten unserer Atmosphäre in dieser Letzteren sich zu erhalten vermögen, als dieses der Natur der Sache nach im Kleinen bei unseren gewöhnlichen physikalischen Versuchen der Fall ist? Diese gegenseitige Vermischung mit den übrigen Luftteilen unserer Atmosphäre kann jedenfalls nur da allmählich vor sich gehen, wo jene metallischen und erdigen Dunstmassen an ihren äußersten Grenzen mit dieser Letzteren unmittelbar in Berührung stehen. Nur von hier aus kann sie allmählich immer weiter nach dem Innern vordringen, und wir dürfen wohl nicht ohne Grund annehmen, dass dieses umso langsamer geschieht, je grösser die Kraft ist, mit welcher die metallischen Gasteilchen gegenseitig sich einander anziehen. Während daher an den äußersten Grenzen solcher metallischen oder erdartigen Dünste und Dämpfe allerdings eine fortwährende Diffusion, eine fortwährende Vermischung mit den übrigen Luftteilen stattfindet und notwendiger Weise stattfinden muss, mag nichtsdestoweniger ihr eigentlicher innerer Kern derselben Vermischung je nach der ursprünglichen Masse und Natur der Stoffe für längere Zeit widerstehen. Schon unsere gewöhnlichen Feuerkugeln scheinen nicht wenig für ein solches Beisammenhalten der sie bildenden gasförmigen Stoffe zu sprechen; wogegen auf der anderen Seite die öfters beobachteten und nach den angestellten Untersuchungen aus denselben Stoffen wie unsere Meteorsteine bestehenden Staubregen* uns höchstwahrscheinlich ein Bild von denjenigen Vorgängen vor die Augen führen, welche eintreten sobald der Übergang aus dem luftförmigen Zustand in den festen nicht wie bei den eigentlichen Meteorsteinen schon vor, sondern erst nach der wirklichen Zerstreuung der ihnen zu Grunde liegenden metallischen und erdartigen Dünste unter die übrigen Luftteile unserer Atmosphäre stattgefunden hat. Auch jener Regen von feinen schwarzen, wahrscheinlich aus Eisenoxydoxydul bestehenden Eisenkügelchen, welche am 14. Nov. 1856 60 geogr. Meilen südlich von Java auf das nordamerikanische Schiff Joshua Bates niedergefallen, und welche von Ehrenberg für Auswürflinge eines Javanischen Vulkanes, von v. Reichenbach aber für die Ergebnisse eines vorüberziehenden, funkensprühenden Eisenmeteores gehalten werden,* dürften vielleicht nicht unwahrscheinlich in ähnlichen Verhältnissen ihre natürlichste Erklärung finden.

So scheint denn nach allen diesen Beispielen und Tatsachen ein innerer und tieferer Zusammenhang zwischen vulkanischer Tätigkeit, Feuerkugeln und Steinfällen wo schwerlich ganz und gar zu verneinen zu sein. Dass Feuerkugeln nicht selten als Begleiter von Erdbeben beobachtet werden,* ist bekannt; in vulkanischen Gegenden werden sie geradezu als die Vorboten von Erderschütterungen betrachte. Wie weit aber der innere Wirkungskreis vulkanischer Tätigkeit, wie diese in den Erdbeben uns entgegentritt, zuweilen von seinem ursprünglichen Sitz und Herde sich entfernt, davon liefert unter Anderem das Erdbeben vom November 1827* ein sprechendes Beispiel. Von Columbia in Südamerika erstreckte es sich durch Europa bis nach Sibirien, also bis in eine Entfernung von nahe 1900 geogr. Meilen. Auch das Erdbeben, welches am 1. Nov. 1755 Lissabon zerstörte, verbreitete sich in seinen Wirkungen von Westindien und Nordafrika bis nach Finnland, also über eine Strecke von nahe 1500 Meilen.* Eine Ausdehnung über so ungeheure Länderstrecken ist aber kaum erklärlich, wenn wir nicht annehmen, dass die erste Ursache der ganzen Erscheinung in einer sehr bedeutenden Tiefe und also auch in einer sehr bedeutenden Entfernung von der Oberfläche unserer Erde ihren eigentlichen Sitz gehabt habe. Und sollte es nun, bei solcher Tiefe, wirklich als eine Unmöglichkeit erscheinen, dass von hier aus auch selbst die schwerflüssigsten Metalle und Gesteine in Gasgestalt sollten emporgeschafft werden können? Dass aber in einem solchen Falle die emporgeschleuderten metallischen und erdigen Gase nicht immer in diesem ihrem gasförmigen Zustand verweilen, sondern dass sie, nach ganz denselben Gesetzen und aus ganz denselben Ursachen wie die in unserer Atmosphäre gelösten wässerigen Dünste, sich endlich wieder verdichten und wie Jene, der freien Anziehung ihrer Teilchen folgend, nun auch zu äußerlich sichtbaren Dunst- und Wolkenmassen sich gestalten müssen: dieses kann wohl Niemanden wundern. Die matte Wolke, die am nächtlichen Himmel sich zeigenden Lichtstreifen, die bis jetzt stets als die ersten Anzeichen eines Meteorsteinfalles beobachtet worden, verraten uns dies erste Stadium der vor sich gehenden Wiederverdichtung. Wie aber die wässerigen Dünste unserer Atmosphäre nicht sogleich und unmittelbar nach ihrem ersten Hervortreten aus der vorigen Gasgestalt auch schon als Regen oder Hagel zu uns herabkommen, sondern noch längere Zeit in gewissen Höhen als Wolken sich zu behaupten vermögen: so scheint ein Gleiches auch bei den metallischen und erdigen Dünsten der Fall zu sein. Dass aber hierdurch ebenso gut für sie wie für die wässerigen Dünste die Möglichkeit gegeben ist, durch Winde und Luftströmungen über beträchtliche Länderstrecken dahingeführt zu werden, und somit die letzten Endergebnisse ihrer wachsenden Verdichtung meist erst in weiter Entfernung von ihrer wahren Heimat wieder zur Erde gelangen zu lassen: dieses ist wohl ebenfalls kaum zu verkennen. Jenes um völlig klaren Himmel plötzlich erscheinende und nun au Umfang immer weiter zunehmende Wölkchen ist schwerlich die eben erst ihren luftförmigen Zustand verlassende, sondern wahrscheinlich nur die in Folge ihrer zunehmenden spezifischen Schwere allmählich aus ihrer vorigen Höhe mehr und mehr sich herabsenkende, schon früher in den blasigen Wolkenzustand übergetretene, aber erst jetzt durch ihre allmähliche Annäherung den Erdbewohnern sichtbar werdende Dunstmasse. Aus den mannigfachsten Stoffen und Materien gebildet, haben hier die chemischen Kräfte mit ihren gegenseitigen Anziehungen den freiesten und ungehindertsten Spielraum. Mehr und mehr muss das Verwandte sich dem Verwandten zugesellen, und ohne Gefahr zu irren, dürfen wir wohl dem Gedanken Raum geben, dass schon hier, in diesen noch dunstförmigen Anhäufungen metallischer und erdiger d. h. chemisch entgegengesetzter Stoffe, im bunten Spiel und wechselnden Kampf der Elemente die erste Grundlage zu jener eigentümlichen Anordnung der Stoffe und zu jenem eigentümlichen natürlichen Gewebe gelegt werde, welche die meisten Meteorsteine ungeachtet der Ähnlichkeit der Bestandteile doch so wesentlich vor den übrigen Gesteinen unserer Feuerberge auszeichnen. In Folge dieser fortschreitenden Verdichtung und der damit Hand in Hand gehenden chemischen Verbindungen müssen nun aber gleichzeitig - je nach der Natur der hierbei tätigen Stoffe - Mengen von Wärme in Freiheit treten, welche das plötzliche Erglühen und Verbrennen der Masse, so wie ihr Zusammenballen zur glühenden Feuerkugel wohl erklärlich machen. Aber auch elektrische und magnetische Kräfte* müssen in Folge aller dieser Vorgänge nicht minder sich regen, und jene Blitze und raketenähnlichen Zuckungen, welche bei solchen Erscheinungen wahrgenommen werden, sind wohl mit Recht als die sprechenden Zeugnisse hierfür zu betrachten. Es ist das Ringen der Materie nach Gestaltung, welches wir hier in großartigster Weise vor Augen haben. Aber während aller dieser rasch aufeinander folgenden Vorgänge verfolgt auch die Feuerkugel, meist mit großer Schnelligkeit, ihren Weg, und stehende oder nur sehr langsam dem Hauptkörper nachziehende, allmählich bald mehr bald minder rasch verschwindende Lichtstreifen bezeichnen gleich einem Lichtschweife* die zurückgelegte Bahn des Meteors. Diese Lichtschweife pflegen zwar in den meisten Fällen schon nach wenigen Sekunden oder Minuten zu verschwinden; doch finden sich auch Beispiele von bedeutend längerem Anhalten. Diejenigen des Meteors von Hraschina (1751) waren noch 3 1/2 Stunden nach dem Herabfallen der Eisenmessen an dem Himmelszelte sichtbar.* Es ist dieses wohl kaum eine andere Erscheinung als diejenige, welche wir unter veränderten und doch ähnlichen Verhältnissen auch bei unseren gewöhnlichen Wolken wahrnehmen. Auch hier bemerken wir bei aufmerksamer Beobachtung ein allmähliches Wiederauflösen und Wiederverschwinden ihrer äußersten Teilchen. Dieselbe Verdunstung, wie sie allenthalben langsam aber ohne Unterbrechung auf unserer Erde stattfindet, findet auch dort statt in jenen höheren Regionen: die äußersten und dadurch mehr vereinzelten Dunstteilchen folgen der auf sie einwirkenden Kapillaranzieheng der sie umgebenden Luftmasse, und zwischen die atmosphärischen Luftteilchen sich eindrängend, nehmen sie hier von Neuem ihre luftförmige Gestalt an. Ganz das Gleiche ist es, was wir auch in dem allmählichen Verschwinden jener feurigen Licht- und Wolkenstreifen vor unseren Augen haben. Der ganze Unterschied besteht allein in der Ungleichheit der dabei tätigen Stoffe.

*) G. 68. 1821. Fol. 350. G. 53. 1816. Fol. 369. G. 64. 1820. Fol. 327.

*) P. 106. 1859. Fol. 476 bis 490.

*) G. 14. 1803. Fol. 55 u. s. w.

*) P. 21. 1831. Fol. 213 u. s. w.

*) Kant, Geschichte des Erdbebens von 1755.

*) WA. 35. 1849. Fol. 11.

*) P. 83. 1851. Fol. 467.

*) WA. 35. 1859. Fol. 384. WA. 37. 1839. Fol. 808-813.

Ebenso ist es nun aber auch natürlich, dass je nach der stofflichen Verschiedenheit der ein solches Gasgemenge bildenden Bestandteile die ganze chemische Tätigkeit und der ganze Akt der Verdichtung ein verschiedenes Endergebnis zur Folge haben muss. Kamen die vulkanischen Gase ursprünglich aus einer sehr beträchtlichen Tiefe, so müssen ohne Zweifel vorzugsweise die Gase metallischer Stoffe, also diejenigen von Eisen und Nickel es sein, die in dem gesamten Gemenge vorherrschen; die Gase erdartiger Substanzen müssen dagegen im Vergleich zu Jenen in Bezug auf ihre Menge zurücktreten. War hingegen die Tiefe, der jene Gase entstammen, eine minderbedeutende, so muss mehr und mehr das umgekehrte Verhältnis stattfinden. Im ersteren Fall werden meteorische Eisenmassen, im anderen basalt- und doleritähnliche Gesteine als das Endergebnis der eintretenden Wiederverdichtung sich bei uns einstellen. In beiden Fällen aber geht aus dem so verschiedenen Wärmefassungsvermögen der zusammenwirkenden Stoffe mit Notwendigkeit hervor, dass nicht alle Bestandteile des werdenden Meteoriten zugleich und auf einmal in den festen Zustand überzugehen im Stande sind. Mit den erdigen Stoffen muss die Wiederverdichtung beginnen; das metallische Eisen und das Nickel müssen sie beschließen. Das innere Gefüge fast aller bis jetzt bekannt gewordenen Meteorsteine und meteorischen Eisenmassen bestätigt die Richtigkeit dieser Vermutung. Denn ein jeder der eisenhaltigeren Meteorsteine zeigt bei gut bewerkstelligter Politur, dass überall die feinen Eisenteilchen die Steinsubstanz umhüllen und sich in die Fugen und spitzen Winkel zwischen ihr hineinlegen; nirgends aber zeigt sich das umgekehrte Verhältnis, nämlich dass die Steinsubstanz das Eisen umfange. Ebenso zeigen auch die meteorischen Eisenmassen, dass allenthalben die Eisenlegierungen schichtenweise sich um die früher erstarrten Olivine herumgeordnet haben. In Folge aller dieser Tatsachen kommt denn auch von Reichenbach zu dem Schluss, dass nicht allein alle Stoffe, aus denen unsere Meteorsteine gebildet, einst in einem völlig gasförmigen Zustand, sondern dass namentlich auch die erdigen Bestandteile unserer gediegenen Eisenmassen einst inmitten einer Atmosphäre von wirklichem Eisengas* sich befunden haben müssen. In gleicher Weise erklärt sich nun aber auch aus allen diesen Verhältnissen, wie trotz der großen Schnelligkeit des Falles die innere Kristallisation, namentlich bei den Gediegen-Eisenmassen, im Allgemeinen mit so großer Regelmäßigkeit von Statten gehen konnte. Je vorherrschender die Metalle, eine umso größere Hitze muss bei dem Übergang aus dem luftförmigen Zustand in den festen sich entwickeln. Darum werden denn auch vorzugsweise die gediegenen Eisenmassen es sein, welche wir, wenn auch nicht wirklich tropfbar flüssig, so doch häufig in einem noch zähen oder halbweichen Zustande zu unserer Erde herabkommen sehen. Das kettenähnliche Herabfällen der Eisenmassen von Hraschina legt hierfür Zeugnis ab. In eben diesem noch halbweichen Zustande und der damit verbundenen ruhigeren Erkaltung müssen wir aber einen Hauptgrund für die so regelmäßige Darstellung des kristallinischen Gefüges erblicken, welches die meteorischen Eisenmassen uns stets in ihrem Innern zeigen. Mit Scheidewasser geätzt und dann poliert, zeigen sie jenes blätterig-kristallinische, aus lauter kleinen vierseitigen, bald völlig würfelförmigen, bald rhomboedrischen Täfelchen gebildete Gefüge, welches unter dem Namen der Widmannstätten'schen Figuren* als eines der hauptsächlichsten Kennzeichen für meteorisches Eisen bekannt ist. Auch die neuerlich bei Hainholz* unweit Borgholz im Paderbornischen aufgefundene gleichsam auf der Grenze zwischen Meteoreisen und Meteorsteinen stehende Gesteinsmasse zeigt in ihrem Inneren Krystalle von einer solchen Größe und Ausbildung, wie sie bis jetzt bei ähnlichen Gebilden noch nicht beobachtet worden. Was nun die wirklich erdigen und basaltähnlichen Gesteine betrifft, so kommen sie zwar ebenfalls meist immerhin heiß, aber fast alle bereits völlig fest und hart auf unserer Erde an. Bis jetzt sind nur wenige Fälle von dem Gegenteil bekannt: der Stein von Weisskirchen* (Belaja-Zerkwa) in Russland (1796), die Steine von Piacenza* in Italien (1808), und diejenigen von Cold Bokkeveld* am Cap der guten Hoffnung (1838). Von Ersterem wird berichtet, dass er geschmolzen und in feuriger Gestalt herabgekommen sei. Die Steine von Piacenza waren brennend heiß auf unserer Erde angelangt, und an einem von ihnen entdeckte man beim Auffinden einen auf der Erde befindlichen Kiesel fest eingeklemmt: ein Beweis, dass er selbst noch nicht völlig fest und hart gewesen sein konnte, als er auf dem Boden mit Letzterem zusammentraf. Eine ähnliche Tatsache ist auch von der Gediegen-Eisenmasse von Bahia* in Südamerika bekannt: auch hier finden sich in Löchern und Höhlungen der Grundfläche fremde Quarzstücke eingekeilt. Die Steine von Cold Bokkeveld endlich waren Anfangs noch sehr weich und wurden erst später etwas fester.

*) P. 108 1859. Fol. 452, 459 u. 464.

*) G. 50. 1815. Fol. 257-263. P. 36. 1835. Fol. 161 u. s. w. WA. 35. 1859. Fol. 361 u. 387.

*) P. 101. 1857. Fol. 311-313.

*) G. 31. 1809. Fol. 307.

*) G. 72. 1822. Fol. 366.

*) WA. 35. 1859. Fol. 11.

Eine Feuerkugel, die unserem Auge etwa von der Größe eines Vollmondes erscheint, muss nach angestellten Berechnungen in Wirklichkeit eine Dicke von mindestens einer Meile besitzen. Wie klein erscheinen dagegen in ihrem Gesamtumfang und in ihrer Gesamtmasse die Steine, welche aus einer solchen Feuerkugel zu uns herabkommen. Dürfte nun aber wohl leicht eine einfachere und natürlichere Erklärung für eine so plötzliche und so bedeutende Verminderung des räumlichen Umfanges sich finden, als diejenige, welche in eben diesem plötzlichen Übergang aus einem so wenig dichten Zustand, wie der der Luft- oder Dunstform ist, in den der Festigkeit in einer so naturgemäßen Weise sich darstellt? Aber nicht allein hierfür - auch noch für eine andere, nicht minder wichtige und auffallende Tatsache in der Geschichte der Meteorsteine dürfte dieses plötzliche Festwerden ihrer vorher noch dunst- oder gasförmige Stoffe uns einen vielleicht nicht unwichtigen Fingerzeig bieten. Nehmen wir an, dass die Meteorsteine bereits fertige, in dem freien Weltraum ihre Bahnen beschreibende kleine Himmelskörper sind: dann müssen wir wohl auch annehmen, dass die Ablenkung aus ihrer ursprünglichen Bahn, welche sie durch die Nähe unserer Erde erleiden sollen, nicht eine plötzliche, sondern nur eine allmähliche sein kann. Die Anziehung unserer Erde wirkt umso schwächer, je weiter der angezogene Körper noch von der Oberfläche unserer Erde entfernt ist; sie wächst in steigendem Grade, je mehr dieser unserer Erde sich nähert. Ein mit planetarischer Geschwindigkeit in der Nähe unserer Erde in einer Planetenbahn an dieser vorüberziehender Körper wird also wohl kaum mit Einem Male in einer fast senkrechten Richtung auf unsere Erde herabstürzen können; sondern in einer allmählich unserer Erde sich nähernden krummen Linie wird er bei uns ankommen müssen. Diese Krümmung nach unserer Erde zu wird allerdings umso stärker werden, und die Richtung der Bahn also auch umso mehr der senkrechten sich nähern, je näher der fallende Körper zu unserer Erde herabkommt, d. h. je mächtiger die Anziehung dieser Letzteren auf ihn einzuwirken im Stande ist. Aber nichtsdestoweniger wird diese mit der Erdnähe zunehmende Krümmung oder Herauslenkung aus der ursprünglichen Bahn eine allmähliche sein und bleiben müssen: sie wird nie die Gestalt eines plötzlichen Buges nach Art eines gebogenen Kniees oder eines gebogenen Ellenbogens annehmen können; aus dem einfachen Grunde, weil auch die Anziehungskraft unserer Erde keine plötzlich und stoßweise, sondern eine allmählich wirkende, darum aber auch nur allmählich und nicht stoßweise zunehmende Kraft ist. Allein die wirkliche Erfahrung, die aufmerksame Untersuchung aller Verhältnisse, wie sie bei wirklich beobachteten Steinfällen stattgefunden, lehrt uns gerade das Gegenteil. Die Feuerkugel, aus welcher am 26. Mai 1751 die beiden Eisenmassen von Hraschina hervorgingen, war auf ihrem Zuge auch schon zu Neustadt an der Aich in der Gegend von Nürnberg beobachtet worden. Von da hatte sie - wie Haidinger in den Sitzungsberichten der Wiener Akademie dargetan und durch eine beigefügte Zeichnung erläutert hat - ihren Weg in fast wagerechter und verhältnismäßig nur wenig gesenkter Richtung bis Hraschina fortgesetzt, wo sie dann plötzlich, etwas östlich von diesem Orte und in demselben Augenblick, wo die donnerähnlichen Explosionen stattfanden, in fast senkrechter Richtung in der Gestalt jener glühenden Eisenmassen zur Erde herabstürzte. Hier gewahren wir also kein allmähliches, in regelrechtem Bogen erfolgendes Herabkommen, sondern ein so plötzliches Verlassen der bis dahin verfolgten Bahn, dass nur ein besonderes und ebenso plötzlich wie diese Umbiegung selbst eingetretenes Ereignis die Ursache und die Veranlassung hierzu sein kann. Und sollten wir dieses Ereignis nicht in jener plötzlichen Verdichtung, in jenem plötzlichen Übergang der vorher noch dunst- oder gasförmigen Meteormasse in den Zustand der Festigkeit zu suchen und zu finden haben? Fand aber ein solcher Übergang, wie nach dem ganzen bisherigen Gedankengang zu vermuten ist, in Wirklichkeit statt: dann konnte er nicht bloß von der entsprechenden Volumverminderung begleitet sein; sondern auch die entsprechende und zwar ebenso plötzliche Zunahme des spezifischen Gewichtes der in dem Feuermeteore enthaltenen Massen musste unausbleiblich damit Hand in Hand gehen. Das fast senkrechte Herabstürzen der aus dieser Verdichtung hervorgegangenen Eisenmassen musste somit als die natürliche und unausbleibliche Folge aller jener Vorgänge sich darstellen.

*) G. 68. 1821. Fol. 343.

*) WA. 35. 1859. Fol. 10 u. 22. - P. 106. 1859. Fol. 486.

*) WA. 35. 1859. Fol. 378.

Aber auch noch eine andere Erscheinung muss eine so plötzliche Verdichtung namhafter inmitten unserer Atmosphäre befindlicher Massen von luft- oder dunstförmigen Stoffen in ihrem Gefolge haben. In demselben Augenblick, wo in dem Innern des Feuermeteores die Verdichtung und die Zusammenziehung der dasselbe bildenden Teile stattfindet, muss auch die das Meteor umgebende atmosphärische Luft mit ihrer ganzen Gewalt in die durch jene Verdichtung frei werdenden Räume eindringen, und so erblicken wir denn auch hierin in naturgemäßer Weise den inneren Grund für jene donnerähnlichen Schläge und für jenes petardenähnliche Krachen, welche bis jetzt bei fast allen Meteorsteinfällen beobachtet worden sind. Je grösser übrigens in solchen Fällen die vorhandenen und in ihrer Umwandlung begriffenen Gasgemenge sein mögen, umso weniger dürfen wir erwarten, dass ihre Verdichtung, auch wenn sie wirklich bereits an irgendeiner Stelle ihren Anfang genommen, sich nun sofort und mit Einem Male über die ganze Masse nach ihrer ganzen Ausdehnung verbreite. Im Gegenteil dürfte es als einleuchtend erscheinen, dass gerade diese plötzliche Verdichtung des Einen Teils und die damit verbundene Wärmeentwicklung dazu beiträgt, andere, in ihrer Verdichtung vielleicht noch minder weit vorangeschrittene Teile nicht nur vorübergehend in ihrer weiteren Verdichtung aufzuhalten, sondern sie auch von Neuem wieder in minder dichte Zustände zurückzuführen, als diejenigen sind, in welchen sie sich eben noch befunden. Während also der Eine Teil in Folge der erlangten Schwere von der Gesamtmasse sich trennt und seinem natürlichen Fall sich überlässt, wird der andere, von Neuem erhitzt und spezifisch erleichtert, von Neuem in die Höhe steigen. Gleichzeitig aber gibt dieser Letztere die neu empfangene Wärme in seinem Emporsteigen auch wieder an die ihn umgebenden kälteren Luftschichten ab: es gehen abermals Teile in den festen Zustand über; er senkt sich von Neuem, und es wiederholt sich dasselbe Schauspiel wie vorher, so lange, bis endlich auch der letzte Rest auf unsere Erde herabstürzt. Während aber dieses Alles in rascher Aufeinanderfolge vor sich geht, schreitet auch das ganze Meteor unaufhaltsam auf seinem luftigen Wege voran. Und dieses unausgesetzte Vorwärtsgehen in Verbindung mit dem dabei stattfindenden sprungweisen Auf- und Niedersteigen ist es nun, welches jene hüpfende und springende Bewegung veranlasst, welche - von der Erde aus gesehen - unter dem Namen des Rikoschettierens* bekannt ist, und von welcher Chladni* seiner Zeit behauptet hatte, dass sie als eine Folge des Abprallens der aus dem Weltraum eindringenden Massen von der äußersten Oberfläche unserer Atmosphäre zu betrachten sei. Aber schon Benzenberg* hat darauf hingewiesen, dass in einer Höhe von 10 Meilen, wo doch gewöhnlich die Grenze unserer Atmosphäre angenommen wird, die Luft notwendig schon eine so dünne sein müsse, dass hier an ein Abprallen von derselben schon aus diesem Grunde gar nicht mehr gedacht werden könne. Außerdem wird aber auch bei Gelegenheit des Steinfalles zu Weston* in Connecticut (1807) ganz ausdrücklich berichtet, dass das scheinbare Verlöschen und das darauffolgende wieder in die Höhe Steigen der Feuerkugel jedesmal nach einer unmittelbar vorhergegangenen Explosion stattfand. Drei Explosionen waren es, welche man hörte. Und ganz in Übereinstimmung mit der oben gegebenen naturgemäßen Erklärung entsprachen ihnen 3 Steinfälle und 3 Bogensprünge. Mit der letzten Explosion erfolgte auch der letzte Steinfall. Mit welch einer ungeheuren Gewalt übrigens diese Explosionen vor sich gehen müssen, dieses erhellt daraus, dass dieselben z. B. bei dem Steinfall zu l'Aigle (1803) noch völlig deutlich in einer Entfernung von 30 Stunden Wegs,* ja bei dem zu Hraschina (1751) selbst noch in einem Umkreise von 40 Quadratmeilen,* wenn auch hier nur als Getöse, vernommen worden sind. Aber ebenso geht auch augenscheinlich daraus hervor, dass die Explosionen, und mit ihnen das sie begleitende Auf- und Abwärtsspringen der Feuerkugel unmöglich außerhalb unserer Atmosphäre vor sich gehen können. Gerade durch sie sind wir berechtigt, den Schauplatz des ganzen Phänomens innerhalb des Bereiches unserer irdischen Atmosphäre zu suchen. Der Ballon, der aus höheren Luftkreisen sich herabsenkt, und nun, seinen Ballast plötzlich auswerfend, wieder von Neuem in die Höhe steigt, unterdes er seinen Weg, vom Winde getrieben, in unveränderter Richtung fortsetzt, ist das deutliche Bild dessen, was dort unter minder einfachen und weit großartigeren Verhältnissen, unter Donnerschlägen und Verbrennungserscheinungen, vor sich geht.

*) G. 57. 1817. Fol. 121.

*) G. 68. 1821. Fol. 369.

Gegen die hier entwickelte Ansicht, dass die Meteorsteine einem Übergang aus dem gasförmigen Zustand in den festen in den höheren Schichten unserer Atmosphäre ihr Dasein zu verdanken hätten, hat man eingewendet, dass die dabei stattfindende Wärmeentwickelung eine ganz ungeheure sein müsse, und dass man dennoch beim Herabkommen der Steine, außer ihrer eigenen Wärme, durchaus nichts davon gewahr werde. Allein wir müssen bedenken, dass jene Umwandlung nicht allein höchst wahrscheinlich in einer sehr bedeutenden Entfernung von der Oberfläche unserer Erde vor sich geht, sondern auch in einem Mittel, das als der allerschlechteste Wärmeleiter bekannt ist. Nur durch Strömungen, nicht durch Leitung, vermag die Wärme in luftförmigen Mitteln mit einiger Geschwindigkeit sich zu verbreiten. Die Strömung der durch Hitze erwärmten und erleichterten Luft geht aber nach bekannten Naturgesetzen nur nach oben, d. h. in unserem Falle, nach der dem freien Weltraum zugekehrten Seite. Also nicht nach unserer Erde zu. Es darf uns daher auch nicht wundern, wenn wir von jenen Wärmemengen, wie sie im Augenblick der Verdichtung notwendig frei werden müssen, bei dem nun unmittelbar erfolgenden Niederfall der Steine auf unserer Erde nichts gewahr werden. Ob aber dann später nicht auch jene Wärme allmählich bis zur Oberfläche unserer Erde sich verbreite, und dann auch hier durch ungewöhnliche und außerordentliche Temperaturverhältnisse sich kundgebe: dieses ist eine Frage, die vielleicht nicht so ganz unbedingt zu verneinen sein dürfte. Im Gegenteil scheint sie manche Wahrscheinlichkeit für sich haben. So fanden z. B. bei uns in Europa in den Monaten August und November des Jahres 1810 die Steinfälle von Tipperary, Chersonville und Cap Matapan statt. Auch aus Ostindien und Nordamerika ward von Solchen berichtet. Das Ende des Monates Dezember zeichnete sich aber in demselben Jahre in fast allen Gegenden Europas durch ungewöhnliche Wärme, durch milde Frühlingsluft und durch zahlreiche, von Blitz und Donner begleitete Gewitterstürme aus. Auch in dem Jahre 1811 gewahren wir ein ähnliches Verhältniss.* Bekannt ist dasselbe durch seinen heißen Sommer und durch seinen warmen Herbst: in den Monaten März und Juli hatten Steinfälle in Russland und in Spanien stattgefunden. Nicht weniger auffallend waren die Temperaturverhältnisse des Jahres 1821. Der Sommer war ein sehr heißer, und selbst Ende Dezember, sowie im Anfang des Januars 1822 war die Luft so mild, dass allenthalben die Vegetation bedeutend vorgeschritten. Am 15. Juni desselben Jahres (1821) hatte der große Steinfall von Juvinas* stattgefunden. Dagegen blieb Europa vom März 1798 an, wo der Steinfall zu Sales bei Lyon statthatte, durch die Jahre 1798, 1799, 1800 und 1801 von ähnlichen Naturerscheinungen gänzlich befreit, und des Winters von 1798 auf 1799 sowohl, als des Winters von 1799 auf 1800* wird als sehr gestrenger Herren Erwähnung getan. Ob diese Tatsachen nun wirklich auf einen tieferen Zusammenhang zwischen Meteorsteinfällen und den Temperaturverhältnissen unserer Erde in der oben erwähnten Weise sich gründen, ist bei den wenigen Beobachtungen, die man bis jetzt noch hierüber zu besitzen scheint, allerdings schwer zu ermitteln. Aber die gegebenen Andeutungen reichen hin, um einen solchen Zusammenhang nicht von vornherein als völlig unmöglich und unwahrscheinlich zu verwerfen.

*) G. 58. 1818. Fol. 289.

*) G. 29. 1808. Fol. 354. -. B. Fol. 27.

*) G. 16. 1804. Fol. 44.

*) WA. 39. 1860. Fol. 522.

*) G. 41. 1812. Fol. 88.

*) G. 72. 1822. Fol. 73.

*) G. 7. 1801. Fol. 33.

Man hat ferner wohl eingewendet, dass wenn die Steine wirklich innerhalb unserer Atmosphäre, also in einem sauerstoffreichen Medium sich gebildet hätten, sie kein reines Eisen, sondern nur Eisenoxyd würden enthalten können. Allein in der Tat finden sich nicht allein stets im Innern gewisse Mengen von Eisenoxyd vor; sondern die äußere Rinde ist auch - namentlich bei den eisenhaltigeren - fast einzig und allein aus dieser Substanz gebildet. Das innerliche Eisenoxyd rührt wohl wahrscheinlich von dem Gasgemenge selbst beigemischten Sauerstoff her. Die Rinde dagegen ist die Folge der Berührung mit dem äußeren Sauerstoff der Luft. In demselben Augenblick, wo durch die eintretende Verdichtung der Masse die bisher in ihr gebundene Wärme in Freiheit trat, und von dem Innern nach außen hin sich verbreitete, trat an der äußersten Grenze in Folge der Berührung mit dem freien Sauerstoff der Luft auch die Verbrennung ein. Dass durch diese aber nur die äußerste Rinde sich bilden, nicht aber auch das übrige Innere sich oxydieren konnte, scheint begreiflich. Denn von dem Augenblick an, wo äußerlich eine, wenn auch noch so dünne Oxydschicht sich gebildet, war auch das Innere durch eben diese Schicht von der Einwirkung des äußeren Sauerstoffs geschützt. Delarive hat bemerkt, dass die Eisenspitze bei dem galvanischen Bogen in gewöhnlicher Luft braunes, in verdünnter aber schwarzes Eisenoxyd liefert. Bei den Meteorsteinen werden sowohl braune als schwarze Oxyde erwähnt. Sollte sich aus diesem Zustande der Rinde daher nicht ein Schluss auf die größere oder geringere Höhe ziehen lassen, in welcher die Verbrennung tatsächlich stattgefunden?

Aber auch für jene eigentümlichen und rätselhaften „Fingereindrücke,“* für jene runden oder sechseckigen Vertiefungen mit ihren erhabenen, bergähnlichen Einfassungen, wie sie auf der Oberfläche so vieler Meteorsteine angetroffen werden, dürfte auf diesem Wege die einfachste und natürlichste Erklärung sich bieten. Denn dass bei vulkanischen Ausbrüchen gleichzeitig mit jenen erdigen und metallischen Dünsten auch noch andere permanente oder schwer zu verdichtende Gase den Kratern entsteigen, ist wohl kaum zu bezweifeln. Was ist aber alsdann wohl natürlicher, als dass derartige Gase, in Gestalt von Blasen zwischen den übrigen Stoffen eingeschlossen, bei eintretender Verdichtung gleich den Luftblasen eines gärenden, halbweichen Breies durch die noch nicht völlig erstarrte Masse nach der Oberfläche sich drängen, hier zerplatzen, und so in den von ihnen aufgeworfenen, bald ebenfalls erstarrenden Rändern, so wie in den durch sie gebildeten Untiefen - unseren scheinbaren Fingereindrücken - die bleibenden Spuren ihrer einstigen Entweichung zurücklassen? Geschah diese Gasentwicklung vereinzelt, so blieben die Blasen und folglich auch die Untiefen mit ihren Einfassungen rund. Geschah sie dagegen tumultuarisch, d. h. gleichzeitig in großer Menge und Blase an Blase sich drängend, dann mussten jene sechseckigen Formen entstehen, die wir so häufig beschrieben finden. Ebenso ist es auch wohl kaum zu bezweifeln, dass solche im Innern der erstarrenden Masse eingeschlossene und in Folge des Festwerdens an ihrem Entweichen gewaltsam verhinderte Gase es sind, welche das öfters beobachtete gewaltsame Zersprengen, dies Bersten der bereits festgewordenen Masse, bewirken. Denn während der eine Teil zu festem Gesteine sich zusammenzieht, müssen die in seinem Innern eingeschlossenen Gase durch die frei gewordene Hitze sich ausdehnen, und durch die gewaltsame Zersprengung des bereits gebildeten Gesteins sich eine Bahn brechen. Die scharfen Ecken und Kanten, mit denen solche Bruchstücke alsdann herabkommen, beweisen, dass jene Zersprengung wirklich im bereits festen und nicht im noch weichen Zustand des Steines stattgefunden habe.

*) P. 85. 1852. Fol. 574 Lixna. — P. 53. 1841. Fol. 172 Grüneburg. — P. 96. 1855. Fol. 626 Bremervörde. — P. 34. 1835. Fol. 340 Seres.

Chladni* - der übrigens hierbei eben sowohl die Meteorsteinfälle als auch die gewöhnlichen Feuerkugeln im Auge hatte - hat seiner Zeit auf das Bestimmteste erklärt, dass diese Erscheinungen an keine geographische Lage gebunden seien. Auch Greg kommt in Folge der von ihm unternommenen Zusammenstellungen zu dem Schlusse, dass die Verteilung der Meteorsteinfälle auf die verschiedenen Länder gleichmäßig geschehe, und dass kein bestimmter Ort, kein größerer Länderkomplex bevorzugt sei vor dem anderen.* Dagegen hat Shepard in seinen 1850 veröffentlichten Bemerkungen über die geographische Verteilung der Meteorsteine darauf aufmerksam gemacht, wie allerdings einzelne Gegenden einen solchen Vorzug voraus zu haben scheinen*; und in der Tat, versuchen wir es - wie dieses auf der beiliegenden Karte 1 und in dem dazu gehörigen Verzeichnis geschehen - diejenigen Meteorsteinfälle und Gediegen-Eisenmassen, welche uns in unserem eigenen Weltteil mit einer gewissen Zuverlässigkeit seit den letzten 160 Jahren bekannt geworden sind, geographisch aufzuzeichnen: so dürften allerdings gewisse Meteorstein-reiche und daneben andere Meteorstein-ärmere Gegenden mit einer kaum zu verkennenden Deutlichkeit uns entgegentreten. Wie auf neueren Karten die Distrikte der Erdbeben und die Gürtel der Vulkanreihen sich verzeichnet finden, so, scheint es, würden sich auch Distrikte für Meteorsteinfälle angeben lassen, namentlich wenn diese Phänomene einmal mit der Zeit allerwärts auf der ganzen Erde mit der gleichen Genauigkeit beobachtet und aufgezeichnet werden. Muss aber ein solches Gebundensein an bestimmte, vorherrschende Gegenden, wenn es wirklich als ein Naturgesetz sich bestätigt, alsdann nicht als ein weiteres Zeugnis für den irdischen Ursprung solcher meteorischen Gesteine betrachtet werden? Denn in der Tat: kämen sie aus dem weiten Weltraum, welch eine eigentümliche Vorliebe müsste es sein, die von diesen Fremdlingen von jeher - namentlich aber seit den letzten 160 Jahren, wo man angefangen, sie genauer zu beobachten - für gewisse Länder und Gegenden an den Tag gelegt worden ist? Ungarn, Böhmen, Mähren und Sachsen auf der einen, Italien, Frankreich und England auf der anderen Seite erscheinen reich damit bedacht. In den diesen angrenzenden Ländern zeigen sie sich dagegen weit seltener vertreten; oft nur wie zufällig durch einzelne dahin verirrte Gäste. Andere Gegenden, wie das Rheinland mit der ganzen Schweiz, mit Baden, Württemberg, Hessen u. s. w., - ebenso Schweden und Dänemark scheinen von jeher beinahe gänzlich von ihnen verschont oder doch nur sehr vereinzelt besucht worden zu sein. Oder sollten wir annehmen, dass diese so auffallenden und merkwürdigen Naturerscheinungen von jeher in Ungarn, Böhmen und Mähren, in Italien, Frankreich und England, oder selbst in Russland, sollten aufmerksamer und genauer beobachtet worden sein, als etwa bei uns in den so reichbevölkerten Rheinlanden? Das Eine scheint in der Tat ebenso unwahrscheinlich als das Andere, und nur die Annahme eines wirklich irdischen Ursprunges dürfte im Stande sein, den Schlüssel zu einer so auffallenden Tatsache zu liefern. Sehen wir uns aber einmal zu dieser Annahme genötigt: dann dürfte wohl auch nichts Anderes übrigbleiben, als denselben in der bisher angedeuteten Weise in der fortgesetzten Tätigkeit unserer irdischen Vulkane zu vermuten, und die weitere Frage dürfte daher nun vorzugsweise die sein: Wo und in welchen Richtungen haben wir - wenigstens für unseren Erdteil - die Krater zu suchen, deren Freigebigkeit wir diese luftigen Zusendungen zu verdanken haben? Bei einem wiederholten Blick auf die beigefügte Karte muss es uns auffallen, dass das ganze Land nördlich oder vielmehr etwas nordwestlich von den Alpen, also namentlich unser ganzes schon oben erwähntes Rheintal, zu allen Zeiten von Meteorsteinen fast völlig frei geblieben ist. Während Italien und namentlich die Gegenden südlich vom Fuße der Alpen von jeher reich damit bedacht worden, scheinen die Schweizer Gebirge mit einem Male sie wie abzuschneiden. Sie scheinen ihnen gleichsam ein gebieterisches „Bis hierher und nicht weiter“ zuzurufen, und damit zugleich alle hinter ihnen liegenden Länder, wenigstens bis in eine gewisse Ferne, vor ihren Heimsuchungen zu bewahren. Alle Nachrichten, die wir in neueren Zeiten von Steinfällen am Rhein, wie z. B. bei Bonn, Düsseldorf, Geißenheim und Mannheim durch Zeitungen empfangen haben, haben keine weitere Bestätigung erhalten. Auch in der Schweiz gehören diese Erscheinungen zu den großen Seltenheiten. Denn bis jetzt besitzen wir nur eine einzige wirklich zuverlässige Nachricht von einem in diesem Lande stattgefundenen Meteorsteinfall, nämlich von demjenigen vom 18 (nicht 19) Mai 1698 zu Hinterschwendi bei Waltringen im Canton Bern. Von demjenigen vom 6. Dezember (nicht Oktober) 1674 im Canton Glarus bleibt es zweifelhaft, ob es wirklich 2 Steine oder nur 2 Feuerkugeln waren, welche vom Himmel auf die Erde herabfielen. Scheuchzer sagt darüber: „dass an jenem Tage sowohl im Canton Glarus als fast in der ganzen Eidgenossenschaft und den angrenzenden Ländern die Erde stark erschüttert worden; alsbald nach diesem seien zu Näfels 2 feurige Kugeln vom Himmel auf den Erdboden gefallen, welches gespürt worden sei.“* Von einem wirklichen Steinfall ist also nicht die Rede, obgleich ein solcher aus dem Nachsatz „dass solches gespürt worden“ wohl zu vermuten ist. Ob der nach Cytasus, Kircher und Scheuchzer im 15. oder 16. Jahrhundert nach Aussage eines Bauern bei Luzern aus einem vorüberfliegenden Drachen zur Erde gefallene und zu Wunderkuren benutzte Stein* ein Meteorstein gewesen, bleibt sehr zweifelhaft. Auch der angebliche Meteorsteinfall vom 8. Dezember 1836 in Ober-Engadin* darf, da alle weiteren Nachrichten darüber fehlen, wohl füglich als ebenso zweifelhaft betrachtet werden. Der angebliche Steinfall vom 21. Oktober 1843 zu Favars im Canton Layssac in der Schweiz* beruht auf einer Verwechselung mit demjenigen, welcher am gleichen Tage zu Lessac im Departement de la Charente in Frankreich stattgefunden. Und der mutmaßliche Meteorsteinfall bei Lugano endlich, vom 15. März 1826,* gehört, der geographischen Lage wegen, in Bezug auf die gegenwärtige Frage mehr zu Italien als zur Schweiz.

*) G. 57 1817. Fol. 121.

*) RPG. Fol. 7. — B. Fol. 154.

*) Shepard, Account of three new American Meteorites; Charleston 1850. Fol. 10. — RPG. Fol. 6.

*) J. J. Scheuchzer, Beschreibung der Naturgeschichte des Schweizerlandes, Zürich 1706. 2. Fol. 75.

*) Ebendaselbst 2. Fol. 72 u. 3. Fol. 30.

*) Ebendaselbst 2. 113.

*) Vierteljahrsschrift der naturforschenden Gesellschaft in Zürich von Dr. R. Wolf. 1856. Fol. 326 nach Starks meteorologischen Jahrbüchern.

In ähnlicher Weise aber, wie bei uns die Alpen, so scheinen auch in Südfrankreich die Sevennen, in Ungarn und Galizien die Karpaten, und in Asien das Himalaja-Gebirge das hinter ihnen liegende Land bis in eine gewisse Entfernung vor Steinfällen zu bewahren. In Bezug auf das Letztere, das Himalaja-Gebirge, könnte man zwar einwenden, dass nur die südlich von ihm gelegenen Länder bis jetzt den Europäern zugänglicher gewesen seien, und dass wir daher auch nur aus diesen einigermaßen vollständige und zuverlässige Nachrichten über besondere Naturereignisse uns erwarten dürften, unterdes aus den nördlichen, von halbwilden Völkerschaften bewohnten Gegenden dieses nicht der Fall sei. Im Allgemeinen wäre gegen einen solchen Einwurf wohl nichts einzuwenden. Allein er verliert seine Schärfe, sobald wir unsere Blicke wieder auf die höheren europäischen Gebirge und namentlich auf die Alpen lenken. Hier kann von einem ähnlichen Unterschiede zwischen Nord und Süd in Bezug auf die Bevölkerung nicht die Rede sein: und dennoch welch ein Unterschied in Bezug auf die Häufigkeit der beobachteten Meteorsteinfälle. Der Unterschied ist so auffallend, dass er seltsam erscheinen könnte, wenn wir nicht wüssten, dass auch in Bezug auf die wässerigen Dünste unserer Atmosphäre hohe Gebirge ähnliche Grenzscheiden bilden. In ganz Süd-Europa ist es bekanntlich der Südwind, der vom Mittelmeere her die wässerigen Dünste dem Festlande zuführt. Und rufen nicht auch hier die hohen Spitzen der Alpen den fremden Ankömmlingen ihr „Halt“ von jeher zu? Es ist dieses umso mehr der Fall, je tiefer die Wolken sich bereits herabgesenkt haben; so dass in unseren Gegenden nur selten die Südwinde es sind, welche uns Regen zuführen. Ganz ähnlich verhält es sich nun auch mit unseren Meteorsteinen. Sehr häufig am südlichen Fuße der Alpen, treffen wir sie nur selten und spärlich in den in nördlicher oder vielmehr in nordwestlicher Richtung, gleichsam im Schatten der Alpen, gelegenen Ländern. Dass dieser Schutz in Bezug auf die Meteorsteine aber bis in keine so bedeutende Entfernung sich erstreckt, als dieses in Bezug auf wässerige Dünste der Fall ist, wird uns nicht wundern, sobald wir die weil größere Höhe berücksichtigen, in welcher die die Meteorsteine erzeugenden Dünste daher ziehen, im Vergleich mit unseren gewöhnlichen Regenwolken. So lange sie aber noch in solch übermäßiger Höhe sich befinden, entziehen sie sich auch leichter der Anziehung der auf der Oberfläche unserer Erde befindlichen Gebirge, und sie vermögen daher auf ihrer luftigen Fahrt, unangefochten von diesen Letzteren, bis in weitere Entfernungen über sie hinaus zu gelangen, bevor sie endlich völlig verdichtet auf unsere Erde herabstürzen. Hat aber ihre innere Verdichtung einmal mehr oder weniger begonnen, - haben sie sich demzufolge bereits in niedrigere, der Oberfläche unserer Erde näher gelegene Regionen unserer Atmosphäre herabgesenkt: dann kann es nicht mehr wundern, wenn auch die Nähe hoher Gebirgszüge ihre Einwirkung nicht verfehlt, wenn diese Letzteren sie immer mächtiger zur Erde herabziehen, und wenn sie, unvermögend dieser Anziehung sich zu entziehen, nun endlich am Fuße solcher Gebirge als völlig verdichtete Massen in reichlicherer Anzahl zu Boden stürzen.

*) P. 4. 1854. 375. - A. 4. 203.

*) P. 18. 1830. 184 u. 316.

So werden wir denn durch alle diese Umstände unwillkürlich nach einer bestimmten Richtung hingewiesen, aus welcher die Meteorsteine zu stammen scheinen; und diese Richtung ist - wenigstens für unser westliches Europa - keine andere als die süd-südöstliche. Befragen wir freilich in dieser Beziehung die Berichte, welche wir über wirklich beobachtete Meteorsteinfälle besitzen, so hat es allerdings den Anschein, als ob diese die eben ausgesprochene Ansicht auch nicht im Entferntesten unterstützten. Nach ihnen scheinen die Meteorsteine so ziemlich aus allen vier Himmelsgegenden bei uns anzukommen. Allein untersuchen wir die Sache etwas näher, so werden wir finden, dass trotzdem eine gewisse vorherrschende Richtung durchaus nicht zu verkennen ist; ohnerachtet es bei diesen Berichten häufig völlig unklar ist, ob bei Angabe einer Richtung diejenige gemeint ist, in der das Meteor selbst daher zog, oder nur diejenige, in welcher die Steine auf die Erde herabfielen. Beides sind aber begreiflicherweise zwei ganz verschiedene Ereignisse, die bei Berichten und Angaben nicht miteinander verwechselt werden sollten. Denn ein Meteor kann z. B. sehr wohl seinen Lauf von Osten hergenommen haben, und dennoch mögen die Steine, deren Niederfall man gerade beobachtet und die durch eine stattgehabte Explosion vielleicht nach allen Richtungen hinausgeschleudert worden sind, von Westen her in den Boden einschlagen. Bei dem Steinfall von Eggenfeld in Bayern (1803) wird ein solches Verhältnis ausdrücklich erwähnt: die Explosion habe man von Osten hergehört; die Steine aber seien von Westen gekommen.

Betrachten wir daher nun, ganz abgesehen hiervon, ausschließlich diejenigen Meteorsteinfälle, bei denen sich genau die Himmelsgegend angegeben findet, aus welcher das die Steine erzeugende Phänomen, d. i. die Wolke oder die Feuerkugel, daher gezogen ist: so erhalten wir für unseren Weltteil für die letzten 160 Jahre das nachstehende Verhältnis:

1\. Von Norden her kamen 4, nämlich 1706 Larissa,* 1722 Schefftlar,* 1810 Charsonville,* 1833 Blansko*;

2\. von Nordwesten her kamen 3, nämlich 1751 Hraschina,* 1814 Agen,* 1824 Zebrak*;

3\. von Südwesten her kamen 3, nämlich 1841 Grüneberg (in Sagan als Feuerkugel gesehen),* 1841 Château-Renard,* 1852 Mezo-Madaras.*

Zusammen 10 Steinfälle.

*) Chladni, über Feuer-Meteore; Wien 1819. Fol. 240.

*) G. 53. 1816. 377.

*) G. 40. 1812. 84.

*) P. 4. 1854. 30.

*) WA. 35. 1859. 17 u. 18.

*) G. 48. 1814. 399.

*) P. 6. 1826. 28.

*) P. 4. 1854. 361.

*) P. 53. 1841. 411.

*) P. 91. 1854. 627.

Dagegen kamen

4\. von Südosten her 9, nämlich 1704 Barcelona,* 1790 Barbotan,* 1798 Sales,* 1803 l'Aigle,* 1812 Erxleben,* 1813 Cutro,* 1820 Lixna,* 1822 Angers,* 1824 Renazzo*;

5\. von Osten her 4, nämlich 1794 Siena,* 1812 Toulouse,* 1813 Adair,* 1840 Ceresetto*;

6\. von Nordosten her 8, nämlich 1780 Beeston,* 1782 Turin,* 1803 Apt (in Genf als Feuerkugel gesehen),* 1808 Stannern,* 1815 Chassigny,* 1847 Braunau,* 1851 Gütersloh,* 1858 Clarac und Aussun.*

Zusammen 21 Steinfälle.

*) P. 8. 1826. 46.

*) G. 57. 1817. 134. - G. 15. 1803. 422 u. 429.

*) G. 18. 1804. 275.

*) G. 15. 1803. 74.

*) G. 40. 1812. 456.

*) Chladni, 377.

*) P. 85. 1852. 574.

*) G. 71. 1822. 351.

*) P. 5. 1825. 122.

*) G. 18. 1804. 285.

*) G. 57. 1817. 134.

*) G. 41. 1812. 447.

*) G. 60. 1818. 233. - P. 4. 1854. 360.

*) K. 3. 276.

*) Chladni, 256.

*) G. 16. 1804. 73.

*) G. 29. 1808. 246.

*) G. 57. 1817. 134. - G. 58. 1817. 171.

*) P. 72. 1847. 170.

*) P. 83. 1851. 465.

*) Harris, the chemical constitution and chronologicâl arrangement of Meteorites; Gött. 1859. Fol. 45.

Also über die Hälfte mehr aus östlichen als aus nicht-östlichen Richtungen. Es ist zwar nur eine geringe Anzahl von Fällen, die dieser Zusammenstellung zu Grunde gelegt werden konnte; allein der sich daraus ergebende Unterschied zwischen denen, die aus östlichen, und denen, die aus nicht-östlichen Richtungen bei uns anlangten, ist ein verhältnismäßig so bedeutender, dass er unmöglich verkannt oder außer Acht gelassen werden kann. Dass dabei immerhin noch Verschiedenheiten obwalten, kann bei den mannigfaltigen regelmäßigen wie unregelmäßigen Winden und Luftströmungen, die unseren Dunstkreis fortwährend bewegen, nicht auffallen. Ein regelmäßiger Luftstrom geht in seinen oberen Schichten unausgesetzt von Süden nach Norden; ein anderer in den tieferen von Norden nach Süden; der mannigfachen sonstigen Winde von mehr lokaler Natur gar nicht weiter zu gedenken. Dass sie alle nicht ohne Einfluss auf den Lauf jener meteorischen Dünste und der aus ihnen hervorgehenden Feuerkugeln bleiben können, leuchtet wohl von selbst ein.

Machen wir nun aber auch noch weiter den Versuch, die seit 1700, also ebenfalls seit den letzten 160 Jahren in unserem Erdteil stattgefundenen 130 Meteorsteinfälle, bei denen Tag oder Monat des Ereignisses angegeben ist, nach den einzelnen 12 Monaten zu ordnen, so erhalten wir nach der am Schlusse dieser Abhandlung befindlichen Zusammenstellung das folgende Verhältnis:

Januar 5  
Februar 5  
März 7  
17  

April 13  
Mai 12  
Juni 16  
41

Juli 17  
August 8  
September 14  
39

Oktober 13  
November 10  
Dezember 10  
33

d. h. auf die 6 Sommermonate ergeben sich etwa um die Hälfte mehr Meteorsteinfälle als auf die 6 Wintermonate. Dabei kommen zugleich von 5 Gediegen-Eisenmassen 4 auf Sommermonate und nur eine Einzige auf einen Wintermonat; unterdessen gleichzeitig die gewöhnlich kältesten 3 Wintermonate, Januar, Februar und März, auch die geringste Anzahl von Steinfällen aufweisen. Auch Kämtz und Greg, indem beide sämtliche, seit den ältesten Zeiten bekannte Meteorsteinfälle zusammenstellten, entgingen diese eben erwähnten Verhältnisse nicht. Auch sie mussten, im Gegensatz zu den früheren Annahmen Chladnis, sowohl jenes Vorwalten einer mehr östlichen Richtung als dieses Überwiegen in der Zahl der Meteorsteinfälle während der Sommerzeit als wirkliche Tatsachen anerkennen. So sagt z. B. Kämtz ganz ausdrücklich: „Das Vorwalten der östlichen Richtung, welches übrigens unbedeutend ist (?), scheint seinen Grund in der Drehung der Erde zu haben“; und weiterhin: „nach Monaten geordnet, scheint allerdings zu folgen, dass die Zahl (der Meteorsteinfälle) im Winter kleiner ist als im Sommer.“*

Wie ganz anders gestaltet sich nun aber das letztere Verhältnis, sobald wir für dieselben letztverflossenen 160 Jahre unsere Blicke auf Asien richten, und die uns aus diesem Weltteil bekannt gewordenen 23 Meteorsteinfälle, von denen die Tage oder Monate ihres Herabkommens uns gegeben sind, nun ebenfalls nach den 12 Monaten des Jahres ordnen. Jetzt erhalten wir gerade das umgekehrte Verhältnis. Nämlich:

Januar 1  
Februar 5  
März 2  
8  

April 2  
Mai 1  
Juni 2  
5

Juli 2  
August 1  
September -  
3

Oktober -  
November 6  
Dezember 1  
7

Sollte dieses etwa ein bloßer Zufall sein? Oder sollte nicht vielleicht auch hier ein und dieselbe tiefere Ursache beiden Verschiedenheiten zu Grunde liegen? Alle Länder der nördlichen Halbkugel haben zu den gleichen Perioden gemeinschaftlich ihre Sommer- und ihre Winterzeit, und wir sehen - wenn wir einen Blick auf die Karte 2 werfen - die Meteorsteinfälle, von den südöstlichsten Grenzen Asiens anfangend, über die nach Nordwesten zu gelegenen Länder bis in unseren eigenen Weltteil am Reichlichsten verbreitet. Sind wir nun aber nach allen bisherigen Auseinandersetzungen nicht ohne Grund versucht, jene meteorischen Gesteine für wirkliche Produkte unseres eigenen Erdkörpers, und zwar für ursprünglich gasförmige Auswürflinge unserer noch tätigen Vulkane zu halten; und werden wir außerdem durch die obigen Aufstellungen unwillkürlich nach dem Osten als ihrer wahren Heimat hingewiesen: dann dürfen wir uns wohl auch nicht ohne Wahrscheinlichkeit der Annahme hingeben, dass wir in jenen zahlreichen, selbst bis in die Neuzeit in fast ununterbrochener Tätigkeit begriffenen Vulkanreihen Ost-Asiens, die fast die ganze östliche uni südöstliche Grenze der alten Welt wie mit einem Feuergürtel umschließen, die eigentlichen und hauptsächlichsten Herde zu suchen haben werden, denen wir - neben den wenigen tätigen Vulkanen in Süd-Europa und in Mittelasien - vorzugsweise jene eigentümlichen und noch immer so rätselhaften Zusendungen zu verdanken haben. In einem solchen Falle darf es uns aber alsdann auch nicht mehr wundern, wenn jene Segler der Lüfte während der wärmeren Sommermonate, wo ihre Abkühlung und Verdichtung notwendig auch langsamer von Statten gehen muss, weit leichter und weit zahlreicher bis zu uns, in den fernen Westen, zu gelangen vermögen, als im Winter. In Letzterem dagegen, wo die strengere Kälte auch ihre innerliche Abkühlung beschleunigt, müssen wir sie aus demselben Grunde größtenteils schon früher, d. h. schon in geringerer Entfernung von ihren ursprünglichen Ausgangspunkten, wieder auf unsere Erde herabfallen sehen. Das heißt aber mit anderen Worten: es muss ganz dasselbe Verhältnis stattfinden, wie es sich aus der obigen Zusammenstellung soeben für uns ergeben hat.

*) K. 3. 304 u. 307. - RPG. 8.

Bevor wir indessen schließen, müssen wir noch eines weiteren Einwurfes gedenken, der gegen die eben dargelegte Ansicht könnte gemacht werden. Er gründet sich auf den Umstand, dass die Ausbrüche vulkanischer Tätigkeit in der vorsündflutlichen Urzeit unserer Erde jedenfalls weit häufiger, großartiger und ausgebreiteter dürften gewesen sein, als dieses gegenwärtig noch der Fall ist. Darnach müssten aber auch die Meteorsteinfälle, wenn die ausgesprochene Ansicht wirklich eine begründete wäre, damals noch weit häufiger und in einer weit ausgedehnteren Weise sich ereignet haben als zu unserer Zeit. Nichtsdestoweniger hat man aber - mit Ausnahme eines einzigen, bis jetzt noch nicht völlig erwiesenen Falles, dessen Reuß und Neumann erwähnen, des Eisens von Chotzen nämlich,* - in den vorsündflutlichen Schichten unserer Erdrinde noch keine Meteorsteine aufgefunden. Dass auch in der Urzeit unserer Erde Meteorsteinfälle stattgefunden haben mögen, ist allerdings sehr wahrscheinlich. Allein dieses muss ganz ebenso der Fall sein, wenn die Meteorsteine aus dem freien Weltraum stammen, als wenn wir sie als selbstständige Erzeugnisse unserer Erde zu betrachten haben. Von Reichenbach, indem er die Ansicht ausspricht, dass die Meteorsteine wahrscheinlich nur als verdichtete und fest gewordene Massen von Kometenstoff zu betrachten sein dürften, hält dafür, dass ganze Berge, die wir jetzt für Gegenstände der Geognosie halten, nichts weiter sind, als zerfallen mächtige Meteoriten.* Dass der Weltraum in jener uns so fernen Urzeit wenigstens reiner und freier von fremden Stoffen sollte gewesen sein als jetzt, ist wohl kaum zu vermuten; und ebenso wenig dürfen wir wohl annehmen, dass die Anziehung unserer Erde damals eine andere sollte gewesen sein, als dieses unter den gegenwärtigen Verhältnissen der Fall ist. Wenn also nichtsdestoweniger in den inneren Schichten unserer Erde gegenwärtig keine oder wenigstens nur zweifelhafte Spuren solcher Ereignisse sich vorfinden: so darf der Grund hiervon gewiss in keinem Fall in der angenommenen Unmöglichkeit eines irdischen Ursprunges unserer Meteorsteine, - sondern gewiss nur in ganz anderen Ursachen und Verhältnissen von uns gesucht werden. Diese Ursachen aufzufinden, scheint aber in der Tat weder sehr schwierig, noch unmöglich. Die Zeiten, welche wir die vordiluvianischen nennen, liegen zum allermindesten viele Tausende von Jahren hinter uns. Ja sie erstrecken sich von da ab in Zeiträume hinein, deren Ausdehnung wir kaum zu mutmaßen, geschweige genauer zu bestimmen im Stande sind. Wir wissen durchaus nicht mehr, ob wir hier noch von Tausenden von Jahren reden dürfen, oder ob wir nicht vielmehr von Millionen von Jahren sprechen müssen, wenn wir nur annähernd die Wahrheit erreichen wollen. Und wenn zu allen jenen Zeiten - seien es nun die ältesten oder jüngsten im Jugendalter unserer Erde, - wirklich Meteorsteine auf diese Letztere herabgeworfen wurden: ist es da zu verwundern, wenn sie längst der Zersetzung anheimgefallen, und als wirklich selbstständige Massen im Innern unserer Erde nun nicht mehr von uns nachgewiesen werden können? Nimmt man in neuester Zeit doch an, dass selbst die Granite und Gneisse keine wirklichen Urgesteine, sondern nur allmähliche, durch die Zeit bewirkte Umgestaltungen anderer Gesteine darstellen; bleiben doch selbst die großartigsten, oft über weite Länderstrecken dahingegossenen Basaltmassen vom Zahn der Zeit nicht unberührt, sondern gehen auch an ihnen, selbst in ihrem tiefsten Innern, fortwährend die mannigfachsten Veränderungen und Umgestaltungen vor sich: wie sollte da, auch nur mit einiger Wahrscheinlichkeit, von uns angenommen werden dürfen, dass verhältnismäßiger kleine Massen, wie unsere Meteorsteine doch meistenteils nur darstellen, solchen Zersetzungsprozessen im Laufe einer so unbestimmbar langen Zeit in Wirklichkeit sollten widerstanden haben? In der Tat, wir glauben nicht, dass dieser Umstand im Ernste als ein Einwurf gegen die Möglichkeit eines irdischen Ursprunges der fraglichen Gebilde dürfte betrachtet werden. Wäre es, er müsste in ganz gleicher Weise auch gegen die Annahme eines außerirdischen Ursprunges seine Geltung haben.

*) WA. 25. 1857. Fol. 545. - Geologische Reichsanstalt; Wien 1857. Fol. 354 - 357.

*) P. 105. 1858. Fol. 438 u. 447.

Nach einer von ihm angestellten Wahrscheinlichkeitsrechnung nimmt v. Reichenbach an, dass jährlich ungefähr 4500 Zentner von Meteorsteinmassen auf unsere Erde herabfällen dürften. In tausend Jahren würde also unsere Erde eine Gewichtszunahme von je 4 1/2 Millionen Zentner zu ertragen haben. Da aber das Gesamtgewicht unseres ganzen Erdballes ungefähr 100,000 Trillionen Zentner betrage, so verschwinde dieser jährliche Zuwachs gegen das wirkliche Gewicht unserer Erde ähnlich wie der Tropfen am Eimer. So sei es denn auch erklärlich, dass ungeachtet dieser von ihm vermuteten jährlichen Gewichtszunahme dennoch seit den frühesten Zeiten, wo Menschen den Lauf der Gestirne beobachteten, auch nicht die geringste Änderung in dem Gleichgewicht und dem Lauf unserer Erde, sowie in ihrer Stellung zu den übrigen Planeten wahrgenommen werden konnte.* Sollte aber eine solche immerhin nicht unbeträchtliche Gewichtszunahme auch in Bezug auf das gegenseitige Verhältnis zwischen unserer Erde und dem ihr viel näheren Mond ohne alle Wirkung bleiben? Diese Frage dürfte wohl einer anderweitigen und eingehenderen Untersuchung wert sein.

Übrigens möchte es hier der Ort sein, um noch einiger anderen Worte Reichenbachs zu erwähnen, welche in Bezug auf die gegenwärtige Frage nicht ohne Interesse sein dürften. Nachdem er es nämlich anerkannt, „dass der Dolerit des Meissners stellenweise so viel Ähnlichkeit des äußeren Ansehens mit manchen Meteorsteinen hat, dass man beide beinahe verwechseln könnte, und dass Kenneraugen dazu gehören, um nicht getäuscht zu werden“;* - nachdem er ferner anerkannt, „dass die hauptsächlichsten Bestandteile es Dolerits fast alle auch in den Meteorsteinen vorkommen, und umgekehrt die Meteoriten nur wenige besitzen, die nicht auch den Doleriten eigen wären“;* und endlich: „dass die Mineralspezies, die sich in den Meteoriten vorfinden, fast alle auch in den vulkanischen und plutonischen Gesteinen des Erdballs vorkommen, und dass ihre Grundstoffe ohne Ausnahme auch auf der Erde vorrätig sind“;* - fährt er also fort: „Es ist gewiss auffallend, dass die Mineralspezies, welche wir in den Meteoriten gewahren, zumeist in den vulkanischen und plutonischen Gebilden sich wiederfinden, und dass damit beide in eine gewisse Nähe geraten, deren Zusammenhang wir noch nicht verstehen. Es müssen also da unten, tief unter den Vulkanen, Gesteinsmassen vorhanden sein, die den näheren Bestandteilen nach fast ganz übereinstimmen mit den Meteoriten, und die in hohem Grade den Verdacht erregen müssen, dass das Innere unserer Erde entweder selbst die mineralische Konstitution eines Meteoriten habe, oder aber, wie nicht ganz unwahrscheinlich, ganz und gar aus einem Aggregat von Meteoriten überhaupt bestehe.“ „Auffallender gibt es wohl kaum Etwas, als dass einige Hundert Analysen, die meisten von unseren ausgezeichnetsten Scheidekünstlern ausgeführt, in keinem einzigen Meteoriten irgendeinen Grundstoff aufgefunden haben, der nicht auf unserer Erde schon vorrätig wäre. Wir sind also einander auf keine Weise fremd, die Meteoriten und die Erde. Wir sind sichtlich Geschwister und kommen von derselben Mutter.“*

*) P. 105. 1858. Fol. 555 - 556.

*) P. 105. 1858. Fol. 558.

*) P. 105. 1858. Fol. 558.

*) P. 105. 1858. Fol. 562.

Sprechen diese Worte nicht wie mit Prophetenstimme für einen wirklich irdischen Ursprungs unserer Meteorsteine? Wohl birgt die Erde in ihrem tiefsten Innern dieselben Stoffe, welche auch diese Letzteren bilden. Alle Tatsachen, die wir kennen, sprechen für die Wahrheit dieses Satzes. Aber nicht als fertige und bereits seit unvordenklichen Zeiten längst erkaltete Meteorsteine oder Anhäufungen von Meteorsteinen dürften sie sich hier befinden; sondern - wenn nicht alle Anzeichen trügen - allein als das noch rohe Material von denjenigen chemischen Ur- und Grundstoffen, welche wir je nach Umständen, je nachdem sie in feurigem Fluss aus dem Innern unserer Feuerberge sich emporwürgen, oder in glühender Dampf- oder Gasgestalt ihren Schloten entsteigen, dort zu Doleriten, Basalten und Laven, - hier zu Meteorsteinen und Meteoreisenmassen der mannigfachsten Abstufungen sich gestalten sehen.

Nicht Geschwister sind sie, unsere Erde und die auf sie herabfallenden meteorischen Gesteine: die Letzteren sind der Ersteren eigene und von ihr selbst erzeugte Kinder. Ihrem mütterlichen Schoosse entstiegen, sehnen diese mit der wachsenden Entfernung von dem festen Erdkörper bald immer mächtiger wieder zu ihrer Mutter Erde sich zurück. Sei es früher, sei es später, sie kehren - wenn auch in veränderter Gestalt - unausbleiblich wieder, ohne dass inzwischen, weder durch ihre vorübergehende Entfernung von dem festen Erdkörper noch durch ihre Wiedervereinigung mit demselben, in den Gewichtsverhältnissen unseres gesamten Erdballes, d. h. sowohl des festen Erdkörpers als auch der ihn umgebenden und zu ihm gehörigen Dunsthülle, jemals auch nur die allergeringste Veränderung vor sich ginge. Hierin liegt denn auch wohl der einfachste und natürlichste Grund, weshalb seit Menschengedenken trotz aller Meteorsteinfälle dennoch noch nie auch nur die allergeringste Veränderung in den Gleichgewichtsverhältnissen unserer Erde sowohl in Bezug auf ihre Mitplaneten als ihren eigenen Lebensgefährten, den Mond, hat können wahrgenommen werden. Aber ebenso löst sich auch hiermit in der allereinfachsten und doch zugleich auch allernatürlichsten Weise jenes sonst so auffallende und so unerklärlich scheinende Rätsel, dass noch in keinem einzigen Meteorstein ein Grundstoff gefunden worden ist, der nicht auch auf unserer eigenen Erde und namentlich nicht in den mineralischen Gebilden unserer Vulkane sich ebenfalls vorfände. Er löst sich in einer Weise, wie dieses kaum bei irgendeiner anderen Annahme über den Ursprung jener rätselhaften Gebilde möglich sein dürfte.

*) P. 105. 1858. Fol. 559 u. 560.

Übrigens soll durch alles dieses durchaus noch nicht gesagt sein, als sei die hier vertretene Ansicht bereits über alle und jede Zweifel und Einwendungen erhaben. Ebenso wenige ist es nach den bis jetzt dafür vorhandenen Anhaltspunkten möglich, schon jetzt ein weiteres und sicheres Naturgesetz darauf zu gründen. Erst dann wird dieses möglich sein, - erst dann wird über alle die Rätsel, die uns auf diesem Felde noch umgeben, ein helleres Licht sich verbreiten, wenn wir einmal im Stande sind, über alle und jede meteorologische und vulkanische Erscheinungen, die fortwährend über den ganzen Erdkreis sich verbreiten, sofort auch vollständige und zuverlässige Nachrichten zu erhalten. Denn ebenso wenig als die Anhänger eines außerirdischen Ursprunges wohl jemals im Stande sein werden, ihre mutmaßlichen Eindringlinge bei ihrem Eintritt in die irdische Atmosphäre tatsächlich zu belauschen: ebenso wenig wird es auf der anderen Seite möglich sein, die unseren Feuerbergen entsteigenden gasförmigen Dünste auf ihrer luftigen Reise zu begleiten und als die wirklichen und unmittelbaren Zeugen ihrer Wiederverdichtung aufzutreten. Nur Vernunftgründe vermögen hier für die größere oder geringere Wahrscheinlichkeit der einen oder der anderen Ansicht zu streiten, und soweit es mit den bis jetzt vorhandenen Mitteln möglich gewesen, ist hier der Versuch gemacht, wenn auch nicht auf die unzweifelhafte Gewissheit, so doch auf die Möglichkeit und selbst auf die große Wahrscheinlichkeit eines tieferen, in dem inneren und verborgenen Gesamtleben unserer Erde begründeten Zusammenhanges zwischen unseren Meteorsteinfällen und der Tätigkeit unserer irdischen Vulkane hinzuweisen. Möchten auch Andere die angeregte Frage einer näheren und vorurteilsfreien Prüfung werthalten.

Dass übrigens eine Arbeit wie die gegenwärtige niemals als eine geschlossene zu betrachten ist, versteht sich wohl von selbst und liegt in der Natur der Sache. Namentlich bedarf die Aufstellung der Karten und Verzeichnisse nicht nur einer fortwährenden Ergänzung und Vervollständigung, sondern auch einer steten Berichtigung, wenn dieselben wirklich einen dauernden Werth besitzen sollen. Es werden daher dem Verfasser Mitteilungen zu diesem Zwecke stets willkommen sein, so wie er auch allen Denen seinen aufrichtigen Dank sagt, welche ihm bisher in seiner Arbeit durch ihre freundlichen Mitteilungen, Berichtigungen und Andeutungen sowie durch sonstige Unterstützung behülflich und förderlich gewesen sind.
\clearpage
\vspace*{\fill}
\section{Europäische Meteorsteinfälle seit dem Jahre 1700, nach den 12 Monaten geordnet.}
\vspace*{\fill}
\clearpage
\begin{landscape}
\begin{table}[!ht]
    \footnotesize
    \centering
    \begin{tabular}{|p{5mm}|p{4mm}|p{13mm}|p{17mm}|p{17mm}|p{4mm}|p{6mm}|p{6mm}|p{6mm}|p{4mm}|p{5mm}|p{4mm}|p{5mm}|p{6mm}|p{5mm}|p{5mm}|p{5mm}|}
    \hline
         &  &  &  &  & Jan. & Febr. & März & April & Mai & Juni & Juli & Aug. & Sept. & Okt. & Nov. & Dez. \\ \hline
        1704 & 24. & Dezember & Barcelona & Spanien & ~ & ~ & ~ & ~ & ~ & ~ & ~ & ~ & ~ & ~ & ~ & 24 \\ \hline
        1706 & 7. & Juni & Larissa & Türkei & ~ & ~ & ~ & ~ & ~ & 7 & ~ & ~ & ~ & ~ & ~ & ~ \\ \hline
        1715 & 11. & April & Schellin & Deutschland & ~ & ~ & ~ & 11 & ~ & ~ & ~ & ~ & ~ & ~ & ~ & ~ \\ \hline
        1722 & 5. & Juni & Schefftlar & Deutschland & ~ & ~ & ~ & ~ & ~ & 5 & ~ & ~ & ~ & ~ & ~ & ~ \\ \hline
        1723 & 22. & Juni & Pleskowitz und Liboschitz & Böhmen & ~ & ~ & ~ & ~ & ~ & 22 & ~ & ~ & ~ & ~ & ~ & ~ \\ \hline
        1725 & 3. & Juli & Mixbury & England & ~ & ~ & ~ & ~ & ~ & ~ & 3 & ~ & ~ & ~ & ~ & ~ \\ \hline
        1731 & 12. & März & Halstead & England & ~ & ~ & 12 & ~ & ~ & ~ & ~ & ~ & ~ & ~ & ~ & ~ \\ \hline
        1740 & 25. & Oktober & Hazargrad & Türkei & ~ & ~ & ~ & ~ & ~ & ~ & ~ & ~ & ~ & 25 & ~ & ~ \\ \hline
        1750 & 1. & Oktober & Nicorps & Frankreich & ~ & ~ & ~ & ~ & ~ & ~ & ~ & ~ & ~ & 1 & ~ & ~ \\ \hline
        1751 & 26. & Mai & Hraschina. Eisen. & Kroatien & ~ & ~ & ~ & ~ & 26 & ~ & ~ & ~ & ~ & ~ & ~ & ~ \\ \hline
        1753 & 3. & Juli & Plan und Strkow & Böhmen & ~ & ~ & ~ & ~ & ~ & ~ & 3 & ~ & ~ & ~ & ~ & ~ \\ \hline
        1753 & 7. & September & Luponnas & Frankreich & ~ & ~ & ~ & ~ & ~ & ~ & ~ & ~ & 7 & ~ & ~ & ~ \\ \hline
        1755 & - & Juli & Terranova & Italien & ~ & ~ & ~ & ~ & ~ & ~ & x. & ~ & ~ & ~ & ~ & ~ \\ \hline
        1766 & M. & Juli & Alboretto & Italien & ~ & ~ & ~ & ~ & ~ & ~ & M. & ~ & ~ & ~ & ~ & ~ \\ \hline
        1768 & 13. & September & Lucé & Frankreich & ~ & ~ & ~ & ~ & ~ & ~ & ~ & ~ & 13 & ~ & ~ & ~ \\ \hline
        1768 & 20. & November & Maurkirchen & Deutschland & ~ & ~ & ~ & ~ & ~ & ~ & ~ & ~ & ~ & ~ & 23 & ~ \\ \hline
        1773 & 17. & November & Sena & Spanien & ~ & ~ & ~ & ~ & ~ & ~ & ~ & ~ & ~ & ~ & 17 & ~ \\ \hline
        1775 & 19. & September & Rodach & Deutschland & ~ & ~ & ~ & ~ & ~ & ~ & ~ & ~ & 19 & ~ & ~ & ~ \\ \hline
        1776 & - & Januar & Sanatoglia & Italien & x. & ~ & ~ & ~ & ~ & ~ & ~ & ~ & ~ & ~ & ~ & ~ \\ \hline
        1780 & 11. & April & Beeston & England & ~ & ~ & ~ & 11 & ~ & ~ & ~ & ~ & ~ & ~ & ~ & ~ \\ \hline
        1782 & - & Juli & Turin & Italien & ~ & ~ & ~ & ~ & ~ & ~ & x. & ~ & ~ & ~ & ~ & ~ \\ \hline
        1785 & 19. & Februar & Wittens & Deutschland & ~ & 19 & ~ & ~ & ~ & ~ & ~ & ~ & ~ & ~ & ~ & ~ \\ \hline
        1787 & 13. & Oktober & Schigailow und Lebedin & Russland & ~ & ~ & ~ & ~ & ~ & ~ & ~ & ~ & ~ & 13 & ~ & ~ \\ \hline
        1790 & 24. & Juli & Barbotan & Frankreich & ~ & ~ & ~ & ~ & ~ & ~ & 24 & ~ & ~ & ~ & ~ & ~ \\ \hline
    \end{tabular}
\end{table}
\end{landscape}
\clearpage
\begin{landscape}
\begin{table}[!ht]
    \footnotesize
    \centering
    \begin{tabular}{|p{5mm}|p{4mm}|p{13mm}|p{22mm}|p{15mm}|p{4mm}|p{6mm}|p{6mm}|p{6mm}|p{4mm}|p{5mm}|p{4mm}|p{5mm}|p{6mm}|p{5mm}|p{5mm}|p{5mm}|}
    \hline
         &  &  &  &  & Jan. & Febr. & März & April & Mai & Juni & Juli & Aug. & Sept. & Okt. & Nov. & Dez. \\ \hline
        1791 & 17. & Mai & Castel-Berardenga & Italien & ~ & ~ & ~ & ~ & 17 & ~ & ~ & ~ & ~ & ~ & ~ & ~ \\ \hline
        1794 & 16. & Juni & Siena & Italien & ~ & ~ & ~ & ~ & ~ & 16 & ~ & ~ & ~ & ~ & ~ & ~ \\ \hline
        1795 & 13. & Dezember & Wold-Cottage & England & ~ & ~ & ~ & ~ & ~ & ~ & ~ & ~ & ~ & ~ & ~ & 13 \\ \hline
        1796 & 4. & Januar & Belaja-Zerkwa & Russland & 4 & ~ & ~ & ~ & ~ & ~ & ~ & ~ & ~ & ~ & ~ & ~ \\ \hline
        1796 & 19. & Februar & Tasquinha & Portugal & ~ & 19 & ~ & ~ & ~ & ~ & ~ & ~ & ~ & ~ & ~ & ~ \\ \hline
        1798 & 12. & März & Sales & Frankreich & ~ & ~ & 12 & ~ & ~ & ~ & ~ & ~ & ~ & ~ & ~ & ~ \\ \hline
        1802 & M. & September & Loch-Tay & Schottland & ~ & ~ & ~ & ~ & ~ & ~ & ~ & ~ & M. & ~ & ~ & ~ \\ \hline
        1803 & 26. & April & l’Aigle & Frankreich & ~ & ~ & ~ & 26 & ~ & ~ & ~ & ~ & ~ & ~ & ~ & ~ \\ \hline
        1803 & 4. & Juli & East-Norton & England & ~ & ~ & ~ & ~ & ~ & ~ & 4 & ~ & ~ & ~ & ~ & ~ \\ \hline
        1803 & 8. & Oktober & Saurette & Frankreich & ~ & ~ & ~ & ~ & ~ & ~ & ~ & ~ & ~ & 8 & ~ & ~ \\ \hline
        1803 & 13. & Dezember & St. Nicolas & Deutschland & ~ & ~ & ~ & ~ & ~ & ~ & ~ & ~ & ~ & ~ & ~ & 13 \\ \hline
        1804 & 5. & April & High-Possil & Schottland & ~ & ~ & ~ & 5 & ~ & ~ & ~ & ~ & ~ & ~ & ~ & ~ \\ \hline
        1805 & - & Juni & Konstantinopel & Türkei & ~ & ~ & ~ & ~ & ~ & x. & ~ & ~ & ~ & ~ & ~ & ~ \\ \hline
        1805 & - & November & Asco & Korsika & ~ & ~ & ~ & ~ & ~ & ~ & ~ & ~ & ~ & ~ & x. & ~ \\ \hline
        1806 & 15. & März & St. Etienne-de-Lolm u. Valence & Frankreich & ~ & ~ & 15 & ~ & ~ & ~ & ~ & ~ & ~ & ~ & ~ & ~ \\ \hline
        1806 & 17. & Mai & Basingstoke & England & ~ & ~ & ~ & ~ & 17 & ~ & ~ & ~ & ~ & ~ & ~ & ~ \\ \hline
        1807 & 13. & März & Timochin & Russland & ~ & ~ & 13 & ~ & ~ & ~ & ~ & ~ & ~ & ~ & ~ & ~ \\ \hline
        1808 & 19. & April & Pieve die Casignano & Italien & ~ & ~ & ~ & 19 & ~ & ~ & ~ & ~ & ~ & ~ & ~ & ~ \\ \hline
        1808 & 22. & Mai & Stannern & Mahren & ~ & ~ & ~ & ~ & 22 & ~ & ~ & ~ & ~ & ~ & ~ & ~ \\ \hline
        1808 & 3. & September & Stratow und Wustra & Böhmen & ~ & ~ & ~ & ~ & ~ & ~ & ~ & ~ & 3 & ~ & ~ & ~ \\ \hline
    \end{tabular}
\end{table}
\end{landscape}
\clearpage
\begin{landscape}
\begin{table}[!ht]
    \footnotesize
    \centering
    \begin{tabular}{|p{5mm}|p{4mm}|p{13mm}|p{17mm}|p{17mm}|p{4mm}|p{6mm}|p{6mm}|p{6mm}|p{4mm}|p{5mm}|p{4mm}|p{5mm}|p{6mm}|p{5mm}|p{5mm}|p{5mm}|}
    \hline
         &  &  &  &  & Jan. & Febr. & März & April & Mai & Juni & Juli & Aug. & Sept. & Okt. & Nov. & Dez. \\ \hline
        1810 & M. & August & Mooresfort & Irland & ~ & ~ & ~ & ~ & ~ & ~ & ~ & M. & ~ & ~ & ~ & ~ \\ \hline
        1810 & 23. & November & Charsonville & Frankreich & ~ & ~ & ~ & ~ & ~ & ~ & ~ & ~ & ~ & ~ & 23 & ~ \\ \hline
        1810 & 28. & November & Cerigo & Greichenland & ~ & ~ & ~ & ~ & ~ & ~ & ~ & ~ & ~ & ~ & 28 & ~ \\ \hline
        1811 & 12. & März & Kuleschowka & Russland & ~ & ~ & 12 & ~ & ~ & ~ & ~ & ~ & ~ & ~ & ~ & ~ \\ \hline
        1811 & 8. & Juli & Berlanguillas & Spanien & ~ & ~ & ~ & ~ & ~ & ~ & 8 & ~ & ~ & ~ & ~ & ~ \\ \hline
        1812 & 10. & April & Toulouse & Frankreich & ~ & ~ & ~ & 10 & ~ & ~ & ~ & ~ & ~ & ~ & ~ & ~ \\ \hline
        1812 & 15. & April & Erxleben & Deutschland & ~ & ~ & ~ & 15 & ~ & ~ & ~ & ~ & ~ & ~ & ~ & ~ \\ \hline
        1812 & 5. & August & Chantonnay & Frankreich & ~ & ~ & ~ & ~ & ~ & ~ & ~ & 5 & ~ & ~ & ~ & ~ \\ \hline
        1813 & 14. & März & Cutro & Italien & ~ & ~ & 14 & ~ & ~ & ~ & ~ & ~ & ~ & ~ & ~ & ~ \\ \hline
        1813 & - & Juli & Malpas & England & ~ & ~ & ~ & ~ & ~ & ~ & x. & ~ & ~ & ~ & ~ & ~ \\ \hline
        1813 & 10. & September & Adair & Irland & ~ & ~ & ~ & ~ & ~ & ~ & ~ & ~ & 10 & ~ & ~ & ~ \\ \hline
        1813 & 13. & Dezember & Lontalax & Finnland & ~ & ~ & ~ & ~ & ~ & ~ & ~ & ~ & ~ & ~ & ~ & 13 \\ \hline
        1814 & 15. & Februar & Bachmut & Russland & ~ & 15 & ~ & ~ & ~ & ~ & ~ & ~ & ~ & ~ & ~ & ~ \\ \hline
        1814 & 5. & September & Agen & Frankreich & ~ & ~ & ~ & ~ & ~ & ~ & ~ & ~ & 5 & ~ & ~ & ~ \\ \hline
        1815 & 3. & Oktober & Chassigny & Frankreich & ~ & ~ & ~ & ~ & ~ & ~ & ~ & ~ & ~ & 3 & ~ & ~ \\ \hline
        1816 & E. & Juli & Glastonbury & England & ~ & ~ & ~ & ~ & ~ & ~ & E. & ~ & ~ & ~ & ~ & ~ \\ \hline
        1818 & 10. & April & Zjaborzyka & Volhynien & ~ & ~ & ~ & 10 & ~ & ~ & ~ & ~ & ~ & ~ & ~ & ~ \\ \hline
        1818 & - & Juni & Seres & Türkei & ~ & ~ & ~ & ~ & ~ & x. & ~ & ~ & ~ & ~ & ~ & ~ \\ \hline
        1818 & 10. & August & Slobodka & Russland & ~ & ~ & ~ & ~ & ~ & ~ & ~ & 10 & ~ & ~ & ~ & ~ \\ \hline
        1819 & E. & April & Massa-Lubrense & Italien & ~ & ~ & ~ & E. & ~ & ~ & ~ & ~ & ~ & ~ & ~ & ~ \\ \hline
        1819 & 13. & Juni & Jonzac und Barbézieux & Frankreich & ~ & ~ & ~ & ~ & ~ & 13 & ~ & ~ & ~ & ~ & ~ & ~ \\ \hline
        1819 & 13. & Oktober & Politz & Deutschland & ~ & ~ & ~ & ~ & ~ & ~ & ~ & ~ & ~ & 13 & ~ & ~ \\ \hline
        1820 & 22. & Mai & Oedenburg & Ungarn & ~ & ~ & ~ & ~ & 22 & ~ & ~ & ~ & ~ & ~ & ~ & ~ \\ \hline
        1820 & 12. & Juli & Lasdany & Russland & ~ & ~ & ~ & ~ & ~ & ~ & 12 & ~ & ~ & ~ & ~ & ~ \\ \hline
        1820 & 29. & November & Cosenza & Italien & ~ & ~ & ~ & ~ & ~ & ~ & ~ & ~ & ~ & ~ & 29 & ~ \\ \hline
    \end{tabular}
\end{table}
\end{landscape}
\clearpage
\begin{landscape}
\begin{table}[!ht]
    \footnotesize
    \centering
    \begin{tabular}{|p{5mm}|p{4mm}|p{13mm}|p{23mm}|p{16mm}|p{4mm}|p{6mm}|p{6mm}|p{6mm}|p{4mm}|p{5mm}|p{4mm}|p{5mm}|p{6mm}|p{5mm}|p{5mm}|p{5mm}|}
    \hline
         &  &  &  &  & Jan. & Febr. & März & April & Mai & Juni & Juli & Aug. & Sept. & Okt. & Nov. & Dez. \\ \hline
        1821 & 15. & Juni & Juvinas & Frankreich & ~ & ~ & ~ & ~ & ~ & 15 & ~ & ~ & ~ & ~ & ~ & ~ \\ \hline
        1821 & 21. & Juni & Mayo. Hagel mit Metallkernen & Irland & ~ & ~ & ~ & ~ & ~ & 21 & ~ & ~ & ~ & ~ & ~ & ~ \\ \hline
        1822 & 3. & Juni & Angers & Frankreich & ~ & ~ & ~ & ~ & ~ & 3 & ~ & ~ & ~ & ~ & ~ & ~ \\ \hline
        1822 & 13. & September & la Baffe & Frankreich & ~ & ~ & ~ & ~ & ~ & ~ & ~ & ~ & 13 & ~ & ~ & ~ \\ \hline
        1824 & 13. & Januar & Renazzo & Italien & 13 & ~ & ~ & ~ & ~ & ~ & ~ & ~ & ~ & ~ & ~ & ~ \\ \hline
        1824 & 14. & Oktober & Praskoles & Böhmen & ~ & ~ & ~ & ~ & ~ & ~ & ~ & ~ & ~ & 14 & ~ & ~ \\ \hline
        1825 & 12. & Mai & Bayden. Eisen & England & ~ & ~ & ~ & ~ & 12 & ~ & ~ & ~ & ~ & ~ & ~ & ~ \\ \hline
        1826 & 19. & Mai & Paulowgrad & Russland & ~ & ~ & ~ & ~ & 19 & ~ & ~ & ~ & ~ & ~ & ~ & ~ \\ \hline
        1827 & 5. & Oktober & Kuasti-Knasti & Russland & ~ & ~ & ~ & ~ & ~ & ~ & ~ & ~ & ~ & 5 & ~ & ~ \\ \hline
        1828 & - & Mai & Tscheroi. Anhydrit. & Türkei & ~ & ~ & ~ & ~ & x. & ~ & ~ & ~ & ~ & ~ & ~ & ~ \\ \hline
        1828 & - & August & Allport & England & ~ & ~ & ~ & ~ & ~ & ~ & ~ & x. & ~ & ~ & ~ & ~ \\ \hline
        1829 & 9. & September & Krasnoi-Ugol & Russland & ~ & ~ & ~ & ~ & ~ & ~ & ~ & ~ & 9 & ~ & ~ & ~ \\ \hline
        1830 & 15. & Februar & Launton & England & ~ & 15 & ~ & ~ & ~ & ~ & ~ & ~ & ~ & ~ & ~ & ~ \\ \hline
        1831 & 18. & Juli & Vouillé & Frankreich & ~ & ~ & ~ & ~ & ~ & ~ & 18 & ~ & ~ & ~ & ~ & ~ \\ \hline
        1831 & 9. & September & Znorow & Mahren & ~ & ~ & ~ & ~ & ~ & ~ & ~ & ~ & 9 & ~ & ~ & ~ \\ \hline
        1833 & 25. & November & Blansko & Mahren & ~ & ~ & ~ & ~ & ~ & ~ & ~ & ~ & ~ & ~ & 25 & ~ \\ \hline
        1833 & 27. & Dezember & Okniny & Volhynien & ~ & ~ & ~ & ~ & ~ & ~ & ~ & ~ & ~ & ~ & ~ & 27 \\ \hline
        1834 & 15. & Dezember & Marsala & Sicilien & ~ & ~ & ~ & ~ & ~ & ~ & ~ & ~ & ~ & ~ & ~ & 15 \\ \hline
        1835 & 18. & Januar & Löbau & Deutschland & 18 & ~ & ~ & ~ & ~ & ~ & ~ & ~ & ~ & ~ & ~ & ~ \\ \hline
        1835 & 4. & August & Cirencester & England & ~ & ~ & ~ & ~ & ~ & ~ & ~ & 4 & ~ & ~ & ~ & ~ \\ \hline
        1835 & 13. & November & Summonod & Frankreich & ~ & ~ & ~ & ~ & ~ & ~ & ~ & ~ & ~ & ~ & 13 & ~ \\ \hline
        1837 & 15. & Januar & Mikolowa & Ungarn & 15 & ~ & ~ & ~ & ~ & ~ & ~ & ~ & ~ & ~ & ~ & ~ \\ \hline
        1837 & 24. & Juli & Groß-Divina & Ungarn & ~ & ~ & ~ & ~ & ~ & ~ & 24 & ~ & ~ & ~ & ~ & ~ \\ \hline
        1837 & - & August & Esnandes & Frankreich & ~ & ~ & ~ & ~ & ~ & ~ & ~ & x. & ~ & ~ & ~ & ~ \\ \hline
    \end{tabular}
\end{table}
\end{landscape}
\clearpage
\begin{landscape}
\begin{table}[!ht]
    \footnotesize
    \centering
    \begin{tabular}{|p{5mm}|p{4mm}|p{13mm}|p{23mm}|p{16mm}|p{4mm}|p{6mm}|p{6mm}|p{6mm}|p{4mm}|p{5mm}|p{4mm}|p{5mm}|p{6mm}|p{5mm}|p{5mm}|p{5mm}|}
    \hline
         &  &  &  &  & Jan. & Febr. & März & April & Mai & Juni & Juli & Aug. & Sept. & Okt. & Nov. & Dez. \\ \hline
        1840 & 12. & Juni & Uden & Holland & ~ & ~ & ~ & ~ & ~ & 12 & ~ & ~ & ~ & ~ & ~ & ~ \\ \hline
        1840 & 17. & Juli & Cereseto & Italien & ~ & ~ & ~ & ~ & ~ & ~ & 17 & ~ & ~ & ~ & ~ & ~ \\ \hline
        1841 & 22. & März & Seifersholz & Deutschland & ~ & ~ & 22 & ~ & ~ & ~ & ~ & ~ & ~ & ~ & ~ & ~ \\ \hline
        1841 & 12. & Juni & Triguères & Frankreich & ~ & ~ & ~ & ~ & ~ & 12 & ~ & ~ & ~ & ~ & ~ & ~ \\ \hline
        1841 & 17. & Juli & Mailand & Italien & ~ & ~ & ~ & ~ & ~ & ~ & 17 & ~ & ~ & ~ & ~ & ~ \\ \hline
        1841 & 5. & November & Roche-Servière & Frankreich & ~ & ~ & ~ & ~ & ~ & ~ & ~ & ~ & ~ & ~ & 5 & ~ \\ \hline
        1842 & 26. & April & Pusinsko-Selo & Kroatien & ~ & ~ & ~ & 26 & ~ & ~ & ~ & ~ & ~ & ~ & ~ & ~ \\ \hline
        1842 & 4. & Juni & Aumières & Frankreich & ~ & ~ & ~ & ~ & ~ & 4 & ~ & ~ & ~ & ~ & ~ & ~ \\ \hline
        1842 & 4. & Juli & Logrono & Spanien & ~ & ~ & ~ & ~ & ~ & ~ & 4 & ~ & ~ & ~ & ~ & ~ \\ \hline
        1842 & 5. & August & Harrowgate & England & ~ & ~ & ~ & ~ & ~ & ~ & ~ & 5 & ~ & ~ & ~ & ~ \\ \hline
        1842 & 5. & Dezember & Eaufromont. Eisen. & Frankreich & ~ & ~ & ~ & ~ & ~ & ~ & ~ & ~ & ~ & ~ & ~ & 5 \\ \hline
        1843 & 2. & Juni & Blaauw-Kapel & Holland & ~ & ~ & ~ & ~ & ~ & 2 & ~ & ~ & ~ & ~ & ~ & ~ \\ \hline
        1843 & 16. & September & Kleinwenden & Deutschland & ~ & ~ & ~ & ~ & ~ & ~ & ~ & ~ & 16 & ~ & ~ & ~ \\ \hline
        1843 & 30. & Oktober & Werchne-Tschirskaja & Russland & ~ & ~ & ~ & ~ & ~ & ~ & ~ & ~ & ~ & 30 & ~ & ~ \\ \hline
        1844 & 29. & April & Killeter & Irland & ~ & ~ & ~ & 29 & ~ & ~ & ~ & ~ & ~ & ~ & ~ & ~ \\ \hline
        1844 & 21. & Oktober & Lessac & Frankreich & ~ & ~ & ~ & ~ & ~ & ~ & ~ & ~ & ~ & 21 & ~ & ~ \\ \hline
        1846 & 8. & Mai & Monte-Milone & Italien & ~ & ~ & ~ & ~ & 8 & ~ & ~ & ~ & ~ & ~ & ~ & ~ \\ \hline
        1846 & 10. & August & County Down. Eisen. & Irland & ~ & ~ & ~ & ~ & ~ & ~ & ~ & 10 & ~ & ~ & ~ & ~ \\ \hline
        1846 & 25. & Dezember & Schönenberg & Deutschland & ~ & ~ & ~ & ~ & ~ & ~ & ~ & ~ & ~ & ~ & ~ & 25 \\ \hline
        1847 & 14. & Juli & Hauptmannsdorf. Eisen. & Böhmen & ~ & ~ & ~ & ~ & ~ & ~ & 14 & ~ & ~ & ~ & ~ & ~ \\ \hline
        1848 & 27. & Dezember & Schie & Norwegen & ~ & ~ & ~ & ~ & ~ & ~ & ~ & ~ & ~ & ~ & ~ & 27 \\ \hline
    \end{tabular}
\end{table}
\end{landscape}
\clearpage
\begin{landscape}
\begin{table}[!ht]
    \footnotesize
    \centering
    \begin{tabular}{|p{5mm}|p{4mm}|p{13mm}|p{23mm}|p{17mm}|p{4mm}|p{6mm}|p{6mm}|p{6mm}|p{4mm}|p{5mm}|p{4mm}|p{5mm}|p{6mm}|p{5mm}|p{5mm}|p{5mm}|}
    \hline
         &  &  &  &  & Jan. & Febr. & März & April & Mai & Juni & Juli & Aug. & Sept. & Okt. & Nov. & Dez. \\ \hline
        1850 & 22. & Juni & Oviedo & Spanien & ~ & ~ & ~ & ~ & ~ & 22 & ~ & ~ & ~ & ~ & ~ & ~ \\ \hline
        1851 & 17. & April & Gütersloh & Deutschland & ~ & ~ & ~ & 17 & ~ & ~ & ~ & ~ & ~ & ~ & ~ & ~ \\ \hline
        1852 & 4. & September & Fekete und Istento & Ungarn & ~ & ~ & ~ & ~ & ~ & ~ & ~ & ~ & 4 & ~ & ~ & ~ \\ \hline
        1852 & 13. & Oktober & Borkut & Ungarn & ~ & ~ & ~ & ~ & ~ & ~ & ~ & ~ & ~ & 13 & ~ & ~ \\ \hline
        1853 & 10. & Februar & Girgenti & Sicilien & ~ & 10 & ~ & ~ & ~ & ~ & ~ & ~ & ~ & ~ & ~ & ~ \\ \hline
        1854 & 5. & September & Linum & Deutschland & ~ & ~ & ~ & ~ & ~ & ~ & ~ & ~ & 5 & ~ & ~ & ~ \\ \hline
        1855 & 11. & Mai & Ösel & Russland & ~ & ~ & ~ & ~ & 11 & ~ & ~ & ~ & ~ & ~ & ~ & ~ \\ \hline
        1855 & 13. & Mai & Bremervörde & Deutschland & ~ & ~ & ~ & ~ & 13 & ~ & ~ & ~ & ~ & ~ & ~ & ~ \\ \hline
        1855 & 7. & Juni & St. Denis-Westrem & Belgien & ~ & ~ & ~ & ~ & ~ & 7 & ~ & ~ & ~ & ~ & ~ & ~ \\ \hline
        1856 & 17. & September & Civita-Vecchia & Italien & ~ & ~ & ~ & ~ & ~ & ~ & ~ & ~ & 17 & ~ & ~ & ~ \\ \hline
        1856 & 12. & November & Trenzano & Italien & ~ & ~ & ~ & ~ & ~ & ~ & ~ & ~ & ~ & ~ & 12 & ~ \\ \hline
        1857 & 15. & April & Kaba & Ungarn & ~ & ~ & ~ & 15 & ~ & ~ & ~ & ~ & ~ & ~ & ~ & ~ \\ \hline
        1857 & 1. & Oktober & les Ormes & Frankreich & ~ & ~ & ~ & ~ & ~ & ~ & ~ & ~ & ~ & 1 & ~ & ~ \\ \hline
        1857 & 10. & Oktober & Ohaba & Siebenburgen & ~ & ~ & ~ & ~ & ~ & ~ & ~ & ~ & ~ & 10 & ~ & ~ \\ \hline
        1858 & 19. & Mai & Kakova & Ungarn & ~ & ~ & ~ & ~ & 19 & ~ & ~ & ~ & ~ & ~ & ~ & ~ \\ \hline
        1858 & 9. & Dezember & Clarae und Aussun & Frankreich & ~ & ~ & ~ & ~ & ~ & ~ & ~ & ~ & ~ & ~ & ~ & 9 \\ \hline
    \end{tabular}
\end{table}
\end{landscape}
\clearpage
\vspace*{\fill}
\section{Asiatische Meteorsteinfälle seit dem Jahre 1700, nach den 12 Monaten geordnet.}
\vspace*{\fill}
\clearpage
\begin{landscape}
\begin{table}[!ht]
    \centering
    \footnotesize
    \begin{tabular}{|l|l|l|l|l|l|l|l|l|l|l|l|l|l|l|l|l|}
    \hline
         & & & & & Jan. & Febr. & März & April & Mai & Juni & Juli & Aug. & Sept. & Okt. & Nov. & Dez. \\ \hline
        1795 & 13. & April & Ceylon & Indien & ~ & ~ & ~ & 13 & ~ & ~ & ~ & ~ & ~ & ~ & ~ & ~ \\ \hline
        1798 & 13. & Dezember & Krak-Hut & Indien & ~ & ~ & ~ & ~ & ~ & ~ & ~ & ~ & ~ & ~ & ~ & 13 \\ \hline
        1805 & 25. & März & Doroninsk & Russland & ~ & ~ & 25 & ~ & ~ & ~ & ~ & ~ & ~ & ~ & ~ & ~ \\ \hline
        1810 & M. & Juli & Shabad & Indien & ~ & ~ & ~ & ~ & ~ & ~ & M. & ~ & ~ & ~ & ~ & ~ \\ \hline
        1811 & 23. & November & Panganoor. Eisen. & Indien & ~ & ~ & ~ & ~ & ~ & ~ & ~ & ~ & ~ & ~ & 23 & ~ \\ \hline
        1814 & 5. & November & Doab & Indien & ~ & ~ & ~ & ~ & ~ & ~ & ~ & ~ & ~ & ~ & 5 & ~ \\ \hline
        1815 & 18. & Februar & Dooralla & Indien & ~ & 18 & ~ & ~ & ~ & ~ & ~ & ~ & ~ & ~ & ~ & ~ \\ \hline
        1822 & 7. & August & Kadonah & Indien & ~ & ~ & ~ & ~ & ~ & ~ & ~ & 7 & ~ & ~ & ~ & ~ \\ \hline
        1822 & 30. & November & Rourpoor & Indien & ~ & ~ & ~ & ~ & ~ & ~ & ~ & ~ & ~ & ~ & 30 & ~ \\ \hline
        1824 & 18. & Februar & Tounkin & Sibirien & ~ & 18 & ~ & ~ & ~ & ~ & ~ & ~ & ~ & ~ & ~ & ~ \\ \hline
        1825 & 16. & Januar & Oriang & Indien & 16 & ~ & ~ & ~ & ~ & ~ & ~ & ~ & ~ & ~ & ~ & ~ \\ \hline
        1827 & 27. & Februar & Mhow & Indien & ~ & 27 & ~ & ~ & ~ & ~ & ~ & ~ & ~ & ~ & ~ & ~ \\ \hline
        1833 & E. & November & Kandahar & Afghanistan & ~ & ~ & ~ & ~ & ~ & ~ & ~ & ~ & ~ & ~ & E. & ~ \\ \hline
        1834 & 12. & Juni & Charwallas & Indien & ~ & ~ & ~ & ~ & ~ & 12 & ~ & ~ & ~ & ~ & ~ & ~ \\ \hline
        1838 & 18. & April & Akburpoor & Indien & ~ & ~ & ~ & 18 & ~ & ~ & ~ & ~ & ~ & ~ & ~ & ~ \\ \hline
        1838 & 6. & Juni & Chandakapoor & Indien & ~ & ~ & ~ & ~ & ~ & 6 & ~ & ~ & ~ & ~ & ~ & ~ \\ \hline
        1840 & 9. & Mai & Kirgisen-Steppe & Russland & ~ & ~ & ~ & ~ & 9 & ~ & ~ & ~ & ~ & ~ & ~ & ~ \\ \hline
        1842 & 30. & November & Jeetala & Indien & ~ & ~ & ~ & ~ & ~ & ~ & ~ & ~ & ~ & ~ & 30 & ~ \\ \hline
        1843 & 26. & Juli & Manjegaon & Indien & ~ & ~ & ~ & ~ & ~ & ~ & 26 & ~ & ~ & ~ & ~ & ~ \\ \hline
        1848 & 15. & Februar & Negloor & Indien & ~ & 15 & ~ & ~ & ~ & ~ & ~ & ~ & ~ & ~ & ~ & ~ \\ \hline
        1850 & 30. & November & Shalka & Indien & ~ & ~ & ~ & ~ & ~ & ~ & ~ & ~ & ~ & ~ & 30 & ~ \\ \hline
        1853 & 6. & März & Segowlee & Indien & ~ & ~ & 6 & ~ & ~ & ~ & ~ & ~ & ~ & ~ & ~ & ~ \\ \hline
        1857 & 28. & Februar & Parnallee & Indien & ~ & 28 ~ & ~ & ~ & ~ & ~ & ~ & ~ & ~ & ~ & ~ & ~ \\ \hline
    \end{tabular}
\end{table}
\end{landscape}
\clearpage
\section{Namen-Verzeichnis zu den auf den Karten 1. 2. u. 3. verzeichneten und für zuverlässig zu erachtenden Meteorstein- und Meteoreisen-Fällen.}
\begin{enumerate}
    \item Ortsnummer auf der betreffenden Karte.
    \item Fallzeit.
    \item Fundort und spezifische Schwere der Gesteine.
    \item Geographische Breite.
    \item Geographische Lange nach Greenwich.
    \item Belege.
    \item[$^\wedge$$^\wedge$$^\wedge$] Orte, deren genaue Lage bis jetzt noch nicht ermittelt werden konnte.
\end{enumerate}
\clearpage
\subsection{Karte 1. - Europa.}
\subsubsection{1. England, Schottland und Irland}
\begin{center}
    \footnotesize
    \begin{longtable}{|p{3mm}|p{6mm}|p{3mm}|p{10mm}|p{33mm}|p{20mm}|p{12mm}|p{12mm}|p{11mm}|}
    \hline
        1. & 2. & 2. & 2. & 3. & 3. & 4. & 5. & 6. \\ \hline
        1. & 1622 & 10. & Januar & Tregnie, angeblich in Devonshire; wahrscheinlich aber Tregony, 16 M. SW. von Bodmin in Cornwallis, da ein Ort jenes Namens in Devonshire nicht zu finden ist. & Cornwallis ? & 50$^\circ$ 16$^\prime$ N. ? & 4$^\circ$ 55$^\prime$ W. ? & G. 50. 1815. 241. \\ \hline
        2. & 1628 & 9. & April & Hatford, 3 M. O. von Faringdon. & Berkshire & 51$^\circ$ 40$^\prime$ N. & 1$^\circ$ 32$^\prime$ W. & G. 54. 1816. 344. \\ \hline
        3. & 1642 & 4. & August & Zwischen Woodbridge und Alborow (Alborough, Aldeburgh oder Aldborough), ONO. von Ipswich. & Suffolk & Zwischen 52$^\circ$ 5$^\prime$ N. und 52$^\circ$ 8$^\prime$ N. & Zwischen 1$^\circ$ 18$^\prime$ O. und 1$^\circ$ 35$^\prime$ O. & G. 54. 1816. 345. \\ \hline
        4. & 1725 & 3. & Juli & Mixbury, 7 M. NNO. von Bicester. & Oxfordshire & 51$^\circ$ 58$^\prime$ N. & 1$^\circ$ 6$^\prime$ W. & RPG. 35. \\ \hline
        5. & 1731 & 12. & März & Halstead, WNW. von Colchester. & Essex & 51$^\circ$ 57$^\prime$ N. & 0$^\circ$ 37$^\prime$ O. & K. 3. 271. \\ \hline
        6. & 1779 & - & - & Pettiswood (oder Petitswood, aber nicht Petriswood), ein Hügel bei Mullingar, Grafschaft Westmeath. & Irland & 53$^\circ$ 31$^\prime$ N. & 7$^\circ$ 19$^\prime$ W. & G. 50. 1815. 250. \\ \hline
        7. & 1780 & 11. & April & Beeston, 3 M. SW. von Nottingham. & Nottinghamshire & 52$^\circ$ 55$^\prime$ N. & 1$^\circ$ 10$^\prime$ W. & K. 3. 276. \\ \hline
        8. & 1795 & 13. & Dezember & Wold-Cottage, 9 M. NNO. von Great-Driffield, S. von Wold-Newton. - Sp.-Gew.: 3,508-4,02. & Yorkshire & 54$^\circ$ 9$^\prime$ N. & 0$^\circ$ 24$^\prime$ W. & G. 13. 1803. 297. und 305. W. 1860. S. 1860. \\ \hline
        9. & 1802 & ~ & Mitte Sept. & Am Loch-Tay. & Schottland & Zwischen 56$^\circ$ 20$^\prime$ N. und 56$^\circ$ 40$^\prime$ N. & Zwischen 3$^\circ$ 55$^\prime$ W. und 4$^\circ$ 25$^\prime$ W. & G. 54. 1816. 352. \\ \hline
        10. & 1803 & 4. & Juli & East-Norton, 9 M. NNO. von Market-Harboro’. & Leicestershire & 52$^\circ$ 25$^\prime$ N. & 0$^\circ$ 51$^\prime$ W. & G. 50. 1815. 252. \\ \hline
        11. & 1804 & 5. & April & High-Possil, 3 M. N. von Glasgow. - Sp.-Gew.: 3,53. & Schottland & 55$^\circ$ 54$^\prime$ N. & 4$^\circ$ 18$^\prime$ W. & G. 24. 1806. 370. W. 1860. \\ \hline
        12. & 1806 & 17. & Mai & Basingstoke, NO. von Winchester. & Hantshire & 51$^\circ$ 17$^\prime$ N. & 1$^\circ$ 6$^\prime$ W. & G. 54. 1816. 353. \\ \hline
        13. & 1810 & ~ & Mitte August & Mooresfort (Moore’s Fort), 5 M. W. von Tipperary, Grafschaft Tipperary. & Irland & 52$^\circ$ 28$^\prime$ N. & 8$^\circ$ 11$^\prime$ W. & G. 63. 1819. 22. W. 1860. S. 1860. \\ \hline
        14. & 1813 & - & Juli oder August & Malpas, SSO. von Chester. & Chestershire & 53$^\circ$ 4$^\prime$ N. & 2$^\circ$ 48$^\prime$ W. & Ann. Of Phil. 2. Nov. 1813. 396. \\ \hline
        15. & 1813 & 10. & September & Adair (Adare), SW. von Limerick; Faha, nahe bei St. Patrickswell, ONO. von Adair; Scough (Scagh), 2 M. NNW. von Rathkeale, WSW. von Adair; und Brasky ($^\wedge$$^\wedge$$^\wedge$). Sammtlich in der Grafschaft Limerick. - Sp.-Gew.: 3,62-4,23. & Irland & 52$^\circ$ 30$^\prime$ N., 52$^\circ$ 32$^\prime$ N., 52$^\circ$ 29$^\prime$ N. & 8$^\circ$ 42 W., 8$^\circ$ 36$^\prime$ W., 8$^\circ$ 50$^\prime$ W. & G. 54. 1816. 355. W. 1860. S. 1860. \\ \hline
        16. & Wahrscheinlich 1813; jedenfalls vor 1819 & ~ & ~ & Pulrose ($^\wedge$$^\wedge$$^\wedge$). & Insel Man & Zwischen 54$^\circ$ 4$^\prime$ N. und 54$^\circ$ 26$^\prime$ N. & Zwischen 4$^\circ$ 15$^\prime$ W. und 4$^\circ$ 44$^\prime$ W. & G. 68. 1821. 333. \\ \hline
        17. & 1816 & ~ & Ende Juli oder Anf. August & Glastonbury, SW. von Wells. & Somersetshire & 51$^\circ$ 9$^\prime$ N. & 2$^\circ$ 42$^\prime$ W. & G. 53. 1816. 384. \\ \hline        
        18. & 1821 & 21. & Juni & Grafschaft Mayo. Hagel mit Metallkernen. & Irland & Zwischen 53$^\circ$ 30$^\prime$ N. und 54$^\circ$ 25$^\prime$ N. & Zwischen 8$^\circ$ 30$^\prime$ W. und 10$^\circ$ 20$^\prime$ W. & G. 72. 1822. 436. \\ \hline
        19. & 1825 & 12. & Mai & Bayden, NW. von Hungerford und NO. von Marlborough. Eisen. & Wiltshire & 51$^\circ$ 30$^\prime$ N. & 1$^\circ$ 36$^\prime$ W. & P. 8. 1826. 49. \\ \hline
        20. & 1828 & - & August & Allport, 5. M. NNW. von Castleton. - Sp.-Gew.: 2,00. & Derbyshire & 53$^\circ$ 24$^\prime$ N. & 1$^\circ$ 48$^\prime$ W. & P. 4. 1854. 43. \\ \hline
        21. & 1830 & 15. & Februar & Launton, 2 M. O. von Bicester. & Oxfordshire & 51$^\circ$ 54$^\prime$ N. & 1$^\circ$ 9$^\prime$ W. & P. 54. 1841. 291. \\ \hline
        22. & 1835 & 4. & August & Cirencester. & Glocestershire & 51$^\circ$ 43$^\prime$ N. & 1$^\circ$ 58$^\prime$ W. & RPG. 37. \\ \hline
        23. & 1842 & 5. & August & Harrowgate, SW. von Leeds und NW. von Sheffield. & Yorkshire & 53$^\circ$ 38$^\prime$ N. & 1$^\circ$ 50$^\prime$ W. & P. 4. 1854. 366. \\ \hline
        24. & 1844 & 29. & April & Killeter (Killeeter, Kelleter oder Killetter), WNW. von Omagh und SSW. von Strabone in North-Tyrone. - Sp.-Gew.: 3,63? & Irland & 54$^\circ$ 44$^\prime$ N. & 7$^\circ$ 40$^\prime$ W. & RPG. 37. P. 107. 1859. 161. S. 1860. \\ \hline
        25. & 1846 & 10. & August & Im Norden der Grafschaft Down. - Eisen. - Sp.-Gew.: 5,9. & Irland & Zwischen 54$^\circ$ 0$^\prime$ N. und 54$^\circ$ 44$^\prime$ N. & Zwischen 5$^\circ$ 30$^\prime$ W. und 6$^\circ$ 30$^\prime$ W. & P. 4. 1854. 434. \\ \hline
    \end{longtable}
\end{center}
\clearpage
\subsubsection{2. Spanien und Portugal}
\begin{table}[!ht]
    \centering
    \footnotesize
    \begin{tabular}{|l|l|l|l|p{40mm}|l|p{14mm}|p{14mm}|p{18mm}|}
    \hline
        1. & 2. & 2. & 2. & 3. & 3. & 4. & 5. & 6. \\ \hline
        1. & 1438 & - & - & Roa, S. von Burgos. & Alt-Kastilien & 41$^\circ$ 42$^\prime$ N. & 3$^\circ$ 56$^\prime$ W. & G. 50. 1815. 235. \\ \hline
        2. & 1520 & - & Mai & Zwischen Oliva und Gandia. & Aragonien & Zwischen 38$^\circ$ 56$^\prime$ N. und 39$^\circ$ 0$^\prime$ N. & Zwischen 0$^\circ$ 6$^\prime$ W. und 0$^\circ$ 10$^\prime$ W. & G. 54. 1816. 342. \\ \hline
        3. & Vor 1603 & - & - & Valencia. & Valencia & 39$^\circ$ 28$^\prime$ N. & 0$^\circ$ 22$^\prime$ W. & G. 50. 1815. 240. \\ \hline
        4. & 1704 & 24. (25.) & Dezember & Barcelona. & Katalonien & 41$^\circ$ 24$^\prime$ N. & 2$^\circ$ 10$^\prime$ O. & P. 8. 1826. 46. \\ \hline
        5. & 1773 & 17. & November & Sena, NW. von Sixena (Sigena). - Sp.-Gew.: 3,63. & Aragonien & 41$^\circ$ 36$^\prime$ N. & 0$^\circ$ 0$^\prime$. & G. 24. 1806. 93. W. 1860. \\ \hline
        6. & 1796 & 19. & Februar & Tasquinha ($^\wedge$$^\wedge$$^\wedge$) bei Evora-Monte (38$^\circ$ 43$^\prime$ N., 7$^\circ$ 27$^\prime$ W.), O. von Lissabon und NO. von Evora; Provinz Alemtejo.* & Portugal & ~ & ~ & G. 13. 1803. 291. R. Southey, Letters u. s. w., 2 fo. 72.* \\ \hline
    \end{tabular}
\end{table}
\clearpage
\begin{table}[!ht]
    \centering
    \footnotesize
    \begin{tabular}{|l|l|l|l|p{40mm}|l|p{14mm}|p{14mm}|p{14mm}|}
    \hline
        1. & 2. & 2. & 2. & 3. & 3. & 4. & 5. & 6. \\ \hline
        7. & 1811 & 8. & Juli & Berlanguillas ($^\wedge$$^\wedge$$^\wedge$), zwischen Aranda und Roa, S. von Burgos. - Sp.Gew.: 3,49. & Alt-Kastilien & Zwischen 41$^\circ$ 40$^\prime$ N. und 41$^\circ$ 42$^\prime$ N. & Zwischen 3$^\circ$ 40$^\prime$ W. und 3$^\circ$ 56$^\prime$ W. & G. 40. 1812. 116. W. 1860. S. 1860. \\ \hline
        8. & 1842 & 4. & Juli & Logrono. & Alt-Kastilien & 42$^\circ$ 23$^\prime$ N. & 2$^\circ$ 30$^\prime$ W. & RPG. 37. \\ \hline
        9. & 1851 & 5. & November & Saragossa.* - Sp.-Gew.: 3,80. & Aragonien & 41$^\circ$ 38$^\prime$ N. & 0$^\circ$ 45$^\prime$ W. & RPG. \\ \hline
    \end{tabular}
\end{table}
\clearpage
\subsubsection{3. Frankreich}
\begin{table}[!ht]
    \centering
    \footnotesize
    \begin{tabular}{|p{5mm}|p{12mm}|p{5mm}|p{15mm}|p{48mm}|p{25mm}|p{12mm}|p{12mm}|p{11mm}|}
    \hline
        1. & 2. & 2. & 2. & 3. & 3. & 4. & 5. & 6. \\ \hline
        1. & Zwischen 1 und 50 & - & - & Im Lande der Vocontier, dem östlichen Teil der heutigen Dauphiné; darinnen die Stadte Die (Dea) und Vaisin (Vasio) liegen. & Dauphiné & Zwischen 44$^\circ$ 15$^\prime$ N. und 44$^\circ$ 40$^\prime$ N. & Zwischen 5$^\circ$ 0$^\prime$ O. und 5$^\circ$ 20$^\prime$ O. & G. 18. 1804. 305. \\ \hline
        2. & 1492 & 7. & November & Ensisheim im Sundgau. - Sp.Gew.: 3,233-3,48. & Ober-Elsass & 47$^\circ$ 51$^\prime$ N. & 7$^\circ$ 22$^\prime$ O. & G. 13. 1803. 295. W. 1860. S. 1860. \\ \hline
        3. & 1634 & 27. & Oktober & Provinz des Charollais (Charolais oder Grafschaft Carolath) in Burgund (Hauptstadt: Charolles). & Dép. de Saone et Loire & Zwischen 46$^\circ$ 20$^\prime$ N. und 46$^\circ$ 45$^\prime$ N. & Zwischen 3$^\circ$ 55$^\prime$ O. und 4$^\circ$ 30$^\prime$ O. & G. 50. 1815. 242. \\ \hline
        4. & 1750 & 1. (11.) & Oktober & Nicor (Nicorps oder Niort), SO. von Coutance; Normandie. & Dép. de la Manche & 49$^\circ$ 2$^\prime$ N. & 1$^\circ$ 26$^\prime$ W. & G. 50. 1815. 248. \\ \hline
        5. & 1753 & 7. & September & Luponnas (oder Luponay-sur-Veyle, nicht Liponas oder Laponas), NNW. von Vonnas und 4 Stunden von Pont-de-Veyle, zwischen dieser Stadt und Bourg-en-Bresse. - Sp.-Gew.: 3,66. & Dép. de l’Ain & 46$^\circ$ 14$^\prime$ N. & 4$^\circ$ 59$^\prime$ O. & G. 13. 1803. 343. W. 1860. \\ \hline
    \end{tabular}
\end{table}
\clearpage
\begin{table}[!ht]
    \centering
    \footnotesize
    \begin{tabular}{|p{4mm}|p{8mm}|p{4mm}|p{15mm}|p{48mm}|p{25mm}|p{12mm}|p{12mm}|p{16mm}|}
    \hline
        1. & 2. & 2. & 2. & 3. & 3. & 4. & 5. & 6. \\ \hline
        6. & 1768 & 13. & September & Luce en Maine, Bezirk von St. Calais. - Sp.-Gew.: 3,47 bis 3,535. & Dép. de la Sarthe & 47$^\circ$ 52$^\prime$ N. & 0$^\circ$ 30$^\prime$ O. & G. 54. 1816. 348. W. 1860. S. 1860. \\ \hline
        7. & 1768 & - & - & Aire en Artois. & Dép. du Pas-de-Calais & 50$^\circ$ 38$^\prime$ N. & 2$^\circ$ 24$^\prime$ O. & G. 54. 1816. 348. \\ \hline
        8. & 1790 & 24. & Juli & Barbotan, ONO. von Cazaubon; und zwischen Créon und Lagrange-de-Julliac, beide W. von Gabarret en Armagnac in der Gascogne. - Sp.-Gew.: 3,62. & Dép. du Gers, Dép. des Landes & 43$^\circ$ 57$^\prime$ N., 43$^\circ$ 59$^\prime$ N. & 0$^\circ$ 4$^\prime$ W., 0$^\circ$ 7$^\prime$ W. & G. 13. 1803. 346. W. 1860. S. 1860. \\ \hline
        9. & 1798 & 12. & März & Sales, NW. von Villefranche bei Lyon. & Dép. du Rhone & 46$^\circ$ 3$^\prime$ N. & 4$^\circ$ 37$^\prime$ O. & G. 18. 1804. 264. und 270. W. 1860. S. 1860. \\ \hline
        10. & 1803 & 26. & April & l’Aigle, zwischen Evreux und Alençon; Fontenil ($^\wedge$$^\wedge$$^\wedge$) bei St. Sulpice-sur-Rille (48$^\circ$ 47, N., 0$^\circ$ 39 O.), NO. von l’Aigle; la Vassolerie ($^\wedge$$^\wedge$$^\wedge$) bei l’Aigle; St. Michel (St. Michel de Sommaire), NW. von l’Aigle; St. Nicolas (St. Nicolas de Sommaire), NNW. von l’Aigle; le Bas-Vernet, NW. von St. Nicolas und NNW. von l’Aigle; Glos, N. von l’Aigle; le Buat, S. von l’Aigle; le Futey (la Futaie), O. von St. Sulpice-sur-Rille und NO. von l’Aigle. - Sp.-Gew.: 3,39-3,49. & Dép. de l’Orne & 48$^\circ$ 45$^\prime$ N., 48$^\circ$ 48$^\prime$ N., 48$^\circ$ 49$^\prime$ N., 48$^\circ$ 49$^\prime$ N., 48$^\circ$ 52$^\prime$ N., 48$^\circ$ 44$^\prime$ N., 48$^\circ$ 47$^\prime$ N. & 0$^\circ$ 38$^\prime$ O., 0$^\circ$ 35$^\prime$ O., 0$^\circ$ 37$^\prime$ O., 0$^\circ$ 35$^\prime$ O., 0$^\circ$ 36$^\prime$ O., 0$^\circ$ 38$^\prime$ O., 0$^\circ$ 40$^\prime$ O. & G. 15. 1803. 74. W. 1860. S. 1860. \\ \hline
    \end{tabular}
\end{table}
\clearpage
\begin{table}[!ht]
    \centering
    \footnotesize
    \begin{tabular}{|p{4mm}|p{7mm}|p{4mm}|p{16mm}|p{48mm}|p{25mm}|l|l|p{16mm}|}
    \hline
        1. & 2. & 2. & 2. & 3. & 3. & 4. & 5. & 6. \\ \hline
        11. & 1803 & 8. & Oktober & Saurette ($^\wedge$$^\wedge$$^\wedge$) bei Apt (43 52 N., 5 23 O.). - Sp.-Gew.: 3,48. & Dép. de Vaucluse & ~ & ~ & G. 16. 1804. 73. W. 1860. S. 1860. \\ \hline
        12. & 1806 & 15. & März & St. Etienne-de-Lolm und Valence, OSO. von Vezenobres und SO. von Alais. - Sp.-Gew.: 1,70-1,94. & Dép. du Gard & 44$^\circ$ 0$^\prime$ N. & 4$^\circ$ 15$^\prime$ O. & G. 54. 1816. 353. W. 1860. S. 1860. \\ \hline
        13. & 1810 & 23. & November & Charsonville, Gemeinde Meung-sur-Loire, WNW. von Orléans und NNW. von Beaugency. - Sp.-Gew.: 3,36-3,75. & Dép. du Loiret & 47$^\circ$ 56$^\prime$ N. & 1$^\circ$ 35$^\prime$ O. & G. 37. 1811. 349. W. 1860. S. 1860. \\ \hline
        14. & 1812 & 10. & April & Burgau (le Bourgaut), 6 St. NW. von Toulouse; Peret ($^\wedge$$^\wedge$$^\wedge$), Gourdas ($^\wedge$$^\wedge$$^\wedge$), Seucourieux ($^\wedge$$^\wedge$$^\wedge$), Permejean ($^\wedge$$^\wedge$$^\wedge$), Pechmeja ($^\wedge$$^\wedge$$^\wedge$); sammtlich in der Gemeinde Grenade (43 46 N., 1 16 O.) NW. von Toulouse; und Las Pradere ($^\wedge$$^\wedge$$^\wedge$) bei Savenes (43 50 N., 1 11 O.), NW. von Toulouse und WSW. von Verdun. - Sp.-Gew.: 3,66-3,73. & Dép. De la Haute-Garonne, Dép. de Tarn et Garonne & 43$^\circ$ 47$^\prime$ N. & 1$^\circ$ 9$^\prime$ O. & G. 41. 1812. 445. Bigot de Morogues fo. 275. W. 1860. \\ \hline
        15. & 1812 & 5. & August & Chantonnay, O. von Bourbon-Vendee. - Sp.-Gew.: 3,44-3,49. & Dép. de la Vendée & 46$^\circ$ 40$^\prime$ N. & 1$^\circ$ 5$^\prime$ W. & G. 63. 1819. 228. W. 1860. S. 1860. \\ \hline
        16. & 1814 & 5. & September & Agen. - Sp.-Gew.: 3,59 bis 3,62. & Dép. du Lot et Garonne & 44$^\circ$ 12$^\prime$ N. & 0$^\circ$ 35$^\prime$ O. & G. 48. 1814. 340. W. 1860. S. 1860. \\ \hline
    \end{tabular}
\end{table}
\clearpage
\begin{table}[!ht]
    \centering
    \footnotesize
    \begin{tabular}{|p{4mm}|p{8mm}|p{4mm}|p{14mm}|p{48mm}|p{25mm}|l|l|p{16mm}|}
    \hline
        1. & 2. & 2. & 2. & 3. & 3. & 4. & 5. & 6. \\ \hline
        17. & 1815 & 3. & Oktober & Chassigny, 4 M. SSO. von Langres. - Sp.-Gew.: 3,55 bis 3,65. & Dép. de la Haute-Marne & 47$^\circ$ 43$^\prime$ N. & 5$^\circ$ 23$^\prime$ O. & G. 53. 1816. 381. W. 1860. S. 1860. \\ \hline
        18. & 1819 & 13. & Juni & Barbezieux, SW. von Angouleme; und Jonzac, W. von Barbezieux. - Sp.-Gew.: 3,08-3,12. & Dép. De la Charente, Dép. de la Charente-Inférieure & 45$^\circ$ 23$^\prime$ N., 45$^\circ$ 26$^\prime$ N. & 0$^\circ$ 11$^\prime$ W., 0$^\circ$ 27$^\prime$ W. & G. 63. 1819. 24. W. 1860. S. 1860. \\ \hline
        19. & 1821 & 15. & Juni & Juvinas (nicht Juvenas), NNW. von Aubenas und WSW. von Privas. Sp.-Gew.: 2,80 bis 3,11. & Dép. De l’Ardeche & 44$^\circ$ 42$^\prime$ N. & 4$^\circ$ 21$^\prime$ O. & G. 71. 1822. 201 und 360. W. 1860 S. 1860. \\ \hline
        20. & 1822 & 3. & Juni & Angers. & Dép. De Maine et Loire & 47$^\circ$ 28$^\prime$ N. & 0$^\circ$ 34$^\prime$ W. & G. 71. 1822. 345 und 361. \\ \hline
        21. & 1822 & 13. & September & la Baffe, O. von Epinal. - Sp.-Gew.: 3,66. & Vogesen & 48$^\circ$ 9$^\prime$ N. & 6$^\circ$ 35$^\prime$ O. & G. 72. 1822. 323. W. 1860. \\ \hline
        22. & 1831 & 18. & Juli & Vouille, WNW. von Poitiers. - Sp.-Gew.: 3,55. & Dép. De la Vienne & 46$^\circ$ 37$^\prime$ N. & 0$^\circ$ 8$^\prime$ O. & P. 34. 1835. 341. W. 1860. \\ \hline
        23. & 1835 & 13. & November & Simonod (Summonod), N. von Belmont, von Virieux-le-Grand und von Belley. - Sp.-Gew.: 1,35. & Dép. de l’Ain & 45$^\circ$ 55$^\prime$ N. & 5$^\circ$ 40$^\prime$ O. & P. 37. 1836. 460. W. 1860. \\ \hline
    \end{tabular}
\end{table}
\clearpage
\begin{table}[!ht]
    \centering
    \footnotesize
    \begin{tabular}{|p{5mm}|p{9mm}|p{5mm}|p{15mm}|p{48mm}|p{25mm}|l|l|p{11mm}|}
    \hline
        1. & 2. & 2. & 2. & 3. & 3. & 4. & 5. & 6. \\ \hline
        24. & 1837 & - & August & Esnandes (nicht Esnaude), N. von la Rochelle. - Sp.-Gew.: 3,47 (?). & Dép. De la Charente-Inférieure & 46$^\circ$ 14$^\prime$ N. & 1$^\circ$ 10$^\prime$ W. & P. 4. 1854. 357. W. 1860. S. 1860. \\ \hline
        25. & 1841 & 12. & Juni & Trigueres, O. von Chateau-Renard und OSO. von Montargis. - Sp.-Gew.: 3,54 bis 3,56. & Dép. du Loiret & 47$^\circ$ 56$^\prime$ N. & 2 58$^\prime$ O. & P. 53. 1841. 411. W. 1860. S. 1860. \\ \hline
        26. & 1841 & 5. & November & Roche-Serviere, N. von Bourbon-Vendee. & Dép. de la Vendée & 46$^\circ$ 56$^\prime$ N. & 1$^\circ$ 30$^\prime$ W. & P. 4. 1854. 92. \\ \hline
        27. & 1842 & 4. & Juni & Aumieres ($^\wedge$$^\wedge$$^\wedge$) bei St. Georges-de-Levejae (44 18 N., 3 13 O.), S. von Canourgue und W. von Florac; Canton Massegros. - Sp.-Gew.: 3,50 (?). & Dép. de la Lozere & ~ & ~ & W. 1860. S. 1860. \\ \hline
        28. & 1842 & 5. & Dezember & Eaufromont, O. von Epinal. Eisen. - Sp.-Gew.: 5,23. & Vogesen & 48$^\circ$ 10$^\prime$ N. & 6$^\circ$ 28$^\prime$ O. & P. 87. 1852. 320. \\ \hline
        29. & 1844 & 21. & Oktober & Lessac, N. von Confolens. & Dép. de la Charente & 46$^\circ$ 4$^\prime$ N. & 0$^\circ$ 38$^\prime$ O. & CR. 19. 1844. fo. 1181. S. 1860. \\ \hline
        30. & 1857 & 1. & Oktober & les Ormes, WSW. von Aillant-sur-Tholon und SSW. von Joigny. & Dép. de l’Yonne & 47$^\circ$ 51$^\prime$ N. & 3$^\circ$ 15$^\prime$ O. & B. 103. \\ \hline
    \end{tabular}
\end{table}
\clearpage
\begin{table}[!ht]
    \centering
    \footnotesize
    \begin{tabular}{|p{5mm}|p{9mm}|p{5mm}|p{15mm}|p{48mm}|p{25mm}|l|l|p{11mm}|}
    \hline
        1. & 2. & 2. & 2. & 3. & 3. & 4. & 5. & 6. \\ \hline
        31. & 1858 & 9. & Dezember & Clarac und Aussun, beide ONO. von Montrejeau u. W. von St. Gaudens. - Sp.-Gew.: 3,30. & Dép. de la Haute-Garonne & 43$^\circ$ 4$^\prime$ N., 43$^\circ$ 5$^\prime$ N. & 0$^\circ$ 35$^\prime$ O., 0$^\circ$ 33$^\prime$ O. & CR. 47. 1858. fo. 1053. W. 1860. S. 1860. \\ \hline
         & & & & Meteor-Eisenmasse, deren Fallzeit unbekannt. & & & & \\ \hline
        32. & - & - & - & la Caille, S. v. St. Auban und NW. von Grasse. 12 Zentner Gefunden 1828. - Sp.-Gew.: 7,642. & Dép. du Var & 43$^\circ$ 47$^\prime$ N. & 6$^\circ$ 43$^\prime$ O. & P. 18. 1830. 187. W. 1860. S. 1860. \\ \hline
    \end{tabular}
\end{table}
\clearpage
\subsubsection{4. Belgien und Holland}
\begin{table}[!ht]
    \centering
    \begin{tabular}{|l|p{17mm}|l|l|p{48mm}|l|l|l|p{13mm}|}
    \hline
        1. & 2. & 2. & 2. & 3. & 3. & 4. & 5. & 6. \\ \hline
        1. & Vor 1520 & - & - & Brüssel. & Belgien & 50$^\circ$ 51$^\prime$ N. & 4$^\circ$ 22$^\prime$ O. & G. 50. 1815. 239. \\ \hline
        2. & 1650 & 6. & August & Dordrecht. & Holland & 51$^\circ$ 48$^\prime$ N. & 4$^\circ$ 40$^\prime$ O. & G. 50. 1815. 243. \\ \hline
        3. & Zwischen 1804 und 1807 & ~ & - & Dordrecht. & Holland & 51$^\circ$ 48$^\prime$ N. & 4$^\circ$ 40$^\prime$ O. & G. 53. 1816. 379. \\ \hline
        4. & 1840 & 12. & Juni & Uden, O. von Herzogenbusch; Nordbrabant. & Holland & 51$^\circ$ 40$^\prime$ N. & 5$^\circ$ 35$^\prime$ O. & P. 59. 1843. 350. \\ \hline
        5. & 1843 & 2. & Juni & Blaauw-Kapel, NNO. von Utrecht. - Sp.-Gew.: 3,57 bis 3,65. & Holland & 52$^\circ$ 8$^\prime$ N. & 5$^\circ$ 8$^\prime$ O. & P. 59. 1843. 348. und 427. W. 1860. S. 1860. \\ \hline
        6. & 1855 & 7. & Juni & St. Denis-Westrem, 1. M. WSW. von Gent. - Sp.-Gew.: 3,29-3,40. & Belgien & 51$^\circ$ 4$^\prime$ N. & 3$^\circ$ 40$^\prime$ O. & P. 99. 1856. 63. \\ \hline
    \end{tabular}
\end{table}
\clearpage
\subsubsection{5. Schweden und Norwegen}
\begin{table}[!ht]
    \centering
    \begin{tabular}{|l|p{17mm}|l|l|p{48mm}|l|l|l|p{13mm}|}
    \hline
        1. & 2. & 2. & 2. & 3. & 3. & 4. & 5. & 6. \\ \hline
        1. & 1848 (1854) ? & 27. & Dezember & Schie, Filial zu Krogstad (59$^\circ$ 56$^\prime$ N., 11$^\circ$ 18$^\prime$ O.), Amt Aggerhuus. - Sp.-Gew.: 3,539. & Norwegen & ~ & ~ & P. 96. 1855. 341. \\ \hline
    \end{tabular}
\end{table}
\clearpage
\subsubsection{6. Dänemark}
\begin{table}[!ht]
    \centering
    \begin{tabular}{|l|l|l|l|l|l|l|l|p{13mm}|}
    \hline
        1. & 2. & 2. & 2. & 3. & 3. & 4. & 5. & 6. \\ \hline
        1. & 1654 & 30. & März & ? & Insel Fühnen & Zwischen 55$^\circ$ 2$^\prime$ N. Und 55$^\circ$ 38$^\prime$ N. & Zwischen 9$^\circ$ 45$^\prime$ O. Und 10$^\circ$ 50$^\prime$ O. & G. 18. 1804. 328. \\ \hline
    \end{tabular}
\end{table}
\clearpage
\subsubsection{7. Deutschland}
\begin{table}[!ht]
    \centering
    \begin{tabular}{|p{5mm}|p{9mm}|p{5mm}|p{15mm}|p{48mm}|p{20mm}|p{18mm}|p{18mm}|p{11mm}|}
    \hline
        1. & 2. & 2. & 2. & 3. & 3. & 4. & 5. & 6. \\ \hline
        1. & 951 & - & - & Augsburg; Kreis Schwaben. & Bayern & 48$^\circ$ 22$^\prime$ N. & 10$^\circ$ 53$^\prime$ O. & G. 47. 1814. 105. \\ \hline
        2. & 998 & - & - & Magdeburg. & Pr. Sachsen & 52$^\circ$ 8$^\prime$ N. & 11$^\circ$ 40$^\prime$ O. & G. 50. 1815. 231. \\ \hline
        3. & 1135 (1136) & - & - & Oldisleben, an der Unstrut; Thüringen. & Sachsen-Weimar & 51$^\circ$ 19$^\prime$ N. & 11$^\circ$ 10$^\prime$ O. & G. 29. 1808. 375. \\ \hline
        4. & 1164 & - & Mai & Im Meissen’schen. Eisen. & Sachsen & Zwischen 50$^\circ$ 30$^\prime$ N. und 51$^\circ$ 30$^\prime$ N. & Zwischen 11$^\circ$ 30$^\prime$ O. und 14$^\circ$ 30$^\prime$ O. & G. 50. 1815. 233. \\ \hline
        5. & 1249 & 26. & Juli & Zwischen Quedlinburg, Blankenburg und Ballenstadt. & Pr. Sachsen & Zwischen 51$^\circ$ 43$^\prime$ N. und 51$^\circ$ 48$^\prime$ N. & Zwischen 10$^\circ$ 58$^\prime$ O. und 11$^\circ$ 14$^\prime$ O. & G. 50. 1815. 234. \\ \hline
        6. & 1304 & 1. & Oktober & Friedland in Brandenburg (oder Vredeland in Vandalia); nach Anderen: Friedeburg an der Saale. & Preußen & 52$^\circ$ 6$^\prime$ N. & 14$^\circ$ 17$^\prime$ O. & G. 50. 1815. 234. \\ \hline
    \end{tabular}
\end{table}
\clearpage
\begin{table}[!ht]
    \centering
    \begin{tabular}{|p{5mm}|p{15mm}|p{5mm}|p{11mm}|p{48mm}|p{21mm}|p{18mm}|p{18mm}|p{11mm}|}
    \hline
        1. & 2. & 2. & 2. & 3. & 3. & 4. & 5. & 6. \\ \hline
        7. & 1379 & 26. & Mai & Münden. & Hannover & 52$^\circ$ 14$^\prime$ N. & 8$^\circ$ 53$^\prime$ O. & G. 54. 1816. 342. \\ \hline
        8. & Zwischen 1540 und 1550 & ~ & ~ & Naunhof (Neuholm), zwischen Leipzig und Grimma. - Eisen. & Sachsen & 51$^\circ$ 17$^\prime$ N. & 12$^\circ$ 36$^\prime$ O. & G. 50. 1815. 237. \\ \hline
        9. & 1552 & 19. & Mai & Schleusingen; Thüringen. & Pr. Sachsen & 50$^\circ$ 31$^\prime$ N. & 10$^\circ$ 45$^\prime$ O. & G. 50. 1815. 238. \\ \hline
        10. & 1561 & 17. & Mai & Torgau, Siptitz, WNW. v. Torgau u. Eilenburg (prope arcem Juliam). & Pr. Sachsen & 51$^\circ$ 33$^\prime$ N., 51$^\circ$ 34$^\prime$ N., 51$^\circ$ 28$^\prime$ N. & 13$^\circ$ 1$^\prime$ O., 12$^\circ$ 56$^\prime$ O., 12$^\circ$ 38$^\prime$ O. & G. 50. 1815. 238. \\ \hline
        11. & 1580 & 27. & Mai & Nörten, zwischen Nordheim und Göttingen. & Hannover & 51$^\circ$ 38$^\prime$ N. & 9$^\circ$ 55$^\prime$ O. & G. 53. 1816. 375. \\ \hline
        12. & 1581 & 26. & Juli & Niederreissen (Nieder-Reusen), S. von Buttstädt in Thüringen. & Sachsen-Weimar & 51$^\circ$ 6$^\prime$ N. & 11$^\circ$ 25$^\prime$ O. & G. 50. 1815. 239. \\ \hline
        13. & 1636 & 6. & März & Zwischen Sagan und Dubrow ($^\wedge$$^\wedge$$^\wedge$). & Pr. Sachsen & 51$^\circ$ 36$^\prime$ N. & 15$^\circ$ 20$^\prime$ O. & G. 50. 1815. 242. \\ \hline
        14. & 1647 & 18. & Februar & Pöhlau (Polau), O. von Zwickau. & Sachsen & 50$^\circ$ 43$^\prime$ N. & 12$^\circ$ 33$^\prime$ O. & G. 53. 1816. 376. \\ \hline
    \end{tabular}
\end{table}
\clearpage
\begin{table}[!ht]
    \centering
    \begin{tabular}{|p{5mm}|p{8mm}|p{5mm}|p{15mm}|p{48mm}|p{21mm}|p{11mm}|p{11mm}|p{18mm}|}
    \hline
        1. & 2. & 2. & 2. & 3. & 3. & 4. & 5. & 6. \\ \hline
        15. & 1647 & - & August & Zwischen Wermsen (Warmsen) und Schameelo ($^\wedge$$^\wedge$$^\wedge$), Vogtei Bomhorst (Bohnhorst), Amt Stolzenau in Westphalen. & Hannover & 52$^\circ$ 28$^\prime$ N. & 8$^\circ$ 49$^\prime$ O. & G. 29. 1808. 215. \\ \hline
        16. & 1671 & 27. & Februar & Oberkirch und Zusenhausen (Zusenhofen?), in der Ortenau; Kreis Schwaben. & Baden & 48$^\circ$ 32$^\prime$ N., 48$^\circ$ 33$^\prime$ N., ? & 8$^\circ$ 7$^\prime$ O., 8$^\circ$ 2$^\prime$ O., ? & G. 50. 1815. 245. \\ \hline
        17. & 1677 & 26. & Mai & Ermendorf, zwischen Dresden und Grossenhain. & Sachsen & 51$^\circ$ 14$^\prime$ N. & 13$^\circ$ 36$^\prime$ O. & G. 50. 1815. 245. \\ \hline
        18. & 1715 & 11. & April & Schellin (nicht Garz), 1 M. W. von Stargard in Pommern. Sp.-Gew.: 3,50? & Preußen & 53$^\circ$ 20$^\prime$ N. & 15$^\circ$ 0$^\prime$ O. & G. 71. 1822. 213. W. 1860. \\ \hline
        19. & 1722 & 5. & Juni & Schefftlar (Scheftlarn) im Freising’schen; N. von Wolfrathshausen an der Isar und SSW. von München; Kreis Oberbayern. & Bayern & 47$^\circ$ 56$^\prime$ N. & 11$^\circ$ 35$^\prime$ O. & G. 53. 1816. 377. \\ \hline
        20. & 1768 & 20. & November & Maurkirchen, SO. von Braunau in Ober-Bayern, jetzt im österreichischen Inn-Viertel. - Sp.-Gew.: 3,45-3,50. & Österreich & 48$^\circ$ 12$^\prime$ N. & 13$^\circ$ 7$^\prime$ O. & G. 18. 1804. 328. W. 1860. S. 1860. \\ \hline
    \end{tabular}
\end{table}
\clearpage
\begin{table}[!ht]
    \centering
    \begin{tabular}{|p{5mm}|p{9mm}|p{5mm}|p{16mm}|p{48mm}|p{22mm}|p{11mm}|p{11mm}|p{19mm}|}
    \hline
        1. & 2. & 2. & 2. & 3. & 3. & 4. & 5. & 6. \\ \hline
        21. & 1775 & 19. & September & Rodach, NW. von Coburg in Thüringen. & Sachsen-Coburg & 50$^\circ$ 21$^\prime$ N. & 10$^\circ$ 46$^\prime$ O. & G. 23. 1806. 93. \\ \hline
        22. & 1785 & 19. & Februar & Im Wittmess (nicht Wittens), Wald 1 ½ Stunde SW. v. Eichstaedt; Kr. Mittelfranken. - Sp.-Gew.: 3,60-3,70. & Bayern & 48$^\circ$ 52$^\prime$ N. & 11$^\circ$ 10$^\prime$ O. & G. 50. 1815. 250. v. Moll* 3 f. 251 bis 259. W. 1860. \\ \hline
        23. & 1803 & 13. & Dezember & St. Nicolas, NNW. Von Massing u. WNW. Von Eggenfelden; Kreis Niederbayern. - Sp.-Gew.: 3,21-3,365. & Bayern & 48$^\circ$ 27$^\prime$ N. & 12$^\circ$ 36$^\prime$ O. & G. 18. 1804. 329. W. 1860. \\ \hline
        24. & 1812 & 15. & April & Erxleben, zwischen Magdeburg und Helmstadt. - Sp.-Gew.: 3,60-3,64. & Pr. Sachsen & 52$^\circ$ 13$^\prime$ N. & 11$^\circ$ 14$^\prime$ O. & G. 40. 1812. 450. W. 1860. S. 1860. \\ \hline
        25. & 1819 & 13. & Oktober & Politz, NNW. Von Köstritz bei Gera. - Sp.-Gew.: 3,37-3,49. & Reuss & 50$^\circ$ 57$^\prime$ N. & 12$^\circ$ 2$^\prime$ O. & G. 63. 1819. 217. W. 1860. S. 1860. \\ \hline
    \end{tabular}
\end{table}
\clearpage
\begin{table}[!ht]
    \centering
    \begin{tabular}{|p{5mm}|p{9mm}|p{5mm}|p{16mm}|p{48mm}|p{22mm}|p{11mm}|p{11mm}|p{19mm}|}
    \hline
        1. & 2. & 2. & 2. & 3. & 3. & 4. & 5. & 6. \\ \hline
        26. & 1835 & 18. & Januar & Löbau in der Ober-Lausitz. & Sachsen & 51$^\circ$ 6$^\prime$ N. & 14$^\circ$ 40$^\prime$ O. & P. 4. 1854. 79. \\ \hline
        27. & 1841 & 22. & März & Seifersholz und Heinrichsau, beide W. von Grüneberg in Schlesien. - Sp.-Gew.: 3,69-3,73. & Preußen & 51$^\circ$ 56$^\prime$ N., 51$^\circ$ 54$^\prime$ N. & 15$^\circ$ 22 O., 15$^\circ$ 25 O. & P. 52. 1841. 495. W. 1860. S. 1860. \\ \hline
        28. & 1843 & 16. & September & Kleinwenden bei Münchenlohra (Mönchlora), 1 ¾ geogr. M. WSW. von Nordhausen und 1 geogr. M. SO. v. Bleicherode, Kreis Nordhausen in Thüringen. Sp.-Gew.: 3,70. & Preußen & 51$^\circ$ 24$^\prime$ N. & 10$^\circ$ 38$^\prime$ O. & P. 60. 1843. 157. W. 1860. S. 1860. \\ \hline
        29. & 1846 & 25. & Dezember & Schönenberg im Mindelthal, NW. von Pfaffenhausen, NNW. von Mindelheim und S. von Burgau; Kreis Schwaben. - Sp.-Gew.: 3,75-3,8. & Bayern & 48$^\circ$ 9$^\prime$ N. & 10$^\circ$ 26$^\prime$ O. & P. 70. 1847. 334. \\ \hline
        30. & 1851 & 17. & April & Gütersloh in Westphalen. - Sp.-Gew.: 3,54. & Preußen & 51$^\circ$ 55$^\prime$ N. & 8$^\circ$ 21$^\prime$ O. & P. 83. 1851. 465. W. 1860. S. 1860. \\ \hline
        31. & 1854 & 5. & September & Linum, SO. von Fehrbellin, Mark Brandenburg. & Preußen & 52$^\circ$ 46$^\prime$ N. & 12$^\circ$ 52$^\prime$ O. & P. 94. 1854. 169. \\ \hline
        32. & 1855 & 13. & Mai & Bremervörde, Landdrostei Stade. - Sp.-Gew.: 3,53. & Hannover & 53$^\circ$ 30$^\prime$ N. & 9$^\circ$ 8$^\prime$ O. & P. 96. 1855. 626. W. 1860. S. 1860. \\ \hline
          &   &   &   & Meteorsteine, deren Fallzeit unbekannt. &   &   &   &   \\ \hline
        33. & - & - & - & Darmstadt. 1 Stein von 16 ¾ Loth. Gefunden vor 1816. & Hessen & 49$^\circ$ 52$^\prime$ N. & 8$^\circ$ 40$^\prime$ O. & G. 53. 1816. 379. \\ \hline
        34. & - & - & - & Hainholz, N. von Borgholz und OSO. von Paderborn; Westphalen. - 1 Stein von 33 Pfund, den Übergang zu Meteoreisen bildend. Gef. 1856. Sp.-Gew.: 4,61. & Preußen & 51$^\circ$ 39$^\prime$ N. & 9$^\circ$ 14$^\prime$ O. & P. 100. 1857. 342. W. 1860. S. 1860. \\ \hline
        35. & - & - & - & Mainz. 1 Stein. Gefunden 1852. Sp.-Gew.: 3,44. & Hessen & 50$^\circ$ 0$^\prime$ N. & 8$^\circ$ 15$^\prime$ O. & B. 104. W. 1860. \\ \hline
          &   &   &   & Meteor-Eisenmassen, deren Fallzeit unbekannt. &   &   &   &   \\ \hline
        36. & - & - & - & Bitburg in der Eifel, NNW. von Trier. 33 Zentner Gefunden 1805. - Sp.-Gew.: 6,14-6,52. & Rhein-Preußen & 49$^\circ$ 59$^\prime$ N. & 6$^\circ$ 30$^\prime$ O. & G. 68. 1821. 342. W. 1860. S. 1860. \\ \hline
        37. & - & - & - & Nauheim. Gefunden 1826. & Kurhessen & 50$^\circ$ 22$^\prime$ N. & 8$^\circ$ 44$^\prime$ O. & B. 117. \\ \hline
        38. & - & - & - & Seeläsgen, WSW. v. Schwiebus in der Mark Brandenburg. 218 Pfund Gefunden 1847. - Sp.-Gew.: 7,59-7,73. & Preußen & 52$^\circ$ 14$^\prime$ N. & 15$^\circ$ 23$^\prime$ O. & P. 73. 1848. 329. W. 1860. S. 1860. \\ \hline
        39. & - & - & - & Schwetz an der Weichsel, N. von Culm. 43 Pfund Gefunden 1850. Sp.-Gew.: 7,77. & Preußen & 53$^\circ$ 24$^\prime$ N. & 18$^\circ$ 26$^\prime$ O. & P. 83. 1851. 594. W. 1860. S. 1860. \\ \hline
        40. & - & - & - & Steinbach, WNW. v. St. Johann-Georgenstadt. Gefunden 1751. - Sp.-Gew.: 6,56-7,50. & Sachsen & 50$^\circ$ 25$^\prime$ N. & 12$^\circ$ 40$^\prime$ O. & G. 50. 1815. 257. W. 1860. S. 1860. \\ \hline
        41. & - & - & - & Tabarz, am Fuß des Inselbergs in Thüringen. 3 Loth. Gefunden 1854. - Sp.-Gew.: 7,737. & Sachsen-Gotha & 50$^\circ$ 53$^\prime$ N. & 10$^\circ$ 31$^\prime$ O. & B. 121. \\ \hline
        42. & - & - & - & (Im Naturalien-Cabinet in Gotha.) & Wahrscheinlich aus Sachsen & - & - & Chladni, Feuer-Met. Fol. 326. \\ \hline
        Böhmen u. Mähren &   &   &   &   &   &   &   &   \\ \hline
        43. & 1618 & - & - & ? Eisen. & Böhmen & - & - & G. 50. 1815. 240. \\ \hline
        44. & 1723 & 22. & Juni & Pleskowitz ($^\wedge$$^\wedge$$^\wedge$) und Liboschitz ($^\wedge$$^\wedge$$^\wedge$), beide etliche Meilen von Reichstadt (50$^\circ$ 41$^\prime$ N., 14$^\circ$ 39$^\prime$ O.), Kreis Bunzlau. & Böhmen & - & - & G. 15. 1803. 309. Chladni, Feuer-Met. Fol. 240. \\ \hline
        45. & 1753 & 3. & Juli & Plan und Strkow, beide SO. von Tabor, ehemaliger Kreis Bechin. Sp.-Gew.: 3,65-4,28. & Böhmen & 49$^\circ$ 21$^\prime$ N., 49$^\circ$ 21$^\prime$ N. & 14$^\circ$ 43$^\prime$ O., 14$^\circ$ 44$^\prime$ O. & G. 50. 1815. 248. W. 1860. S. 1860. \\ \hline
        46. & 1808 & 22. & Mai & Stannern, S. von Iglau. - Sp.-Gew.: 2,95-3,19. & Mähren & 49$^\circ$ 18$^\prime$ N. & 15$^\circ$ 36$^\prime$ O. & G. 30. 1808. 358. W. 1860. S. 1860. \\ \hline
        47. & 1808 & 3. & September & Stratow und Wustra, beide OSO. von Lissa, Kreis Bunzlau. - Sp.-Gew.: 3,50-3,56. & Böhmen & 50$^\circ$ 12$^\prime$ N., 50$^\circ$ 10$^\prime$ N. & 14 54 O., 14 53 O. & G. 30. 1808. 358. W. 1860. S. 1860. \\ \hline
        48. & 1824 & 14. & Oktober & Praskoles, OSO. von Zebrak (Schebrak) und NO. von Horzowitz, Kreis Beraun. - Sp.-Gew.: 3,60. & Böhmen & 49$^\circ$ 52$^\prime$ N. & 13$^\circ$ 55$^\prime$ O. & P. 6. 1826. 28. W. 1860. S. 1860. \\ \hline
        49. & 1831 & 9. & September & Znorow, SW. von Wessely, Kreis Hradisch. - Sp.-Gew.: 3,66-3,70. & Mähren & 48$^\circ$ 54$^\prime$ N. & 17$^\circ$ 21$^\prime$ O. & P. 34. 1835. 342. W. 1860. S. 1860. \\ \hline
        50. & 1833 & 25. & November & Blansko, N. von Brunn und SSW. von Boskowitz. - Sp.-Gew.: 3,70. & Mähren & 49$^\circ$ 20$^\prime$ N. & 16$^\circ$ 38$^\prime$ O. & P. 34. 1835. 343. W. 1860. S. 1860. \\ \hline
        51. & 1847 & 14. & Juli & Hauptmannsdorf, NW. von Braunau, Kreis Koniggratz. - Eisen. - Sp.-Gew.: 7,714. & Böhmen & 50$^\circ$ 36$^\prime$ N. & 16$^\circ$ 19$^\prime$ O. & P. 72. 1847. 170. W. 1860. S. 1860. \\ \hline
          &   &   &   & Meteor-Eisenmassen, deren Fallzeit unbekannt. &   &   &   & ~ \\ \hline
        52. & - & - & - & Bohumilitz bei Alt-Skalitz, SW. von Wollin und NNO. von Winterberg, Kr. Prachin. 103 Pfund Gefunden 1829. - Sp.-Gew.: 7,146-7,71. & Böhmen & 49$^\circ$ 6$^\prime$ N. & 13$^\circ$ 49$^\prime$ O. & P. 34. 1835. 344. W. 1860. S. 1860. \\ \hline
        53. & - & - & - & Ellbogen, Kreis Ellbogen. 191 Pfund Gefunden 1811. - Sp.-Gew.: 7,2-7,83. & Böhmen & 50$^\circ$ 12$^\prime$ N. & 12$^\circ$ 44$^\prime$ O. & G. 42. 1812. 197. W. 1860. S. 1860. \\ \hline
        54. & - & - & - & ? (1 Stück gediegenes Eisen, fruher in der Born’schen, jetzt in der Greville’schen Sammlung). & Böhmen & - & - & Chladni, Feuer-Met. Fol. 324. \\ \hline
        Illyrien &   &   &   &   &   &   &   &   \\ \hline
        55. & 1112 & - & - & Aquileja (Aglar). & Illyrien & 45$^\circ$ 46$^\prime$ N. & 13$^\circ$ 24$^\prime$ O. & G. 50. 1815. 232. \\ \hline
    \end{tabular}
\end{table}
\clearpage
\subsubsection{8. Schweiz}
\begin{table}[!ht]
    \centering
    \begin{tabular}{|l|l|l|l|l|l|l|l|l|}
    \hline
        1. & 2. & 2. & 2. & 3. & 3. & 4. & 5. & 6. \\ \hline
        1. & 1698 & 18. (nicht 19.) & Mai & Hinterschwendi ($^\wedge$$^\wedge$$^\wedge$) bei Waltringen (47$^\circ$ 5$^\prime$ N., 7$^\circ$ 45$^\prime$ O.), NO. von Bern und ONO. von Burgdorf. & Canton Bern & - & - & G. 50. 1815. 246. \\ \hline
    \end{tabular}
\end{table}
\clearpage
\subsubsection{9. Italien und Korsika}
\begin{table}[!ht]
    \centering
    \begin{tabular}{|l|l|l|l|l|l|l|l|l|}
    \hline
        1. & 2. & 2. & 2. & 3. & 3. & 4. & 5. & 6. \\ \hline
          & vor Christus & - & - &   &   &   &   &   \\ \hline
        1. & 654 (644 oder 642) & - & - & Albaner Gebirge (Mons Albanus), SO. von Rom. & Kirchenstaat & 41$^\circ$ 40$^\prime$ N. & 12$^\circ$ 40$^\prime$ O. & G. 50. 1815. 228. P. 4. 1854. 7. \\ \hline
        2. & 206 (205) & - & - & ~ & Italien ? & - & - & A. 4. 185. \\ \hline
        3. & 176 (174) & - & - & Mars-See ($^\wedge$$^\wedge$$^\wedge$, Lacus Martis) im Gebiet von Crustumerium in Sabinien, unweit Veji (42$^\circ$ 0$^\prime$ N., 12$^\circ$ 26$^\prime$ O.) in Etrurien. & Kirchenstaat & - & - & P. 4. 1854. 8. \\ \hline
        4. & 90 (89) & - & - & ? & Italien & - & - & G. 54. 1816. 339. \\ \hline
        5. & 56 (54 oder 52) & - & - & Provinz Lucanien - Eisen. & Neapel & Zwischen 39$^\circ$ 35$^\prime$ N. und 40$^\circ$ 50$^\prime$ N. & Zwischen 15$^\circ$ 0$^\prime$ O. und 17$^\circ$ 0$^\prime$ O. & G. 50. 1815. 229. \\ \hline
          & nach Christus &   &   &   &   &   &   &   \\ \hline
        6. & 650 & - & - & ? & Italien ? & - & - & P. 4. 1854. 8. \\ \hline
        7. & 921 & - & - & Narni, SW. von Spoleto. & Kirchenstaat & 42$^\circ$ 32$^\prime$ N. & 12$^\circ$ 30$^\prime$ O. & P. 2. 1824. 151. \\ \hline
        8. & 956 & - & - & ? & Italien & - & - & P. 4. 1854. 8. \\ \hline
        9. & 963 & - & - & ? & Italien & - & - & P. 4. 1854. 8. \\ \hline
        10. & Zwischen 964 und 972 & - & - & ? & Italien & - & - & G. 50. 1815. 231. P. 4. 1854. 8. \\ \hline
        11. & 1474 & - & - & Viterbo. & Kirchenstaat & 42$^\circ$ 27$^\prime$ N. & 12$^\circ$ 6$^\prime$ O. & G. 68. 1821. 332. \\ \hline
        12. & 1491 & 22. & März & Rivolta de’ Bassi, NW. von Crema und O. von Mailand. & Lombardei & 45$^\circ$ 28$^\prime$ N. & 9$^\circ$ 30$^\prime$ O. & G. 50. 1815. 235. \\ \hline
        13. & 1496 & 26. (28.) & Januar & Zwischen Cesena und Bertinoro, W. von Cesena und SO. von Forli, und bei Valdinoce, SO. von Cesena und S. von Bertinoro. & Kirchenstaat & Zwischen 44$^\circ$ 8$^\prime$ N. und 44$^\circ$ 7$^\prime$ N., 44$^\circ$ 4$^\prime$ N. & Zwischen 12$^\circ$ 14$^\prime$ O. und 12$^\circ$ 7$^\prime$ O., 12$^\circ$ 6$^\prime$ O. & G. 50. 1815. 236. \\ \hline
        14. & 1511 & 4. & September & Crema, unweit der Adda. & Lombardei & 45$^\circ$ 21$^\prime$ N. & 9$^\circ$ 42$^\prime$ O. & G. 50. 1815. 237. \\ \hline
        15. & Zwischen 1550 und 1570 & - & - & ? Eisen. & Piemont & - & - & G. 50. 1815. 239. \\ \hline
        16. & 1583 & 9. & Januar & Castrovillari in Calabrien. & Neapel & 39$^\circ$ 45$^\prime$ N. & 16$^\circ$ 15$^\prime$ O. & G. 50. 1815. 240. \\ \hline
        17. & 1583 & 2. & März & ? & Piemont & - & - & G. 50. 1815. 240. \\ \hline
        18. & 1596 & 1. & März & Crevalcore, W. von Cento u. WSW. von Ferrara. & Kirchenstaat & 44$^\circ$ 43$^\prime$ N. & 11$^\circ$ 8$^\prime$ O. & G. 50. 1815. 240. \\ \hline
        19. & 1635 & 7. & Juli & Calce im Vicentinischen (vielleicht Colze, 45$^\circ$ 28 N., 11$^\circ$ 38 O., und SO. von Vicenza?). & Venezien & - & - & G. 18. 1804. 307. \\ \hline
        20. & 1637 (1617) ? & 27. (29.) & November & Mont Vaisien (mons Vasonum), zwischen Pesne (Pedona) und Guilleaume (Guilielmo), unweit Nizza, im Flussgehiet des Var in der ehemaligen Provence. - Sp.-Gew.: 3,6. & Piemont; gegenwartig in Frankreich & Zwischen 44$^\circ$ 7$^\prime$ N. und 44$^\circ$ 5$^\prime$ N. & Zwischen 6$^\circ$ 54$^\prime$ O. und 6$^\circ$ 51$^\prime$ O. & G. 50. 1815. 242. \\ \hline
        21. & 1660 & - & - & Mailand. & Lombardei & 45$^\circ$ 28$^\prime$ N. & 9$^\circ$ 11$^\prime$ O. & G. 50. 1815. 246. \\ \hline
        22. & 1668 (nicht 1662, 1663 oder 1672) & 19. (21.) & Januar & Vago, O. von Verona und SSW. von Trignano. & Venezien & 45$^\circ$ 25$^\prime$ N. & 11$^\circ$ 8$^\prime$ O. & G. 50. 1815. 244. \\ \hline
        23. & 1697 & 13. & Januar & Pentolina, SW. von Siena, Menzano, W. von Siena, und Capraja ($^\wedge$$^\wedge$$^\wedge$). & Toskana & 43$^\circ$ 12$^\prime$ N., 43$^\circ$ 19$^\prime$ N. & 11$^\circ$ 10$^\prime$ O., 11$^\circ$ 3$^\prime$ O. & G. 50. 1815. 246. \\ \hline
        24. & 1755 & - & Juli & Am Fluss Crati, unweit Terranova in Calabrien. & Neapel & 39$^\circ$ 38$^\prime$ N. (nach Fata: 39$^\circ$ 50$^\prime$ N.) & 16$^\circ$ 30$^\prime$ O. & G. 50. 1815. 248. \\ \hline
        25. & 1766 & - & Mitte Juli & Alboretto, NO. v. Modena. & Modena & 44$^\circ$ 41$^\prime$ N. & 10$^\circ$ 57$^\prime$ O. & G. 50. 1815. 249. \\ \hline
        26. & 1776 (1777) & - & Januar & Sanatoglia (San Anatoglia), S. von Fabriano. & Kirchenstaat & 43$^\circ$ 15$^\prime$ N. & 12$^\circ$ 54$^\prime$ O. & G. 50. 1815. 250. \\ \hline
        27. & 1782 & - & Juli & Turin. & Piemont & 45$^\circ$ 4$^\prime$ N. & 7$^\circ$ 41$^\prime$ O. & G. 57. 1817. 134. \\ \hline
        28. & 1791 & 17. & Mai & Castel-Berardenga, ONO. von Siena. & Toskana & 43$^\circ$ 21$^\prime$ N. & 11$^\circ$ 29$^\prime$ O. & G. 50. 1815. 251. \\ \hline
        29. & 1794 & 16. & Juni & Siena. - Sp.-Gew.: 3,34-3,418. & Toskana & 43$^\circ$ 20$^\prime$ N. & 11$^\circ$ 20$^\prime$ O. & G. 6. 1800. 156. W. 1860. \\ \hline
        30. & 1805 & - & November & Asco, OSO. von Calvi. - Sp.-Gew.: 3,66. & Korsika & 42$^\circ$ 28$^\prime$ N. & 9$^\circ$ 2$^\prime$ O. & P. 4. 1854. 11. W. 1860. \\ \hline
        31. & 1808 & 19. & April & Borgo San Donino, zwischen Parma und Piacenza; und Pieve di Casignano, S. von Borgo San Donino. - Sp.-Gew.: 3,39-3,40. & Parma & 44$^\circ$ 47$^\prime$ N., 44$^\circ$ 52$^\prime$ N. & 10$^\circ$ 4$^\prime$ O., 10$^\circ$ 4$^\prime$ O. & G. 50. 1815. 254. W. 1860. S. 1860. \\ \hline
        32. & 1813 & 14. & März & Cutro, zwischen Crotone und Catanzaro in Calabrien. & Neapel & 38$^\circ$ 58$^\prime$ N. & 17$^\circ$ 2$^\prime$ O. & G. 53. 1816. 381. \\ \hline
        33. & 1819 & - & Ende April & Massa Lubrense (Massa oder Massa di Sorento); Furstenthum Salerno. & Neapel & 40$^\circ$ 38$^\prime$ N. & 14$^\circ$ 18$^\prime$ O. & G. 71. 1822. 359. \\ \hline
        34. & 1820 & 29. & November & Cosenza in Calabrien. & Neapel & 39$^\circ$ 15$^\prime$ N. & 16$^\circ$ 18$^\prime$ O. & P. 4. 1854. 520. \\ \hline
        35. & 1824 & 13. (15.) & Januar & Renazzo (Atenazzo), 4 ital. M. N. von Cento, Prov. Ferrara. - Sp.-Gew.: 3,24-3,28. & Kirchenstaat & 44$^\circ$ 47$^\prime$ N. & 11$^\circ$ 18$^\prime$ O. & P. 18. 1830. 181. W. 1860. S. 1860. \\ \hline
        36. & 1834 & 15. & Dezember & Marsala. & Sicilien & 37$^\circ$ 51$^\prime$ N. & 12$^\circ$ 24$^\prime$ O. & P. 4. 1854. 34. \\ \hline
        37. & 1840 & 17. & Juli & Cereseto, SW. von Casale-Montferrat u. NNW. von Ottiglio (nicht Offiglia), ebenfalls SW. von Casale. - Sp.-Gew.: 3,49? & Piemont & 45$^\circ$ 4$^\prime$ N. & 8$^\circ$ 20$^\prime$ O. & P. 50. 1840. 668. W. 1860. S. 1860. \\ \hline
        38. & 1841 & 17. & Juli & Mailand. & Lombardei & 45$^\circ$ 28$^\prime$ N. & 9$^\circ$ 11$^\prime$ O. & P. 4. 1854. 364. \\ \hline
        39. & 1846 & 8. & Mai & Monte-Milone an der Potenza, SW. von Macerata und NO. von Tolentino; Mark Ancona. - Sp.-Gew.: 3,55? & Kirchenstaat & 43$^\circ$ 16$^\prime$ N. & 13$^\circ$ 21$^\prime$ O. & P. 4. 1854. 375. W. 1860. S. 1860. \\ \hline
        40. & 1853 & 10. & Februar & Girgenti. - Sp.-Gew.: 3,76. & Sicilien & 37$^\circ$ 17$^\prime$ N. & 13$^\circ$ 34$^\prime$ O. & W. 1860. S. 1860. \\ \hline
        41. & 1856 & 17. & September & Bei Civita Vecchia. Ins Meer. & Kirchenstaat & Ungefähr 42$^\circ$ 7$^\prime$ N. & Ungefähr 11$^\circ$ 46$^\prime$ O. & P. 99. 1856. 645. \\ \hline
        42. & 1856 & 12. & November & Trenzano, WSW. von Brescia und SO. von Chiari. & Lombardei & 45$^\circ$ 28$^\prime$ N. & 10$^\circ$ 2$^\prime$ O. & WA. 41. 1860. 569. \\ \hline
    \end{tabular}
\end{table}
\clearpage
\subsubsection{10. Ungarn, Kroatien und Siebenbürgen}
\begin{table}[!ht]
    \centering
    \begin{tabular}{|l|l|l|l|l|l|l|l|l|}
    \hline
        1. & 2. & 2. & 2. & 3. & 3. & 4. & 5. & 6. \\ \hline
        1. & 1559 & - & - & Miskolcz, Gespannschaft Borschod. & Ungarn & 48$^\circ$ 6$^\prime$ N. & 20$^\circ$ 47$^\prime$ O. & G. 47. 1814. 97. \\ \hline
        2. & 1618 & - & Ende August & Bezirk Muraköz (Mur-Insel), an der Grenze von Steyermark, zwischen der Mur und der Drau; Gespannschaft Salad. & Ungarn & Zwischen 46$^\circ$ 20$^\prime$ N. und 46$^\circ$ 32$^\prime$ N. & Zwischen 16$^\circ$ 15$^\prime$ O. und 16$^\circ$ 52$^\prime$ O. & G. 50. 1815. 240. P. 4. 1854. 33 u. 40. \\ \hline
        3. & 1642 & 12. ? & Dezember ? & Zwischen Ofen und Gran. Wahrscheinlich Eisen. & Ungarn & Zwischen 47$^\circ$ 30$^\prime$ N. und 47$^\circ$ 48$^\prime$ N. & Zwischen 19$^\circ$ 3$^\prime$ O. und 18$^\circ$ 44$^\prime$ O. & G. 56. 1817. 379. \\ \hline
        4. & 1751 & 26. & Mai & Hraschina (nicht Hradschina), SW. von Warasdin und 5 M. NO. von Agram, Gespannschaft Agram. - Eisen. - Sp.-Gew.: 7,72-7,82. & Kroatien & 46$^\circ$ 6$^\prime$ N. & 16$^\circ$ 20$^\prime$ O. & WA. 35. 1859. 361. \\ \hline
        5. & 1820 & 22. & Mai & Oedenburg, Gespannschaft Oedenburg. & Ungarn & 47$^\circ$ 41$^\prime$ N. & 16$^\circ$ 36$^\prime$ O. & G. 68. 1821. 337. \\ \hline
        6. & 1834 & - & - & Szala, Gespannschaft Salad. & Ungarn & 46$^\circ$ 50$^\prime$ N. & 16$^\circ$ 52$^\prime$ O. & P. 4. 1854. 33. \\ \hline
        7. & 1836 & - & - & Am Platten-See. & Ungarn & Zwischen 46$^\circ$ 30$^\prime$ N. und 47$^\circ$ 10$^\prime$ N. & Zwischen 17$^\circ$ 0$^\prime$ O. und 18$^\circ$ 20$^\prime$ O. & P. 4. 1854. 355. \\ \hline
        8. & 1837 & 15. & Januar & Mikolowa ($^\wedge$$^\wedge$$^\wedge$), Gespannschaft Salad (vielleicht Mihalyfa zwischen Lövő und Szala? Oder Mihalyfa zwischen Turgye und Sümeg?) & Ungarn & Zwischen 46$^\circ$ 20$^\prime$ N. und 47$^\circ$ 8$^\prime$ N. & Zwischen 16$^\circ$ 10$^\prime$ O. und 18$^\circ$ 0$^\prime$ O. & P. 4. 1854. 356. \\ \hline
        9. & 1837 & 24. & Juli & Gross-Divina ($^\wedge$$^\wedge$$^\wedge$) nächst Budetin (49$^\circ$ 15$^\prime$ N., 18$^\circ$ 44$^\prime$ O.) bei Sillein, Gespannschaft Trentschin. - Sp.-Gew.: 3,55-3,56. & Ungarn & - & - & P. 4. 1854. 356. W. 1860. \\ \hline
        10. & 1842 & 26. & April & Pusinsko-Selo, 1 M. S. von Milena (Melyan, W. von Warasdin), Gespannsch. Warasdin. - Sp.-Gew.: 3,54. & Kroatien & 46$^\circ$ 11$^\prime$ N. & 16$^\circ$ 4$^\prime$ O. & P. 56. 1842. 349. W. 1860. S. 1860. \\ \hline
        11. & 1852 & 4. & September & Fekete und Istento, 1 M. W. von Mezo-Madaras, im bergischen Haidlande Mezőség. - Sp.-Gew.: 3,50. & Siebenbürgen & 46$^\circ$ 37$^\prime$ N. & 24$^\circ$ 19$^\prime$ O. & WA. 11. 1853. 674. P. 91. 1854. 627. W. 1860. S. 1860. \\ \hline
        12. & 1852 & 13. & Oktober & Borkut, 5 D. M. NO. von Szigeth, an der Schwarzen Theiss, Gespannschaft Marmaros. - Sp.-Gew.: 3,24. & Ungarn & 48$^\circ$ 7$^\prime$ N. & 24$^\circ$ 17$^\prime$ O. & B. 101. W. 1860. \\ \hline
        13. & 1857 & 15. & April & Kaba, SW. von Debreczin, Gespannschaft Nord-Bihar. - Sp.-Gew.: 3,39? & Ungarn & 47$^\circ$ 22$^\prime$ N. & 21$^\circ$ 16$^\prime$ O. & P. 105. 1858. 329. W. 1860. \\ \hline
        14. & 1857 & 10. & Oktober & Ohaba, O. von Carlsburg, Bezirk Blasendorf. - Sp.-Gew.: 3,11. & Siebenbürgen & 46$^\circ$ 4$^\prime$ N. & 23$^\circ$ 50$^\prime$ O. & P. 105. 1858. 334. W. 1860. S. 1860. \\ \hline
        15. & 1858 & 19. & Mai & Kakova, NW. v. Oravitza, Gespannschaft Kraschow (Krasso), Temeser Banat. - Sp.-Gew.: 3,384. & Ungarn & 45$^\circ$ 6$^\prime$ N. & 21$^\circ$ 38$^\prime$ O. & WA. 34. 1859. 11. W. 1860. S. 1860. \\ \hline
          &   &   &   & Meteor-Eisenmassen, deren Fallzeit unbekannt. &   &   & ~ &   \\ \hline
        16. & - & - & - & Lenarto, W. von Bartfeld, Gespannschaft Sarosch. 194 Pfund Gefunden 1815. - Sp.-Gew.: 7,72-7,83. & Ungarn & 49$^\circ$ 18$^\prime$ N. & 21$^\circ$ 4$^\prime$ O. & G. 50. 1815. 272. W. 1860. S. 1860. \\ \hline
        17. & - & - & - & Gebirg Magura, SW. von Szlanicza. (49$^\circ$ 26$^\prime$ N., 19$^\circ$ 33$^\prime$ O.), Gespannschaft Arva. Gefunden 1844. - Sp.-Gew.: 7,01-7,22 oder 7,76-7,814. & Ungarn & Ungefähr 49$^\circ$ 20$^\prime$ N. & Ungefähr 19$^\circ$ 29$^\prime$ O. & P. 61. 1844. 675. W. 1860. S. 1860. \\ \hline
    \end{tabular}
\end{table}
\clearpage
\subsubsection{11. Polen und Russland}
\begin{table}[!ht]
    \centering
    \begin{tabular}{|l|l|l|l|l|l|l|l|l|}
    \hline
        1. & 2. & 2. & 2. & 3. & 3. & 4. & 5. & 6. \\ \hline
        1. & Zwischen 1251 und 1360 & - & - & Welikoi-Ustiug (Ustjug-Weliki, Gross-Ustiug). & Gouv. Wologda & 60$^\circ$ 45$^\prime$ N. & 46$^\circ$ 16$^\prime$ O. & G. 50. 1815. 234. \\ \hline
        2. & 16.. & - & - & Warschau. & Polen & 52$^\circ$ 13$^\prime$ N. & 21$^\circ$ 5$^\prime$ O. & G. 50. 1815. 244. \\ \hline
        3. & 1775 (1776) & - & - & Obruteza (Owrutsch, Owrucz?). & Gouv. Volhynien & 51$^\circ$ 23$^\prime$ N. & 28$^\circ$ 40$^\prime$ O. & G. 31. 1809. 306. \\ \hline
        4. & 1787 & 13. & Oktober & Schigailow ($^\wedge$$^\wedge$$^\wedge$), Kreis Achtyrka (50$^\circ$ 17$^\prime$ N., 35$^\circ$ 10$^\prime$ O.), 10 Werst von Bobrik im Kreis Sumi; und Lebedin, Kreis Achtyrka. - Sp.-Gew.: 3,49. & Gouv. Charkow (Slobodsko-Ukrain) & ?, 50$^\circ$ 33$^\prime$ N. & ?, 34$^\circ$ 50$^\prime$ O. & G. 31. 1809. 311. W. 1860. \\ \hline
        5. & 1796 & 4. & Januar & Belaja-Zerkwa (Biala-Cerkow, Weisskirchen). & Gouv. Kiew & 49$^\circ$ 50$^\prime$ N. & 30$^\circ$ 6$^\prime$ O. & G. 31. 1809. 307. \\ \hline
        6. & 1807 & 13. & März & Timochin ($^\wedge$$^\wedge$$^\wedge$), Kreis Juchnow (54$^\circ$ 48$^\prime$ N., 35$^\circ$ 10$^\prime$ O.) Sp.-Gew.: 3,60-3,70. & Gouv. Smolensk & - & - & G. 26. 1807. 238. W. 1860. \\ \hline
        7. & 1809 & - & - & Kikina ($^\wedge$$^\wedge$$^\wedge$), Wiasemker Kreis (Wjasma: 55$^\circ$ 17$^\prime$ N., 34$^\circ$ 13$^\prime$ O.). Sp.-Gew.: 3,58? & Gouv. Smolensk & - & - & W. 1859. W. 1860. \\ \hline
        8. & 1811 & 12. (13.) & März & Kuleschowka ($^\wedge$$^\wedge$$^\wedge$), Kreis Romen (50$^\circ$ 43$^\prime$ N., 33$^\circ$ 45$^\prime$ O.). Sp.-Gew.: 3,47-3,49. & Gouv. Pultawa & - & - & G. 38. 1811. 120. W. 1860. S. 1860. \\ \hline
        9. & 1813 (1814) ? & 13. & Dezember (Mitte März) ? & Lontalax ($^\wedge$$^\wedge$$^\wedge$) bei Switaipola (Sowaitopola oder Savitaipal, 61$^\circ$ 13$^\prime$ N., 27$^\circ$ 49$^\prime$ O.), NW. von Willmanstrand und NNO. von Friedrichsham in Finnland. - Sp.-Gew.: 3,07. & Gouv. Wiborg & - & - & G. 68. 1821. 340. W. 1860. \\ \hline
        10. & 1814 & 15. & Februar & Distrikt Bachmut (48$^\circ$ 34$^\prime$ N., 37$^\circ$ 52$^\prime$ O.). - Sp.-Gew.: 3,42. & Gouv. Jekaterinoslaw & - & - & G. 50. 1815. 256. W. 1860. S. 1860. \\ \hline
        11. & 1818 & 10. (11.) & April & Zjaborzyka (Saborytz oder Zabortch) am Slucz (Slutsch), S. von Nowgrad-Volhynsk (Nowgrad-Vollhynskoi oder Nowgrad-Wolinsk), W. von Shitomir (Zytomir) und NNO. von Staro-Konstantino. - Sp.-Gew.: 3,40. & Gouv. Volhynien & 50$^\circ$ 15$^\prime$ N. & 27$^\circ$ 30$^\prime$ O. (27$^\circ$ 44$^\prime$) & G. 75. 1823. 230. W. 1860. S. 1860. \\ \hline
        12. & 1818 & 10. & August & Slobodka ($^\wedge$$^\wedge$$^\wedge$), Kreis Juchnow (54$^\circ$ 48$^\prime$ N., 35$^\circ$ 10$^\prime$ O.). - Sp.-Gew.: 3,47. & Gouv. Smolensk & - & - & G. 75. 1823. 266. W. 1860. S. 1860. \\ \hline
        13. & 1820 & 12. & Juli & Lasdany ($^\wedge$$^\wedge$$^\wedge$) bei Lixna (oder Liksen: 56$^\circ$ 0$^\prime$ N., 26$^\circ$ 25$^\prime$ O.), N. von Dunaburg. - Sp.-Gew.: 3,66-3,76. & Gouv. Witepsk & - & - & G. 68. 1821. 337. W. 1860. S. 1860. \\ \hline
        14. & 1826 & 19. & Mai & Distrikt Paulowgrad (48$^\circ$ 32$^\prime$ N., 35$^\circ$ 52$^\prime$ O.). - Sp.-Gew.: 3,77. & Gouv. Jekaterinoslaw & - & - & P. 18. 1830. 185. W. 1860. S. 1860. \\ \hline
        15. & 1827 & 5. (8.) & Oktober & Kuasti-Knasti ($^\wedge$$^\wedge$$^\wedge$), 2 Stunden von Bialystock (Belostok, 53$^\circ$ 12$^\prime$ N., 23$^\circ$ 10$^\prime$ O.). - Sp.-Gew.: 3,17. & Gouv. Bialystock & - & - & P. 18. 1830. 185. W. 1860. S. 1860. \\ \hline
        16. & 1829 & 9. & September & Krasnoi-Ugol (Krasnyi-Ugol) ($^\wedge$$^\wedge$$^\wedge$), Kreis Saposhok (Sapozok, Sapojok oder Sapojek, 53$^\circ$ 56$^\prime$ N., 40$^\circ$ 28$^\prime$ O.). - Sp.-Gew.: 3,49. & Gouv. Rjasan & - & - & P. 54. 1841. 291. W. 1860. \\ \hline
        17. & 1833 & 27. & Dezember & Okniny (Okaninah) ($^\wedge$$^\wedge$$^\wedge$) bei Kremenetz (50$^\circ$ 6$^\prime$ N., 25$^\circ$ 40$^\prime$ O.). - Sp.-Gew.: 3,63? & Gouv. Volhynien & - & - & W. 1859. W. 1860. P. 107. 1859. 161. \\ \hline
        18. & 1843 & 30. & Oktober & Werschne-Tschirskaja-Stanitza (Werschn Czirskaia) am Don. Sp.-Gew.: 3,58. & Gouv. der Donischen Kosaken & 48$^\circ$ 25$^\prime$ N. & 43$^\circ$ 10$^\prime$ O. & P. 72. 1848. Sup. 366. \\ \hline
        19. & 1855 & 11. & Mai & Insel Oesel. - Sp.-Gew.: 3,668. & Ostsee & Zwischen 58$^\circ$ 0$^\prime$ N. und 58$^\circ$ 40$^\prime$ N. & Zwischen 21$^\circ$ 50$^\prime$ O. und 23$^\circ$ 20$^\prime$ O. & P. 99. 1856. 642. W. 1860. \\ \hline
          &   &   &   & Meteorsteine, deren Fallzeit unbekannt. &   &   &   &   \\ \hline
        20. & - & - & - & Czartoria (Czartorysk). Sp.-Gew.: 3,49? & Gouv. Volhynien & 51$^\circ$ 14$^\prime$ N. & 25$^\circ$ 49$^\prime$ O. & P. 107. 1859. 161. \\ \hline
        21. & - & - & - & ? Gefunden 1845. - Sp.-Gew.: 3,55. & Gouv. Kursk & Zwischen 50$^\circ$ 20$^\prime$ N. und 52$^\circ$ 25$^\prime$ N. & Zwischen 33$^\circ$ 40$^\prime$ O. und 38$^\circ$ 30$^\prime$ O. & W. 1860. P. 107. 1859. 161. \\ \hline
        22. & - & - & - & ? Gefunded 1845. - Sp.-Gew.: 3,33. & Gouv. Pultawa & Zwischen 48$^\circ$ 40$^\prime$ N. und 51$^\circ$ 10$^\prime$ N. & Zwischen 30$^\circ$ 40$^\prime$ O. und 36$^\circ$ 0$^\prime$ O. & W. 1860. P. 107. 1859. 161. \\ \hline
          &   &   &   & Meteor-Eisenmasse, deren Fallzeit unbekannt. &   &   &   &   \\ \hline
        23. & - & - & - & Rokicky ($^\wedge$$^\wedge$$^\wedge$) bei Brahin (51$^\circ$ 46$^\prime$ N., 30$^\circ$ 10$^\prime$ O.), Kreis Retschitz (Rseczytza), Distrikt Mozyrz, am Zusammenfluss des Daiepr und Prypetz. 2 Stuck von zusammen 200 Pfund Gefunden 1822. - Sp.-Gew.: 6,2-7,58. & Gouv. Minsk & - & - & G. 68. 1821. 342. W. 1860. \\ \hline
        24. & - & - & - & Tula; an der Strasse nach Moskau. Gefunden 1857. & Gouv. Tula & 54$^\circ$ 35$^\prime$ N. & 37$^\circ$ 34$^\prime$ O. \\ \hline
    \end{tabular}
\end{table}
\clearpage
\subsubsection{12. Dalmatien, Europäische Türkei und Griechenland}
\begin{table}[!ht]
    \centering
    \begin{tabular}{|l|l|l|l|l|l|l|l|l|}
    \hline
        1. & 2. & 2. & 2. & 3. & 3. & 4. & 5. & 6. \\ \hline
          & vor Christus &   &   &   &   &   &   &   \\ \hline
        1. & Um 1478 & - & - & Cybelische Berge. & Insel Creta & 35$^\circ$ 15$^\prime$ N. & 24$^\circ$ 50$^\prime$ O. & G. 54. 1816. 336. \\ \hline
        2. & 1200 & - & - & Stein, der zu Orchomenos in Böotien war aufbewahrt worden. & Griechenland & 38$^\circ$ 33$^\prime$ N. & 22$^\circ$ 58$^\prime$ O. & G. 54. 1816. 338. \\ \hline
        3. & 476 (468, 465, 464, 462, 405 oder 403) & - & - & Am Ziegen-Fluss (Aegos Potamos) im Thrakischen Chersonnes, in der Gegend des heutigen Gallipoli. & Thrakien & 40$^\circ$ 24$^\prime$ N. & 26$^\circ$ 36$^\prime$ O. & G. 50. 1815. 228. \\ \hline
        4. & 465 & - & - & Theben in Böotien. & Griechenland & 38$^\circ$ 17$^\prime$ N. & 23$^\circ$ 17$^\prime$ O. & G. 54. 1816. 339. \\ \hline
          & nach Christus &   &   &   &   &   &   &   \\ \hline
        5. & 452 & - & - & ? & Thrakien & - & - & G. 50. 1815. 230. \\ \hline
        6. & 1706 & 7. & Juni & Larissa in Thessalien. & Türkei & 39$^\circ$ 38$^\prime$ N. & 22$^\circ$ 35$^\prime$ O. & G. 50. 1815. 247. \\ \hline
        7. & 1740 (nicht 1770) & 25. & Oktober & Hazargrad (Rasgrad), zwischen Schumla (Dsjumla) und Rustschuck in Bulgarien. & Türkei & 43$^\circ$ 23$^\prime$ N. & 26$^\circ$ 12$^\prime$ O. & G. 50. 1815. 247. \\ \hline
        8. & 1805 & - & Juni & Konstantinopel. - Sp.-Gew.: 3,17. & Türkei & 41$^\circ$ 0$^\prime$ N. & 28$^\circ$ 58$^\prime$ O. & G. 50. 1815. 253. W. 1860. \\ \hline
        9. & 1810 & 28. & November & Zwischen der Insel Cerigo und Cap Matapan. & Griechenland & Zwischen 36$^\circ$ 0$^\prime$ N. und 36$^\circ$ 20$^\prime$ N. & Zwischen 22$^\circ$ 30$^\prime$ O. und 22$^\circ$ 50$^\prime$ O. & P. 24. 1832. 223. \\ \hline
        10. & 1818 & - & Juni & Seres in Makedonien. Sp.-Gew.: 3,60-3,71. & Türkei & 41$^\circ$ 3$^\prime$ N. & 23$^\circ$ 33$^\prime$ O. & P. 34. 1835. 340. W. 1860. S. 1860. \\ \hline
        11. & 1828 & - & Mai & Tscheroi ($^\wedge$$^\wedge$$^\wedge$), zwischen Widdin und Krajowa; Wallachei. Anhydrit. & Türkei & Zwischen 44$^\circ$ 5$^\prime$ N. und 44$^\circ$ 43$^\prime$ N. & Zwischen 22$^\circ$ 55$^\prime$ O. und 23$^\circ$ 50$^\prime$ O. & P. 28. 1833. 574. P. 34. 1815. 341. \\ \hline
          &   &   &   & Meteorsteine, deren Fallzeit unbekannt. &   &   &   &   \\ \hline
        12. & - & - & - & Stein, der zu Cassandria (Potidaea) war aufbewahrt worden. & Makedonien & 40$^\circ$ 10$^\prime$ N. & 23$^\circ$ 20$^\prime$ O. & A. 4. 185. \\ \hline
          &   &   &   & Meteor-Eisenmasse, deren Fallzeit unbekannt. &   &   &   &   \\ \hline
        13. & - & - & - & ? & Makedonien & - & - & P. 18. 1830. 190. \\ \hline
    \end{tabular}
\end{table}
\clearpage
\subsection{Karte 2. - Oeftliche Halbkugel.}
\subsubsection{A. Europa. Siehe Karte 1.}
\subsubsection{B. Afrika.}
\begin{table}[!ht]
    \centering
    \begin{tabular}{|l|l|l|l|l|l|l|l|l|}
    \hline
        1. & 2. & 2. & 2. & 3. & 3. & 4. & 5. & 6. \\ \hline
        1. & 481 & - & - & ? & Afrika & - & - & P. 8. 1826. 45. \\ \hline
        2. & 856 & - & Dezember & Sowaida (Sowadi), S. von Cairo. & Ägypten & 28$^\circ$ 0$^\prime$ N. & 31$^\circ$ 20$^\prime$ O. & G. 50. 1815. 231. \\ \hline
        3. & 1801 & - & - & Isle des Tonneliers, durch eine Brücke mit Isle de France (20$^\circ$ 30 S., 58$^\circ$ 0 O.) verbunden. & Indisches Meer & - & - & G. 60. 1818. 246. \\ \hline
        4. & 1838 & 13. & Oktober & Im Kalten Bokkeveld, 15 engl. M. N. von Tulbagh und 70 engl. M. von der Kapstadt. - Sp.-Gew.: 2,69-2,94. & Süd-Afrika & Zwischen 32$^\circ$ 0$^\prime$ S. und 33$^\circ$ 0$^\prime$ S. & Zwischen 19$^\circ$ 0$^\prime$ O. und 20$^\circ$ 0$^\prime$ O. & P. 47. 1839. 384. W. 1860. S. 1860. \\ \hline
        5. & 1849 & - & August & In den Kumadau-See (Kumatao-Bassin). & Süd-Afrika & 21$^\circ$ 25$^\prime$ S. & 25$^\circ$ 20$^\prime$ O. & L. 1. Fol. 85 und 2. Fol. 257* \\ \hline
        6. & 1849 & 13. & November & Tripolis. & Nord-Afrika & 32$^\circ$ 50$^\prime$ N. & 13$^\circ$ 25$^\prime$ O. & P. 4. 1854. 382. \\ \hline
        7. & 1850 & 25. & Januar & Tripolis. & Nord-Afrika & 32$^\circ$ 50$^\prime$ N. & 13$^\circ$ 25$^\prime$ O. & P. 4. 1854. 382. \\ \hline
        8. & 1852 & - & Zwischen Juni und Dezember & Am Großen Tschuai (Gr. Tschui), NO. von Kuruman und Metito. & Süd-Afrika & 26$^\circ$ 30$^\prime$ S. & 25$^\circ$ 20$^\prime$ O. & L. 2. 257. \\ \hline
        9. & 1852 & - & Zwischen Juni und Dezember & Kuruman (Neu-Lattuku), am oberen Lauf des Kuruman-Flusses. & Süd-Afrika & 27$^\circ$ 25$^\prime$ S. & 24$^\circ$ 10$^\prime$ O. & Desgl. \\ \hline
          &   &   &   & Meteor-Eisenmassen, deren Fallzeit unbekannt. &   &   &   &   \\ \hline
        10. & - & - & - & Im Lande Bambuk und im Lande Siwatik (Siratik) ($^\wedge$$^\wedge$$^\wedge$), nicht weit vom rechten Ufer des oberen Senegal. In vielen großen und kleinen Stücken herumliegend. Gefunden 1763. - Sp.-Gew.: 7,34-7,72. & West-Afrika & Zwischen 13$^\circ$ 0$^\prime$ N. und 15$^\circ$ 0$^\prime$ N. & Zwischen 10$^\circ$ 0$^\prime$ W. und 12$^\circ$ 0$^\prime$ W. & G. 50. 1815. 271. W. 1860. S. 1860. \\ \hline
        11. & - & - & - & Am Löwen-Fluss, dem oberen, östlichen Arm des Aub oder großen Fischflusses, der in den Gariep oder Oranjefluss sich ergiesst; Groß-Namaqualand. - 1 Eisenmasse von 178 Pfund und mehrere kleinere. Gefunden 1853. - Sp.-Gew.: 7,45. & Süd-Afrika & Zwischen 22$^\circ$ 30$^\prime$ S. und 24$^\circ$ 50$^\prime$ S. & Zwischen 17$^\circ$ 20$^\prime$ O. und 17$^\circ$ 50$^\prime$ O. & B. 128. W. 1860. S. 1860. \\ \hline
        12. & - & - & - & Am Oranje-Fluss (Gariep); Kapland. Gefunden 1856. - Sp.-Gew.: 7,3. & Süd-Afrika & Zwischen 28$^\circ$ 10$^\prime$ S. und 31$^\circ$ 0$^\prime$ S. & Zwischen 16$^\circ$ 30$^\prime$ O. und 28$^\circ$ 35$^\prime$ O. & SJ. 2. 21. 1856. 213. W. 1860. S. 1860. \\ \hline
        13. & - & - & - & Im NO. des Großen Schwarzkopf-Flusses ($^\wedge$$^\wedge$$^\wedge$), zwischen dem Sonntags- und Boschemans-Fluss; Kapland. 300 Pfund Gefunden 1793. - Sp.-Gew.: 6,63-7,94. & Süd-Afrika & Zwischen 33$^\circ$ 20$^\prime$ S. und 34$^\circ$ 40$^\prime$ S. & 27$^\circ$ 30$^\prime$ O. & P. 4. 1854. 397. W. 1860. S. 1860. \\ \hline
        14. & - & - & - & Am Großen Fischfluss, Distrikt von Graaf-Reynet (32$^\circ$ 10$^\prime$ S., 24$^\circ$ 50$^\prime$ O.); Kapland. Große Menge von Eisen, darunter eine Masse von 3 Zentner Gefunden 1838. - & Süd-Afrika & Zwischen 32$^\circ$ 0$^\prime$ S. und 32$^\circ$ 30$^\prime$ S. & Zwischen 25$^\circ$ 0$^\prime$ O. und 26$^\circ$ 50$^\prime$ O. & G. 50. 1815. 264. \\ \hline
        15. & - & - & - & St. Augustines Bay. Gefunden 1843. & Insel Madagascar & 23$^\circ$ 30$^\prime$ S. & 44$^\circ$ 20$^\prime$ O. & SJ. 2. 15. 1853. 22. S. 1860. \\ \hline
    \end{tabular}
\end{table}
\clearpage
\subsubsection{C. Asien.}
1. Kleinasien, Arabien, Persien und Afghanistan.
\begin{table}[!ht]
    \centering
    \begin{tabular}{|l|l|l|l|l|l|l|l|l|}
    \hline
        1. & 2. & 2. & 2. & 3. & 3. & 4. & 5. & 6. \\ \hline
        1. & 5.. & - & - & Gebirge Libanon. & Syrien & Ungefähr 34$^\circ$ 0$^\prime$ N. & Ungefähr 36$^\circ$ 0$^\prime$ O. & G. 54. 1816. 340. \\ \hline
        2. & 5.. & - & - & Emesa. & Syrien & 34$^\circ$ 40$^\prime$ N. & 37$^\circ$ 50$^\prime$ O. & G. 54. 1816. 340. \\ \hline
        3. & 852 & - & Juli (August) & Provinz Tabarestan (Taberistan) oder Provinz Masanderan, an der Südküste des Kaspischen Meeres. & Persien & Zwischen 35$^\circ$ 0$^\prime$ N. und 37$^\circ$ 0$^\prime$ N. & Zwischen 50$^\circ$ 0$^\prime$ O. und 57$^\circ$ 0$^\prime$ O. & G. 50. 1815. 230. \\ \hline
        4. & 893 (892, 897, 898, 899 oder 908) & - & - & Ahmed-Abad (Ahmed-Dad) ($^\wedge$$^\wedge$$^\wedge$) bei Kufah (32$^\circ$ 0$^\prime$ N., 45$^\circ$ 0$^\prime$ O.), S. von Bagdad und von Helle, und SO. von Mesched-Ali. & Mesopotamien & 37$^\circ$ 0$^\prime$ N. & 57$^\circ$ 0$^\prime$ O. & G. 50. 1815. 231. \\ \hline
        5. & Zwischen 999 und 1030; wahrscheinlich um 1009 & - & - & Provinz Tschurdschan (Djouzdjan, Dschuzzan, oder Dsjordsjan) in Khorasan, an der Ostküste des Kaspischen Meeres. Eisen. & Persien & Ungefähr 37$^\circ$ 0$^\prime$ N. & Zwischen 53$^\circ$ 50$^\prime$ O. und 55$^\circ$ 50$^\prime$ O. & G. 50. 1815. 232. \\ \hline
        6. & 1151 & - & - & ? & Im Orient & - & - & P. 24. 1832. 222. \\ \hline
        7. & Um 1340 (nicht 1440) & - & - & Birki (Bireki oder Birgeh), NNO. von Güzelhissar (Aidin oder Tralles), SSW. von Sardes (Sart) und OSO. von Smyrna; Provinz Aidin. & Klein-Asien & 38$^\circ$ 16$^\prime$ N. & 27$^\circ$ 57$^\prime$ O. & P. 4. 1854. 10. Ibn Batuta Fol. 72* \\ \hline
        8. & 1833 (1834) & - & Ende November (Ende April) & Kandahar. & Afghanistan & 32$^\circ$ 40$^\prime$ N. & 65$^\circ$ 15$^\prime$ O. & P. 4. 1854. 33. \\ \hline
          &   &   &   & Meteorsteine, deren Fallzeit unbekannt. &   &   &   &   \\ \hline
        9. & - & - & - & Stein in der Kaaba in Mekka eingemauert. & Arabien & 21$^\circ$ 30$^\prime$ N. & 39$^\circ$ 50$^\prime$ O. & G. 54. 1816. 332. \\ \hline
        10. & - & - & - & Stein, der zu Emesa (jetzt Hems oder Hims) verehrt und durch Heliogabal nach Rom war gebracht worden. & Syrien & 34$^\circ$ 40$^\prime$ N. & 37$^\circ$ 50$^\prime$ O. & G. 54. 1816. 331. \\ \hline
        11. & - & - & - & Stein zu Pessinus in Phrygien gefallen, und 204 v. Chr. Nach Rom gebracht. & Klein-Asien & 39$^\circ$ 24$^\prime$ N. & 31$^\circ$ 20$^\prime$ O. & G. 54. 1816. 330. \\ \hline
        12. & - & - & - & Stein, der zu Abydos war aufbewahrt worden. & Klein-Asien & 40$^\circ$ 18$^\prime$ N. & 26$^\circ$ 20$^\prime$ O. & P. 2. 1824. 156. \\ \hline
    \end{tabular}
\end{table}
\clearpage
2. Vorder- und Hinter-Indien.
\begin{table}[!ht]
    \centering
    \begin{tabular}{|l|l|l|l|l|l|l|l|l|}
    \hline
        1. & 2. & 2. & 2. & 3. & 3. & 4. & 5. & 6. \\ \hline
        1. & 1421 & - & - & ? & Java & Zwischen 6$^\circ$ 0$^\prime$ S. und 9$^\circ$ 0$^\prime$ S. & Zwischen 105$^\circ$ 0$^\prime$ O. und 115$^\circ$ 0$^\prime$ O. & G. 63. 1819. 17. \\ \hline
        2. & 1621 (nicht 1650 oder 1652) & 17. & April & Tschalinda (Dschallinder oder Jalendher), 20 geogr. M. OSO. von Lahore. Eisen. & Pendsjab (Punjab) & 31$^\circ$ 24$^\prime$ N. & 75$^\circ$ 34$^\prime$ O. & G. 50. 1815. 241. \\ \hline
        3. & 1795 & 13. & April & Provinz Carnawelpattu ($^\wedge$$^\wedge$$^\wedge$), 4 M. von Multetiwu (Moeletivoe, 9$^\circ$ 14$^\prime$ N., 80$^\circ$ 54$^\prime$ O.). & Insel Ceylon & - & - & G. 54. 1816. 351. \\ \hline
        4. & 1798 & 13. (19.) & Dezember & Krak-Hut, an der Nordseite des Goomty (Gumti), ungefähr 14 engl. M. von Benares und 12 engl. M. Von Jounpoor (Juanpoor oder Dschaunpur) in Bengalen. - Sp.-Gew.: 3,35-3,36. & Hindostan & 25$^\circ$ 38$^\prime$ N. & 83$^\circ$ 0$^\prime$ O. & G. 13. 1803. 298. W. 1860. S. 1860. \\ \hline
        5. & 1802 & - & - & Allahabad in Bengalen. - Sp.-Gew.: 3,5. & Hindostan & 25$^\circ$ 23$^\prime$ N. & 81$^\circ$ 49$^\prime$ O. & P. 24. 1832. 223. \\ \hline
        6. & 1808 & - & - & Mooradabad, Provinz Rohilcund in Delhi. & Hindostan & 28$^\circ$ 50$^\prime$ N. & 78$^\circ$ 48$^\prime$ O. & P. 24. 1832. 223. \\ \hline
        7. & 1810 & - & Mitte Juli & Shabad ($^\wedge$$^\wedge$$^\wedge$), 30 engl. M. von Futtehpore (Futtypoor), oder nach anderer Angabe bei Futtyghur, jenseits des Ganges. & Hindostan & - & - & P. 8. 1826. 47. \\ \hline
        8. & 1811 & 23. & November & Panganoor in Carnatic. Eisen. & Dekan & 13$^\circ$ 22$^\prime$ N. & 78$^\circ$ 38$^\prime$ O. & P. 4. 1854. 396. RPG. 36. \\ \hline
        9. & 1814 & 5. & November & Bezirk Lapk ($^\wedge$$^\wedge$$^\wedge$); Bezirk Bhaweri ($^\wedge$$^\wedge$$^\wedge$), zum Bezirk Bezum-Sumro ($^\wedge$$^\wedge$$^\wedge$) gehörig; Bezirk Chal ($^\wedge$$^\wedge$$^\wedge$), zum Pergunnah de Schawlif ($^\wedge$$^\wedge$$^\wedge$), gehörig; und Bezirk Kaboul ($^\wedge$$^\wedge$$^\wedge$), ebendahin gehörend. Sämmtlich in der Provinz Doab. & Hindostan & Zwischen 26$^\circ$ 0$^\prime$ N. und 28$^\circ$ 15$^\prime$ N. & Zwischen 77$^\circ$ 30$^\prime$ O. und 82$^\circ$ 0$^\prime$ O. & G. 53. 1816. 381. \\ \hline
        10. & 1815 & 18. & Februar & Dooralla (Duralla) ($^\wedge$$^\wedge$$^\wedge$), im Gebiet des Pattialah Rajah, 16 bis 18 engl. M. von Umballa und 18 engl. M. von Loodianah (Ludeana oder Loodheeana) in Lahore. & Hindostan & 30$^\circ$ 30$^\prime$ N. (ungefähr) & 76$^\circ$ 4$^\prime$ O. & G. 68. 1821. 333. \\ \hline
        11. & 1822 & 7. & August & Kadonah ($^\wedge$$^\wedge$$^\wedge$), Distrikt von Agra (27$^\circ$ 12$^\prime$ N., 78$^\circ$ 3$^\prime$ O.); Provinz Doab. & Hindostan & - & - & P. 4. 1854. 33. \\ \hline
        12. & 1822 & 30. & November & Rourpoor ($^\wedge$$^\wedge$$^\wedge$) bei Fattehpore (25$^\circ$ 57$^\prime$ N., 80$^\circ$ 50$^\prime$ O.); 72 M. von Allahabad, auf dem Wege nach Cawnpoor; Provinz Doab. - Sp.-Gew.: 3,352-3,526. & Hindostan & - & - & P. 18. 1830. 179. SJ. 2. 11. 1851. 36. WA. 41. 1860. 747. W. 1860. S. 1860. \\ \hline
        13. & 1825 & 16. & Januar & Oriang ($^\wedge$$^\wedge$$^\wedge$) in Malwa, N. vom oberen Lauf des Nerbada-(Nerbudda-)Flusses & Hindostan & ungefähr zwischen 22$^\circ$ 30$^\prime$ N. und 23$^\circ$ 30$^\prime$ N. & ungefähr zwischen 77$^\circ$ 0$^\prime$ O. und 81$^\circ$ 0$^\prime$ O. & P. 6. 1826. 32. \\ \hline
        14. & 1827 & 27. & Februar & Mhow (Mow), Distrikt Azim-Gesh, NNO. von Ghazeepoor (am Ganges) und OSO. von Azimgur. - Sp.-Gew.: 3,5. & Hindostan & 25$^\circ$ 57$^\prime$ N. & 83$^\circ$ 36$^\prime$ O. & P. 24. 1832. 226. RPG. 37. \\ \hline
        15. & 1834 & 12. & Juni & Charwallas ($^\wedge$$^\wedge$$^\wedge$), 30 M. von Hissar (29$^\circ$ 12$^\prime$ N., 75$^\circ$ 40$^\prime$ O.) und 40 M. von Delhi. - Sp.-Gew.: 3,38. & Hindostan & - & - & P. 4. 1854. 33. SJ. 2. 11. 1851. Fol. 36. S. 1860. \\ \hline
        16. & 1838 & 18. & April & Akburpoor, WSW. von Cawnpoor, zwischen dem Ganges und dem Jumna. & Hindostan & 26$^\circ$ 25$^\prime$ N. & 79$^\circ$ 57$^\prime$ O. & RPG. 37. \\ \hline
        17. & 1838 & 6. & Juni & Chandakapoor ($^\wedge$$^\wedge$$^\wedge$) in Berar (Hauptstadt: Nagpoor, 21$^\circ$ 10$^\prime$ N., 79$^\circ$ 10$^\prime$ O.). - Sp.-Gew.: 3,49? & Dekan & - & - & W. 1860. S. 1860. \\ \hline
        18. & 1842 & 30. & November & Zwischen Jeetala ($^\wedge$$^\wedge$$^\wedge$) und Mor-Monree ($^\wedge$$^\wedge$$^\wedge$) in Myhee-Counta ($^\wedge$$^\wedge$$^\wedge$), NO. von Ahmedabad (23$^\circ$ 2$^\prime$ N., 72$^\circ$ 38$^\prime$ O.). - Sp.-Gew.: 3,360. & Hindostan & - & - & P. 4. 1854. 366. Edinb. Phil. Journ. 47. 1849. 55. \\ \hline
        19. & 1843 & 26. & Juli & Manjegaon ($^\wedge$$^\wedge$$^\wedge$) bei Eidulabad ($^\wedge$$^\wedge$$^\wedge$) in Khandeish (vielleicht Mallygaum, 20$^\circ$ 32$^\prime$ N., 74$^\circ$ 35$^\prime$ O., und NO. von Bombay?). - Sp.-Gew.: 4,0-4,5. & Dekan & - & - & P. 4. 1854. 370. \\ \hline
        20. & 1848 & 15. & Februar & Negloor (Nerulgee oder Neralgi), wenige M. vom Zusammenfluss des Wurda (Warada) mit dem Toombooda (Tumbudra, Toongabudra oder Tunga-Bhadra), Gootul-Division des Ranee-Bednoor-Talook des Dharwar-Collectorates in Beejapoor. - Sp.-Gew.: 3,512. & Dekan & 14$^\circ$ 55$^\prime$ N. & 75$^\circ$ 44$^\prime$ O. & P. 4. 1854. 380. Edinb. Phil. Journ. 47. 1849. 53. \\ \hline
        21. & 1850 & 30. & November & Shalka (Sháluka, Shalkà oder Sulker) ($^\wedge$$^\wedge$$^\wedge$), bei Bissempur (Bissunpoor, 23$^\circ$ 5$^\prime$ N., 87$^\circ$ 22$^\prime$ O., 10 engl. M. von Bancoorah) in West-Burdwan, WNW. von Calcutta. - Sp.-Gew.: 3,412-3,66. & Hindostan & - & - & WA. 41. 1860. 253. P. 4. 1854. 382. W. 1860. \\ \hline
        22. & 1853 & 6. & März & Segowlee (Soojonlee oder Sugouli), N. von Patna in Bahar, und 17 engl. M. O. von Bettiah. - Sp.-Gew.: 3,425. & Hindostan & 26$^\circ$ 45$^\prime$ N. & 84$^\circ$ 48$^\prime$ O. & WA. 41. 1860. 754. W. 1860. \\ \hline
        23. & 1857 & 28. & Februar (?) & Parnallee ($^\wedge$$^\wedge$$^\wedge$) bei Madras (13$^\circ$ 5$^\prime$ N., 80$^\circ$ 20$^\prime$ O.). - & Dekan & - & - & Brit. Ass. Reports (?) \\ \hline
        24. & 1857 & 27. & Dezember & Quenggouk bei Bassein in Pegu. - Sp.-Gew.: 3,737. & Birma & Ungefähr 17$^\circ$ 30$^\prime$ N. & Ungefähr 95$^\circ$ 0$^\prime$ O. & WA. 41. 1860. 750. u. 42. 301. W. 1860.* \\ \hline
        25. & 1860 & 14. & Juli & Dhurmsala ($^\wedge$$^\wedge$$^\wedge$) bei Kangra (31$^\circ$ 57$^\prime$ N., 76$^\circ$ 5$^\prime$ O.), ONO. von Lahore. & Pendsjab (Punjab) & - & - & WA. 42. 1816. Fol. 305.* \\ \hline
        26. & 1860 & - & - & Bhurtpore (Bhurtpoor), W. von Agra. & Hindostan & 27$^\circ$ 14$^\prime$ N. & 77$^\circ$ 30$^\prime$ O. & H. \\ \hline
          &   &   &   & Meteorsteine, deren Fallzeit unbekannt. &   &   &   & ~ \\ \hline
        27. & - & - & - & ? Gefunden 1846. - Sp.-Gew.: 3,792. & Wahrscheinlich aus Assam & Zwischen 25$^\circ$ 0$^\prime$ N. und 27$^\circ$ 30$^\prime$ N. & Zwischen 90$^\circ$ 0$^\prime$ O. und 95$^\circ$ 0$^\prime$ O. & WA. 41. 1860. 752. W. 1860. \\ \hline
          &   &   &   & Meteor-Eisenmasse, deren Fallzeit unbekannt. &   &   &   &   \\ \hline
        28. & - & - & - & Singhur (Singurh), SW. von Poonah in Beejapoor. - 31 Pfund Gefunden 1847. - Sp.-Gew.: 4,72-4,90. & Dekan & 18$^\circ$ 20$^\prime$ N. & 73$^\circ$ 48$^\prime$ O. & P. 4. 1854. 396. \\ \hline
    \end{tabular}
\end{table}
\clearpage
3. Asiatisches Russland
\begin{table}[!ht]
    \centering
    \footnotesize
    \begin{tabular}{|p{3mm}|p{5mm}|p{4mm}|p{13mm}|p{22mm}|p{14mm}|p{10mm}|p{10mm}|p{13mm}|}
    \hline
        1. & 2. & 2. & 2. & 3. & 3. & 4. & 5. & 6. \\ \hline
        1. & 1805 & 25. & März & Doroninsk, nahe am Indoga, Gouv. Irkutsk. - Sp.-Gew.: 3,63. & Sibirien & 50$^\circ$ 30$^\prime$ N. & 112$^\circ$ 20$^\prime$ O. & G. 31. 1809. 308. W. 1860. S. 1860. \\ \hline
        2. & 1824 & 18. & Februar & Tounkin (Tungin, Tunginsk oder Tunga), 216 Werste WSW. von Irkutsk, Gouv. Irkutsk. - Sp.-Gew.: 3,72? & Sibirien & 51$^\circ$ 50$^\prime$ N. & 105$^\circ$ 50$^\prime$ O. & P. 24. 1832. 224. P. 107. 1859. 162. \\ \hline
        3. & 1840 & 9. & Mai & Am Fluss Karokol ($^\wedge$$^\wedge$$^\wedge$). & Kirgisen-Steppe & Zwischen 45$^\circ$ 0$^\prime$ N. und 55$^\circ$ 0$^\prime$ N. & Zwischen 70$^\circ$ 0$^\prime$ O. und 110$^\circ$ 0$^\prime$ O. & P. 4. 1854. 360. RPG. 37. \\ \hline
          &   &   &   & Meteorstein, dessen Fallzeit unbekannt. &   &   &   &   \\ \hline
        4. & - & - & - & Gouv. Simbirsk (54$^\circ$ 30$^\prime$ N., 48$^\circ$ 20$^\prime$ O.). Gefunden 1845. - Sp.-Gew.: 3,51-3,55. & Königreich Kasan & - & - & W. 1860. \\ \hline
          &   &   &   & Meteor-Eisenmassen, deren Fallzeit unbekannt. &   &   &   &   \\ \hline
        5. & - & - & - & Zwischen Krasnojarsk und Abakansk auf einem Berg zwischen dem Ubei und dem Sisim, 2 Nebenflüssen des Jenisei, Gouv. Jeniseisk. - 1600 Pfund Pallas’sche Masse. Gefunden 1749. - Sp.-Gew.: 6,487-7,84. & Sibirien & Zwischen 56$^\circ$ 30$^\prime$ N. und 54$^\circ$ 30$^\prime$ N. & Zwischen 93$^\circ$ 0$^\prime$ O. und 91$^\circ$ 0$^\prime$ O. & G. 50. 1815. 257. W. 1860. S. 1860. B. 48. \\ \hline
        6. & - & - & - & Alasej’scher Bergrücken, der das Flussgebiet des Alasej (Alazeia) von dem der Indigirka trennt; 100 Werste von Orinkino. & Sibirien & Zwischen 66$^\circ$ 30$^\prime$ N. und 71$^\circ$ 0$^\prime$ N. & Zwischen 143$^\circ$ 20$^\prime$ O. und 155$^\circ$ 20$^\prime$ O. & P. 4. 1854. 396. \\ \hline
        7. & - & - & - & Goldseife Petropawlowsk ($^\wedge$$^\wedge$$^\wedge$) am Altai, Bezirk des Mrasa-Flusses; Gouv. Omsk. - 17 ½ Pfund Gefunden 1841. - Sp.-Gew.: 7,76. & Sibirien & 57$^\circ$ 7$^\prime$ N. & 87$^\circ$ 27$^\prime$ O. & P. 61. 1844. 675. Clark Fol. 72* W. 1860. \\ \hline
        8. & - & - & - & ? Sp.-Gew.: 7,55. & Kamtschatka & - & - & P. 107. 1859. 162. \\ \hline
        9. & - & - & - & 30 Werste von Sarepta, an der Wolga; Gouv. Saratow. & Königreich Astrachan & 48$^\circ$ 28$^\prime$ N. & 44$^\circ$ 29$^\prime$ O. & RPG. \\ \hline
    \end{tabular}
\end{table}
4. Tibet
\begin{table}[!ht]
    \centering
    \footnotesize
    \begin{tabular}{|p{3mm}|p{5mm}|p{4mm}|p{13mm}|p{22mm}|p{14mm}|p{10mm}|p{10mm}|p{13mm}|}
    \hline
        1. & 2. & 2. & 2. & 3. & 3. & 4. & 5. & 6. \\ \hline
          &   &   &   & Meteor-Eisenmasse, deren Fallzeit unbekannt. &   &   &   &   \\ \hline
        1. & - & - & - & Die eiserne Keule, im Lama-Kloster Sera ($^\wedge$$^\wedge$$^\wedge$) bei Lhassa (H’Lassa oder Lassa, 29$^\circ$ 30$^\prime$ N., 91$^\circ$ 50$^\prime$ O.) aufbewahrt. & Tibet & - & - & P. 24. 1832. 233. \\ \hline
    \end{tabular}
\end{table}
5. China und Korea.
\begin{center}
    \footnotesize
    \begin{longtable}{|p{3mm}|p{5mm}|p{4mm}|p{13mm}|p{22mm}|p{14mm}|p{10mm}|p{10mm}|p{13mm}|}
    \hline
        1. & 2. & 2. & 2. & 3. & 3. & 4. & 5. & 6. \\ \hline
          & vor Christus &   &   &   &   &   &   &   \\ \hline
        1. & 645 (644 Frühjahr) & 24. & Dezember & In dem ehemaligen Königreich Song (Soung), jetzt der östliche Teil der Provinz Ho-nan, darin Song (Soung) im Bezirk von Ho-nan-fou. & Provinz Ho-nan & 34$^\circ$ 10$^\prime$ N. & 112$^\circ$ 8$^\prime$ O. & MS. 135. AR. 1. 190. EB. 189. u. 40. G. 50. 1815. 228. \\ \hline
        2. & 211 & - & - & Tong-kien (Tong-kiun, Toung-kiun oder Toung-tch'ang-fou). & Provinz Chan-toung (Shan-toong) & 36$^\circ$ 32$^\prime$ O. & 116$^\circ$ 10$^\prime$ O. & MS. 135. AR. 1. 190. EB. 251 u. 252. G. 50. 1815. 229. \\ \hline
        3. & 192 & - & - & Mian-tchou (Mien-tchou), Bezirk von Mien-tcheou. & Provinz Sse-tchuen (Szu-tchhuan) & 31$^\circ$ 17$^\prime$ O. & 104$^\circ$ 16$^\prime$ O. & MS. 135. AR. 1. 191. EB. 127. G. 50. 1815. 229. \\ \hline
        4. & 89 & 9. & März & Yong (Young, Yoong oder Young-cheou), nahe bei der ehemaligen Haupstadt Tchang-ngan, jetzt im Bezirk von Singan-fou. & Provinz Chen-si (Shen-si) & 34$^\circ$ 48$^\prime$ O. & 108$^\circ$ 3$^\prime$ O. & MS. 135. AR. 1. 191. EB. 294, 198 u. 172. G. 50. 1815. 229. \\ \hline
        5. & 38 & 13. & März & In ehemal. Konigreich Leang (Liang), Gegend des heutigen Khai-foung-fou. & Provinz Ho-nan & Ungefähr 34$^\circ$ 52$^\prime$ N. & Ungefähr 114$^\circ$ 33$^\prime$ O. & MS. 136. AR. 1. 191. EB. 101 u. 59. G. 50. 1815. 229. \\ \hline
        6. & 29 & 29. & Februar & Khao (Khao-tch’ing) im Bezirk von Tching-ting-fou (Tchin-ting-fou); und zu Fei-lo (Fei-tch’ing), unter 36 39 ebenfalls in Pe-tchi-li.* & Provinz Pe-tchi-li & 38$^\circ$ 5$^\prime$ N. & 114$^\circ$ 59$^\prime$ O. & MS. 136. AR. 1. 192. EB. 60 u. 209. G. 50. 1815. 230. DG. 1. 146. \\ \hline
        7. & 22 & 12. & April & Pe-ma, im Distrikt von Toung-kien (Toung-kiun) bei Hoa, Bezirk von Thai-ming-fou (oder Ta-ming). & Provinz Pe-tchi-li & Ungefähr 35$^\circ$ 38$^\prime$ N. & Ungefähr 114$^\circ$ 48$^\prime$ O. & MS. 136. AR. 1. 192. EB. 157, 43, 223 u. 251. G. 50. 1815. 230. \\ \hline
        8. & 19 & 16. & Juni & Tu-yan (Tou-yan oder Tou-yen) bei Nan-yang (Nan-yang-fou). & Provinz Ho-nan & Ungefähr 33$^\circ$ 6$^\prime$ N. & Ungefähr 112$^\circ$ 35$^\prime$ O. & MS. 137. AR. 1. 192. EB. 136. G. 50. 1815. 230. \\ \hline
        9. & 12 & - & Ungefähr im April & Tu-ku-an (Tou-kouan, Chang-yang oder Chan-yang), Bezirk von Chang-tcheou. & Provinz Chen-si (Shen-si) & 33$^\circ$ 29$^\prime$ N. & 110$^\circ$ 1$^\prime$ O. & MS. 137. AR. 1. 192. EB. 2, 5 u. 172. G. 50. 1815. 230. \\ \hline
        10. & 9 & - & - & ? & China & - & - & DG. 1. Fol. 250. G. 50. 1815. 230. \\ \hline
        11. & 6 & 4. & März & Ning-tschu (Ning-tcheou), Bezirk von Pe-ti (oder Khing-yang-fou), früher in der Provinz Chen-si (Shen-si), jetzt Provinz Kan-sou. & Provinz Kan-sou (Kan-soo) & 35$^\circ$ 35$^\prime$ N. & 107$^\circ$ 51$^\prime$ O. & MS. 137. AR. 1. 192. DG. 1. 250. EB. 144, 156 u. 64. G. 50 1815. 230. \\ \hline
        12. & 6 & 27. & Oktober & Yu (Ju) bei Ngan-y, im ehemaligen Königreich Liang (Leang), jetzt Bezirk Kiai-tcheou, Provinz Chan-si. & Provinz Chan-si (Shan-si) & Ungefähr 35$^\circ$ 5$^\prime$ N. & Ungefähr 110$^\circ$ 58$^\prime$ O. & MS. 137. AR. 1. 192. EB. 142, 71 u. 164. G. 50. 1815. 230. \\ \hline
          & nach Christus &   &   &   &   &   &   &   \\ \hline
        13. & 2 & - & - & Kiu-lu (Kiou-lou oder Kiu-lo), Bezirk von Chun-t-fou (Shun-te). & Provinz Pe-tchi-li & 37$^\circ$ 17$^\prime$ N. & 115$^\circ$ 11$^\prime$ O. & MS. 137. AR. 1. 192. EB. 82 u. 14. P. 4. 1854. 450. \\ \hline
        14. & 106 & - & - & Tschin-lieu (Tschin-lieou, Tch’in-lieou-fou oder Tchhin-liu), Bezirk von Khai-foung-fou. & Provinz Ho-nan & 34$^\circ$ 45$^\prime$ N. & 114$^\circ$ 40$^\prime$ O. & MS. 141. AR. 1. 193. EB. 212 u. 59. P. 4. 1854. 450. \\ \hline
        15. & 154 (164) & 1. & April & Yeu-fu-fung (Yeou-fou-foung oder Foung-thsiang-fou). & Provinz Chen-si (Shen-si) & 34$^\circ$ 25$^\prime$ N. & 107$^\circ$ 30$^\prime$ O. & MS. 141. AR. 1. 194. EB. 286. u. 22. P. 4. 1854. 450. \\ \hline
        16. & 154 (164) & - & - & Khien (Khiang, Khian, Kiang oder Khien-kiang), Bezirk Tchoung-khing-fou. & Provinz Sse-tchuen (Szu-tchhuan) & 29$^\circ$ 21$^\prime$ N. & 106$^\circ$ 23$^\prime$ O. & MS. 141. AR. 1. 194. EB. 63 u. 218. P. 4. 1854. 450. \\ \hline
        17. & 310 & 23. & Oktober & Wahrscheinlich in der Nähe von Phing-yang (P’ing-yang-fou). & Provinz Chan-si (Shan-si) & Wahrscheinlich 36$^\circ$ 6$^\prime$ N. & Wahrscheinlich 111$^\circ$ 33$^\prime$ O. & MS. 143. AR. 1. 195. EB. 164. P. 4. 1854. 450. \\ \hline
        18. & 333 & - & - & 6 franz. M. NO. von Ye (oder Lin-tch’ang), Bezirk von Tchang-te-fou. & Provinz Ho-nan & 36$^\circ$ 22$^\prime$ N. & 114$^\circ$ 48$^\prime$ O. & MS. 143. AR. 1. 195. EB. 283, 106 u. 202. P. 4. 1854. 450. \\ \hline
        19. & 616 & 28. & Mai & U-kien (Ou-kiun oder Son-tcheon-fou) in der ehemaligen Provinz Ou, dem östlichen Teil der ehemaligen Provinz Kiang-nan; jetzt Provinz Kiang-sou. & Provinz Kiang-sou (Kiang-soo) & 31$^\circ$ 23$^\prime$ N. & 120$^\circ$ 29$^\prime$ O. & MS. 147. AR. 1. 197. EB. 186 u. 73. P. 4. 1854. 450. \\ \hline
        20. & 1057 & - & - & Provinz Hoang-hai (Hauptstadt: Hoang-tcheou, Hoang-liei). & Korea & 34$^\circ$ 54$^\prime$ N. & 127$^\circ$ 0$^\prime$ O. & AR. 1. 205. P. 6. 1826. 23. \\ \hline
        21. & 1358 & - & - & Thai-ming, Bezirk von Thai-ming-fou. & Provinz Pe-tchi-li & 36$^\circ$ 18$^\prime$ N. & 115$^\circ$ 20$^\prime$ O. & MS. 328. EB. 223. A. 4. 189. \\ \hline
        22. & 1491 & 15. & November & Kouang-chan (Kwang-shan), Bezirk von Jou-ning-fou. & Provinz Ho-nan & 32$^\circ$ 8$^\prime$ N. & 114$^\circ$ 51$^\prime$ O. & MS. 333. EB. 86 u. 53. \\ \hline
        23. & 1516 & - & - & Schun-king-fu (Chun-khing-fou). & Provinz Sse-tchuen (Szu-tchhuan) & 30$^\circ$ 49$^\prime$ N. & 106$^\circ$ 7$^\prime$ O. & AR. 1. 208. EB. 13. P. 4. 1854. 451. \\ \hline
        24. & 1540 & 14. & Juni & Tsao-khiang, bei Ki-tcheou, Bezirk von Tchin-ting-fou. & Provinz Pe-tchi-li & Ungefähr 37$^\circ$ 38$^\prime$ N. & Ungefähr 115$^\circ$ 42$^\prime$ O. & MS. 336. EB. 254, 67 u. 209. A. 4. 190. \\ \hline
        25. & 1575 (nicht 1565) & 3. & Juli & King-tcheou (King-tcheou-fou), ehemals Prov. Hou-kouang, jetzt Provinz Hou-pe. & Provinz Hou-pe (Hoo-pe) & 30$^\circ$ 27$^\prime$ N. & 112$^\circ$ 5$^\prime$ O. & MS. 336. EB. 81 u. 50. A. 4. 190. \\ \hline
        26. & 1618 & 12. & November & Nan-king (Cour du midi oder Kiang-ning-fou), ehemals Provinz Kiang-nan, jetzt Provinz Kiang-sou. & Provinz Kiang-sou (Kiang-soo) & 32$^\circ$ 5$^\prime$ N. & 118$^\circ$ 47$^\prime$ O. & MS. 339. EB. 133, 72 u. 73. A. 4. 191. \\ \hline
\end{longtable}
\end{center}
\clearpage
\subsection{Karte 3. - Westliche Halbkugel.}
\subsubsection{1. Stilles Meer.}
\begin{table}[!ht]
    \centering
    \footnotesize
    \begin{tabular}{|p{3mm}|p{5mm}|p{4mm}|p{13mm}|p{22mm}|p{14mm}|p{10mm}|p{10mm}|p{13mm}|}
    \hline
        1. & 2. & 2. & 2. & 3. & 3. & 4. & 5. & 6. \\ \hline
        1. & 1825 & 14. & September & Hanaruru (Honolulu), auf der Insel Oahu (Wahu oder Waohoo). Sp.-Gew.: 3,39. & Sandwichs-Inseln & 21$^\circ$ 30$^\prime$ N. & 158$^\circ$ 0$^\prime$ W. & P. 18. 1830. 184. W. 1860. S. 1860. \\ \hline
    \end{tabular}
\end{table}
\subsubsection{2. Grönland und Nordisches Eismeer}
\begin{table}[!ht]
    \centering
    \footnotesize
    \begin{tabular}{|p{3mm}|p{5mm}|p{3mm}|p{11mm}|p{28mm}|p{12mm}|p{10mm}|p{12mm}|p{14mm}|}
    \hline
        1. & 2. & 2. & 2. & 3. & 3. & 4. & 5. & 6. \\ \hline
        1. & 1850 & 3. & Dezember & Prince-of-Wales-Strait. & Eismeer & 73$^\circ$ 31$^\prime$ N. & 114$^\circ$ 30$^\prime$ W. (nach M.'s Karte etwa 117$^\circ$ W.) & Miertsching. Fol. 64 u. 67.* \\ \hline
          &   &   &   & Meteor-Eisenmassen, deren Fallzeit unbekannt. &   &   &   &   \\ \hline
        2. & - & - & - & Niakornak, zwischen Rittenbeck und Jacobshavn. 21 Pfund Gefunden 1819. - Sp.-Gew.: 7,073. & Grönland & 69$^\circ$ 25$^\prime$ N. & 50$^\circ$ 30$^\prime$ W. & P. 93. 1854. 155. \\ \hline
        3. & - & - & - & Sowallick, eine Gegend der nördlichen Küste der Baffinsbai. - Sp.-Gew.: 7,23-7,72. & Grönland & 76$^\circ$ 22$^\prime$ N. & 58$^\circ$ 0$^\prime$ W. & G. 63. 1819. 29. W. 1860. \\ \hline
        4. & - & - & - & Eine 3te Masse in Süd-Grönland. & Grönland & - & - & P. 93. 1854. 155. \\ \hline
    \end{tabular}
\end{table}
\subsubsection{3. Canada}
\begin{table}[!ht]
    \centering
    \footnotesize
    \begin{tabular}{|p{3mm}|p{4mm}|p{11mm}|p{7mm}|p{22mm}|p{14mm}|p{10mm}|p{10mm}|p{13mm}|}
    \hline
        1. & 2. & 2. & 2. & 3. & 3. & 4. & 5. & 6. \\ \hline
          &   &   &   & Meteor-Eisenmasse, deren Fallzeit unbekannt. &   &   &   &   \\ \hline
        1. & - & - & - & Madoc ($^\wedge$$^\wedge$$^\wedge$), am St. Lorenzo-Strom, zwischen Montreal und dem Joronto-See. 370 Pfund Gefunden. 1854. - Sp.-Gew.: 7,88? & Ober-Canada & - & - & SJ. 2. 19. 1855. 417. W. 1860. S. 1860. \\ \hline
    \end{tabular}
\end{table}
\subsubsection{4. Vereinigte Staaten von Nord-Amerika.}
\begin{center}
\footnotesize
\begin{longtable}{|p{3mm}|p{6mm}|p{9mm}|p{11mm}|p{22mm}|p{17mm}|p{10mm}|p{10mm}|p{13mm}|}
    \hline
        1. & 2. & 2. & 2. & 3. & 3. & 4. & 5. & 6. \\ \hline
        1. & 1780 & - & - & Kinsdale ($^\wedge$$^\wedge$$^\wedge$), zwischen West-River-Mountain und Connecticut in New-England. Eisen. & ? & - & - & P. 2. 1824. 152. \\ \hline
        2. & 1807 & 14. & Dezember & Weston, Fairfield-County (Hauptstadt: Fairfield), NW. von Fairfield und 53 M. SW. von Hartford. - Sp.-Gew.: 3,3-3,6. & Connecticut & 41$^\circ$ 15$^\prime$ N. & 73$^\circ$ 34$^\prime$ W. & G. 29. 1808. 354. W. 1860. S. 1860. \\ \hline
        3. & 1809 & 17. (20.) & Juni & Zwischen Block-Island und St. Bart. & Ost-Küste von Nord-Amerika & 30$^\circ$ 58$^\prime$ N. & 70$^\circ$ 25$^\prime$ W. & G. 50. 1815. 254. Shepard, Rep. On Am. Met. F. 18* \\ \hline
        4. & 1810 & 4. (7.) (30.) & Januar & Caswell-County (Hauptstadt: Yanceyville, 60 M. NW. von Raleigh). & North-Carolina & Zwischen 36$^\circ$ 15$^\prime$ N. und 36$^\circ$ 30$^\prime$ N. & Zwischen 79$^\circ$ 16$^\prime$ W. und 79$^\circ$ 40$^\prime$ W. & G. 50. 1815. 255. Shepard, Rep. On Am. Met. Fol. 18. \\ \hline
        5. & 1823 & 7. & August & Nobleborough, Lincoln-County (Hauptstadt: Warren), W. von Warren und 23 M. SO. von Augusta. - Sp.-Gew.: 2,08(?)-3,09. & Maine & 44$^\circ$ 5$^\prime$ N. & 69$^\circ$ 40$^\prime$ W. & P. 2. 1824. 153. W. 1860. S. 1860. \\ \hline
        6. & 1825 & 10. & Februar & Nanjemoy, Charles-County (Hauptstadt: Port-Tobacco), WSW. von Port-Tobacco und 47 M. SW. von Annapolis. - Sp.-Gew.: 3,66. & Maryland & 38$^\circ$ 28$^\prime$ N. & 77$^\circ$ 16$^\prime$ W. & P. 6. 1826. 33. W. 1860. S. 1860. \\ \hline
        7. & 1826 (1827) & - & Sommer & Waterloo am Seneca-River, Hauptstadt von Seneca-County, 166 M. WNW. von Albany. - Sp.-Gew.: 2,30. & New-York & 42$^\circ$ 54$^\prime$ N. & 77$^\circ$ 8$^\prime$ W. & P. 88. 1853. 176. S. 1860. \\ \hline
        8. & 1826 & - & September & Waterville am Kennebec-River, Kennebec-County (Hauptstadt: Augusta), 17 M. NNO. von Augusta. & Maine & 44$^\circ$ 35$^\prime$ N. & 69$^\circ$ 65$^\prime$ W. & P. 4. 1854. 24. \\ \hline
        9. & 1827 & 9. (22.) & Mai & Drake Creek ($^\wedge$$^\wedge$$^\wedge$), 18 M. von Nashville (36$^\circ$ 9$^\prime$ N. u. 87$^\circ$ 0$^\prime$ W.), Hauptstadt von Davidson-County; nach Shepard in Sumner-County (Hauptstadt: Gallatin, 23 M. NO. von Nashville). - Sp.-Gew.: 3,485-3,58. & Tennessee & - & - & P. 24. 1832. 226. B. 89 u. 90. Shepard, Rep. On Am. Met. Fol. 18. W. 1860. S. 1860. \\ \hline
        10. & 1828 & 4. & Juni & 7 M. SW. von Richmond, Hauptstadt von Henrico-County (nicht Chesterfield-County). - Sp.-Gew.: 3,29-3,47. & Virginia & 37$^\circ$ 32$^\prime$ N. & 77$^\circ$ 35$^\prime$ W. & P. 17. 1829. 380. W. 1860. S. 1860. \\ \hline
        11. & 1829 & 8. & Mai & Forsyth, Hauptstadt von Monroe-County, 47 M. W. von Milledgeville. - Sp.-Gew.: 3,37-3,52. & Georgia & 33$^\circ$ 0$^\prime$ N. & 84$^\circ$ 13$^\prime$ W. & P. 24. 1832. 227. W. 1860. S. 1860. \\ \hline
        12. & 1829 & 14. & August & Deal ($^\wedge$$^\wedge$$^\wedge$) bei Long-Branch (40$^\circ$ 17$^\prime$ N., 47$^\circ$ 12$^\prime$ O.), Monmouth-County (Hauptstadt: Freehold), ONO. von Freehold und 38 M. O. von Trenton. & New-Jersey & - & - & P. 24. 1832. 228. S. 1860. \\ \hline
        13. & 1835 & 31. & Juli & Charlotte, Hauptstadt von Dickson-County, 33 M. W. von Nashville. - Eisen. Sp.-Gew.: 7,88? & Tennessee & 36$^\circ$ 13$^\prime$ N. & 87$^\circ$ 36$^\prime$ W. & P. 73. 1848. 332. S. 1860. \\ \hline
        14. & 1837 & 5. & Mai & East-Bridgewater, Plymouth-County (Hauptstadt: Plymouth), W. von Plymouth und 22 M. S. von Boston. - Sp.-Gew.: 2,159-2,815. & Massachusetts & 41$^\circ$ 58$^\prime$ N. & 71$^\circ$ 8$^\prime$ W. & P. 4. 1854. 83. \\ \hline
        15. & 1839 & 13. & Februar & Pine-Bluff am Gasconade-River, 10 M. SW. von Little-Piney, Pulasky-County (Hauptstadt: Waynesville), 10 M. NO. von Waynesville und 43 M. S. von Jeffersoncity. - Sp.-Gew.: 3,5. & Missouri & 37$^\circ$ 55$^\prime$ N. & 92$^\circ$ 5$^\prime$ W. & P. 4. 1854. 359. Shepard, Rep. On Am. Met. Fol. 41. SJ. 2. 37. 1839. 385. W. 1860. S. 1860. \\ \hline
        16. & 1840 (1846) ? & - & Oktober & Concord, Hauptstadt von Merrimae-County. & New-Hampshire & 43$^\circ$ 12$^\prime$ N. & 71$^\circ$ 38$^\prime$ W. & P. 4. 1854. 376. S. 1860. \\ \hline
        17. & 1843 & 25. & März & Bishopville, Sumter-Distrikt (Hauptstadt: Sumterville), NNO. von Sumterville und 63 M. ONO. von Columbia. - Sp.-Gew.: 3,02-3,11. & South-Carolina & 34$^\circ$ 12$^\prime$ N. & 80$^\circ$ 12$^\prime$ W. & P. 4. 1854. 367. W. 1860. S. 1860. \\ \hline
        18. & 1846 (1847) ? & - & Juli & Richland-Distrikt, 20 M. O. von dessen Hauptstadt Columbia. - Sp.-Gew.: 2,32. & South-Carolina & 34$^\circ$ 0$^\prime$ N. & 80$^\circ$ 45$^\prime$ W. & P. 4. 1854. 376. S. 1860. \\ \hline
        19. & 1847 & 25. & Februar & Hartford, Linn-County, 9 M. S. von dessen Hauptstadt Marion (23 M. N. von Jowa-City). Sp.-Gew.: 3,58. & Jowa & 41$^\circ$ 58$^\prime$ N. & 91$^\circ$ 57$^\prime$ W. & P. 4. 1854. 378. SJ. 2. 4. 1847. 429. W. 1860. S. 1860. \\ \hline
        20. & 1847 & 8. & Dezember & Foresthill ($^\wedge$$^\wedge$$^\wedge$). & Arkansas & - & - & P. 4. 1854. 380. \\ \hline
        21. & 1848 & 20. & Mai & Castine, Hauptstadt von Hancock-County, 48 M. O. von Augusta. Sp.-Gew.: 3,456. & Maine & 44$^\circ$ 29$^\prime$ N. & 68$^\circ$ 57$^\prime$ W. & P. 4. 1854. 381. S. 1860. \\ \hline
        22. & 1849 & 31. & Oktober & Cabarras-County, 18 bis 20 M. von dessen Haupstadt Concord (102 M. WSW. von Raleigh) und 22 M. O. von Charlotte (Haupstadt von Mecklenburg-County, SW. von Concord). - Sp.-Gew.: 3,60-3,66. & North-Carolina & 35$^\circ$ 15$^\prime$ N. & 80$^\circ$ 28$^\prime$ W. & P. 4. 1854. 381. Shepard, Account of 3 new Am. Met. Fol. 4.* W. 1860. S. 1860. \\ \hline
        23. & 1855 & 5. & August & Petersburg, Lincoln-County (Haupstadt: Fayetteville), NNW. von Fayetteville), NNW. von Fayetteville und 56 M. SSO. von Nashville. - Sp.-Gew.: 3,20. & Tennessee & 35$^\circ$ 20$^\prime$ N. & 86$^\circ$ 50$^\prime$ W. & P. 103. 1858. 434. W. 1860. S. 1860. \\ \hline
        24. & 1859 & 26. & März & Harrison-County (Hauptstadt: Cynthiana, 39 M. ONO. von Frankfort). & Kentucky & Zwischen 38$^\circ$ 16$^\prime$ N. und 38$^\circ$ 38$^\prime$ N. & Zwischen 84$^\circ$ 15$^\prime$ W. und 84$^\circ$ 45$^\prime$ W. & S. 1860. \\ \hline
        25. & 1859 & 11. & August & Bethlehem, Albany-County, 5 M. S. von Albany. & New-York & 42$^\circ$ 27$^\prime$ N. & 74$^\circ$ 0$^\prime$ W. & S. 1860. \\ \hline
        26. & 1860 & 1. & Mai & New-Concord, Muskingum-County (Hauptstadt: Zanesville), NO. von Zanesville und 65 M. ONO. von Columbus; und Claysville, SO. von Cambridge, der Hauptstadt von Guernsey-County, u. 68 M. N. v. Columbus. & Ohio & Ungefähr 40$^\circ$ 10$^\prime$ N. & Ungefähr 81$^\circ$ 30$^\prime$ W. & WA. 41. 1860. 572. S. 1860. \\ \hline
          &   &   &   & Meteor-Eisenmassen, deren Fallzeit unbekannt. &   &   &   &   \\ \hline
        27. & - & - & - & White-Mountains, O. von Franconia, Grafton-County (Haupstadt: Haverhill), NO. von Haverhill und 68 M. N. von Concord. - 20 Pfund Beschrieben 1846. & New-Hampshire & Zwischen 44$^\circ$ 4$^\prime$ N. und 44$^\circ$ 15$^\prime$ N. & Zwischen 71$^\circ$ 10$^\prime$ W. und 71$^\circ$ 40$^\prime$ W. & P. 4. 1854. 404. \\ \hline
        28. & - & - & - & Burlington, Otsego-County (Hauptstadt: Cooperstown), W. von Cooperstown und 68 M. W. von Albany. - 150 Pfund Gefunden 1819. - Sp.-Gew.: 7,501-7,728. & New-York & 42$^\circ$ 42$^\prime$ N. & 75$^\circ$ 25$^\prime$ W. & P. 4. 1854. 402. W. 1860. S. 1860. \\ \hline
        29. & - & - & - & Cambria, Niagara-County (Hauptstadt: Lockport), W. von Lockport und 248 M. W. von Albany. - 36 Pfund Gefunden 1818. - Sp.-Gew.: 7,32-7,525. & New-York & 43$^\circ$ 9$^\prime$ N. & 79$^\circ$ 7$^\prime$ W. & P. 67. 1846. 124. W. 1860. S. 1860. \\ \hline
        30. & - & - & - & Otsego-County (Haupstadt: Cooperstown, 58 M. W. von Albany). - 276 Gran. Gefunden 1845. & New-York & Zwischen 42$^\circ$ 20$^\prime$ N. und 42$^\circ$ 55$^\prime$ N. & Zwischen 74$^\circ$ 55$^\prime$ W. und 75$^\circ$ 40$^\prime$ W. & P. 4. 1854. 410. S. 1860. \\ \hline
        31. & - & - & - & Seriba am Ontario-See, Oswego-County (Hauptstadt: Oswego), 4 M. NO. von Oswego, 152 M. und NW. von Albany. - 8 Pfund Gefunden 1834. - Sp.-Gew.: 7,50. & New-York & 43$^\circ$ 27$^\prime$ N. & 76$^\circ$ 43$^\prime$ W. & P. 4. 1854. 399. \\ \hline
        32. & - & - & - & Bei Seneca-Falls (Seneca-County, Hauptstadt: Waterloo), 44 M. OSO. von Rochester und 162 M. WNW. von Albany; auf der zu Cayuga-County gehörigen Seite des Seneca-River. 8 bis 10 Pfund Gefunden 1850. - Sp.-Gew.: 7,337. & New-York & Ungefähr 42$^\circ$ 55$^\prime$ N. & Ungefähr 77$^\circ$ 0$^\prime$ W. & SJ. 2. 14. 1852. Fol. 439. SJ. 2. 15. 1853. Fol. 363. W. 1860. S. 1860. \\ \hline
        33. & - & - & - & Bedford-County (Hauptstadt: Bedford, 94 M. WSW. von Harrisburg). - Einige Unzen. Gefunden 1828. - Sp.-Gew.: 6,915. & Pennsylvanien & Zwischen 39$^\circ$ 40$^\prime$ N. und 40$^\circ$ 20$^\prime$ N. & Zwischen 78$^\circ$ 15$^\prime$ W. und 78$^\circ$ 55$^\prime$ W. & P. 4. 1854. 409. \\ \hline
        34. & - & - & - & Pittsburg, Hauptstadt von Alleghany-County. Gefunden 1850. - Sp.-Gew.: 7,380. & Pennsylvanien & 40$^\circ$ 28$^\prime$ N. & 80$^\circ$ 8$^\prime$ W. & S. 1860. SJ. 2. 11. 1851. 40. \\ \hline
        35. & - & - & - & 20 engl. M. von Fort-Pierre (44$^\circ$ 21$^\prime$ N. und 100$^\circ$ 15$^\prime$ W.), zwischen Council-Bluffs und Fort-Union, am Missouri. - 35 Pfund Gefunden 1856. & Nebraska & - & - & WA. 41. 1860. Fol. 571. S. 1860. \\ \hline
        36. & - & - & - & Grayson-County (Hauptstadt: Greenville, WSW. von Richmond). & Virginia & Zwischen 36$^\circ$ 32$^\prime$ N. und 36$^\circ$ 48$^\prime$ N. & Zwischen 80$^\circ$ 50$^\prime$ W. und 82$^\circ$ 0$^\prime$ W. & P. 4. 1854. 404. \\ \hline
        37. & - & - & - & Roanoke-County (Hauptstadt: Salem, 145 M. W. von Richmond). - & Virginia & Zwischen 37$^\circ$ 10$^\prime$ N. und 37$^\circ$ 26$^\prime$ N. & Zwischen 79$^\circ$ 55$^\prime$ W. und 80$^\circ$ 25$^\prime$ W. & P. 4. 1854. 404. \\ \hline
        38. & - & - & - & Marshall-County (Hauptstadt: Benton, 212 M. WSW. von Frankfort). Gefunden 1856. & Kentucky & Zwischen 36$^\circ$ 48$^\prime$ N. und 37$^\circ$ 5$^\prime$ N. & Zwischen 88$^\circ$ 24$^\prime$ W. und 88$^\circ$ 47$^\prime$ W. & S. 1860. \\ \hline
        39. & - & - & - & Nelson-County (Hauptstadt: Bardstown, 42 M. SW. von Frankfort). - Gefunden 1856. & Kentucky & Zwischen 37$^\circ$ 35$^\prime$ N. und 38$^\circ$ 0$^\prime$ N. & Zwischen 85$^\circ$ 14$^\prime$ W. und 86$^\circ$ 0$^\prime$ W. & S. 1860. \\ \hline
        40. & - & - & - & Salt-River. Gefunden 1850. Sp.-Gew.: 6,835. & Kentucky & Zwischen 37$^\circ$ 50$^\prime$ N. und 38$^\circ$ 5$^\prime$ N. & Zwischen 85$^\circ$ 5$^\prime$ W. und 86$^\circ$ 10$^\prime$ W. & W. 1860. S. 1860. SJ. 2. 11. 1851. 40. \\ \hline
        41. & - & - & - & Smithland, Livingston-County (Hauptstadt: Salem), SW. von Salem und 205 M. WSW. von Frankfort. Gefunden 1840 oder 1841. - Sp.-Gew.: 7,56.? & Kentucky & 37$^\circ$ 10$^\prime$ N. & 88$^\circ$ 40$^\prime$ W. & P. 4. 1854. 401. \\ \hline
        42. & - & - & - & Forsyth am White-River, Hauptstadt von Taney-County, 142 M. SSW. von Jeffersoncity. Gefunden 1854. & Missouri & 36$^\circ$ 42$^\prime$ N. & 93$^\circ$ 18$^\prime$ W. & S. 1860. \\ \hline
        43. & - & - & - & Ashe-County (Hauptstadt: Jefferson, 158 M. WNW. von Raleigh). - & North-Carolina & Zwischen 36$^\circ$ 10$^\prime$ N. und 36$^\circ$ 32$^\prime$ N. & Zwischen 80$^\circ$ 56$^\prime$ W. und 81$^\circ$ 54$^\prime$ W. & SJ. 43. 1842. Fol. 169. \\ \hline
        44. & - & - & - & Bairds Plantation, nahe bei French-Broad-River, 6 M. N. von Asheville (Ashville), Hauptstadt von Buncombe-County, 218 M. W. von Raleigh. - 30 Pfund Gefunden 1839. - Sp.-Gew.: 6,5-8,0. & North-Carolina & 35$^\circ$ 38$^\prime$ N. & 82$^\circ$ 38$^\prime$ W. & P. 4. 1854. 403. Shepard, Rep. On Am. Met. Fol. 24. W. 1860. S. 1860. \\ \hline
        45. & - & - & - & Black-Mountain, am Ursprung des Swannanoah-River, 15 M. NO. von Asheville, der Hauptstadt von Buncombe-County. - 22 Unzen. Gefunden 1835. Sp.-Gew.: 7,261-7,5. & North-Carolina & 35$^\circ$ 45$^\prime$ N. & 82$^\circ$ 25$^\prime$ W. & P. 4. 1854. 407. S. 1860. \\ \hline
        46. & - & - & - & Guilford-County (Hauptstadt: Greensborough, 75 M. WNW. von Raleigh). 28 Pfund Gefunden 1828. - Sp.-Gew.: 7,67. & North-Carolina & Zwischen 35$^\circ$ 54$^\prime$ N. und 36$^\circ$ 14$^\prime$ N. & Zwischen 79$^\circ$ 40$^\prime$ W. und 80$^\circ$ 10$^\prime$ W. & P. 4. 1854. 403. W. 1860. S. 1860. \\ \hline
        47. & - & - & - & Haywood-County (Hauptstadt: Waynesville, 248 M. W. von Raleigh). - Gefunden zwischen 1850 und 1854. - Sp.-Gew.: 7,419. & North-Carolina & Zwischen 35$^\circ$ 8$^\prime$ N. und 35$^\circ$ 45$^\prime$ N. & Zwischen 82$^\circ$ 50$^\prime$ W. und 83$^\circ$ 25$^\prime$ W. & SJ. 2. 17. 1854. Fol. 327. S. 1860. \\ \hline
        48. & - & - & - & Pisgah-Mountain, Hommoney-(oder Hammoney-)Creek, 10 M. W. von Asheville (Hauptstadt von Buncombe-County) und 232 M. W. von Raleigh. - 27 Pfund Gefunden 1845. - Sp.-Gew.: 7,32. & North-Carolina & Ungefähr 35$^\circ$ 30$^\prime$ N. & Ungefähr 82$^\circ$ 17$^\prime$ W. & P. 4. 1854. 405. Shepard, Rep. On Am. Met. Fol. 25. \\ \hline
        49. & - & - & - & Jewell-Hill ($^\wedge$$^\wedge$$^\wedge$), Madison-County (NW. von Asheville). - Gefunden 1856. & North-Carolina & Zwischen 35$^\circ$ 40$^\prime$ N. und 36$^\circ$ 0$^\prime$ N. & Zwischen 82$^\circ$ 40$^\prime$ W. und 83$^\circ$ 10$^\prime$ W. & S. 1860. \\ \hline
        50. & - & - & - & Randolph-County (Hauptstadt: Ashboro, 69 M. W. von Raleigh). - 2 Pfund Gefunden 1822. - Sp.-Gew.: 7,618. & North-Carolina & Zwischen 35$^\circ$ 30$^\prime$ N. und 35$^\circ$ 55$^\prime$ N. & Zwischen 79$^\circ$ 42$^\prime$ W. und 80$^\circ$ 10$^\prime$ W. & P. 4. 1854. 409. \\ \hline
        51. & - & - & - & Babbs-Mill, 10 M. N. von Greenville (222 M. O. von Nashville), Hauptstadt von Greene-County, 13 Pfund und 6 Pfund Gefunden 1842. - Sp.-Gew.: 7,548-7,839. & Tennessee & 36$^\circ$ 9$^\prime$ N. & 83$^\circ$ 0$^\prime$ W. & P. 4. 1854. 400. W. 1860. S. 1860. Clark, Fol. 65. \\ \hline
        52. & - & - & - & Campbell-County (Hauptstadt: Jacksboro, 148 M. O. von Nashville). - 4 Unzen. Gefunden 1856. - Sp.-Gew.: 7,05. & Tennessee & Zwischen 36$^\circ$ 10$^\prime$ N. und 36$^\circ$ 30$^\prime$ N. & Zwischen 84$^\circ$ 0$^\prime$ W. und 84$^\circ$ 50$^\prime$ W. & B. 131. S. 1860. \\ \hline
        53. & - & - & - & Carthago, Hauptstadt von Smith-County, 46 M. O. von Nashville. 280 Pfund Gefunden 1846. - Sp.-Gew.: 7,82? & Tennessee & 36$^\circ$ 17$^\prime$ N. & 86$^\circ$ 12$^\prime$ W. & P. 4. 1854. 404. W. 1860. S. 1860. \\ \hline
        54. & - & - & - & Cosby-Creek, Cocke-County (Hauptstadt: Newport, 204 M. O. von Nashville), S. von Newport. 20 Zentner Auch Sevier-Eisen gennant. Gefunden 1840. - Sp.-Gew.: 6,22-7,26. & Tennessee & Zwischen 35$^\circ$ 40$^\prime$ N. und 35$^\circ$ 50$^\prime$ N. & Ungefähr 83$^\circ$ 25$^\prime$ W. & P. 4. 1854. 408. P. 107. 1859. 162. W. 1860. S. 1860. \\ \hline
        55. & - & - & - & DeKalb-County (Hauptstadt: Smithville, 53 M. OSO. von Nashville). 36 Pfund Gefunden 1845. & Tennessee & Zwischen 35$^\circ$ 53$^\prime$ N. und 36$^\circ$ 8$^\prime$ N. & Zwischen 85$^\circ$ 45$^\prime$ W. und 86$^\circ$ 20$^\prime$ W. & P. 4. 1854. 403. S. 1860. \\ \hline
        56. & - & - & - & Jackson County (Hauptstadt: Gainesboro, 61 M. ONO. von Nashville). - Beschrieben 1846. & Tennessee & Zwischen 36$^\circ$ 15$^\prime$ N. und 36$^\circ$ 35$^\prime$ N. & Zwischen 85$^\circ$ 45$^\prime$ W. und 86$^\circ$ 5$^\prime$ W. & P. 4. 1854. 404. \\ \hline
        57. & - & - & - & Long-Creek, Jefferson-County (Hauptstadt: Dandridge, 35$^\circ$ 57$^\prime$ N., 83$^\circ$ 37$^\prime$ W., und 192 M. O. von Nashville). - 2 ½ Pfund Sp.-Gew.: 7,43. & Tennessee & - & - & B. 133. \\ \hline
        58. & - & - & - & Murfreesboro, Hauptstadt von Rutherford-County, 28 M. SO. von Nashville. - & Tennessee & 35$^\circ$ 50$^\prime$ N. & 86$^\circ$ 38$^\prime$ W. & P. 4. 1854. 409. \\ \hline
        59. & - & - & - & Tazewell, Hauptstadt von Claiborne-County, 183 M. O. von Nashville. - 55 Pfund Gefunden 1853 oder 1854. - Sp.-Gew.: 7,30-7,91. & Tennessee & 36$^\circ$ 25$^\prime$ N. & 83$^\circ$ 38$^\prime$ W. & B. 137. W. 1860. S. 1860. \\ \hline
        60. & - & - & - & Chesterville (Chester), Hauptstadt von Chester-Distrikt, 59 M. NNW. von Columbia. Gefunden 1847. & South-Carolina & 36$^\circ$ 40$^\prime$ N. & 81$^\circ$ 7$^\prime$ W. & W. 1860. S. 1860. \\ \hline
        61. & - & - & - & Am Columbia-Fluss ($^\wedge$$^\wedge$$^\wedge$). - Gefunden ungefähr 1850; soll jedoch nach neuerer Angabe einerlei mit Nr. 18, Richland-Distrikt, sein. & South-Carolina & - & - & P. 4. 1854. 409. \\ \hline
        62. & - & - & - & Ruffs-Mountain, Newberry-Distrikt (Hauptstadt: Newberry, 47 M. WNW. von Columbia). - 117 Pfund Gefunden 1841. - Sp.-Gew.: 7,01-7,10. (außen: 5,97-6,80.) & South-Carolina & Zwischen 34$^\circ$ 3$^\prime$ N. und 34$^\circ$ 28$^\prime$ N. & Zwischen 81$^\circ$ 20$^\prime$ W. und 82$^\circ$ 0$^\prime$ W. & P. 4. 1854. 405. W. 1860. S. 1860. \\ \hline
        63. & - & - & - & Putnam-County (Hauptstadt: Eatonton, 24 M. NNW. von Milledgeville). 72 Pfund Gefunden 1839. - Sp.-Gew.: 7,69. & Georgia & Zwischen 33$^\circ$ 10$^\prime$ N. und 33$^\circ$ 25$^\prime$ N. & Zwischen 83$^\circ$ 22$^\prime$ W. und 83$^\circ$ 47$^\prime$ W. & B. 131. W. 1860. S. 1860. \\ \hline
        64. & - & - & - & Union-County (Hauptstadt: Blairsville, 118 M. NNW. von Milledgeville). - 15 Pfund Gefunden 1853. - Sp.-Gew.: 7,07. & Georgia & Zwischen 34$^\circ$ 37$^\prime$ N. und 35$^\circ$ 0$^\prime$ N. & Zwischen 83$^\circ$ 54$^\prime$ W. und 84$^\circ$ 30$^\prime$ W. & B. 135. W. 1860. S. 1860. \\ \hline
        65. & - & - & - & Claiborne, Hauptstadt von Monroe-County (nicht Clarke-County), 90 M. SW. von Montgomery. 40 Pfund Gefunden 1834. - Sp.-Gew.: 5,75-6,82. & Alabama & 31$^\circ$ 32$^\prime$ N. & 87$^\circ$ 45$^\prime$ W. & P. 1840. Sup. 371. W. 1860. S. 1860. \\ \hline
        66. & - & - & - & Walker-County (Hauptstadt: Jasper, 116 M. NNW. von Montgomery). - 165 Pfund Gefunden 1832. - Sp.-Gew.: 7,265. & Alabama & Zwischen 33$^\circ$ 30$^\prime$ N. und 34$^\circ$ 0$^\prime$ N. & Zwischen 87$^\circ$ 5$^\prime$ W. und 87$^\circ$ 50$^\prime$ W. & P. 4. 1854. 399. \\ \hline
        67. & - & - & - & Oktibbeha-County (Hauptstadt: Starksville, 116 M. NO. von Jackson). - 5 ½ Unzen. Gefunden zwischen 1850 und 1854. - Sp.-Gew.: 6,854. & Mississippi & Zwischen 33$^\circ$ 15$^\prime$ N. und 33$^\circ$ 38$^\prime$ N. & Zwischen 88$^\circ$ 52$^\prime$ W. und 89$^\circ$ 16$^\prime$ W. & B. 130. S. 1860. \\ \hline
        68. & - & - & - & Am Red River, nahe dem Ursprung von Trinity-River, einige M. W. von den Cross-Timbers in Dallas-County (zwischen 32$^\circ$ 35$^\prime$ N., 96$^\circ$ 35$^\prime$ W., und 33$^\circ$ 0$^\prime$ N., 97$^\circ$ 0$^\prime$ W.), 100 M. Oberhalb Natchitochez, Provinz Copuila, welche in Louisiana Texas begranzt; am Fusse des Berges San-Saba, ungefähr 70 engl. M. NNO. von Rio-Grande oder Bravo und 170 engl. M. vom nächsten Ende des zu Texas gehörigen Rio Brasos (Brazos). - 1635 Pfund Gefunden 1808. - Sp.-Gew.: 7,40-7,82. & Texas & 32$^\circ$ 7$^\prime$ N. Oder nach Gehlers Phys. Worterbuch 32$^\circ$ 20$^\prime$ N. & 95$^\circ$ 10$^\prime$ W. Oder nach Gehlers Phys. Worterbuch 97$^\circ$ 0$^\prime$ W. & G. 68. 1821. 343. Clark, 59. W. 1860. S. 1860. \\ \hline
        69. & - & - & - & An der östlichen Seite des Rio-Brazos. - 320 Pfund Gefunden 1856. & Texas & Ungefähr 34$^\circ$ 0$^\prime$ N. & Ungefähr 100$^\circ$ 0$^\prime$ W. & WA. 41. 1860. 571. S. 1860. \\ \hline
        70. & - & - & - & Denton-County (Hauptstadt: Alton, 208 M. NNW. von Austin-City). Ursprünglich 40 Pfund Gefunden 1856. - Sp.-Gew.: 7,669. & Texas & Zwischen 32$^\circ$ 58$^\prime$ N. und 33$^\circ$ 25$^\prime$ N. & Zwischen 96$^\circ$ 55$^\prime$ W. und 97$^\circ$ 25$^\prime$ W. & WA. 41. 1860. 572. S. 1860. \\ \hline
        71. & - & - & - & Rogue-River-Mountains, nahe bei Port-Oxford (Hauptstadt von Umpqua-County und 160 M. SSW. von Salem), am großen Ocean. & Oregon & 42$^\circ$ 35$^\prime$ N. & Zwischen 123$^\circ$ 0$^\prime$ W. und 124$^\circ$ 0$^\prime$ W. & WA. 41. 1860. 572. \\ \hline
        72. & - & - & - & ? Sp.-Gew.: 8,13. & New-Mexico & - & - & SJ. 2. 17. 1854. 239. \\ \hline
        73. & - & - & - & Caryfort ($^\wedge$$^\wedge$$^\wedge$). - Sp.-Gew.: 7,38? & ? & - & - & P. 107. 1859. 162. \\ \hline
    \end{longtable}
\end{center}
\subsubsection{5. Staaten von Mexico und Mittel-Amerika.}
\begin{center}
    \footnotesize
    \begin{longtable}{|p{3mm}|p{4mm}|p{11mm}|p{7mm}|p{22mm}|p{14mm}|p{10mm}|p{10mm}|p{13mm}|}
    \hline
        1. & 2. & 2. & 2. & 3. & 3. & 4. & 5. & 6. \\ \hline
        1. & 1858 & Ungefähr 1. & August & Heredin (Eredia). - Sp.-Gew.: 3,70? & Costa-Rica & 8$^\circ$ 45$^\prime$ N. & 83$^\circ$ 25$^\prime$ W. & P. 107. 1859. 162. Harris 99.* \\ \hline
          &   &   &   & Meteor-Eisenmassen, deren Fallzeit unbekannt. &   &   &   &   \\ \hline
        2. & - & - & - & Cañada de Hierro (Eisen-Thal) in den Santa-Rita Bergen, und von da nach dem 30 M. N. gelegenen Tuczon gebracht. - 6 Zentner, 10 Zentner und 12 Zentner Gefunden zwischen 1850 und 1854. - Sp.-Gew.: 6,52-7,13. & Sonora & 32$^\circ$ 58$^\prime$ N. & 111$^\circ$ 10$^\prime$ W. & B. 147. SJ. 2. 13. 1852. 289. SJ. 2. 18. 1854.369. S. 1860. \\ \hline
        3. & - & - & - & Landgut Conception ($^\wedge$$^\wedge$$^\wedge$), 10 M. von Zatapa, SO. von Chihuahua (28$^\circ$ 36$^\prime$ N., 106$^\circ$ 12$^\prime$ W.). 40 Zentner Vielleicht gleichen Ursprungs mit dem Folgenden. & Chihuahua & - & - & B. 145. \\ \hline
        4. & - & - & - & Sierra Blanca ($^\wedge$$^\wedge$$^\wedge$), 3 M. von Villa nueva di Huaxuquilla (27$^\circ$ 15$^\prime$ N., 105$^\circ$ 4$^\prime$ W., und SSO. von Chihuahua); 12 M. von Valle di San-Bartolomo und 48 M. NNW. von Durango. - Eisenmassen von 20, 30 und mehr Zentner Gefunden 1784. & Chihuahua & - & - & G. 56. 1817. 383. P. 4. 1854. 412. Chladni 339. \\ \hline
        5. & - & - & - & Südwest-Ecke des Balson de Malpini (Bolson de Mapimi), auf der Strasse nach den Minen von Parral (Parras?). - 2 Tonnen schwer. & Chihuahua & Ungefähr 26$^\circ$ 15$^\prime$ N. & Ungefähr 105$^\circ$ 0$^\prime$ W. & B. 144. \\ \hline
        6. & - & - & - & San-Gregorio ($^\wedge$$^\wedge$$^\wedge$), ungefähr 70 M. S. von Chihuahua. - Eine kleine Eisenmasse. & Chihuahua & Ungefähr 27$^\circ$ 30$^\prime$ N. & Ungefähr 105$^\circ$ 0$^\prime$ W. & RPG. 40. \\ \hline
        7. & - & - & - & Im Staate Cohahuila von dem Fundorte nach dem 11 bis 12 M. Davon entfernten Saltillo (25$^\circ$ 30$^\prime$ N., 101$^\circ$ 5$^\prime$ W.), zwischen Durango und Matamoros, gebracht. - 252 Pfund - Sp.-Gew.: 7,81. & Cohahuila & - & - & B. 144. S. 1860. (?) \\ \hline
        8. & - & - & - & Durango. - 380 Zentner Gefunden 1811. - Sp.-Gew.: 7,88. & Durango & 24$^\circ$ 12$^\prime$ N. & 103$^\circ$ 56$^\prime$ W. & P. 4. 1854. 411. W. 1860. S. 1860. \\ \hline
        9. & - & - & - & Alamos de Catorze, 50 M. O. von Durango. - Mehrere Eisenmassen. & San-Luis-Potosi & 23$^\circ$ 45$^\prime$ N. & 100$^\circ$ 16$^\prime$ W. & B. 144. \\ \hline
        10. & - & - & - & Santa-Maria de los Charcas, 10 M. SSW. von Catorze. - 8 bis 9 Zentner Gefunden 1792 und angeblich schon fruher von dem 7 M. von Charcas entfernten Meierhof San-José del Sitio dahin gebracht. & San-Luis-Potosi & 23$^\circ$ 12$^\prime$ N. & 100$^\circ$ 28$^\prime$ W. & G. 50. 1815. 270. \\ \hline
        11. & - & - & - & Zacatecas. - 20 Zentner Gefunden 1792, aber angeblich schon fruher aus dem N. Dahin gebracht. - Sp.-Gew.: 7,2-7,625. & Zacatecas & 22$^\circ$ 51$^\prime$ N. & 102$^\circ$ 0$^\prime$ W. & G. 50. 1815. 269. W. 1860. S. 1860. \\ \hline
        12. & - & - & - & Xiquipilco ($^\wedge$$^\wedge$$^\wedge$), in der Gerichtsbarkeit von Ixtlahuaca (19$^\circ$ 37$^\prime$ N., 99$^\circ$ 34$^\prime$ W.), 10 Leguas NNW. von Toluca und WNW. von Mexico; und Bata (Beta), eine Schlucht, ½ Stunde von Xiquipilco el nuevo ($^\wedge$$^\wedge$$^\wedge$) entfernt. - Eisenmassen von mehreren Zentner bis zu wenigen Unzen. Gefunden seit 1784. - Sp.-Gew.: 7,60-7,72. & Mexico & - & - & G. 56. 1817. 384. Chladni 339. B. 139. W. 1860. S. 1860. \\ \hline
        13. & - & - & - & Ocatitlan (Ocotitlan), N. von Ixtlahuaca. - 27 Pfund Sp.-Gew.: 6,50-7,67? & Mexico & 19$^\circ$ 45$^\prime$ N. & 99$^\circ$ 32$^\prime$ W. & P. 100. 1857. 250. P. 107. 1859. 162. \\ \hline
        14. & - & - & - & Tejupilco, WSW. von Toluca. - Sp.-Gew.: 6,50-7,67? & Mexico & 18$^\circ$ 56$^\prime$ N. & 100$^\circ$ 6$^\prime$ W. & P. 100. 1857. 250. P. 107. 1859. 162. \\ \hline
        15. & - & - & - & Manji (Hacienda Mañi ($^\wedge$$^\wedge$$^\wedge$)) im Thal von Toluca. - Sp.-Gew.: 6,50-7,67? & Mexico & - & - & P. 100. 1857. 250. P. 107. 1859. 162. \\ \hline
        16. & - & - & - & In der Mistecà ($^\wedge$$^\wedge$$^\wedge$) im Staat Oaxaca (Oaxaca: 16$^\circ$ 45$^\prime$ N., 97$^\circ$ 4$^\prime$ W.). - Gefunden 1843. - Sp.-Gew.: 7,2-7,62. & Oaxaca & - & - & P. 100. 1857. 246. W. 1860. S. 1860. \\ \hline
    \end{longtable}
\end{center}
\subsubsection{6. Süd-Amerika.}
\begin{center}
    \footnotesize
    \begin{longtable}{|p{3mm}|p{6mm}|p{5mm}|p{12mm}|p{22mm}|p{14mm}|p{10mm}|p{10mm}|p{13mm}|}
        \hline
        1. & 2. & 2. & 2. & 3. & 3. & 4. & 5. & 6. \\ \hline
        1. & 1810 & 20. (21.) & April & Hügel von Tocavita, 1 M. von Santa-Rosa, das ungefähr 20 franz. M. NO. von Santa-Fé de Bogotá auf dem halben Wege von dieser Stadt nach Pamplona. Eisen. 15 Zentner - Sp.-Gew.: 7,30. & Neu-Granada & 5$^\circ$ 40$^\prime$ N. & 73$^\circ$ 20$^\prime$ W. & P. 4. 1854. 412. A. 4. 196. B. 117 u. 130. WA. 8. 1852. 496. \\ \hline
        2. & 1836 & 11. & November & Macao, am Fluss Assu (Açu oder Amargoro), nicht weit von dessen Ausfluss in das Meer, W. von Anaçu und fast N. von Villa nova da Prinzeza und von Açu; Prov.: Rio Grande do Norte. - Sp.-Gew.: 3,72-3,74. & Brasilien & 4$^\circ$ 55$^\prime$ S. & 37$^\circ$ 10$^\prime$ W. & P. 42. 1837. 592. W. 1860. S. 1860. \\ \hline
        3. & 1844 & - & Januar & Caritas-Paso am Fluss Mocorita, nahe an der Grenze der Provinz Entre-Rios auf der Ostseite des Parana, S. von Corrientes. Eisen. & Corrientes (Rio de la Plata Staaten) & 30$^\circ$ 10$^\prime$ S. & 58$^\circ$ 30$^\prime$ W. & B. 120. WA. 40. 1860. 528. \\ \hline
          &   &   &   & Meteor-Eisenmassen, deren Fallzeit unbekannt. &   &   &   &   \\ \hline
        4. & - & - & - & Rasgata ($^\wedge$$^\wedge$$^\wedge$), bei den Salinen von Zipaquira (4$^\circ$ 50$^\prime$ N., 74$^\circ$ 10$^\prime$ W.), NNO. von Santa-Fé de Bogotá. - 45 Pfund Und 84 Pfund Gefunden 1824. - Sp.-Gew.: 7,33-7,77. & Neu-Granada & - & - & P. 4. 1854. 412. A. 4. 206. B. 117 u. 130. WA. 8. 1852. 496. W. 1860. S. 1860. \\ \hline
        5. & - & - & - & Wüste Tarapaca ($^\wedge$$^\wedge$$^\wedge$), 80 engl. M. NO. von Talcahuaxa ($^\wedge$$^\wedge$$^\wedge$) u. 46 engl. M. von Hemalga ($^\wedge$$^\wedge$$^\wedge$). - 17 Pfund Gefunden 1840. - Sp.-Gew.: 6,50. & Chili (Peru?) & 19$^\circ$ 57$^\prime$ S. ? oder 37$^\circ$ 0$^\prime$ S. ? & 69$^\circ$ 40$^\prime$ W. ? oder 73$^\circ$ 0$^\prime$ W. ? & P. 96. 1855. 176. SJ. 44. 1843. Fol. 1. W. 1860. S. 1860. \\ \hline
        6. & - & - & - & Potosi. Beschrieben 1839. - Sp.-Gew.: 7,736. & Bolivia & 19$^\circ$ 40$^\prime$ S. & 67$^\circ$ 40$^\prime$ W. & P. 47. 1839. 470. \\ \hline
        7. & - & - & - & San Pedro (San-Pedro Atacama), an dem nördlichen Ende des Sees Salina de Atacama in der Wüste Atacama, 20 Leguas O. von Cobija. - Nahe an 3000 Stückchen ohne die größeren Stücke von 120 bis 150 Pfund, die schon früher fortgebracht worden. Gefunden 1827. - Sp.-Gew.: 6,687-7,66. & Bolivia & 22$^\circ$ 25$^\prime$ S. & 69$^\circ$ 2$^\prime$ W. & P. 14. 1828. 469. B. 105. W. 1860. S. 1860. \\ \hline
        8. & - & - & - & Nahe am Fluss Vermejo, Prov. Grand-Chaco-Gualamba, 15 M. von Otumpa ($^\wedge$$^\wedge$$^\wedge$) in Tucuman. 300 Zentner Gefunden 1788. - Sp.-Gew.: 7,54-7,65. & San Jago del Estero (Rio de la Plata Staaten) & Ungefähr 25$^\circ$ 0$^\prime$ S. bis 26$^\circ$ 0$^\prime$ S. (27-28°?) & Ungefähr 60$^\circ$ 0$^\prime$ W. bis 62$^\circ$ 0$^\prime$ W. & G. 50. 1815. 266. W. 1860. S. 1860. \\ \hline
        9. & - & - & - & Am Bache Bemdegó (Bendegó), der in den Rio San-Francisco fällt, 10 Leguas N. von Monte-Santo und 50 Leguas von Bahia; Capitanie Bahia. - 140 bis 170 Zentner Gefunden 1784. Auch Eisen von Sergipe oder Wollaston-Eisen genannt. - Sp.-Gew.: 7,48-7,88. & Brasilien & 10$^\circ$ 20$^\prime$ S. & 40$^\circ$ 10$^\prime$ W. & G. 68. 1821. 343. SJ. 2. 15. 1853. 12. W. 1860. S. 1860. \\ \hline
    \end{longtable}
\end{center}
\clearpage
\section{Zeitfolge sämtlicher, sowohl zuverlässiger als zweifelhafter Meteorstein- und Meteoreisen-Fälle.}
\begin{enumerate}
    \item Ordnungsnummer der Zeitfolge.

    \item Ortsnummer auf den betreffenden Karten 1., 2. u. 3.

    \item Fallzeit.

    \item Fundort.

    \item Geographische Lage; die Längengrade nach Greenwich.

    \item Belege.

    \item Größere oder geringere Beglaubigung der einzelnen Fälle.
\end{enumerate}
\paragraph{}
Die mit größerer Schrift gedruckten Zeiten bedeuten die mehr oder weniger für zuverlässig zu erachtenden und auf den Karten 1., 2. und 3. geographisch verzeichneten Meteorstein- und Meteoreisen-Fälle; die mit kleinerer Schrift gedruckten dagegen die nur mutmaßlichen und mehr oder weniger zweifelhaften, auf den Karten nicht verzeichneten Fälle. In Betreff der Ersteren sind alle genaueren Angaben über die geographische Lage, das spezifische Gewicht, so wie endlich alle diejenigen Meteorsteine und Meteor-Eisenmassen, deren Fallzeit unbekannt ist, aus den zu den Karten gehörigen Verzeichnissen zu ersehen.

In den Chinesischen Aufzeichnungen ist häufig von einem Niederfallen von „Sternen“ die Rede, ohne dass dabei irgend eines Auffindens wirklicher Steine Erwähnung geschähe. Chladni sagt hierüber in seinem Werke über Feuermeteore u. s. w. Fol. 189 und 190, dass die Chinesen in späteren Zeiten wahrscheinlich ebenso wenig wie die Abendländer an ein Herabfallen von Steinen geglaubt hätten, und dieses dürfte denn auch wohl allerdings die natürlichste Ursache sein, weshalb viele Jahrhunderte hindurch zwar von vielen, selbst unter donnerähnlichem Getöse herabgefallenen „Sternen“ oder „Sternschnuppen,“ aber von keinem einzigen wirklichen „Steinfall“ die Rede ist; unterdessen doch ein so plötzliches Aufhören dieser Letzteren in einem so weitausgedehnten Reiche kaum anzunehmen sein durfte. Man fand keine Meteorsteine, weil man nicht an dieselben glaubte und daher auch nicht nach denselben suchte. Aus diesem Grunde sind denn auch in dem gegenwärtigen Verzeichnis alle diejenigen Ereignisse, wo von einem wirklichen Herabfallen und nicht bloß von einem Erscheinen und Wiederverlöschen solcher Sterne oder Sternschnuppen berichtet wird, der Vollständigkeit wegen mit unter die Zahl der zweifelhaften Meteorsteinfälle aufgenommen. Denn wenn auf der einen Seite auch wohl anzunehmen ist, dass unter diesen fallenden Sternen, diesen Sternregen, namentlich wenn das Ereignis bei Nacht stattfand, häufig nur unsere gewöhnlichen Sternschnuppen in der gegenwärtigen Bedeutung des Wortes zu verstehen sein dürften: so geht doch auf der anderen Seite ebenso sehr aus der oft ganz ungewöhnlichen Größe dieser angeblichen, unter donnerndem Getöse herabfallenden Sterne und Sternschnuppen auf das Deutlichste hervor, dass ganz andere Erscheinungen darunter gemeint sind als diejenigen, die wir jetzt als Sternschnuppen zu bezeichnen pflegen. So heißt es z. B. von einer 616 n. Chr. herabgefallenen Sternschnuppe, dass sie Wagen zertrümmert und Menschen getötet habe: ein Beweis, dass wir hier gewiss weit eher berechtigt sind, an einen wirklichen Meteorsteinfall, als an eine bloße Sternschnuppe im jetzigen Sinn dies Wortes zu denken.

Ebenso bleibt es zweifelhaft, ob die von Lycosthenes zu verschiedenen Malen erwähnten „Erdregen,“ selbst wenn sie auf Wahrheit und nicht etwa auf bloßer Dichtung beruhen, vulkanischer Staub und Asche oder leicht zerreibliche wirkliche Meteorsteine gewesen. Das Ähnliche ist der Fall mit den nach den Aufzeichnungen von Plinius und Anderen von ihm erwähnten „Steinregen.“ Ob dieselben aus wirklichen Meteorsteinen oder vielleicht in vielen Fällen nur aus gewöhnlichem Hagel bestanden, muss dahingestellt bleiben. Nichts desto weniger dürfen diese Berichte und Tatsachen in einem auch die zweifelhaften Steinfälle umfassenden Verzeichnisse nicht übergangen werden.

Was endlich die sowohl in dem vorgehenden als in dem gegenwärtigen Verzeichnisse angegebenen Länge- und Breitegrade betrifft, so können dieselben in vielen Fällen - namentlich, wo es sich um ganz kleine und wenig bekannte Orte handelt - nur eine annähernde Gültigkeit besitzen; einmal wegen der Schwierigkeit, solche kleine Orte wirklich auf Karten verzeichnet zu finden; zum Andern aber auch aus dem Grunde, weil - namentlich bei außer-europäischen Ländern - die geographischen Lagen selbst der größeren Städte auf den verschiedenen zu dieser Arbeit benutzten Karten nicht immer vollkommen übereinstimmten. Im Allgemeinen sind jedoch die Lagen nach den Karten des großen Stieler'schen Atlasses zu Grunde gelegt.
\clearpage
\begin{center}
    \footnotesize
    \begin{longtable}{| p{3mm} | p{3mm} | p{15mm} | p{25mm} | p{16mm} | p{12mm} | p{13mm} | p{20mm} |}
    \hline
        1. & 2. & 3. & 4. & 4. & 5. & 6. & 7. \\ \hline
          &   & Vor Christus &   &   &   &   &   \\ \hline
        1. & - & 1984. - - & Sodom, Gomorra, Adama und Zeboim. & Palästina & Ungefähr 31$^\circ$ 0' N. 36$^\circ$ 0' O. & 1. Moses 19. v. 24 u. 25. 5. Moses 29. v. 23. & Zerstörung der 4 Städte durch Schwefel und Feuer, welche vom Himmel gefallen. \\ \hline
        2. & - & 1808. (1807.) - - & ? & China & - & Chou-king Fol. 76.* & In der Nacht fiel ein Stern wie Regen. \\ \hline
        3. & - & 1768. - - & ? & China & - & Quetelet 1841. 21.* & Man sah Sterne fallen. \\ \hline
        4. & 1. & Um 1479. - - & Cybelische Berge. & Insel Creta & Ungefähr 35$^\circ$ 15' N. 24$^\circ$ 50' O. & C. 174.* & Vom Himmel gefallener Stein der Cybele. \\ \hline
        5. & - & 14.. (1451.) - - & Von Beth-Horon (Beth-Eron), NNW. von Gibeon (N. von Jerusalem), bis Aseka (Azecha), SW. von Jerusalem und WSW. von Bethlehem. & Palästina & Von 31$^\circ$ 58' N. 35$^\circ$ 15' O. Bis 31$^\circ$ 38' N. 35$^\circ$ 0' O. & Josua 10. v. 10 und 11. & Hagel von Steinen; doch ungewiss, ob wirkliche Steine oder gewöhnlicher Hagel. \\ \hline
        6. & - & Um 1403. - - & Berg Ida. & Insel Creta & 35$^\circ$ 15' N. 24$^\circ$ 50' O. & C. 175. & Mutmaßlicher Niederfall von Eisen. \\ \hline
        7. & 2. & Um 1200. - - & ? & Griechenland & 38$^\circ$ 33' N. 22$^\circ$ 58' O. & C. 175. & Vom Himmel gefallener Stein, s. Z. Zu Orchomenos aufbewahrt. \\ \hline
        8. & - & 1149. - - & Po ($^\wedge$$^\wedge$$^\wedge$). & China & - & Chou-king Fol. 134. & Erd-Regen. \\ \hline
        9. & - & 1081. - - & Hien-Yang, Bezirk von Si-ngan-fou, Prov. Chen-si. & China & 34$^\circ$ 20' N. 108$^\circ$ 38' O. & Chou-king Fol. 185. EB. 33 u. 172. & Angeblicher Gold-Regen. \\ \hline
        10. & - & 707. (705.) (704.) - - & Rom. & Italien & 41$^\circ$ 54' N. 12$^\circ$ 26' O. & C. 175. Lycosthenes 57.* & Angebliches Herabfallen eines ehernen Schildes; vielleicht eine schildförmige Eisenmasse. \\ \hline
        11. & - & 687. 23. März & ? & China & - & AR. 1. 190. MS. 134. & Wahrend der Nacht fiel ein Stern (nach MS. Sterne) in Gestalt von Regen. \\ \hline
        12. & - & 686. - - & ? & China & - & Quetelet 1841. 21. & Die Meteore fielen wie ein Regen; vermutlich Sternschnuppen. \\ \hline
        13. & 1. & 654. (644.) (642.) - - & Albaner Gebirge (Mons Albanus). & Italien & 41$^\circ$ 40' N. 12$^\circ$ 40' O. & C. 176. & Steinregen, mit einem Hagelwetter verglichen. \\ \hline
        14. & 1. & 645. (644. Frühjahr) 24. Dezember & Ehemaliges Königreich Song, jetzt in der Provinz Ho-nan. & China & ungefähr 34$^\circ$ 10' N. 112$^\circ$ 8' O. & MS. 135. AR. 1. 190. C. 176. & Sterne fielen als 5 Steine hernieder. \\ \hline
        15. & - & Um 538. - - & ? & ? & - & Chron. Magn. Schedelii Bl. 69. S. 2.* & In einem Hagel sind rechte harte Steine gefallen; vielleicht aber auch nur große Schlossen. \\ \hline
        16. & 3. & 476. (468, 465, 464, 462, 405 oder 403.) - - & Am Ziegen-Fluss (Aegos Potamos). & Thrakien & 40$^\circ$ 24' N. 26$^\circ$ 36' O. & C. 176. & 1 großer vom Himmel gefallener Stein, den Plinius noch gesehen. \\ \hline
        17. & 4. & 465. - - & Theben in Bootien. & Griechenland & 38$^\circ$ 17' N. 23$^\circ$ 17' O. & C. 178. & 1 unter Feuer und Getöse vom Himmel gefallener, als Mutter der Götter verehrter Stein. \\ \hline
        18. & - & 461. (459.) - - & Provinz Picenum (jetzt Mark Ancona). & Italien & Ungefähr 43$^\circ$ 0' N. 13$^\circ$ 30' O. & P. 4. 1854. 7. Lycosthenes 76. & Es regnete Steine; doch ungewiss, ob nicht bloßer Hagel. \\ \hline
        19. & - & Um 356. - - & ? & Italien & - & Chron. Magn. Schedelii Bl. 82. S. 2. & Es fielen Felsen von den Wolken und hagelte mit eingemengten Steinen. \\ \hline
        20. & - & 343. (341.) - - & Rom. & Italien & 41$^\circ$ 54' N. 12$^\circ$ 26' O. & P. 4. 1854. 7. Lycosthenes 89. & Es regnete Steine; vielleicht nur Hagel. \\ \hline
        21. & - & 334. (332.) - - & ? & ? & - & P. 4. 1854. 7. Lycosthenes 92. & Als Alexander den Göttern opferte, ließ ein Vogel seinen Klauen einen Stein entfallen. \\ \hline
        22. & - & 297. (295.) - - & ? & Italien & - & Lycosthenes Fol. 96. & Angeblicher Erdregen. \\ \hline
        23. & - & 216. (214.) - - & Provinz Picenum (jetzt Mark Ancona). & Italien & Ungefähr 43$^\circ$ 0' N. 13$^\circ$ 30' O. & P. 4. 1854. 7. Lycosthenes 114. & Es regnete Steine; doch ungewiss, ob nicht bloßer Hagel. \\ \hline
        24. & - & 216. (214.) - - & Auf dem Aventin, einem der 7 Hügel Roms, und gleichzeitig zu Aricia in Latium, 10 Rom M. SO. von Rom. & Italien & 41$^\circ$ 54' N. 12$^\circ$ 26' O. Und 41$^\circ$ 49' N. 12$^\circ$ 30' O. & P. 4. 1854. 7. Lycosthenes 116. & Desgleichen. \\ \hline
        25. & - & 215. (213.) - - & Lanuvium in Latium, SO. von Rom und S. von Aricia. & Italien & 41$^\circ$ 40' N. 12$^\circ$ 40' O. & Lycosthenes 116 u. 117. & Desgleichen. \\ \hline
        26. & - & 214. (212.) - - & Cales in Terra di Lavoro in Campanien, NW. von Capua. & Italien & 41$^\circ$ 13' N. 14$^\circ$ 6' O. & Lycosthenes 119. & Es regnete Kreide. \\ \hline
        27. & - & 211. (209.) - - & Albaner Gebirge (Mons Albanus). & Italien & 41$^\circ$ 40' N. 12$^\circ$ 40' O. & P. 4. 1854. 7. Lycosthenes 121. & Es regnete, angeblich wahrend zweier Tage, Steine; und zu Reate in Sabinien sah man einen großen Felsen am Himmel fliegen. \\ \hline
        28. & 2. & 211. - - & Tong-kien (Tong-kiun), Provinz Chan-toung. & China & 36$^\circ$ 32' N. 116$^\circ$ 10' O. & MS. 135. AR. 1. 190. C. 178. & 1 gefallener Stern verwandelte sich in einen Stein. \\ \hline
        29. & - & 210. (208.) - - & Eretum in Sabinien, NO. von Rom. & Italien & 42$^\circ$ 3' N. 12$^\circ$ 40' O. & Lycosthenes 123. & Es regnete Steine; doch ungewiss, ob nicht bloßer Hagel. \\ \hline
        30. & - & 207. (206.) (205.) - - & Veji in Etrurien, 10 Rom. M. N. von Rom. & Italien & 42$^\circ$ 0' N. 12$^\circ$ 25' O. & P. 4. 1854. 8. Lycosthenes 128. & Desgleichen. \\ \hline
        31. & - & 207. (205.) - - & Armilustrum, ein Waffenplatz in Rom. & Italien & 41$^\circ$ 54' N. 12$^\circ$ 26' O. & Lycosthenes 128. & Desgleichen. \\ \hline
        32. & 2. & 206. (205.) - - & ? & Italien (?) & - & C. 179. & Es fielen feurige Steine. \\ \hline
        33. & - & 205. (203.) - - & ? & Italien & - & P. 4. 1854. 8. Lycosthenes 129. & Es regnete häufig Steine; doch wahrscheinlich nur großer Hagel. \\ \hline
        34. & - & 202. (200.) - - & Cumae in Campanien, W. von Neapel. & Italien & 40$^\circ$ 52' N. 14$^\circ$ 0' O. & P. 4. 1854. 8. Lycosthenes 132. & Es regnete Steine; doch ungewiss, ob nicht bloßer Hagel. \\ \hline
        35. & - & 202. (200.) - - & Auf dem Palatium, einem der 7 Hügel Roms. & Italien & 41$^\circ$ 54' N. 12$^\circ$ 26' O. & P. 4. 1854. 8. Lycosthenes 133. & Desgleichen. \\ \hline
        36. & - & 193. (191.) - - & Im Gebiet von Adria (Hadria), in Venezia. & Italien & 45$^\circ$ 0' N. 12$^\circ$ 5' O. & P. 4. 1854. 8. Lycosthenes 141. & Desgleichen. \\ \hline
        37. & - & 193. (191.) - - & Rom. & Italien & 41$^\circ$ 54' N. 12$^\circ$ 26' O. & Lycosthenes 141. & Es regnete einige Mal Erde; doch wahrscheinlich in Folge eines vulkanischen Ausbruches. \\ \hline
        38. & 3. & 192. - - & Mian-tchou, bei Mien-tcheou, Prov. Sse-tchouen. & China & 31$^\circ$ 17' N. 104$^\circ$ 16' O. & MS. 135. AR. 1. 191. C. 179. & Es fiel ein Stein vom Himmel. \\ \hline
        39. & - & 192. (190.) - - & Aricia in Latium, 10 Rom. M. SO. von Rom. & Italien & 41$^\circ$ 49' N. 12$^\circ$ 30' O. & P. 4. 1854. 8. Lycosthenes 143. & Es regnete Steine; doch ungewiss, ob nicht bloßer Hagel. \\ \hline
        40. & - & 192. (190.) - - & Lanuvium in Latium, SO. von Rom und S. von Aricia. & Italien & 41$^\circ$ 40' N. 12$^\circ$ 40' O. & Lycosthenes 143. & Desgleichen. \\ \hline
        41. & - & 192. (190.) - - & Auf dem Aventin, einem der 7 Hügel Roms. & Italien & 41$^\circ$ 45' N. 12$^\circ$ 26' O. & P. 4. 1854. 8. Lycosthenes 143. & Desgleichen. \\ \hline
        42. & - & 191. (189.) - - & Amiternum in Sabinien, NO. von Rom. & Italien & 42$^\circ$ 15' N. 13$^\circ$ 40' O. & Lycosthenes 145. & Es regnete Erde. \\ \hline
        43. & - & 190. (188.) - - & Terracina in Latium, zwischen Rom u. Neapel. & Italien & 41$^\circ$ 16' N. 13$^\circ$ 12' O. & P. 4. 1854. 8. Lycosthenes 146. & Es regnete Steine; doch ungewiss, ob nicht bloßer Hagel. \\ \hline
        44. & - & 190. (188.) - - & Amiternum in Sabinien, NO. von Rom. & Italien & 42$^\circ$ 15' N. 13$^\circ$ 40' O. & P. 4. 1854. 8. Lycosthenes 146. & Desgleichen. \\ \hline
        45. & - & 189. (187.) - - & Tusculum, bei Rom. & Italien & 41$^\circ$ 48' N. 12$^\circ$ 40' O. & Lycosthenes 147. & Es regnete Erde. \\ \hline
        46. & - & 187. (185.) - - & Auf dem Aventin, einem der 7 Hügel Roms & Italien & 41$^\circ$ 54' N. 12$^\circ$ 26' O. & P. 4. 1854. 8. Lycosthenes 148. & Es regnete Steine; doch ungewiss, ob nicht bloßer Hagel. \\ \hline
        47. & 3. & 176. (174.) - - & In den Mars-See (Lacus Martis) bei Crustumerium in Etrurien. & Italien & Ungefähr 42$^\circ$ 0' N. 12$^\circ$ 25' O. & C. 179. & 1 ungeheurer, vom Himmel gefallener Stein. \\ \hline
        48. & - & 172. (170.) - - & Apud Rementem ($^\wedge$$^\wedge$$^\wedge$) im Vejentischen, N. von Rom. & Italien & Ungefähr 42$^\circ$ 0' N. 12$^\circ$ 25' O. & Lycosthenes 156 u. 157. & Es fielen Steine; doch wahrscheinlich nur Hagel. \\ \hline
        49. & - & 171. (169.) - - & Oxinus ($^\wedge$$^\wedge$$^\wedge$). & Italien & - & Lycosthenes 158. & Es regnete Erde. \\ \hline
        50. & - & 168. (166.) - - & Reate in Sabinien, NO. von Rom. & Italien & 42$^\circ$ 25' N. 12$^\circ$ 50' O. & Lycosthenes 159. & Es regnete Steine; doch ungewiss, ob nicht bloßer Hagel. \\ \hline
        51. & - & 166. (164.) - - & Anagnia in Latium, OSO. von Rom. & Italien & 41$^\circ$ 45' N. 13$^\circ$ 7' O. & Lycosthenes 161. & Es regnete Erde. \\ \hline
        52. & - & 165. (163.) - - & Provinz Campanien (Gegend von Neapel). & Italien & - & Lycosthenes 162. & Desgleichen. \\ \hline
        53. & - & 162. (160.) - - & Wahrscheinlich auf der Insel Cephalonien. & Jonische Inseln & 38$^\circ$ 15' N. 20$^\circ$ 40' O. & Lycosthenes 164. & Desgleichen. \\ \hline
        54. & - & 151. (149.) - - & Aricia in Latium, 10 Rom. M. SO. von Rom. & Italien & 41$^\circ$ 49' N. 12$^\circ$ 30' O. & P. 4. 1854. 8. Lycosthenes 167. & Es regnete Steine; doch ungewiss, ob nicht bloßer Hagel. \\ \hline
        55. & - & 133. (131.) - - & Ardea in Latium, SO. von Rom. & Italien & 41$^\circ$ 37' N. 12$^\circ$ 32' O. & Lycosthenes 174. & Es regnete Erde. \\ \hline
        56. & - & 124. (122.) - - & Arpi in Apulien. & Italien & 41$^\circ$ 24' N. 15$^\circ$ 37' O. & Lycosthenes 180. & Es regnete 3 Tage lang Steine; daher vermutlich bloss Hagel. \\ \hline
        57. & - & 106. (104.) - - & ? & Italien & - & Lycosthenes 187 u. 188. & Getöse ward in der Luft gehört, und man sah eine Keule vom Himmel fallen. \\ \hline
        58. & - & 102. (100.) - - & In Etrurien (Toskana). & Italien & - & Lycosthenes 192. & Es regnete Steine; doch ungewiss, ob nicht bloßer Hagel. \\ \hline
        59. & - & 98. (96.) - - & Rom. & Italien & 41$^\circ$ 54' N. 12$^\circ$ 26' O. & Lycosthenes 195. & Es regnete weiße Kreide. \\ \hline
        60. & - & 94. (92.) - - & Im Lande der Volsker, in Latium, SO. von Rom, in der Gegend von Terracina. & Italien & Ungefähr 41$^\circ$ 30' N. 12$^\circ$ 50' O. & P. 4. 1854. 8. Lycosthenes 199. & Es regnete Steine; doch ungewiss, ob nicht bloßer Hagel. \\ \hline
        61. & - & 94. (92.) - - & Im Lande der Vestiner, NO. von Rom, S. von der Prov. Picenum, am Adriatischen Meere. & Italien & Ungefähr 42$^\circ$ 30' N. 13$^\circ$ 50' O. & P. 4. 1854. 8. Lycosthenes 199. & Desgleichen. \\ \hline
        62. & - & 91. (89.) - - & Im Lande der Vestiner, NO. von Rom, S. von der Prov. Picenum, am Adriatischen Meere. & Italien & Ungefähr 42$^\circ$ 30' N. 13$^\circ$ 50' O. & Lycosthenes 203 u. 204. & Es regnete 7 Tage lang Steine und Muscheln; vielleicht in Folge eines Vulkan-Ausbruches auf der Insel Aenaria (Ischia). \\ \hline
        63. & 4. & 90. (89. 50. 48.) - - & Carissanum Castellum ($^\wedge$$^\wedge$$^\wedge$). & Italien & - & C. 179. Lycosthenes 215. & Vom Himmel gefallene gebräunte Steine. \\ \hline
        64. & 4. & 89. 9. März & Yong (Young), Bezirk Si-ngan-fou, Provinz Chen-si. & China & 34$^\circ$ 48' N. 108$^\circ$ 3' O. & MS. 135. AR. 1. 191. C. 179. & Unter starkem Getöse 2 von Himmel gefallene Steine. \\ \hline
        65. & - & 87. - - & Athen. & Griechenland & 37$^\circ$ 58' N. 23$^\circ$ 44' O. & P. 6. 1826. 21. & Sehr zweifelhafter Steinfall. \\ \hline
        66. & - & Zwischen 86 u. 81. & Im Lande Yen ($^\wedge$$^\wedge$$^\wedge$), im Norden der Provinz Petchi-li. & China & - & MS. 135. & Eine Sternschnuppen fiel auf den Palast von Wang-tsai. \\ \hline
        67. & - & 75. (73.) - - & Otryae ($^\wedge$$^\wedge$$^\wedge$) in Phrygien (wahrscheinlich einerlei mit Otryae oder Otroea in Bithynien oberhalb des Sees Ascania). & Klein-Asien & - & Lycosthenes 211. Pauly 5. 1027.* & Ein fassgrösser, feuriger, silberglänzender Körper fiel wahrend der Schlacht zwischen Lucullus und Mithridates zwischen die zwei streitenden Hecre. \\ \hline
        68. & 5. & 56. (54 oder 52.) - - & Provinz Lucanien, OSO. von Neapel. & Italien & Ungefähr 40$^\circ$ 10' N. 16$^\circ$ 0' O. & C. 180. & Vom Himmel gefallenes schwammiges Eisen. \\ \hline
        69. & - & 52. (51.) - - & ? & Italien & - & P. 6. 1826. 22. & Feuerkugel mit Stein- und Erdfall; vielleicht einerlei mit dem Vorstehenden? \\ \hline
        70. & - & 46. (45.) - - & Acilia (Acilla, Acolla, Acholla oder Achilla) bei Thapsus, S. von Carthago. & Nord-Afrika & Ungefähr 35$^\circ$ 30' N. 11$^\circ$ 20' O. & C. 180. Lycosthenes 217. & Steinregen; doch vielleicht nur Hagel. \\ \hline
        71. & - & 43. (41.) - - & Rom (?) & Italien & 41$^\circ$ 54' N. 12$^\circ$ 26' O. & P. 4. 1854. 8. Lycosthenes 228. & Desgleichen. \\ \hline
        72. & 5. & 38. 13. März & Im ehemaligen Königreich Leang, jetzt in der Provinz Ho-nan. & China & Ungefähr 34$^\circ$ 52' N. 114$^\circ$ 33' O. & MS. 136. AR. 1. 191. C. 180. & 6 vom Himmel gefallene Steine. \\ \hline
        73. & 6. & 29. 29. Februar & Khao, Provinz Petchi-li; und Feï-lo (nach anderer Angabe: Po), Provinz Pe-tchi-li. & China & 38$^\circ$ 5' N. 114$^\circ$ 59' O. & MS. 136. AR. 1. 192. DG. 1. 246. C. 180. & 1 oder 2 vom Himmel gefallene Steine am ersten Ort und 4 am zweiten Ort. \\ \hline
        74. & 7. & 22. 12. April & Pe-ma, Bezirk Thaï-ming-fou, Provinz Pe-tchi-li. & China & Ungefähr 35$^\circ$ 38' N. 114$^\circ$ 48' O. & MS. 136. AR. 1. 192. C. 180. & 8 vom Himmel gefallene Steine. \\ \hline
        75. & 8. & 19. 16. Juni & Tu-yan, bei Nan-yang-fou, Provinz Ho-nan. & China & Ungefähr 33$^\circ$ 6' N. 112$^\circ$ 35' O. & MS. 137. AR. 1. 192. C. 180. & 3 desgleichen. \\ \hline
        76. & - & 15. 27. März & ? & China & - & MS. 137. AR. 1. 192. & 1 Stern (nach MS. Sterne) fiel wahrend der Nacht in Gestalt von Regen. \\ \hline
        77. & 9. & 12. - April & Tu-ku-an (Tou-kouan), Bezirk Chang-tcheou, Prov. Chen-si. & China & 33$^\circ$ 29' N. 110$^\circ$ 1' O. & MS. 137. AR. 1. 192. C. 180. & 1 vom Himmel gefallener Stein. \\ \hline
        78. & - & 12. 24. Mai & ? & China & - & MS. 137. AR. 1. 192. & 1 Stern fiel bei Tage in Gestalt von Regen und unter wiederholtem donnerähnlichem Getöse. \\ \hline
        79. & 10. & 9. - - & ? & China & - & DG. 1. 250. C. 180. & 2 vom Himmel gefallene Steine. \\ \hline
        80. & 11. & 6. 4. März & Ning-tschu, Bezirk von Pe-ti, Provinz Kan-sou. & China & 35$^\circ$ 35' N. 107$^\circ$ 51' O. & MS. 137. AR. 1. 192. C. 180. & 10 oder 16 desgleichen. \\ \hline
        81. & 12. & 6. 27. Oktober & Yu (Ju), Bezirk Kiaï-tscheou, Provinz Chan-si. & China & Ungefähr 35$^\circ$ 5' N. 110$^\circ$ 58' O. & MS. 137. AR. 1. 192. C. 180. & 2 desgleichen. \\ \hline
          &   & Nach Christus &   &   &   &   &   \\ \hline
        82. & 1. & Zwischen 1 und 50 - & Im Lande der Vocontier; Gegend von Die und Vaisin in der heutigen Dauphiné. & Frankreich & Ungefähr 44$^\circ$ 25' N. 5$^\circ$ 15' O. & C. 186. & 1 vom Himmel gefallene Steine. \\ \hline
        83. & 13. & 2. - - & Kiu-lu, Bezirk Chun-te-fou, Provinz Pe-tchi-li. & China & 37$^\circ$ 17' N. 115$^\circ$ 11' O. & MS. 137. AR. 1. 192. C. 187. & 2 vom Himmel gefallene Steine. \\ \hline
        84. & - & 7. - - & ? & Japan & - & Quetelet 1841. 21. & Ein Sternregen fiel vom Himmel; wahrscheinlich nur Sternschnuppen. \\ \hline
        85. & - & 60. - - & In Cantabrien. & Spanien & Ungefähr 43$^\circ$ 0' N. 3 bis 6 W. & Schweigger 14 (44). 1825. Fol. 357.* Beccheri Ph. Subt. 603.* Merula 294.* Suetonius 2. 162.* & Der Blitz fiel in einen See worauf man 12 Beile fand. (Ob die von Becher erwähnten 6 eisernen Beile noch ein anderer Fall sind als dieser von 12 Beilen, muss dahingestellt bleiben). \\ \hline
        86. & 14. & 106. - - & Tschin-lieu, Bezirk Khaï-foung-fou, Prov. Ho-nan. & China & 34$^\circ$ 45' N. 114$^\circ$ 40' O. & MS. 141. AR. 1. 193. C. 187. & 4 Sterne fielen als 4 Steine. \\ \hline
        87. & 15. & 154. (164.) 1. April & Yeou-fu-fung, (Foung-thsiang-fou), Provinz Chen-si. & China & 34$^\circ$ 25' N. 107$^\circ$ 30' O. & MS. 141. AR. 1. 194. C. 187. & 1 Stein fiel unter donnerndem Getöse. \\ \hline
        88. & 16. & 154. (164.) - - & Khien, Bez. Tchoung-khing-fou, Prov. Sse-tchouen. & China & 29$^\circ$ 21' N. 106$^\circ$ 23' O. & MS. 141. AR. 1. 194. C. 187. & 2 desgleichen. \\ \hline
        89. & - & 235. - - & Wei-nan, Bezirk von Singan-fou, Prov. Chen-si. & China & 34$^\circ$ 29' N. 109$^\circ$ 27' O. & MS. 142. EB. 266 u. 173. & 1 Stern fiel in das Kriegslager. \\ \hline
        90. & - & 238. 26. September & Siang-p’ing ($^\wedge$$^\wedge$$^\wedge$). & China & - & MS. 142. & 1 große Sternschnuppe fiel in der Nacht im SO. der Stadt. \\ \hline
        91. & - & 268. - - & ? & China & - & MS. 142. AR. 1. 194. & 1 Stern fiel als Regen (nach MS. Sterne). \\ \hline
        92. & - & 288. 26. September & ? & China & - & MS. 142. AR. 1. 194. & Desgleichen. \\ \hline
        93. & - & 303. 5. Dezember & ? & China & - & MS. 143. AR. 1. 194. & 1 Stern fiel bei hellem Tage mit donnerähnlicher Explosion. \\ \hline
        94. & - & 304. 15. September & ? & China & - & MS. 143. AR. 1. 194. & 1 Stern fiel mit Geräusch (nach MS. Sterne). \\ \hline
        95. & - & 305. - - & ? & China & - & MS. 143. AR. 1. 194. & Desgleichen. \\ \hline
        96. & 17. & 310. 23. Oktober & Wahrscheinlich in der Nahe von Phing-yang, Prov. Chan-si. & China & 36$^\circ$ 6' N. 111$^\circ$ 33' O. ? & MS. 143. AR. 1. 195. C. 178. & Es fiel 1 Stern, dessen Bruchstucke nach Phing-yang gesandt wurden. \\ \hline
        97. & 18. & 333. - - & 6 franz. M. NO. von Ye, Bezirk Tchang-te-fou, Provinz Ho-nan. & China & 36$^\circ$ 22' N. 114$^\circ$ 48' O. & MS. 143. AR. 1. 195. C. 187. & Es fiel 1 brennender Stern, worauf man 1 Stein fand. \\ \hline
        98. & - & 369. 10. Dezember & ? & China & - & MS. 144. AR. 1. 195. & 1 Stern fiel unter donnerndem Getöse. \\ \hline
        99. & - & 388. - - & ? & China & - & MS. 144. AR. 1. 195. & 1 himmlischer Hund (Meteor) fiel mit Geräusch. \\ \hline
        100. & - & 394. - - & In der ehemaligen Provinz Ho-pe, im Norden des Gelben Flusses. & China & - & MS. 145. AR. 1. 196. & 1 Stern fiel mit donnerndem Getöse. \\ \hline
        101. & - & 452. - - & ? & China & - & AR. 1. 196. & 1 Stern fiel mit 6-7fachem Getöse. \\ \hline
        102. & 5. & 452. - - & ? & Thrakien & - & C. 188. & 3 vom Himmel gefallene große Steine. \\ \hline
        103. & 1. & 481. - - & ? & Afrika & - & P. 8. 1826. 45. & Vom Himmel gefallene feurige Steine. \\ \hline
        104. & 1. & 5.. - - & Gebirge Libanon. & Syrien & Ungefähr 34$^\circ$ 0' N. 36$^\circ$ 0' O. & C. 188. & Viele vom Himmel gefallene Steine (Batylia). \\ \hline
        105. & 2. & 5.. - - & Emesa. & Syrien & 34$^\circ$ 40' N. 37$^\circ$ 50' O. & C. 188. & 1 Stein aus einer Feuerkugel. \\ \hline
        106. & - & 532. 28. August & ? & China & - & MS. 145. AR. 1. 196. & 1 Stern fiel als Regen (nach MS. Sterne). \\ \hline
        107. & - & 545. (546.) 22. Oktober & Ju-pi, wahrscheinlich der ehemalige Bezirk Pi-tcheou in der Provinz Sse-tchouen. & China & - & MS. 145. AR. 1. 196. EB. 159. & 1 Stern fiel in das kaiserliche Kriegslager. \\ \hline
        108. & - & 549. - - & Wou (Wou-kiun) ($^\wedge$$^\wedge$$^\wedge$). & China & - & MS. 146. & 1 große Sternschnuppe fiel in die Stadt. \\ \hline
        109. & - & 552. - Dezember & Ou-kiun (Sou-tcheou-fou), Prov. Kiang-nan. & China & 31$^\circ$ 23' N. 120$^\circ$ 29' O. & MS. 146. AR. 1. 196. EB. 186. & Es fiel 1 Stern. \\ \hline
        110. & - & 554. - November & Kiang-ling (King-tcheou-fou), ehemals Provinz Hou-kouang, jetzt Provinz Ho-nan. & China & 30$^\circ$ 27' N. 112$^\circ$ 5' O. & MS. 146. AR. 1. 196. EB. 72, 80, 81 u. 49. & 1 Stern (Sternschnuppe) fiel in die Stadt. \\ \hline
        111. & - & 570. - - & Beder (Beddr). & Arabien & 23$^\circ$ 30' N. 39$^\circ$ 35' O. & C. 188. & Steinregen, der in der Schlacht die Feinde tötete; vielleicht nur Hagel. \\ \hline
        112. & - & 585. 23. (6.) September & ? & China & - & MS. 147. & Einige 100 Sternschnuppen fielen und zerstreuten sich nach allen Seiten. (Wohl wirkliche Sternschnuppen). \\ \hline
        113. & - & 599. 26. Dezember & Po-haï, ehemaliger Distrikt der Provinzen Pe-tchi-li und Chang-toung, darinnen Pin-tcheou und Ho-kien-fou. & China & - & MS. 147. & Regen von Sternen; vielleicht auch in das Meer von Pe-tchi-li, welches ebenfalls Po-hai genannt wird. \\ \hline
        114. & - & 615. - - & Tse-lou (Tse-lo, Thse-lo), Bezirk von Pao-ting-fou, Provinz Pe-tchi-li. & China & 38$^\circ$ 53' N. 115$^\circ$ 36' O. & AR. 1. 197. EB. 255, 237 u. 154. & Es fiel 1 Stern. \\ \hline
        115. & - & 616. 14. Januar & ? & China & - & MS. 147. & 1 große Sternschnuppe fiel in das Lager von Ming-youe, zertrümmerte Wagen und tötete 10 Mann. \\ \hline
        116. & 19. & 616. 28. Mai & U-kien (Ou-kiun oder Sou-tcheou-fou), Prov. Kiang-sou. & China & 31$^\circ$ 23' N. 120$^\circ$ 29' O. & MS. 147. AR. 1. 197. C. 189. & 1 große Feuerkugel (Sternschnuppe) fiel und verwandelte sich in 1 Stein. \\ \hline
        117. & - & 617. 11. Juni & Kiang-tou (Yang-tcheou-fou), Prov. Kiang-nan. & China & 32$^\circ$ 26' N. 119$^\circ$ 24' O. & MS. 148. AR. 1. 197. EB. 73 u. 280. & Es fiel 1 Stern (große Sternschnuppe). \\ \hline
        118. & - & 620. 29. November & Toung-tou (Ho-nan-fou), Provinz Ho-nan. & China & 34$^\circ$ 43' N. 112$^\circ$ 28' O. & MS. 148. AR. 1. 197. EB. 253 u. 40. & 1 Stern fiel unter mehrmaligem donnerndem Getöse. \\ \hline
        119. & - & 628. - - & Hia-tcheou (Ning-hia-fou), Prov. Kan-sou, jetzt östlicher Teil der Provinz Chen-si. & China & 38$^\circ$ 33' N. 106$^\circ$ 7' O. & MS. 148. AR. 1. 197. EB. 30, 145 u. 55. & 1 himmlischer Hund (Meteor) fiel in die Stadt. \\ \hline
        120. & - & 640. - September & Kao-tch’ang, ehemalige Hauptstadt der Uiguren (Ost-Turken oder Turkomannen), im Norden von Cha-tcheou, ein Distrikt 80 Lieues O. von So-tcheou-fou (Provinz Kan-sou, jetzt östlicher Teil der Prov. Chen-si). & China & Ungefähr 39$^\circ$ 40' N. 94$^\circ$ 50' O. & MS. 148. AR. 1. 197. EB. 308. 307 u. 55. & Es fiel ein Stern (nach MS. Sterne). \\ \hline
        121. & - & 648. - - & Konstantinopel. & Europäischen Türkei & 41$^\circ$ 0' N. 28$^\circ$ 58' O. & C. 190. & 1 Stein wie ein feuriger Ambos soll herabgefallen sein, und gleichzeitig will man einen feurigen Drachen (Feuerkugel) durch die Luft haben fliegen sehen. \\ \hline
        122. & - & 653. - November & In der Gegend von Mou-tcheou (Mo-tcheou oder Yen-tcheou-fou) und von Ou-tcheou (Kin-hoa-fou), beide Provinz Tche-kiang. & China & Zwischen 29$^\circ$ 37' N. 119$^\circ$ 33' O. Und 29$^\circ$ 11' N. 119$^\circ$ 51' O. & MS. 148. AR. 1. 198. EB. 285 u. 78. & 1 Stern fiel in das Lager der Aufrührer. \\ \hline
        123. & - & 708. 16. März & ? & China & - & MS. 149. AR. 1. 198. & 1 großer Stern fiel unter donnerndem Getöse. \\ \hline
        124. & - & 713. (708.) - Juli & Yieou ($^\wedge$$^\wedge$$^\wedge$), im N. der Provinz Pe-tchi-li. & China & - & MS. 149. AR. 1. 198. & 1 großer Stern fiel in das Kriegslager. \\ \hline
        125. & - & 744. 4. April & ? & China & - & MS. 150. AR. 1. 198. & 1 Stern von der Große des Mondes fiel unter donnerndem Getöse. \\ \hline
        126. & - & 757. 19. Mai & Nan-yang (Nan-yang-fou), Provinz Ho-nan. & China & 33$^\circ$ 6' N. 112$^\circ$ 35' O. & MS. 150. AR. 1. 198. EB. 137 u. 136. & 1 großer Stern fiel in das Lager der Aufrührer. \\ \hline
        127. & - & 764. 4. Juli & Fen-tcheou (Fen-tcheou-fou), Provinz Chan-si. & China & 37$^\circ$ 19' N. 111$^\circ$ 41' O. & MS. 150. AR. 1. 199. EB. 17. & Es fiel 1 Stern. \\ \hline
        128. & - & 769. - Mai & ? & Arabien, Mesopotamien oder Persien & - & Abd. Allatif par S. de Sacy. 505 (notes).* Assemani Bibl. Or. 2. 114.* & Regen von schwarzen Steinen, wie sie sonst in der Gegend ihres Niederfalles nicht angetroffen werden, und von denen 70 Jahre später noch welche zu sehen waren.* \\ \hline
        129. & - & 783. 16. September & Tchang-ngan (Si-ngan-fou), Prov. Chen-si. & China & 34$^\circ$ 17' N. 108$^\circ$ 58' O. & MS. 151. AR. 1. 199. EB. 198 u. 172. & 1 Stern fiel in die Stadt. \\ \hline
        130. & - & 784. 10. Juli & ? & China & - & MS. 151. & Sterne fielen in Haufen von 5 oder 10. \\ \hline
        131. & - & 787. 15. Juli & ? & China & - & MS. 151. & Es fiel ein schlangenförmiges Meteor. \\ \hline
        132. & - & 798. 20. Juni & ? & China & - & MS. 152. AR. 1. 199. & 1 Stern fiel unter donnerndem Getöse. \\ \hline
        133. & - & 811. 30. März & Zwischen Youan (Yen-tcheou, Yen-tcheou-fou) und Yun (Yun-tching), Bezirk Thsao-tcheou-fou, Provinz Chan-toung. & China & Zwischen 35$^\circ$ 42' N. 117$^\circ$ 3' O. Und 35$^\circ$ 45' N. 116$^\circ$ 14' O. & MS. 152. AR. 1. 199. EB. 285, 304 u. 237. & 1 Stern (große Sternschnuppe) fiel mit großem Getöse. \\ \hline
        134. & - & 817. 26. Oktober & Zwischen Tchin (Tchin-tcheou, Tchin-tcheou-fou) und Thsai (Jou-ning-fou), beide Provinz Ho-nan. & China & Zwischen 33$^\circ$ 46' N. 115$^\circ$ 2' O. Und 33$^\circ$ 1' N. 114$^\circ$ 21' O. & MS. 152. EB. 212 u. 53. & 1 große Sternschnuppe fiel unter 3maligen donnerndem Getöse. \\ \hline
        135. & - & 821. - - & Ou (Sou-tcheou-fou), Provinz Kiang-nan. & China & 31$^\circ$ 23' N. 120$^\circ$ 29' O. & MS. 153. EB. 186. & 1 großer Stern fiel unter Geräusch in die Stadt. \\ \hline
        136. & - & 822. 30. Juli & ? & China & - & MS. 154. & Es fiel 1 kleiner Stern. \\ \hline
        137. & - & 823. (822.) - - & Im Gau von Frisatz (Frisazi, Frihsazi, Firihsazi, Fiusazi, Firichsare oder Virsedi) ($^\wedge$$^\wedge$$^\wedge$) in Sachsen. & Deutschland & - & C. 191. P. 4. 1854. 450. Ann. Fuld. (Pertz 1. 358.)* & Bei hellem, heiterem Himmel werden 23 Dorfer durch vom Himmel gefallenes Feuer angezündet. \\ \hline
        138. & - & 823. (822.) - - & ? & ? & - & Ann. Fuld. (Pertz 1. 358.) & Hagel mit wahren Steinen von großem Gewicht; doch vielleicht ebenfalls nur sehr große Schlossen. \\ \hline
        139. & - & 823. 23. September & ? & China & - & MS. 154. & 1 große Sternschnuppe fiel in der Nacht unter Geräusch auf die Erde. \\ \hline
        140. & - & 824. - Mai & ? & China & - & MS. 154. & Es fielen viele Sterne. \\ \hline
        141. & - & 828. (829.) - - & ? & ? & - & Schnurrer 1. 175.* & Fallende Sterne sollen Menschen und Tiere getötet haben. \\ \hline
        142. & - & 837. - - & In Sachsen (?) & Deutschland (?) & - & P. 4. 1854. 8. Lycosthenes 348. & Man glaubt, dass unter dem Hagel Steine vom Himmel fielen; doch vielleicht auch nur große Schlossen. \\ \hline
        143. & - & 837. 18. Dezember & Hing-Youen (Hang-tchong-fou, Han-tchoung-fou), Provinz Chen-si. & China & 32$^\circ$ 56' N. 107$^\circ$ 12' O. & MS. 156. B. 36 u. 27. & 1 großer Stern fiel auf das Schlafgemach des Statthalters. \\ \hline
        144. & - & 839. - - & Provinz Isumo (Hauptstadt: Isumi) an der Ostkuste der Bay von Osaka im W. der Insel Nipon (Niphon). & Japan & Ungefähr 34$^\circ$ 40' N. 134$^\circ$ 0' O. & C. 191. AR. 1. 201. & Nach 10tagigem Donnern und Regen fielen viele weiße und rote Steine wie Pfeile und kleine Äxte. \\ \hline
        145. & - & 844. 1. Oktober & ? & China & - & MS. 157. & Es fiel 1 großer Stern. \\ \hline
        146. & - & 844. - - & ? & Frankreich & - & Chron. Magn. Schedelii Bl. 191. S. 2. & Hagel mit harten Kernen. \\ \hline
        147. & 3. & 852. - Juli (August) & Provinz Tabarestan oder Masanderan, am Kaspischen Meer. & Persien & Ungefähr 36$^\circ$ 0' N. 53$^\circ$ 0' O. & C. 191. & 1 Stein von 13 Tb., der dem Kalifen gesandt ward. \\ \hline
        148. & 2. & 856. - Dezember & Sowaida (Sowadi), S. von Kairo. & Ägypten & 28$^\circ$ 0' N. 31$^\circ$ 20' O. & C. 192. & 5 Steine, deren 4 nach Fossat und 1 nach Tennis gebracht wurden. \\ \hline
        149. & - & 872. Frühjahr & ? & China & - & MS. 157. & Es fielen 2 Sterne. \\ \hline
        150. & - & 876. - - & ? & China & - & MS. 157. & Bei hellem Tage fiel ein Stern. \\ \hline
        151. & - & 881. 10. bis 18. September & ? & China & - & MS. 158. & In der Nacht fielen Sterne wie Regen. \\ \hline
        152. & - & 883. Ende November (Anf. Dezember) & ? & China & - & MS. 158. & Desgleichen. \\ \hline
        153. & - & 884. (886.) - Oktober & Yang-tcheou-fou, Provinz Kiang-nan. & China & 32$^\circ$ 26' N. 119$^\circ$ 24' O. & MS. 158. AR. 1. 201. EB. 280. & 1 Stern fiel mit großem Getöse. \\ \hline
        154. & - & 885. (887.) - Juni & Pian-tcheou (Pien-tcheou, Khaï-foung-fou), Provinz Ho-nan. & China & 34$^\circ$ 52' N. 114$^\circ$ 33' O. & MS. 158. AR. 1. 201. EB. 160 u 59. & 1 Stern fiel unter donnerndem Getöse in das Lager. \\ \hline
        155. & - & 885. - - & Akiden (Akinda), Provinz Dewa, auf der NW. Seite der Insel Nipon (Niphon). & Japan & 40$^\circ$ 10' N. 139$^\circ$ 50' O. & C. 192. AR. 1. 201. & Eckige Steine wie Pfeilspitzen, doch vielleicht nur Hagel. \\ \hline
        156. & - & 886. - - & ? & Japan & - & C. 192. AR. 1. 201. & Desgleichen. \\ \hline
        157. & - & 886. 16. November & ? & China & - & MS. 158. & Es fiel ein Stern. \\ \hline
        158. & 4. & 893. (892.) (897.) (898.) (899.) (908.) - - & Ahmed-Abad (Ahmed-Bad) bei Kufah, S. von Bagdad und von Helle. & Mesopotamien & Ungefähr 32$^\circ$ 0' N. 45$^\circ$ 0' O. & C. 192. & Unter Regen und Donnerschlagen weiße und schwarze Steine, die zum Teil nach Bagdad gebracht wurden. \\ \hline
        159. & - & 894. Sommer & Youe (Chao-hing-fou), Provinz Tche-kiang. & China & 30$^\circ$ 6' N. 120$^\circ$ 33' O. & MS. 158. EB. 291 u 6. & Es fiel 1 Stern. \\ \hline
        160. & - & 896. - Juli & ? & China & - & AR. 1. 201. MS. 158. & 1 Stern fiel mit Geräusch. \\ \hline
        161. & - & 898. 27. November & ? & China & - & MS. 159. & Es fiel 1 großer Stern. \\ \hline
        162. & - & 905. - - & ? & China & - & AR. 1. 202. & Viele kleine Sterne fielen als Regen. \\ \hline
        163. & 7. & 921. - - & Narni, SW. von Spoleto, N. von Rom; Kirchenstaat. & Italien & 42$^\circ$ 32' N. 12$^\circ$ 30' O. & P. 2. 1824. 151. & Viele Steine, deren größter in den Fluss Narnus gefallen und später noch darin zu sehen war. \\ \hline
        164. & - & 925. 27. April & ? & Arabien & - & L’Institut 6. 350.* & Ein Stern fiel unter heftigem donnerähnlichem Getöse. \\ \hline
        165. & - & 925. (926.) 7. Oktober & ? & China & - & MS. 160. AR. 1. 203. & 1 himmlischer Hund (Meteor) fiel mit großem Geräusch. \\ \hline
        166. & - & 930. 24. November & ? & China & - & MS. 160. & Es fielen gleichzeitig viele kleine Sterne. \\ \hline
        167. & - & 944. - - & ? & ? & - & Quetelet 1841. 29. & Feuersbrunste durch herabgefallene Feuerkugeln veranlasst. \\ \hline
        168. & 1. & 951. (950.) (952.) (953.) - - & Augsburg, Kreis Schwaben. & Deutschland & 48$^\circ$ 22' N. 10$^\circ$ 53' O. & C. 193. & 1 großer glühender, von Westen kommender und wie glühendes Eisen aussehender Stein fiel vom Himmel. \\ \hline
        169. & - & 954. 20. Februar & ? & China & - & MS. 162. AR. 1. 203. & 1 großer Stern fiel mit großem Getöse. \\ \hline
        170. & 8. & 956. (963.) (zwischen 964 u. 972) - - & ? & Italien & - & P. 4. 1854. 8. A. 4. 187. Lycosthenes 362. & Unter Sturm und Donner fiel ein großer Stein vom Himmel. \\ \hline
        171. & - & 962. 13. Juni & ? & China & - & MS. 163. & Es fiel ein himmlischer Hund (Meteor). \\ \hline
        172. & - & 970. - - & ? & Arabien & - & L’Institut 6. 350. & 1 Stern fiel unter donnerndem Getöse. \\ \hline
        173. & - & 990. 30. November & ? & China & - & MS. 168. AR. 1. 203. & 1 Stern (Sternschnuppe) fiel mit Getöse auf die Erde. \\ \hline
        174. & - & 995. 31. Mai & ? & China & - & MS. 169. & Es fiel 1 Stern. \\ \hline
        175. & - & 996. 21. Mai & ? & China & - & MS. 169. AR. 1. 204. & 1 Stern fiel mit Geräusch. \\ \hline
        176. & - & 996. 28. Juni & ? & China & - & MS. 169. & 1 Stern fiel ohne Geräusch auf die Erde. \\ \hline
        177. & - & 997. 19. Oktober & ? & China & - & MS. 170. & Es fielen 2 Sterne. \\ \hline
        178. & 2. & 998. - - & Magdeburg, Preuss. Sachsen. & Deutschland & 52$^\circ$ 8' N. 11$^\circ$ 40' O. & C. 193. & 2 große glühende Steine, deren einer in die Stadt fiel. \\ \hline
        179. & - & 1002. 12. Oktober & ? & China & - & MS. 170. AR. 1. 204. & 1 großer Stern und viele kleine fielen mit großem Geräusch. \\ \hline
        180. & - & 1002. 23. Oktober & ? & China & - & MS. 170. & Es fiel 1 Stern am hellen Tage. \\ \hline
        181. & - & 1004. 25. Januar & Wei (Wei-tcheou), Bezirk von Tch’ing-tou-fou, Provinz Sse-tchouen. & China & 31$^\circ$ 25' N. 103$^\circ$ 40' O. & MS. 170. EB. 265 u. 215. & 1 Stern fiel im NO. der Stadt unter 3fachem donnerndem Getöse. \\ \hline
        182. & - & 1004. 12. Dezember & Thien-Hioung (Thaï-ming-fou), Provinz Pe-tchi-li. & China & 36$^\circ$ 21' N. 115$^\circ$ 22' O. & MS. 170. EB. 231 u. 223. & Es fiel 1 Stern. \\ \hline
        183. & 5. & Zwischen – 999 u. 1030; etwa 1009. & Provinz Tschurdschan am Kaspischen Meer. & Persien & Ungefähr 37$^\circ$ 0' N. 54$^\circ$ 30' O. & C. 194. & Eisenmasse, daraus man vergeblich versuchte, Schwerter zu schmieden. \\ \hline
        184. & - & 1021. (1020.) - Juli (August) & Provinz Afrika (Africa proprie dicta). & Nord-Afrika & Zwischen 33 u. 37 N. 5 u. 11 O. & C. 196. P. 4. 1854. 8. 450 u. 449. & Viele Steine bis zu 5 Tb. schwer, aus einer mit Blitz und Donner geladenen Wolke, die viele Menschen töteten; vielleicht nur Hagel. \\ \hline
        185. & - & 1021. - - & ? & Persien & - & P. 4. 1854. 450. & Vielleicht einerlei mit Tschurdschan Nr. 183. \\ \hline
        186. & - & 1029. - Juli (August) & ? & Arabien & - & L’Institut 6. 350. Quetelet 1841. 30. & Es fielen viele Sterne mit großem Getöse, welches vielleicht von einem Steinfall oder von Feuermeteoren herrührte. \\ \hline
        187. & 20. & 1057. - - & Provinz Hoang-haï (Hoang-liei). & Korea & 34$^\circ$ 54' N. 127$^\circ$ 0' O. & C. 196. AR. 1. 205. & Unter Donnerschlag fiel 1 Stein, der an den Hof gesandt ward. \\ \hline
        188. & - & 1057. - - & ? & ? & - & P. 4. 1854. 9. Lycosthenes 380. Quetelet 1841. 30. & Hagel mit großen Steinen; vielleicht ebenfalls Hagel. \\ \hline
        189. & - & 1076. - - & ? & Dänemark & - & P. 4. 1854. 9. Lycosthenes 383. & 1 Wurfgeschoss, das wahrend der Schlacht in der Luft umherirrend gesehen ward, stürzte auf den Harquinus und tötete ihn. \\ \hline
        190. & - & 1093. (1094.) (1095.) (1096.) 4. April (10. März) & ? & Frankreich & - & P. 6. 1826. 23. K. 3. 265. A. 4. 187. Lycosthenes 387. Quet. 1841. 31. & Viele Sternschnuppen, deren Eine, sehr große, auf dem Boden gefunden ward; mit Wasser begossen, zischte sie auf. \\ \hline
        191. & - & 1099. - - & ? & ? & - & Rivander 215.* & Sterne sah man vom Himmel auf die Erde fallen (wahrscheinlich nur Sternschnuppen und vielleicht einerlei mit dem Vorigen). \\ \hline
        192. & - & 1103. (1104.) Ungefähr 24. Juni & Würzburg; Fränkischer Kreis. & Deutschland & - & Schnurrer 1. 229. & Hagel mit Steinen, deren einer, in 4 Stucke zerteilt, von 4 Mannern kaum getragen werden konnte; doch vielleicht ebenfalls nur ein sehr großes Stuck Eis. \\ \hline
        193. & - & 1110. - - & In den See Van; Provinz Vaspuragan. & Armenien & Ungefähr 38$^\circ$ 20' N. 42$^\circ$ 50' O. & C. 191. & Feuermeteor mit mutmaßlichem Meteorsteinfall. \\ \hline
        194. & - & 1111. 27. Juni & ? & China & - & MS. 306. & Es fiel 1 Stern bei Tage. \\ \hline
        195. & 55. & 1112. - - & Aquileja (Aglar). & Illyrien & 45$^\circ$ 46' N. 13$^\circ$ 24' O. & C. 197. & Glühende Steine; vielleicht Eisen. \\ \hline
        196. & - & 1126. 10. Juli & ? & China & - & MS. 308. & 1 Stern fiel unter donnerndem Getöse. \\ \hline
        197. & - & 1128. - - & ? & Deutschland ? & - & Chron. Magn. Schedelii Bl. 222. S. 2. & Sterne fielen auf die Erde, und als man Wasser darauf goss, gaben sie einen Hail (Feuerkugelmaterie?). \\ \hline
        198. & - & 1130. (nicht 1138.) 8. März & Mosul, am Tigris. & Mesopotamien & 36$^\circ$ 24' N. 43$^\circ$ 20' O. & C. 197. Abulfaradsch (B. Hebraeus) Chr. Syr. 314.* & Nach einem Gewitter fielen feurige Kohlen, die viele Hauser anzündeten. \\ \hline
        199. & - & 1131. 6. Mai & ? & China & - & MS. 309. & Es fiel 1 Stern bei Tage. \\ \hline
        200. & 3. & 1135. (1130.) (1136.) - - & Oldisleben (Oldesleb, Aldessleben), in Thüringen. & Deutschland & 51$^\circ$ 19' N. 11$^\circ$ 10' O. & C. 197. & 1 großer Stein, der aufbewahrt worden. \\ \hline
        201. & - & 1137. 30. August & Pien-king (Khaï-foung-fou); Provinz Ho-nan. & China & 34$^\circ$ 52' N. 114$^\circ$ 33' O. & MS. 310. EB. 160 u. 59. & Es fiel 1 Stern. \\ \hline
        202. & - & Zwischen 1100 und 1160 - - & Kaswin (Casbine), S. vom Kaspischen Meer. & Persien & 36$^\circ$ 10' N. 49$^\circ$ 35' O. & Fundgruben des Orients 6. 307 u. 308.* & Aus einer Wolke fielen unter Donner nach einander 2 Steine.* \\ \hline
        203. & - & Zwischen 1100 und 1160 - - & In einer von Kaswin entfernteren Gegen und etwas später als der vorige Steinfall. & Persien ? & - & Fundgruben des Orients 6. 307 u. 308. & Es soll Steine geregnet haben, wobei viele Leute zu Grunde gegangen sein sollen. \\ \hline
        204. & 4. & 1164. - Mai & Im Meissen’schen Sachsen. & Deutschland & Ungefähr 51$^\circ$ 0' N. 13$^\circ$ 0' O. & C. 198. & 1 vom Himmel gefallene Eisenmasse. \\ \hline
        205. & - & 1186. (1187.) 8. Juli (30. Juni) & Mons. & Belgien & 50$^\circ$ 26' N. 3$^\circ$ 57' O. & P. 4. 1854. 9. & Hagel von Steinen von über 1 Tb.; doch ungewiss, ob nicht große Schlossen. \\ \hline
        206. & - & 1190. (1189.) (1191.) (1194.) - - & Zwischen Clermont (Claurus mons) und Compiegne (Compennium), OSO. von Beauvais (in Beauvoisin, pago Beluacensi); Départ. de l’Oise. & Frankreich & Zwischen 49$^\circ$ 23' N. 2$^\circ$ 25' O. Und 49$^\circ$ 25' N. 2$^\circ$ 5' O. & C. 198. A. 4. 188. Lycosthenes 425. P. 6. 1826. 23. & Bei starkem Regen fielen viereckige Steine von der Große von Huhnereiern, und gleichzeitig wurden schwarze Vogel (Raben) in der Luft fliegend gesehen, mit glühenden Kohlen in den Schnabeln, welche sie auf die Hauser fallen ließen, und durch welche sie diese anzündeten. \\ \hline
        207. & - & 1197. - - & ? & Italien & - & A. 4. 188. Lycosthenes 426. & Steine fielen unter Regen; vielleicht nur Hagel. \\ \hline
        208. & - & 1198. 8. Juni (Juli) & Zwischen Chelles (Kala, Chiele oder Challe), 2 Stunden O. von Paris, und Tremblai (Tremblaco), Dép. de Seine et Oise. & Frankreich & Ungefähr 48$^\circ$ 23' N. 2$^\circ$ 36' O. & C. 198. Lycosthenes 427. & Nuss- und eigroße Steine, selbst noch größere, fielen wahrend eines Sturmes; wahrscheinlich nur Hagel. \\ \hline
        209. & - & 1210. 18. November & ? & China & - & MS. 319. & 1 Stern fiel bei Nacht. \\ \hline
        210. & - & 1213. 13. Juni & ? & China & - & MS. 319. & 1 Stern fiel bei Tage. \\ \hline
        211. & - & 1213. 21. September & ? & China & - & MS. 319. & 1 Stern fiel bei Nacht. \\ \hline
        212. & - & 1213. 5. Oktober & ? & China & - & MS. 319. & 1 Stern fiel bei Tage. \\ \hline
        213. & - & 1214. 18. Januar & ? & China & - & MS. 319. & Desgleichen. \\ \hline
        214. & - & 1219. 20. August & ? & China & - & MS. 326. & 1 Stern fiel unter trommelähnlichem Getöse. \\ \hline
        215. & - & 1226. - - & ? & ? & - & P. 6. 1826. 23. Schnurrer 1. 273. & Eigroße viereckige Hagelsteine und gleichzeitig wieder schwarze Vogel (Raben) mit glühenden Kohlen in den Schnäbeln, welche sie auf die Hauser fallen ließen. Auch feurige Drachen (Hellebrande) wurden gesehen. Sehr wahrscheinlich ein und dasselbe, nur von manchen Chronikenschreibern ohne Ortsangabe in eine spätere Zeit versetzte Ereignis, wie Nr. 206: 1190 (1191, 1194) Beauvais. \\ \hline
        216. & - & 1228. 10. Juli & ? & China & - & MS. 321. & 1 Stern fiel bei Tage. \\ \hline
        217. & - & 1230. 25. Dezember & ? & China & - & MS. 321. & Desgleichen. \\ \hline
        218. & - & 1231. 18. Oktober & ? & China & - & MS. 322. & Desgleichen. \\ \hline
        219. & - & 1232. 22. August & ? & China & - & MS. 322. & 1 Stern fiel bei Nacht. \\ \hline
        220. & - & 1235. 5. Juli & ? & China & - & MS. 322. & 1 Stern fiel bei Tage. \\ \hline
        221. & - & 1235. 26. Juli & ? & China & - & MS. 322. & Desgleichen. \\ \hline
        222. & - & 1236. 12. Juli & ? & China & - & MS. 322. & Desgleichen. \\ \hline
        223. & - & 1237. 5. März & ? & China & - & MS. 322. & 1 Stern fiel bei Nacht. \\ \hline
        224. & - & 1238. 13. Juli & ? & China & - & MS. 322. & 1 Stern fiel bei Tage. \\ \hline
        225. & - & 1238. 6. September & ? & China & - & MS. 322. & 1 Stern fiel bei Tage. \\ \hline
        226. & - & 1239. 9. April & ? & China & - & MS. 322. & Desgleichen. \\ \hline
        227. & - & 1240. 1. März & ? & China & - & MS. 323. & Desgleichen. \\ \hline
        228. & - & 1240. 12. April & ? & China & - & MS. 323. & Desgleichen. \\ \hline
        229. & - & 1241. 1. August & ? & China & - & MS. 323. & Desgleichen. \\ \hline
        230. & - & 1243. 27. August & ? & China & - & MS. 323. & Desgleichen. \\ \hline
        231. & 5. & 1249. 26. Juli & Zwischen Quedlinburg, Blankenburg und Ballenstadt; am Harz. & Deutschland & Ungefähr 51$^\circ$ 45' N. 11$^\circ$ 6' O. & C. 199. & Unter Hagel graue Steine, die nach Schwefel rochen. \\ \hline
        232. & - & 1250. 4. Mai & ? & China & - & MS. 323. & 1 Stern fiel bei Nacht. \\ \hline
        233. & - & 1251. 19. August & ? & China & - & MS. 324. & 1 Stern fiel bei Tage. \\ \hline
        234. & 1. & Zwischen 1251 und 1360. - - & Welikoi-Ustiug (Groß-Ustiug), Gouv. Wologda. & Russland & 60$^\circ$ 45' N. 46$^\circ$ 16' O. & C. 200. & Viele Steine unter donnerartigem Getöse und Geprassel. \\ \hline
        235. & - & 1276. - - & ? & China & - & MS. 326. & Es fiel 1 Stern. \\ \hline
        236. & - & 1278. - - & ? & China & - & MS. 327. & 1 Stern fiel unter donnerndem Getöse in das Meer. \\ \hline
        237. & - & 1278. - - & Kouang-tcheou (Canton), Provinz Kouang-toung. & China & 23$^\circ$ 8' N. 113$^\circ$ 16' O. & MS. 327. EB. 86 u. 87. & Es fiel 1 Stern unter trommelähnlichem Getöse. \\ \hline
        238. & - & 1280. - - & Alexandrien. & Ägypten & 31$^\circ$ 13' N. 29$^\circ$ 50' O. & C. 200. & Der Blitz fiel auf einen Stein und verbrannte ihn. \\ \hline
        239. & - & 1300.? - - & Aragonien. & Spanien & - & P. 2. 1824. 152. & Vom Himmel gefallener Stein von der Große eines Fasses. \\ \hline
        240. & 6. & 1304. 1. Oktober & Friedland in der Mark Brandenburg (Fredtlandt oder Urdeland; auch Vredeland in Vandalia). & Deutschland & 52$^\circ$ 6' N. 14$^\circ$ 17' O. & C. 200. Krantz, Sax. Bl. 190. S. 1.* & Viele feurige Steine, wie Hagel, welche Hauser und Dorfer, samt Allem, was sie erreichten, anzündeten. \\ \hline
        241. & - & 1304. - - & Friedeburg an der Saale, NW. von Halle und S. von Bernburg. & Deutschland & 51$^\circ$ 37' N. 11$^\circ$ 45' O. & C. 200. Rivander 360. Spangenberg Bl. 324. S. 2.* Dresser 312.* & In einem Donnerwetter fielen glühend heiße Steine, kohlschwarz und hart wie Eisen, welche, wo sie hinfielen, das Gras versengten.* \\ \hline
        242. & - & 1323. (1328.) 9. Januar & Provinzen Mortahiah ($^\wedge$$^\wedge$$^\wedge$) und Dakhahiah (Dakhalia) ($^\wedge$$^\wedge$$^\wedge$). & Ägypten & - & C. 201. & Hagel mit sehr großen Steinen; doch vielleicht ebenfalls Hagelmassen. \\ \hline
        243. & - & 1339. 13. Juli & Schlesien. & Deutschland & - & C. 201. & 300 Donnerkeile bei einem Gewitter; doch ungewiss, ob Meteorsteine oder bloße Donnerschlage. \\ \hline
        244. & 7. & Um 1340. (nicht 1440.) - - & Birki (Bireki oder Birgeh), OSO. von Smyrna, und NNO. von Guzelhissar (Aidin); Provinz Aidin. & Klein-Asien & 38$^\circ$ 16' N. 27$^\circ$ 57' O. & P. 4. 1854. 10. Ibn Batuta Fol. 72 u. 2. & 1 von Himmel gefallener, sehr harter Stein von 112 oder 120 Tb., der aufbewahrt und dem Ibn Batuta zu Birki war vorgezeigt worden. \\ \hline
        245. & 21. & 1358. - - & Thaï-ming, Bezirk von Thaï-ming-fou, Provinz Pe-tchi-li. & China & 36$^\circ$ 18' N. 115$^\circ$ 20' O. & MS. 328. & Es fiel 1 Stern wie eine Flamme, drang in die Erde und ward 1 Stein. \\ \hline
        246. & - & 1360. - - & Yorkshire. & England & - & RPG. & ? \\ \hline
        247. & - & 1368. - - & Wahrscheinlich in der Nahe von Blexen, am Ausflusse der Weser, NNO. von Oldenburg. & Deutschland & 53$^\circ$ 33' N. 8$^\circ$ 30' O. & C. 201. & Eine eiserne Keule erschien in der Luft, tötete wahrend der Schlacht viele Feinde, und ward, 200 Tb. Schwer, in der Blexer Kirche aufbewahrt. Meteoreisen? \\ \hline
        248. & 7. & 1379. 26. Mai & Han. Munden. & Deutschland & 52$^\circ$ 14' N. 8$^\circ$ 53' O. & C. 202. & Steinfall aus einer Feuerkugel. \\ \hline
        249. & 1. & 1421. - - & ? & Java & Ungefähr 7$^\circ$ 30' S. 110$^\circ$ 0' O. & C. 202. & Unter Blitz und Donner 1 Stein, der dem Oberhaupt gebracht ward. \\ \hline
        250. & - & 1427. 12. Januar & ? & China & - & MS. 331. & 1 Stern fiel unter donnerndem Getöse. \\ \hline
        251. & 1. & 1438. - - & Roa, S. von Burgos und W. von Aranda, in Alt-Kastilien. & Spanien & 41$^\circ$ 42' N. 3$^\circ$ 56' W. & C. 203. & Großer Steinfall von ganz leichten, schwammigen, weißen Steinen, deren 4 dem Könige gebracht wurden. \\ \hline
        252. & 11. & 1474. - - & Viterbo, NNW. von Rom; Kirchenstaat. & Italien & 42$^\circ$ 27' N. 12$^\circ$ 6' O. & G. 68. 1821. 332. & 2 große, nach Schwefel riechende Steine. \\ \hline
        253. & - & 1476. 11. Dezember & ? & China & - & MS. 333. & Es fielen 2 Sterne, der eine in einen Kanal, der andere auf einen Wall. \\ \hline
        254. & - & 1478. - - & ? & Schweiz & - & Lycosthenes 493. & Feurige Kugeln fielen auf die Erde und hinterließen hier Spuren ihres Brandes. \\ \hline
        255. & - & 1480. - - & Sachsen oder Böhmen. & Deutschland & - & RPG. 34. & Angeblich 1 Stein (?). \\ \hline
        256. & - & 1484. 3. Juni & Fan-iu (die eine der 2 Städte, welche Canton oder Kouang-tcheou-fou bilden), Provinz Kouang-toung. & China & 23$^\circ$ 8' N. 113$^\circ$ 16' O. & MS. 333. EB. 15 u. 86. & 1 großer Stern fiel unter donnerndem Getöse im SO. von der Stadt. \\ \hline
        257. & 12. & 1491. 22. März & Rivolta de Bassi, NW. von Crema; Lombardei. & Italien & 45$^\circ$ 28' N. 9$^\circ$ 30' O. & C. 204. & Unter donnerndem Getöse fiel 1 Stein, davon 1 Bruchstuck nach Venedig gebracht ward. \\ \hline
        258. & 22. & 1491. 15. November & Kouang-chan, Bezirk Jou-ning-fou, Provinz Ho-nan. & China & 32$^\circ$ 8' N. 114$^\circ$ 51' O. & MS. 333. & 1 Stern fiel unter trommelähnlichem Getöse in die Stadt und verwandelte sich in 1 Stein. \\ \hline
        259. & - & 1491. 2. Dezember & Tchin-ting (Tchin-ting-fou); Provinz Pe-tchi-li. & China & 38$^\circ$ 11' N. 114$^\circ$ 45' O. & MS. 334. EB. 209. & 1 Stern fiel unter trommelähnlichem Getöse in NW. von der Stadt. \\ \hline
        260. & 2. & 1492. 7. November & Ensisheim, im Sundgau; Ober-Elsass. & Gegenwärtig in Frankreich & 47$^\circ$ 51' N. 7$^\circ$ 22' O. & C. 205. Chron. Magn. Sehedelli Bl. 300. S. 1. & Aus einem Feuermeteor 1 Stein von ursprünglich 300 Tb., der in der Kirche aufbewahrt ward. \\ \hline
        261. & - & 1494. - - & Siouen-fou (Siouen-hoa oder Nan-ning-fou), Prov. Kouang-si; ebenso in den Provinzen Chan-si und Ho-nan. & China & 22$^\circ$ 43' N. 108$^\circ$ 3' O. & MS. 334. EB. 183 u. 134. & Es fielen Sterne bei hellem Tage. \\ \hline
        262. & - & 1495. 12. Mai & Yen-chan, Bezirk von Thien-tsin-fou; Provinz Pe-tchi-li. & China & 38$^\circ$ 7' N. 117$^\circ$ 16' O. & MS. 334. EB. 283 u. 231. & 1 Stern fiel unter donnerndem Getöse in die Stadt. \\ \hline
        263. & 13. & 1496. 26. (28.) Januar & Zwischen Cesena und Bertinoro, und zu Valdinoce; Kirchenstaat. & Italien & 44$^\circ$ 8' N. 12$^\circ$ 10' O. Und 44$^\circ$ 4' N. 12$^\circ$ 6' O. & C. 207. & 3 unter donnerndem Getöse vom Himmel gefallene Steine. \\ \hline
        264. & - & 1496. 13. Juli & Munchberg (Munchpergk), SSW. von Hof im Voigtlande; Bayern. & Deutschland & 50$^\circ$ 12' N. 11$^\circ$ 47' O. & C. 209. & 3eckige und hühnereiförmige Steine; wahrscheinlich nur Hagel. \\ \hline
        265. & - & 1497. 11. Februar & Ning-hia (Ning-hia-fou); Provinz Chen-si. & China & 38$^\circ$ 33' N. 106$^\circ$ 7' O. & MS. 334. EB. 145. & 1 Stern fiel unter donnerndem Getöse im NW. der Stadt. \\ \hline
        266. & - & 1497. 26. (nicht 25.) Juli & Langres; Dép. de la Haute-Marne (Langer in Hoch-Burgund). & Frankreich & 47$^\circ$ 52' N. 5$^\circ$ 20' O. & C. 209. Gotz v. Berl. 17.* & Wahrend eines Unwetters fielen Steine, so groß wie Hühnereier; wer über die Gasse lief und ward von einem Stein getroffen, den warf derselbe nieder. Vermutlich aber Alles nur große Schlossen. \\ \hline
        267. & - & 1497. 2. Oktober & Young-p’ing (Young-p’ing-fou); Provinz Pe-tchi-li. & China & 39$^\circ$ 56' N. 118$^\circ$ 54' O. & MS. 334. EB. 297. & 1 Stern fiel unter großem Geräusch. \\ \hline
        268. & - & 1498. 17. Februar & So-tcheou (So-tcheou-fou); Provinz Chen-si. & China & 39$^\circ$ 46' N. 99$^\circ$ 7' O. & MS. 334. EB. 185. & Eine hausgrosse Sternschnuppe fiel unter donnerndem Getöse. \\ \hline
        269. & - & 14.. - - & Luzern. & Schweiz & 47$^\circ$ 3' N. 8$^\circ$ 18' O. & C. 209. Cysat. 176. u. s. w.* & 1 angeblich aus einem fliegenden Drachen herabgefallener und zu Wunderkuren gebrauchter Stein. \\ \hline
        270. & - & 1501. 18. August & Cheou-kouang, Bezirk von Thsing-tcheou-fou; Provinz Chan-toung. & China & 36$^\circ$ 55' N. 119$^\circ$ 0' O. & MS. 334. EB. 8 u. 241. & 1 großer Stern fiel unter trommelahnlichem Getöse. \\ \hline
        271. & - & 1503. 9. März & Nan-king (Cour du midi oder Kiang-ning-fou); Provinz Kiang-nan. & China & 32$^\circ$ 4' N. 118$^\circ$ 47' O. & MS. 335. EB. 133 u. 72. & Es fiel ein Stern bei hellem Tage. \\ \hline
        272. & - & 1507. 8. Januar & Ning-hia (Ning-hia-fou); Provinz Chen-si. & China & 38$^\circ$ 33' N. 106$^\circ$ 7' O. & MS. 335. EB. 145. & 1 Stern fiel mitten in die Stadt. \\ \hline
        273. & - & 1507. 4. Oktober & Distrikt von Ning-hia; Provinz Chen-si. & China & 38$^\circ$ 33' N. 106$^\circ$ 7' O. & MS. 335. EB. 145. & 1 großer Stern fiel im SW. \\ \hline
        274. & - & 1509. - - & In Schwaben. & Deutschland & - & Surius, Comment. 62.* & Hagel mit eigrossen Steinen; doch wahrscheinlich ebenfalls nur große Schlossen. \\ \hline
        275. & 14. & 1511. 4. September & Crema, unweit der Adda; Lombardei. & Italien & 45$^\circ$ 21' N. 9$^\circ$ 42' O. & C. 209. & Viele nach Schwefel riechende große Steine, darunter von 120 und 260 Tb.; einer von 100 Tb. Ward nach Mailand gebracht. \\ \hline
        276. & - & 1511. 17. September & Thsoung-king (Thsoung-khing-tcheou), Bezirk von Tch’ing-tou-fou; Provinz Sse-tchouen. & China & 30$^\circ$ 36' N. 103$^\circ$ 43' O. & MS. 335. EB. 245 u. 215. & 1 große Sternschnuppe fiel unter donnerndem Getöse in die Stadt. \\ \hline
        277. & 23. & 1516. - - & Schun-king-fu; Provinz Sse-tchouen. & China & 30$^\circ$ 49' N. 106$^\circ$ 7' O. & C. 211. AR. 1. 208. & 6 Steine von 10 Unzen bis zu 10 u. 17 Tb. \\ \hline
        278. & 1. & Vor 1520. - - & Brussel. & Belgien & 50$^\circ$ 51' N. 4$^\circ$ 22' O. & C. 208. & 1 vom Himmel gefallener Stein, den Alb. Durer noch gesehen. \\ \hline
        279. & - & 1520. 6. Februar & Loung-tchouen; Provinz Chan-si (oder Prov. Kouang-toung?). & China & ? & MS. 335. EB. 121. & Es fiel ein Stern. \\ \hline
        280. & - & 1520. 15. Mai & Koung-tch’ang-fou; fruher Provinz Chen-si, jetzt Provinz Kan-sou. & China & 34$^\circ$ 56' N. 104$^\circ$ 43' O. & MS. 335. EB. 94. & 1 großer Stern fiel unter trommelahnlichem Getöse. \\ \hline
        281. & 2. & 1520. - Mai & Zwischen Oliva und Gandia: Aragonien. & Spanien & 38$^\circ$ 58' N. 0$^\circ$ 8' W. & C. 211. & Aus einem Feuermeteor 3 Steine von 25 Tb., deren einer aufbewahrt worden. \\ \hline
        282. & - & 15.. - - & Zwischen Cicuic und Quivira, 2 Orte in Neu-Spanien (jetzt in New-Mexico?), deren Lage und Dasein jedoch nach Humboldt sehr zweifelhaft ist. & Nord-Amerika & Ungefähr 35$^\circ$ 0' N. 105$^\circ$ 0' W. ? & C. 209. & Angeblicher Steinfall; doch vielleicht nur Hagel. \\ \hline
        283. & - & 15.. (?) - - & Thal von Gagona ($^\wedge$$^\wedge$$^\wedge$). & Amerika & - & Majolus 11.* & Regen von Steinen; doch vielleicht nur Hagel. \\ \hline
        284. & - & 1525. 28. (29.) Juni & Mailand; Lombardei. & Italien & 45$^\circ$ 28' N. 9$^\circ$ 11' O. & G. 50. 1815. 237. & Feuerkugel, die ein Pulver-Magazin in Brand steckte; doch ungewiss, ob dabei ein Stein fiel. \\ \hline
        285. & - & 1528. 29. Juni (19. Juli) & Augsburg, Kreis Schwaben. & Deutschland & 48$^\circ$ 22' N. 10$^\circ$ 53' O. & C. 212. Lycosthenes 535. & Große, wie aus Buchsen geschossene Steine während eines Gewitters; vielleicht nur großer Hagel. \\ \hline
        286. & - & 1540. 28. April & Les Eglises (St. Laurent-des-Eglises, NO. von Limoges?), Provinz Limousin; Dép. de la Haute-Vienne. & Frankreich & 45$^\circ$ 57' N. 1$^\circ$ 29' O. (?) & C. 212. & Unter Hagel 1 Stein von der Große eines Fasses, der 2 Ellen tief in die Erde eingedrungen und mit Hebebaumen herausgeholt worden sein soll. \\ \hline
        287. & 24. & 1540. 14. Juni & Tsao-khiang, bei Ki-tcheou; Provinz Pe-tchi-li. & China & Ungefähr 37$^\circ$ 38' N. 115$^\circ$ 42' O. & MS. 336. & Es fiel 1 Stern und verwandelte sich in 4 Steine. \\ \hline
        288. & 8. & Zwischen 1540 und 1550 - - & Naunhof (Neuholm), zwischen Grimma und Leipzig; Sachsen. & Deutschland & 51$^\circ$ 17' N. 12$^\circ$ 36' O. & C. 212. & Große vom Himmel gefallene Eisenmasse. \\ \hline
        289. & 15. & Zwischen 1550 und 1570 - - & An mehreren Orten in Piemont. & Italien & - & C. 213. & Niederfall von Eisen, wovon Scaliger ein Stuck in Handen gehabt. \\ \hline
        290. & 9. & 1552. 19. Mai & Schleusingen in Thüringen. & Deutschland & 50$^\circ$ 31' N. 10$^\circ$ 45' O. & C. 213. & Unter Blitzen und Donnern viele Steine, deren Spangenberg mehrere nach Eisleben brachte. \\ \hline
        291. & - & 1558. 10. Mai & In Thüringen. & Deutschland & - & Rivander 502. Spangenberg Bl. 477. S. 2. & Es fiel Schwefel vom Himmel, den man einzeln hin und wieder hat aufheben konnen. \\ \hline
        292. & 1. & 1559. - - & Miskolez; Gespanschaft Borschod. & Ungarn & 48$^\circ$ 6' N. 20$^\circ$ 47' O. & C. 214. & 5 große Stein- oder Eisenmassen, deren vier nach Wien gebracht wurden. \\ \hline
        293. & - & 1560. 24. Dezember & Lillebonne (Juliobona), O. von Hàvre; Dép. de la Seine-Infériure. & Frankreich & 49$^\circ$ 32' N. 0$^\circ$ 31' O. & C. 364. & Feuermeteor mit Niederfall einer roten und vielleicht auch einer festen Stein-Masse. \\ \hline
        294. & 10. & 1561. 17. Mai & Torgau, Siptitz, WNW. von Torgau, und Eilenburg (prope arcem Juliam); Preuss. Sachsen. & Deutschland & 51$^\circ$ 33' N. 13$^\circ$ 1' O. Und 51$^\circ$ 28' N. 12$^\circ$ 38' O. & C. 215. & Mehrere Stein- oder Eisenmassen, harter als Basalt. \\ \hline
        295. & - & 1564. 1. März & Zwischen Brussel und Mecheln. & Belgien & Ungefähr 51$^\circ$ 0' N. 4$^\circ$ 25' O. & C. 215. & Angeblicher Steinfall, darunter Steine von 5-6 Tb., wie Marmorsteine. \\ \hline
        296. & - & 1569. 14. (15.) September & Venedig. & Italien & 45$^\circ$ 26' N. 12$^\circ$ 20' O. & Dresser Sachs. Chr. 670. & Sterne und Feuer fielen vom Himmel und schlugen in zwei Pulverthurme und einen Schwefelthurm. \\ \hline
        297. & - & 1572. 9. Januar & Thorn; West-Preußen. & Deutschland & 53$^\circ$ 1' N. 18$^\circ$ 37' O. & C. 216. & Es hagelte zehnpfundige Steine unter einem Wolkenbruch; wahrscheinlich nur große Schlossen. \\ \hline
        298. & 25. & 1575. (nicht 1565.) 3. Juli & King-tcheou, Provinz Hou-kouang; jetzt Prov. Hou-pe. & China & 30$^\circ$ 27' N. 112$^\circ$ 5' O. & MS. 336. AR. 4. 190. & Mit trommelahnlichem Getöse fielen 2 Sterne und verwandelten sich in schwarze Steine. \\ \hline
        299. & - & 1576. 25. November & Pii-hien (P’i), Bezirk von Y-tcheou-fou; Provinz Chan-toung. & China & 35$^\circ$ 18' N. 118$^\circ$ 5' O. & MS. 336. EB. 159 u. 278. & Es fielen 4 Sterne. \\ \hline
        300. & - & 1577. - - & Meaco (Miaco), auf der Insel Nipon (Niphon). & Japan & 34$^\circ$ 55' N. 135$^\circ$ 20' O. & Majolus 11. & während eines Gotzenfestes fiel aus heiterem Himmel und unter lautem Getöse ein Regen von Felsen, vor welchem jedoch alle anwesenden Christen verschont blieben. \\ \hline
        301. & - & 1579. 21. Mai & Stendal; Preußisch Sachsen. & Deutschland & 52$^\circ$ 37' N. 11$^\circ$ 50' O. & Engelius Rer. March Brev. 163.* & Schwefel-Regen, dass Straßen und Äcker voll zermalmten Schwefelpulvers lagen. \\ \hline
        302. & 11. & 1580. 27. Mai & Norten, zwischen Nordheim und Göttingen; Hannover. & Deutschland & 51$^\circ$ 38' N. 9$^\circ$ 55' O. & C. 217. & Viele Steine, die zum Teil aufbewahrt oder versandt wurden. \\ \hline
        303. & - & 1580. 13. August & Wiehe, WSW. von Merseburg und N. von Buttstadt; und auf der Finne; Thüringen. & Deutschland & 51$^\circ$ 16' N. 11$^\circ$ 24' O. & Bangen Bl. 188. S. 2.* & Hagel von der Große von Hühnereiern, voll langer Zacken und inwendig voll scharfer weißer Steine. \\ \hline
        304. & 12. & 1581. 26. Juli & Niederreißen (Nieder-Reusen), S. von Buttstadt; Thüringen. & Deutschland & 51$^\circ$ 6' N. 11$^\circ$ 25' O. & C. 218. & Unter Donnerschlag 1 Stein von 39 oder 49 Tb., der nach Weimar und von da nach Dresden gebracht worden. \\ \hline
        305. & 16. & 1583. 9. Januar & Castrovillari in den Abruzzen; Neapel. & Italien & 39$^\circ$ 45' N. 16$^\circ$ 15' O. & C. 219. & Unter donnerndem Getöse ein eisenähnlicher Stein von 33 Tb. \\ \hline
        306. & 17. & 1583. 2. März & In Piemont. & Italien & - & C. 219. & Aus einer donnernden Wolke 1 Stein, der dem Herzog von Savoyen gebracht wurde. \\ \hline
        307. & - & 1585. - - & ? & Italien & - & G. 18. 1804. 307. & 1 bleifarbiger Stein metallischer Masse von 30 Tb.; wahrscheinlich einerlei mit No. 305: Castrovillari. \\ \hline
        308. & - & 1585. 28. Juli & Mien (Mien-tcheou); Provinz Sse-tchouen. & China & 31$^\circ$ 28' N. 104$^\circ$ 52' O. & MS. 337. EB. 127. & 1 großer Stern fiel unter trommelähnlichem Getöse. \\ \hline
        309. & - & 1587. 3. Juli & Ping-yang (P’ing-yang-fou); Provinz Chan-si. & China & 36$^\circ$ 6' N. 111$^\circ$ 33' O. & MS. 337. EB. 164. & Es fiel 1 Stern am hellen Tage. \\ \hline
        310. & - & 1587. 4. Juli & Ping-yn, Bezirk von Thaï-ngan-fou; Provinz Chan-toung. & China & 36$^\circ$ 23' N. 116$^\circ$ 34' O. & MS. 337. EB. 165 u. 226. & Am Tage fiel 1 Stern unter donnerndem Getöse. \\ \hline
        311. & - & 1589. 16. Februar & Si-ning-wei (Si-ning-fou?) im W. von Chen-si. & China & 36$^\circ$ 39' N. 101$^\circ$ 48' O. ? & MS. 337. EB. 172. & Unter donnerndem Getöse fiel 1 Stern von der Große des Mondes. \\ \hline
        312. & - & 1591. 9. Juni & Kuhnersdorf, in der Mark Brandenburg. & Deutschland & 52$^\circ$ 24' N. 15$^\circ$ 0' O. & G. 50. 1815. 240. G. 54. 1816. 344. A. 4. 190. Engelius Rer. March. Brev. 177. & während eines Unwetters große und sehr eckige Hagelsteine, wobei auch ganze Stucke Feuer aus den Wolken gefallen sein sollen. Wahrscheinlich nur große Schlossen mit heftigen Blitzschlagen. \\ \hline
        313. & - & 1592. - - & Min (Fou-tcheou-fou), Provinz Fo-kien. & China & 26$^\circ$ 2' N. 119$^\circ$ 29' O. & MS. 337. EB. 128 u. 19. & 3 Sterne fielen im SO. der Stadt. \\ \hline
        314. & 18. & 1596. 1. März & Crevalcore, W. von Cento, Bezirk Ferrara; Kirchenstaat. & Italien & 44$^\circ$ 43' N. 11$^\circ$ 8' O. & C. 220. & Niederfall vieler Steine, ähnlich wie Feuerflammen. \\ \hline
        315. & - & 1599. 5. April & Kai-tcheou (Kai), Provinz Liao-toung. & China & 40$^\circ$ 30' N. 122$^\circ$ 30' O. & MS. 337. EB. 55. & 3 Sterne fielen unter trommelähnlichem Getöse. \\ \hline
        316. & 3. & Vor 1603. - - & Valencia. & Spanien & 39$^\circ$ 28' N. 0$^\circ$ 22' W. & C. 220. & Niederfall einer metallischen Masse, wahrscheinlich Eisen. \\ \hline
        317. & - & 1605. 18. Oktober & Nan-king (Cour du midi, Kiang-ning-fou), Provinz Kiang-nan. (Im 9ten Mond.) & China & 32$^\circ$ 4' N. 118$^\circ$ 47' O. & MS. 338. EB. 133 u. 72. & Es fiel 1 Stern auf die Erde. \\ \hline
        318. & - & 1605. - - & ? (Im 10ten Mond.) & China & - & MS. 338. & 1 Stern fiel zur Erde. \\ \hline
        319. & - & 1605. - - & Nan-king (Cour du midi, Kiang-ning-fou), Provinz Kiang-nan. (Im 11ten Mond.) & China & 32$^\circ$ 4' N. 118$^\circ$ 47' O. & MS. 338. EB. 133 u. 72. & 1 Stern fiel auf ein Gebäude, drang in die Erde, und hinterließ keine Spur. \\ \hline
        320. & - & 1605. 12. Dezember & King-yang und Chun-hao, Distrikt von Pintcheou; beide im Bezirk von Si-ngan-fou, Provinz Chen-si. & China & 34$^\circ$ 30' N. 108$^\circ$ 45' O. Und 34$^\circ$ 55' N. 108$^\circ$ 30' O. & MS. 338. EB. 80, 15, 160 u. 172. & Es fielen unter donnerndem Getöse Sterne von der Große von Radern. \\ \hline
        321. & - & 1610. 11. März & Yang-kio (Yang-khio oder Thaï-youen-fou), Provinz Chan-si. & China & 37$^\circ$ 53' N. 112$^\circ$ 33' O. & MS. 338. EB. 280 u. 225. & 1 Stern fiel unter trommelähnlichem Getöse im NW. der Stadt. \\ \hline
        322. & - & 1613. 21. Januar & Ting-hing, Bezirk von Pao-ting-fou; Provinz Pe-tchi-li. & China & 39$^\circ$ 17' N. 115$^\circ$ 56' O. & MS. 338. EB. 248 u. 154. & Bei hellem Tage fiel eine Sternschnuppe unter trommelähnlichem Getöse. \\ \hline
        323. & - & 1615. 19. Mai & Thsing-foung, Bezirk von Thaï-ming-fou, Provinz Pe-tchi-li. & China & 35$^\circ$ 58' N. 115$^\circ$ 21' O. & MS. 338. EB. 242 u. 223. & Bei hellem Tage fiel ein Stern unter donnerndem Getöse im O. der Stadt. \\ \hline
        324. & 43. & 1618. - - & ? & Böhmen & - & C. 221. & Niederfall einer metallischen Masse, wahrscheinlich Eisen. \\ \hline
        325. & - & 1618. 7. März & Paris. & Frankreich & 48$^\circ$ 53' N. 2$^\circ$ 20' O. & C. 79, 99 u. 220. & Herabgefallene brennende Masse (Stern), die einen Palast anzündete. \\ \hline
        326. & 2. & 1618. Ende Aug. & Murakoz (Mur-Insel), an der Grenze von Steiermark; Gespanschaft Salad. & Ungarn & Ungefähr 46$^\circ$ 25' N. 16$^\circ$ 30' O. & C. 220. & Unter Donnerschlagen aus einer Feuerkugel 3 Zentner schwere Steine und eine rote, schlammige Masse. \\ \hline
        327. & 26. & 1618. 12. November & Nan-king (Cour du midi oder Kiang-ning-fou); Provinz Kiang-sou. & China & 32$^\circ$ 5' N. 118$^\circ$ 47' O. & MS. 339. & Unter donnerndem Getöse fiel 1 Stern und verwandelte sich in einen Stein von 21 Tb. \\ \hline
        328. & 2. & 1621. (1620.) (nicht 1650 oder 1652.) 17. April & Tschalinda (Dschallinder oder Jalendher), 20 M. OSO. von Lahore; Pendsjab. Eisen. & Ost-Indien & 31$^\circ$ 24' N. 75$^\circ$ 34' O. & C. 221. & Unter gewaltigem Getöse eine 5 Tb. Schwere Eisenmasse, daraus unter Zusatz von anderem Eisen Waffen geschmiedet wurden. \\ \hline
        329. & 1. & 1622. 10. Januar & Tregnie, angeblich in Devonshire; wahrscheinlich Tregony in Cornwallis. & England & 50$^\circ$ 16' N. 4$^\circ$ 55' O. ? & C. 222. & Unter donnerähnlichem Krachen 1 Stein, der als Wunder gezeigt ward. \\ \hline
        330. & - & 1623. 10. Oktober & Kou-youen (Kou-youen-tcheou), im Bezirk von P’ing-liang-fou; Provinz Chen-si. & China & 36$^\circ$ 3' N. 106$^\circ$ 21' O. & MS. 339. EB. 84 u. 162. & Sterne fielen wie Regen. \\ \hline
        331. & 2. & 1628. 9. April & Hatford, 3 M. O. von Faringdon; Berkshire. & England & 51$^\circ$ 40' N. 1$^\circ$ 32' W. & C. 223. & Unter vielem Getöse ein innen noch weicher Stein, davon der Sherif 1 Stuck erhielt. \\ \hline
        332. & 3. & 1634. 27. Oktober & Provinz Charollais (Grafschaft Carolath); im ehemaligen Herzogtum Burgund. & Frankreich & Ungefähr 46$^\circ$ 30' N. 4$^\circ$ 10' O. & C. 223. & Aus einem Feuermeteor viele Steine, darunter von 5 8 Tb. \\ \hline
        333. & - & 1635. 21. Juni & Vago, O. von Verona; Venezien. & Italien & 45$^\circ$ 25' N. 11$^\circ$ 8' O. & A. 4. 191. C. 233. Bigot de Morogues Fol. 79 (nach Fr. Carli)* & 1 großer Stein; wahrscheinlich jedoch einerlei mit No. 353, dem Steinfall von 1668, von welchem viele falsche Jahreszahlen angegeben worden. \\ \hline
        334. & 19. & 1635. 7. Juli & Calce (Colze, SO. von Vicenza?) im Vicentinischen; Venezien. & Italien & 45$^\circ$ 28' N. 11$^\circ$ 38' O. ? & C. 224. & Unter Hagel 1 Stein von 11 Unzen, den Valisnieri aufbewahrt hatte. \\ \hline
        335. & 13. & 1636. 6. März & Zwischen Sagan und Dubrow; Preuss. Schleisen. & Deutschland & 51$^\circ$ 36' N. 15$^\circ$ 20' O. & C. 225. & Unter großem Krachen ein leicht zerreiblicher Stein, der innen voll metallischer Teile. \\ \hline
        336. & 20. & 1637. (1627.) (1617.) 27. (29.) November & Mont Vaisien (Mons Vasonum), zwischen Guilleaume u. Pesne, bei Nizza, in der ehemaligen Provence; Piemont. & Italien (Gegenwärtig in Frankreich) & Ungefähr 44$^\circ$ 6' N. 6$^\circ$ 52' O. & C. 225. & Unter heftigem Krachen 1 Stein von 38 Tb. und von metallischem Ansehen, welcher in Aix war aufbewahrt worden. \\ \hline
        337. & - & 1642. - Juni & Magdeburg, Lohberg u. s. w.; Preuss. Sachsen. & Deutschland & 52$^\circ$ 8' N. 11$^\circ$ 40' O. & C. 367. & Es sollen faustgroße Schwefelklumpen gefallen sein. \\ \hline
        338. & 3. & 1642. 4. August & Zwischen Woodbridge und Alborow; Suffolk. & England & Ungefähr 52$^\circ$ 6' N. 1$^\circ$ 25' O. & C. 226. & Unter anhaltendem Getöse ein noch heißer Stein von 4 Tb. \\ \hline
        339. & 3. & 1642. 12. ? Dezember ? & Zwischen Ofen und Gran. & Ungarn & Ungefähr 47$^\circ$ 40' N. 18$^\circ$ 50' O. & C. 100. & Unter schrecklicher Explosion aus einer Feuerkugel angeblich Blei und Zinn; wahrscheinlich weiches Eisen. \\ \hline
        340. & - & 1643. (1644.) - - & Auf ein Schiff. & Ost-Indisches Meer & - & C. 227. A. 4. 191. & Angeblich einige harte Steine. \\ \hline
        341. & - & 1644. 17. April & In den Yu-ho (Kaiserlichen Kanal). & China & - & MS. 338. & Niederfall von Sternen. \\ \hline
        342. & - & 1646. 16. Mai & Kopenhagen. & Dänemark & 55$^\circ$ 40' N. 30$^\circ$ 15' O. & Olaus Worm 28.* & Vom Himmel gefallener pulverförmiger Schwefel, welcher zum Teil gesammelt u. Aufbewahrt wurde. \\ \hline
        343. & 14. & 1647. 18. Februar & Pohlau (Polau), O. von Zwickau; Sachsen. & Deutschland & 50$^\circ$ 43' N. 12$^\circ$ 33' O. & C. 227. & Aus einem Feuermeteor ein nach Schwefel riechender, Eisenschlakken-ähnlicher Stein von 50 Tb., der nach Dresden gesandt ward. \\ \hline
        344. & - & 1647. Pfingsten & Insel Falster. & Dänemark & Ungefähr 54$^\circ$ 55' N. 12$^\circ$ 0' O. & G. 50. 1815. 243. & Steine zur Zeit eines Hagelfalles; vielleicht ebenfalls nur Hagel. \\ \hline
        345. & 15. & 1647. - August & Zwischen Wermsen u. Schameelo, Vogtei Bomhorst, Amt Stolzenau; Westphalen. & Deutschland & Ungefähr 52$^\circ$ 28' N. 8$^\circ$ 49' O. & C. 227. & Unter kanonenähnlichem Donner 1 Stein, davon ein Bruchstuck nach Nienburg gesandt ward. \\ \hline
        346. & - & Zwischen 1647 u. 1654. - - & Auf ein Schiff. & Ost-Indisches Meer & - & C. 228. & 1 Kugel von 8 Tb., welche auf dem Schiff 2 Menschen tötete. \\ \hline
        347. & - & 1649. 11. Mai & Zu Dombach, Ebersheim und Munster im Elsass. & Gegenwärtig in Frankreich & Ungefähr 48$^\circ$ 3' N. 7$^\circ$ 8' O. & G. 29. 1808. 216. C. 101. & Großes Getöse und Sausen in der Luft, vielleicht von einem Meteorsteinfall herrührend. \\ \hline
        348. & 2. & 1650. 6. August & Dordrecht. & Holland & 51$^\circ$ 48' N. 4$^\circ$ 40' O. & C. 228. & 1 noch heißer, von einem Blitzschlag begleiteter Stein, der zu Leyden war aufbewahrt worden. \\ \hline
        349. & 2. & 16.. - - & Warschau. & Polen & 52$^\circ$ 13' N. 21$^\circ$ 5' O. & C. 229. & 1 nach Schwefel riechender Stein, der den Thurm eines Gefängnisses zerstörte. \\ \hline
        350. & 1. & 1654. 30. März & Insel Fuhnen. & Dänemark & Ungefähr 55$^\circ$ 20' N. 10$^\circ$ 20' O. & C. 228. & Unter Blitz und Donner wahrend eines Regens mehrere Steine, deren einer nach Kopenhagen gesandt ward. \\ \hline
        351. & 21. & Um 1660. - - & Mailand; Lombardei. & Italien & 45$^\circ$ 28' N. 9$^\circ$ 11' O. & C. 230. & 1 nach Schwefel riechender Stein von ¼ Unze, der einen Monch tötete und nachher aufbewahrt ward. \\ \hline
        352. & - & 1667. - - & Chiras. & Persien & 29$^\circ$ 38' N. 53$^\circ$ 8' O. & C. 231. & Angeblicher Niederfall einer sehr lockeren, aber steinartigen Substanz. \\ \hline
        353. & 22. & 1668. (nicht 1662, 1663 oder 1672.) 19. (21.) Juni & Vago, O. von Verona; Venezien. & Italien & 45$^\circ$ 25' N. 11$^\circ$ 8' O. & C. 223. & Viele Steine aus einem Feuermeteor, davon 1 in einer Kirche war aufbewahrt und 2 von 200 und 300 Tb. waren nach Verona gesandt worden. \\ \hline
        354. & 16. & 1671. 27. Februar & Oberkirch und Zusenhausen (Zusenhofen?) in der Ortenau, Baden. & Deutschland & 48$^\circ$ 32' N. 8$^\circ$ 7' O. Und 48$^\circ$ 33' N. 8$^\circ$ 2' O. ? ? & C. 236. & Unter donnerndem Getöse und Sausen 1 Stein von 10 Tb. bei ersterem und 1 Stein von 9 Tb. bei letzterem Ort. \\ \hline
        355. & - & 1673. - - & Dietlingen, 2 Stunden OSO. von Ettlingen; Baden. & Deutschland & 48$^\circ$ 54' N. 8$^\circ$ 36' O. & C. 236. & 15 angebliche Schlossensteine in der Brakenhofer’schen Sammlung; nach Chladni sehr zweifelhaft. \\ \hline
        356. & - & 1674. 6. Dezember (nicht Oktober) & Nafels, Canton Glarus. & Schweiz & 47$^\circ$ 6' N. 9$^\circ$ 3' O. & C. 237. Scheuchzer 2. Fol. 72 und 3. Fol. 30. & 2 feurige Kugeln, welche auf den Erdboden gefallen und gespurt worden. \\ \hline
        357. & - & Zwischen 1675 und 1677. - - & Bei der Insel Copinsha auf ein Schiff. & Orkaden & Ungefähr 58$^\circ$ 48' N. 2$^\circ$ 30' W. & C. 237. & Angeblich 1 Stein. \\ \hline
        358. & - & 1676. 31. März & Bei Livorno, in der Richtung nach Korsika, wahrscheinlich ins Meer. & Italien & Ungefähr 43$^\circ$ 30' N. 10$^\circ$ 0' O. & C. 102. P. 4. 1854. 33. & Mutmaßlicher Meteorsteinfall aus einer großen, von Dalmatien hergekommenen Feuerkugel, welche mit Krachen und Erschütterung zersprang. \\ \hline
        359. & 17. & 1677. 26. Mai & Ermendorf, zwischen Dresden und Grossenhain; Sachsen. & Deutschland & 51$^\circ$ 14' N. 13$^\circ$ 36' O. & C. 237. & Aus einem Feuermeteor viele angeblich kupferhaltige Steine. \\ \hline
        360. & 23. & 1697. 13. Januar & Pentolina, SW. von Siena; Menzano, W. von Siena; und Capraja; sämtlich in Toskana. & Italien & 43$^\circ$ 12' N. 11$^\circ$ 10' O. Und 43$^\circ$ 19' N. 11$^\circ$ 3' O. & C. 239. & Unter donnerähnlichem Getöse viele Steine, deren einer, noch heiß und nach Schwefel riechend, von 13 Unzen. \\ \hline
        361. & 1. & 1698. 18. (nicht 19.) Mai & Hinterschwendi bei Waltringen, ONO. von Burgdorf; Canton Bern. & Schweiz & Ungefähr 47$^\circ$ 5' N. 7$^\circ$ 45' O. & C. 239. & Unter vielem Getöse ein großer schwarzer Stein, der in Bern war aufbewahrt worden. \\ \hline
        362. & - & 1700. - - & Insel Jamaica. & Westindien & Ungefähr 18$^\circ$ 10' N. 42$^\circ$ 0' O. & C. 105. & Eine Feuerkugel schlug tiefe Locher in den Boden; nach Steinen ist aber nicht gesucht worden. \\ \hline
        363. & 4. & 1704. 24. (25.) Dezember & Barcelona; Katalonien. & Spanien & 41$^\circ$ 24' N. 2$^\circ$ 10' O. & P. 8. 1826. 46. & Feuerkugel mit Steinfall. \\ \hline
        364. & 6. & 1706. 7. Juni & Larissa in Thessalien. & Europäischen Türkei & 39$^\circ$ 28' N. 22$^\circ$ 35' O. & C. 240. & Aus einer kleinen Wolke ein Stein von 72 Tb., wie Eisenschlacke, von dem ein Stuck dem Sultan gesandt ward. \\ \hline
        365. & 18. & 1715. 11. April & Schellin (nicht Garz), 1 M. W. von Stargard, in Pommern. & Deutschland & 53$^\circ$ 20' N. 15$^\circ$ 0' O. & G. 71. 1822. 213. & Unter donnerähnlichem Getöse 2 Steine von 15 Tb. Und 1 kleinerer, welche aufbewahrt worden. \\ \hline
        366. & - & 1721. - - & Riga. & Russland & 56$^\circ$ 55' N. 25$^\circ$ 50' O. & C. 108. & Brennende oder glühende Meteormasse, die einen Brand in der Peterskirche verursachte. \\ \hline
        367. & - & 1721. - - & Braunschweig. & Deutschland & 52$^\circ$ 15' N. 10$^\circ$ 33' O. & Soldani 122.* & Regen von brennendem Schwefel. \\ \hline
        368. & 19. & 1722. 5. Juni & Schefftlar (Scheftlarn), im Freising’schen; N. von Wolfrathshausen, Bayern. & Deutschland & 47$^\circ$ 56' N. 11$^\circ$ 35' O. & C. 240. & Aus einer kleinen Wolke unter großem Getöse mehrere nach Schwefel riechende Steine, wovon 3 von ¾ Tb. \\ \hline
        369. & 44. & 1723. 22. Juni & Pleskowitz und Liboschitz; beide etliche M. von Reichstadt; Kreis Bunzlau. & Böhmen & Ungefähr 50$^\circ$ 41' N. 14$^\circ$ 39' O. & C. 240. & Aus einer kleinen Wolke unter starkem Krachen 8 nach Schwefel riechende Steine am ersten und 25 am zweiten Ort. \\ \hline
        370. & 4. & 1725. 3. Juli & Mixbury, 7 M. NNO. von Bicester; Oxfordshire. & England & 51$^\circ$ 58' N. 1$^\circ$ 6' W. & RPG. 35. & 1 Stein von 20 Tb. \\ \hline
        371. & 5. & 1731. 12. März & Halstead, WNW. von Colchester; Essex. & England & 51$^\circ$ 57' N. 0$^\circ$ 37' O. & C. 111. & Explosion bei heiterem Himmel, wonach man Etwas wie einen glühenden Muhlstein, nachdem es einen Pfahl zerschlagen, in einen Kanal fallen sah. \\ \hline
        372. & - & 1732. 15. August & Springfield; 1 M. NO. von Chelmsford; Essex. & England & 51$^\circ$ 46' N. 0$^\circ$ 27' O. & P. 66. 1845. 476. K. 3. 271. & Feuermeteor, aus dem Etwas in einen Kanal fiel. \\ \hline
        373. & - & Vor 1736. - - & ? & England & - & C. 371. & 1 fast zollgroßes Stuck Schwefel, welches wahrscheinlich vom Himmel gefallen. \\ \hline
        374. & - & 1737. 21. Mai & Zwischen Lissa u. Monopoli. ($^\wedge$$^\wedge$$^\wedge$) & Adriatisches Meer & Ungefähr 43$^\circ$ 0' N. 16$^\circ$ 10' O. & G. 68. 1821. 350. & Niederfall einer Erde, die ganz vom Magneten angezogen ward (fein verteiltes Meteor-Eisen?). \\ \hline
        375. & - & 1738. 18. Oktober & Carpentras u. Champfort bei Avignon; Dép. de Vaucluse. & Frankreich & 44$^\circ$ 3' N. 5$^\circ$ 3' O. & C. 241. & Mutmaßlicher Meteorsteinfall. Eine unter starker Explosion fallende Feuerkugel schlug tiefe Locher in die Erde, doch ohne dass man nach Steinen gesucht hatte. \\ \hline
        376. & - & 1740. 23 Februar & Toulon, Dép. du Var. & Frankreich & Ungefähr 43$^\circ$ 0' N. 6$^\circ$ 0' O. & P. 66. 1845. 476. K. 3. 272. & Feuerkugel, von der man unter heftigem Donner Stucke ins Meer fallen sah. \\ \hline
        377. & - & 1740. (1741.) Winter & ? & Gronland & 69$^\circ$ 4' N. ? ? W. & C. 242. & Steinfall nach Aussage von Grönländern; aber wahrscheinlich nur ein von einem Berg herabgerollter Felsblock. \\ \hline
        378. & 7. & 1740. (nicht 1770.) 25. Oktober & Hazargrad (Rasgrad), zwischen Schumla u. Rustschuck; Bulgarien. & Europäischen Türkei & 43$^\circ$ 23' N. 26$^\circ$ 12' O. & C. 242. & Unter donnerähnlichem Getöse 2 Steine von ungefähr 43 u. 4 ½ Tb., welche dem Sultan gesandt wurden. \\ \hline
        379. & - & 1749. 4. November & Auf ein Schiff. & Atlantisches Meer & 42$^\circ$ 48' N. 9$^\circ$ 3' W. & C. 114. & 1 Stuck einer Feuerkugel zerschlug unter heftiger Explosion den mittleren Toppmast und warf fünf Menschen nieder; von Steinen ist nicht die Rede. \\ \hline
        380. & - & 1750. 9. Februar & Schlesien. & Deutschland & - & P. 66. 1845. 476. K. 3. 272. & Feuerkugel, die unter starkem Getöse in 4 Stucke zersprang, welche herabgefallen sein sollen. \\ \hline
        381. & 4. & 1750. 1. (11.) Oktober & Nicor (Nicorps, Niort), SO. von Coutance; Dép. de la Manche. & Frankreich & 49$^\circ$ 2' N. 1$^\circ$ 26' W. & C. 243. & Unter donnerähnlichem Getöse ein nach Schwefel riechender Stein, dessen größtes Bruchstuck von 20 Tb. \\ \hline
        382. & 4. & 1751. 26. Mai & Hraschina (nicht Hradschina), SW. von Warasdin, und 5 M. NO. von Agram; Gespanschaft Agram. & Kroatien & 46$^\circ$ 6' N. 16$^\circ$ 20' O. & C. 245. & Aus einer Feuerkugel 2 Eisenmassen von 16 und 71 Tb., deren Letztere nach Wien gesandt ward. \\ \hline
        383. & 45. & 1753. 3. Juli & Plan und Strkow, beide SO. von Tabor; Kreis Bechin. & Böhmen & 49$^\circ$ 21' N. 14$^\circ$ 43' O. Und 49$^\circ$ 21' N. 14$^\circ$ 44' O. & C. 246. & Unter donnerähnlichem Getöse viele eisenhaltige Steine, deren größter von 13 Tb. \\ \hline
        384. & 5. & 1753. 7. September & Luponnas (nicht Laponas oder Liponas) bei Pont-de-Veyle; Dép. de l’Ain. & Frankreich & 46$^\circ$ 14' N. 4$^\circ$ 59' O. & C. 248. & Unter kanonenähnlichem Getöse 2 Steine von 20 und 11 ½ Tb., deren Ersterer nach Dijon kam. \\ \hline
        385. & - & 1755. 19. Mai & Mallow (Malow), NNW. von Cork, Cork-County. & Irland & 52$^\circ$ 9' N. 8$^\circ$ 37' W. & Soldani 122. & Regen von Schwefel, welcher in Masse gesammelt ward. \\ \hline
        386. & 24. & 1755. - Juli & Am Fluss Crati bei Terranova; Kalabrien. & Italien & 39$^\circ$ 38' N. 16 30 (50) O. & C. 248. & Unter starkem Knall 1 Stein von 9 Tb., den Tata besessen, der sich aber nach 9 Jahren schon zersetzt hatte. \\ \hline
        387. & - & 1753. 4. November & Im Bourbonnais. & Frankreich & - & C. 116. & Feuerkugel, deren Stucke unter heftigem Knall in einen Sumpf fielen. \\ \hline
        388. & - & 1756. - - & ? & Frankreich & - & RPG. 40. & Angeblich 1 Stein; vielleicht einerlei mit dem Vorigen oder dem Folgenden? \\ \hline
        389. & - & 1759. 13. Juni & Captieux, S. von Bazar; Dép. de la Gironde. & Frankreich & 44$^\circ$ 18' N. 0$^\circ$ 16' W. & C. 120. & Eine Feuerkugel soll ein Haus angezündet haben. \\ \hline
        390. & - & 1761. 11. (12.) November & Chamlans ($^\wedge$$^\wedge$$^\wedge$) bei Dijon; Dép. de la Côte d’or. & Frankreich & Ungefähr 47$^\circ$ 20' N. 5$^\circ$ 2' O. & C. 121. & 1 Stuck eines großen Feuermeteors zündete ein Haus an. \\ \hline
        391. & 25. & 1766. Mitte Juli & Alboretto, NO. von Modena. & Italien & 44$^\circ$ 41' N. 10$^\circ$ 57' O. & C. 250. & Unter kanonenähnlichem Getöse 1 noch heißer Stein, der aber verloren gegangen. \\ \hline
        392. & - & 1766. 15. August & Novellara bei Modena. & Italien & 44$^\circ$ 48' N. 10$^\circ$ 45' O. & C. 251. & Wahrscheinlich nur ein vom Blitz zersprengter und geschmolzener Stein. \\ \hline
        393. & - & 1768. 22. (23.) (24.) Juli & Siarhi ($^\wedge$$^\wedge$$^\wedge$), Pudaturei Wolur ($^\wedge$$^\wedge$$^\wedge$) und Sendenfudi ($^\wedge$$^\wedge$$^\wedge$), sämtlich bei Tranquebar; Dekan. & Ost-Indien & Ungefähr 11$^\circ$ 0' N. 79$^\circ$ 57' O. & Schnurrer 2. 349. Knapp 2. 172 u. 182.* & Am hellen Mittage zündete vom Himmel gefallenes Feuer, wie Sternschnuppen, mehrere Gebäude an. \\ \hline
        394. & 6. & 1768. 13. September & Lucé en Maine, Arr. von St. Calais; Dép. de la Sarthe. & Frankreich & 47$^\circ$ 52' N. 0$^\circ$ 30' O. & C. 251. & Unter Donnerschlag und Getöse ein noch heißer Stein von 7 ½ Tb., der nach Paris gesandt ward. \\ \hline
        395. & 7. & 1768. - - & Aire en Artois; Dép. du Pas-de-Calais. & Frankreich & 50$^\circ$ 38' N. 2$^\circ$ 24' O. & C. 251. & 1 Stein von 8 Tb., ebenfalls nach Paris gesandt. \\ \hline
        396. & 20. & 1768. 20. November & Maurkirchen, SO. von Braunau, im osterr. Inn-Viertel. & Deutschland & 48$^\circ$ 12' N. 13$^\circ$ 7' O. & C. 252. & Unter starkem Krachen und Brausen 1 Stein von 38 Tb. \\ \hline
        397. & 5. & 1773. 17. November & Sena, NW. von Sigena (Sixena) in Aragonien. & Spanien & 41$^\circ$ 36' N. 0$^\circ$ 0' & C. 253. & Unter Krachen wie Kanonenschusse 1 noch heißer, nach Schwefel riechender Stein von 9 Tb., der nach Madrid gesandt ward. \\ \hline
        398. & 21. & 1775. 19. September & Rodach, NW. von Coburg; Thüringen. & Deutschland & 50$^\circ$ 21' N. 10$^\circ$ 46' O. & C. 254. & Unter Gewehrfeuerähnlichem Getöse ein Stein von 6 ½ Tb., welcher in Coburg war aufbewahrt worden. \\ \hline
        399. & 3. & 1775. (1776.) - - & Obruteza (Owrutsch, Owruez?); Gouv. Volhynien. & Russland & 51$^\circ$ 23' N. 28$^\circ$ 40' O. ? ? & C. 255. & Einige Steine, deren einer in einer Kirche aufbewahrt ward. \\ \hline
        400. & 26. & 1776. (1777.) - Januar & Sanatoglia, S. von Fabriano; Kirchenstaat. & Italien & 43$^\circ$ 15' N. 12$^\circ$ 54' O. & C. 255. & Unter vielem Geräusch Steine, denen von Siena ähnlich. \\ \hline
        401. & 6. & 1779. - - & Pettiswood, Hügel bei Mullingar; Grafschaft Westmeath. & Irland & 53$^\circ$ 31' N. 7$^\circ$ 19' W. & C. 255. & Unter Donnerschlag und Schwefeldampf ein Stein, von welchem 2 Bruchstucke 3 ½ Unze wogen. \\ \hline
        402. & - & 1779. 15. Juni & Ostrog Peter und Paul (Peter-Pauls Hafen). & Kamtschatka & 52$^\circ$ 30' N. 157$^\circ$ 20' O. & Cooks 3te Reise; 4. Fol. 182.* & Stein- und Staubregen wahrend eines Vulkan-Ausbruches (des Awatscha?) und wahrscheinlich nur in unmittelbarer Folge desselben. \\ \hline
        403. & - & 1780. - - & Lahore; Pendsjab. & Indien & - & RPG. 38. & Angeblicher Eisenfall. \\ \hline
        404. & 7. & 1780. 11. April & Beeston, 3 M. SW. von Nottingham. & England & 52$^\circ$ 55' N. 1$^\circ$ 10' W. & C. 256. & Steine aus einem Feuermeteor. \\ \hline
        405. & 1. & Um 1780. - - & Kinsdale, zwischen West-River-Mountain und Connecticut. & Nord-Amerika & ? & P. 2. 1824. 152. & Mehrere Eisenmassen nach einer Explosion. \\ \hline
        406. & 27. & 1782. - Juli & Turin; Piemont. & Italien & 45$^\circ$ 4' N. 7$^\circ$ 41' O. & C. 256. & Weißliche, kalkähnliche Masse aus einer Feuerkugel. \\ \hline
        407. & - & 1783. 18. August & ? & England & - & RPG. 40. & Angeblicher Steinregen. \\ \hline
        408. & 22. & 1785. 19. Februar & Im Wittmess (nicht Wittens), 1 ½ Stunde SW. von Eichstaedt. & Deutschland & 48$^\circ$ 52' N. 11$^\circ$ 10' O. & C. 257. v. Moll, Annalen 3. Fol. 251. & Nach heftigem Donnerschlag 1 Stein von 5 ½ Tb. \\ \hline
        409. & - & 1785. 13. August & Frankfurt a. M. & Deutschland & 50$^\circ$ 7' N. 8$^\circ$ 52' O. & P. 4. 1854. 431. Belli-Gontard 7. Fol. 68.* & Gleichzeitiger Brand zweier Hauser, von welchem man vermutet, dass er durch Meteorsteine sei veranlasst worden. \\ \hline
        410. & 4. & 1787. 13. Oktober & Schigailow und Lebedin, beide im Kreis Achtyrka; Gouv. Charkow. & Russland & Ungefähr 50$^\circ$ 17' N. 35$^\circ$ 10' O. Und 50$^\circ$ 33' N. 34$^\circ$ 50' O. & C. 257. & Unter prasselndem Getöse mehrere Steine, deren einer nach St. Petersburg gesandt worden. \\ \hline
        411. & - & 1788. 13. Juli & ? & Frankreich & - & A. 4. 194. & Angeblich mehrere Steine; vielleicht bloß Verwechselung mit No. 413: Barbotan 1790. 24. Juli? \\ \hline
        412. & - & 1789. Sommer & Worms; Rheinhessen. & Deutschland & 49$^\circ$ 38' N. 11$^\circ$ 22' O. & v. Dalberg Fol. 51.* & Feuerkugel mit donnerndem Getöse u. Mutmaßlichem Meteorsteinfall. \\ \hline
        413. & 8. & 1790. (nicht 1789.) 24. Juli & Barbotan, ONO. von Cazaubon; Depart. du Gers; und zwischen Creon u. Lagrange-de-Julliac in Armagnac; Dép. des Landes. & Frankreich & 43$^\circ$ 57' N. 0$^\circ$ 4' W. Und 43$^\circ$ 59' N. 0$^\circ$ 7' W. & C. 258. & Aus einem Feuermeteor viele Steine, darunter von 1 bis 50 Tb.; einer von 18 Tb. Ward nach Paris gesandt. \\ \hline
        414. & 28. & 1791. 17. Mai & Castel-Berardenga, ONO. von Siena; Toskana. & Italien & 43$^\circ$ 21' N. 11$^\circ$ 29' O. & C. 260. & Unter donnerähnlichem Getöse mehrere Steine aus einem Feuermeteor. \\ \hline
        415. & 29. & 1794. 16. Juni & Cosona, SO. von Siena und WNW. von Pienza; Lucignan d’Asso (Lucignanello? SO. von Siena, NNW. v. Cosona und S. von S. Giovanni d’Asso?); u. Pienza, SO. von Siena; sämtlich in Toskana.* & Italien & 43$^\circ$ 7' N. 11$^\circ$ 36' O. 43$^\circ$ 8' N. 11$^\circ$ 35' O. ? und 43$^\circ$ 5' N. 11$^\circ$ 41' O. & C. 261. Soldani 12, 32 u. 33. Tata 11 u. 12.* & Unter starker Explosion etwa 12 Steine aus einem Feuermeteor, deren größter 7 Tb. \\ \hline
        416. & - & 1794. 30. Juni & Zwischen Torre del Greco, Bosco und Torre dell’ Annunziata, SO. von Neapel. & Italien & Ungefähr 40$^\circ$ 50' N. 14$^\circ$ 22' O. & G. 6. 1800. 168. Soldani 189 bis 191. Tata 28 u. s. w.* & Steinregen aus einer dem Vesuv bei dessen Ausbruch entstiegenen Feuerkugel.* \\ \hline
        417. & 3. & 1795. 13. April & Provinz Carnawelpattu, 4 Meilen von Multetiwu, auf der Insel Ceylon. & Ost-Indien & Ungefähr 9$^\circ$ 15' N. 80$^\circ$ 50' O. & C. 262. & Unter donnerähnlichem Getöse mehrere noch heiße Steine, die dem Oberhaupte gebracht wurden. \\ \hline
        418. & 8. & 1795. 13. Dezember & Wold-Cottage, 9 M. NNO. von Great-Driffield; Yorkshire. & England & 54$^\circ$ 9' N. 0$^\circ$ 24' W. & C. 263. & Unter Pistolenschussähnlichem Getöse ein Stein von 56 Tb., den man in London sehen ließ. \\ \hline
        419. & 5. & 1796. 4. Januar & Belaja-Zerkwa (Weisskirchen); Gouv. Kiew. & Russland & 49$^\circ$ 50' N. 30$^\circ$ 6' O. & C. 264. & 1 großer feuriger Stein im geschmolzenen Zustand. \\ \hline
        420. & 6. & 1796. 19. Februar & Tasquinha bei Evora-Monte; Prov. Alemtejo. & Portugal & 38$^\circ$ 43' N. 7$^\circ$ 27' W. & C. 264. & Mit vielem Getöse ein Stein von 10 Tb. \\ \hline
        421. & 9. & 1798. 12. März & Sales, 1 ½ Stunde NW. von Villefranche bei Lyon; Dép. du Rhone. & Frankreich & 46$^\circ$ 3' N. 4$^\circ$ 37' O. & C. 265. & 1 Stein von 20 Tb. Aus einer Feuerkugel. \\ \hline
        422. & 4. & 1798. 13. (15.) Dezember & Krak-Hut, 14 engl. M. von Benares und 12 engl. M. von Juanpoor; Hindostan. & Ost-Indien & 25$^\circ$ 38' N. 83$^\circ$ 0' O. & C. 266. & Aus einer Feuerkugel unter 3 Explosionen und starkem Getöse mehrere Steine, darunter von 4 Unzen bis zu 10 Tb. \\ \hline
        423. & - & 1800. 1. April & Steeple-Bumstead, 2 M. S. von Haverhill und 23 M. N. von Chelmsfort; Essex. & England & 52$^\circ$ 3' N. 0$^\circ$ 27' O. & C. 139. & Mutmaßlicher Meteorsteinfall. Eine Feuerkugel schlug unter Explosion in die Erde, ohne dass man jedoch weiter nach einem Stein gesucht hatte. \\ \hline
        424. & - & 1800. (1799.) 5. April & Baton-Rouge am Mississippi; Louisiana. & Nord-Amerika & 30$^\circ$ 23' N. 91$^\circ$ 23' W. & C. 139. G. 13. 1803. 315. & Desgleichen. \\ \hline
        425. & 3. & 1801. - - & Isle-des-Tonneliers bei Isle-de-France. & Indischer Ocean & 20$^\circ$ 30' S. 58$^\circ$ 0' O. & C. 268. & 3 Steine aus einer Feuerkugel mit Explosion. \\ \hline
        426. & - & 1801. 23. Oktober & Boury St. Edmunds in Suffolk; NNW. von Colchester in Essex. & England & 52$^\circ$ 15' N. 0$^\circ$ 40' O. & C. 141. & Herabgefallene Stucke einer Feuerkugel zündeten ein Haus an. \\ \hline
        427. & 9. & 1802. Mitte September & Am Loch-Tay. & Schottland & Ungefähr 56$^\circ$ 30' N. 4$^\circ$ 10' W. & C. 268. & Niederfall von Steinen, deren mehrere gefunden wurden. \\ \hline
        428. & 5. & 1802. - - & Allahabad; Hindostan. & Ost-Indien & 25$^\circ$ 23' N. 81$^\circ$ 49' O. & P. 24. 1832. 223. & Steine, denen von Mhow (1827) ganz ähnlich. \\ \hline
        429. & 10. & 1803. 26. April & l’Aigle, zwischen Evreux und Alençon; Dép. de l’Orne. & Frankreich & 48$^\circ$ 45' N. 0$^\circ$ 38' O. & C. 269. & Aus einem Feuermeteor unter heftiger Explosion 2000-3000 Steine von nur 2 Quäntchen bis zu 17 Tb. \\ \hline
        430. & 10. & 1803. 4. Juli & East-Norton, 9 M. NNO. von Market-Harboro’; Leicestershire. & England & 52$^\circ$ 25' N. 0$^\circ$ 51' W. & C. 272. & Stein aus einer Feuerkugel, welcher Teile eines Hauses zerstörte. \\ \hline
        431. & 11. & 1803. 8. Oktober & Saurette bei Apt; Dép. de Vaucluse. & Frankreich & Ungefähr 43$^\circ$ 52' N. 5$^\circ$ 23' O. & C. 273. & Unter heftigem Krachen 1 Stein von über 7 Tb., welcher nach Paris kam. \\ \hline
        432. & 23. & 1803. 13. Dezember & St. Nicolas, WNW. v. Eggenfelden; Bayern. & Deutschland & 48$^\circ$ 27' N. 12$^\circ$ 36' O. & C. 273. & Unter 9-10 fachem Knalle ein noch heißer Stein von 3 ¼ Tb., der nach München kam. \\ \hline
        433. & 11. & 1804. 5. April & High-Possil, 3 M. N. Von Glasgow. & Schottland & 55$^\circ$ 54' N. 4$^\circ$ 18' W. & C. 275. & Unter kanonenähnlichem Getöse 2 Bruchstucke eines Steines. \\ \hline
        434. & 3. & Zwischen 1804 und 1807. - - & Dortrecht. & Holland & 51$^\circ$ 48' N. 4$^\circ$ 40' O. & C. 275. & 1 feuriger Stein fiel unter vielem Getöse in die Stadt. \\ \hline
        435. & - & 1805. 17. Februar & Sigmaringen. & Deutschland & 48$^\circ$ 5' N. 9$^\circ$ 13' O. & Schnurrer 2. 463. & Erderschutterung mit starkem Knall, welche für die Folge eines Meteorsteinfalles gehalten wurde. \\ \hline
        436. & 1. & 1805. 25. März & Doroninsk, im Werneudinski’schen Distrikte, nahe am Indoga; Gouv. Irkutsk. & Sibirien & 50$^\circ$ 30' N. 112$^\circ$ 20' O. & C. 276. & Unter Getöse ein glühender Stein in 2 Bruchstucken von 2 ½ und 7 Tb. \\ \hline
        437. & 8. & 1805. - Juni & Konstantinopel. & Europäischen Türkei & 41$^\circ$ 0' N. 28$^\circ$ 58' O. & C. 278. & Mehrere nach Schwefel riechende Steine fielen in die Stadt. \\ \hline
        438. & 30. & 1805. - November & Asco, OSO. von Calvi. & Korsika & 42$^\circ$ 28' N. 9$^\circ$ 2' O. & P. 4. 1854. 11. & 1 Stein, der in der Kirche aufbewahrt ward. \\ \hline
        439. & 12. & 1806. 15. März & St. Etienne-de-Lolm und Valence, beide SO. von Alais; Dép. du Gard. & Frankreich & 44$^\circ$ 0' N. 4$^\circ$ 15' O. & C. 278. & Unter Explosionen und donnerndem Getöse 2 noch heiße Steine von 4 und 8 Tb. \\ \hline
        440. & 12. & 1806. 17. Mai & Basingstoke; Hantshire. & England & 51$^\circ$ 17' N. 1$^\circ$ 6' W. & C. 280. & Unter Donner 1 noch heißer Stein von 2 ½ Tb. \\ \hline
        441. & 6. & 1807. 13. März & Timochin, Kreis Juchnow, Gouv. Smolensk. & Russland & Ungefähr 54$^\circ$ 48' N. 35$^\circ$ 10' O. & C. 280. & Unter donnerndem Getöse 1 Stein von 140 (160) Tb., der nach Petersburg kam. \\ \hline
        442. & 2. & 1807. 14. Dezember & Weston, Fairfield-County; Connecticut. & Nord-Amerika & 41$^\circ$ 15' N. 73$^\circ$ 34' W. & C. 282. & Aus einer Feuerkugel unter 3-maligen Explosionen viele Steine von zusammen etwa 300 Tb., der größte von 35 Tb. \\ \hline
        443. & 31. & 1808. 19. April & Borgo-San-Donino und Pieve di Casignano, S. von Borgo-San-Donino; Parma. & Italien & 44$^\circ$ 47' N. 10$^\circ$ 4' O. 44$^\circ$ 52' N. 10$^\circ$ 4' O. & C. 284. & Unter 2 Explosionen mehrere Steine, deren einige nach Parma und Paris kamen. \\ \hline
        444. & 46. & 1808. 22. Mai & Stannern, S. von Iglau. & Mahren & 49$^\circ$ 18' N. 15$^\circ$ 36' O. & C. 286. & Aus einer Feuerkugel unter heftigem Knalle 200 bis 300 Steine, im Gesamtgewicht von etwa 150 Tb., meist von 2 ½ Quäntchen bis zu 3 Tb., deren mehrere nach Wien kamen; der größte 11 Tb. \\ \hline
        445. & 47. & 1808. 3. September & Stratow u. Wustra, beide OSO. von Lissa; Kreis Bunzlau. & Böhmen & 52$^\circ$ 12' N. 14$^\circ$ 54' O. Und 50$^\circ$ 10' N. 14$^\circ$ 53' O. & C. 289. & Unter vielem Getöse mehrere Steine von 2 ½ bis 5 Tb. \\ \hline
        446. & 6. & 1808. - - & Mooradabad bei Delhi; Hindostan. & Ost-Indien & 28$^\circ$ 50' N. 78$^\circ$ 48' O. & P. 24. 1832. 223. & Steine, denen von Allahabad (1802) ganz ähnlich. \\ \hline
        447. & 7. & 1809. - - & Kikina, Wiasemsk’er Kreis; Gouv. Smolensk. & Russland & Ungefähr 55$^\circ$ 17' N. 34$^\circ$ 13' O. & W. 1860. & 1 Stein im Wiener Hofkabinet. \\ \hline
        448. & 3. & 1809. 17. (20.) Juni & Zwischen Block-Island und St. Bart; Küste v. Nord-Amerika. & Atlantisches Meer & 30$^\circ$ 58' N. 70$^\circ$ 25' W. & C. 290. & Wahrend eines Gewitters 1 Stein auf ein Schiff und mehrere ins Meer; der Erstere ward aufbewahrt. \\ \hline
        449. & 4. & 1810. 4. (7.) (30.) Januar & Caswell-County (Hauptstadt: Yanceyville); North-Carolina. & Nord-Amerika & Ungefähr 36$^\circ$ 25' N. 79$^\circ$ 30' W. & C. 291. & Unter Explosion mehrere Steine, darunter 1 noch heißer mit magnetischer Polarität. \\ \hline
        450. & 1. & 1810. 20. (21.) April & Hügel von Tacavita, 1 Meile von Santa-Rosa; Neu-Granada. & Sud-Amerika & 5$^\circ$ 40' N. 73$^\circ$ 20' W. & A. 4. 196. B. 117 u. 130. & Eisenmasse von 15 Ctr. \\ \hline
        451. & 7. & 1810. Mitte Juli & Shabad, 30 engl. M. N. von Futty-Ghur (oder v. Futtehpore?), jenseits des Ganges; Hindostan. & Ost-Indien & ? & C. 292. & Aus einer Feuerkugel 1 Stein, welcher aufbewahrt ward. \\ \hline
        452. & 13. & 1810. Mitte August & Mooresfort (Moores Fort); Grafschaft Tipperary. & Irland & 52$^\circ$ 28' N. 8$^\circ$ 11' W. & C. 292. & Unter donnerähnlichem Getöse 1 noch heißer Stein von 7 ¾ Tb. \\ \hline
        453. & 13. & 1810. 23. November & Charsonville, WNW. von Orleans; Dép. du Loiret. & Frankreich & 47$^\circ$ 56' N. 1$^\circ$ 35' O. & C. 293. & Unter donnerndem Getöse aus einer Feuerkugel 3 Steine, wovon 2 von 20 und 40 Tb. Gefunden wurden. \\ \hline
        454. & 9. & 1810. 28. November & Zwischen der Insel Cerigo und dem Cap Matapan. & Griechenland & Ungefähr 36$^\circ$ 10' N. 22$^\circ$ 40' O. & P. 24. 1832. 223. & In das Meer: Steinfall aus einer Feuerkugel. \\ \hline
        455. & - & 1810. - - & ? & Frankreich & - & RPG. 40. & Angeblicher Steinfall; wahrscheinlich einerlei mit No. 453: Charsonville. \\ \hline
        456. & 8. & 1811. 12. (13.) März & Kuleschowka, Kreis Romen; Gouv. Pultawa. & Russland & Ungefähr 50$^\circ$ 43' N. 33$^\circ$ 45' O. & C. 296. & Unter 3 Explosionen 1 noch heißer Stein von 13 (15) Tb. \\ \hline
        457. & 7. & 1811. 8. Juli & Berlanguillas, zwischen Aranda und Roa; Alt-Kastilien. & Spanien & Ungefähr 41$^\circ$ 41' N. 3$^\circ$ 48' W. & C. 296. & Unter donnerndem Krachen mehrere noch heiße Steine, deren einer von 4 bis 6 Tb. nach Paris gesandt ward. \\ \hline
        458. & 8. & 1811. 23. November & Panganoor in Dekan. & Ost-Indien & 13$^\circ$ 22' N. 78$^\circ$ 38' O. & RPG. 36. P. 4. 1854. 396. & Niederfall einer Eisenmasse. \\ \hline
        459. & 14. & 1812. 10. April & Burgau (le Bourgaut), 6 Stunden von Toulouse, und 5 andere Orte, sämtlich bei Grenade, Dép. de la Haute-Garonne; und Las-Pradere bei Savenes, Dép. de Tarn et Garonne. & Frankreich & 43$^\circ$ 47' N. 1$^\circ$ 9' O. Und ungefähr 43$^\circ$ 50' N. 1$^\circ$ 11' O. & C. 297. Bigot de Morogues Fol. 275. & Unter donnerndem Getöse mehrere Steine aus einer Feuerkugel; die gefundenen nur von 6-8 Unzen. \\ \hline
        460. & 24. & 1812. 15. April & Erxleben, zwischen Magdeburg und Helmstadt; Preuss. Sachsen. & Deutschland & 52$^\circ$ 13' N. 11$^\circ$ 14' O. & C. 299. & Unter kanonenähnlichem Getöse ein Stein von 4 ½ Tb. \\ \hline
        461. & 15. & 1812. 5. August & Chantonnay, zwischen Nantes und la Rochelle; Dép. de la Vendée. & Frankreich & 46$^\circ$ 40' N. 1$^\circ$ 5' W. & C. 301. & Aus einem Feuermeteor unter starker Explosion 1 Stein von 69 Tb. \\ \hline
        462. & 32. & 1813. 14. März & Cutro, zwischen Crotone und Catanzaro; Kalabrien. & Italien & 38$^\circ$ 58' N. 17$^\circ$ 2' O. & C. 303 u. 377. & Aus einer roten Wolke unter Donnerschlagen roter Regen, Staub und mehrere Steine. \\ \hline
        463. & 14. & 1813. - Juli (August) & Malpas, SSO. von Chester; Chestershire. & England & 53$^\circ$ 4' N. 2$^\circ$ 48' W. & C. 303. & Aus einer lichten Wolke viele heiße, anfangs noch weiche Steine. \\ \hline
        464. & 15. & 1813. 10. September & Adair (Adare), Faha, Scouph und Brasky; sämtlich in der Grafschaft Limerick. & Irland & Ungefähr 52$^\circ$ 30' N. 8$^\circ$ 42' W. & C. 303. & Aus einer Wolke unter kanonenähnlichem Getöse noch heiße und nach Schwefel riechende Steine von 17, 24 u. 65 Pfund. \\ \hline
        465. & 9. & 1813. 13. Dezember (1814. Mitte März) ? ? ? & Lontalax bei Switaipola, NNO. von Friedrichsham, Gouv. Wiborg; Finnland. & Russland & Ungefähr 61$^\circ$ 13' N. 27$^\circ$ 49' O. & C. 304. & Mehrere Steine. \\ \hline
        466. & 16. & Wahrscheinlich 1813; - - jedenfalls vor 1819. & Pulrose; Insel Man. & England & Ungefähr 54$^\circ$ 15' N. 4$^\circ$ 30' W. & G. 68. 1821. 333. & 1 Stein. \\ \hline
        467. & 10. & 1814. 15. Februar & Distrikt Bachmut; Gouv. Jekaterinoslaw. & Russland & Ungefähr 48$^\circ$ 34' N. 37$^\circ$ 52' O. & C. 304. & Unter Explosion 1 noch heißer Stein von 40 Pfund in zwei Bruchstucken, deren eines von 20 Pfund nach Charkow gesandt ward. \\ \hline
        468. & 16. & 1814. 5. September & Monclar, NNW. von Agen; und le Temple, S. von Monclar und O. von Tonneins; beide im Dép. du Lot et Garonne.* & Frankreich & 44$^\circ$ 26' N. 0$^\circ$ 31' O. Und 44$^\circ$ 23' N. 0$^\circ$ 31' O. & C. 305. Schnurrer 2. 523. & Unter starken Explosionen mehrere Steine, deren größter etwa 18 Pfund. \\ \hline
        469. & 9. & 1814. 5. November & Bezirke Lapk, Bhaweri, Chal und Kaboul, Prov. Doab; Hindostan. & Ost-Indien & Ungefähr 27$^\circ$ 0' N. 80$^\circ$ 0' O. & C. 306. & Unter donnerndem Getöse viele Steine bis zu 30 Pfund; 25 derselben wurden gesammelt. \\ \hline
        470. & 10. & 1815. 18. Februar & Dooralla im Gebiet des Pattialah Rajah; Hindostan. & Ost-Indien & Ungefähr 30$^\circ$ 30' N. 76$^\circ$ 4' O. & G. 68. 1821. 333. & Unter kanonenähnlicher Explosion 1 Stein von 25 Pfund, der nach London kam. \\ \hline
        471. & 17. & 1815. 3. Oktober & Chassigny, 4 M. SSO. von Langres; Dép. de la Haute-Marne. & Frankreich & 47$^\circ$ 43' N. 5$^\circ$ 23' O. & C. 307. & Unter rollendem Getöse und Pfeifen 1 Stein in etwa 60 Bruchstucken von zusammen 8 Pfund. \\ \hline
        472. & 17. & 1816. Ende Juli oder Anf. August & Glastonbury, SW. von Wells; Somersetshire. & England & 51$^\circ$ 9' N. 2$^\circ$ 42' W. & C. 309. & Unter donnerndem Getöse 1 noch heißer Stein mit schwefligem Geruch. \\ \hline
        473. & - & 1816. - - & Confolens; Dép. de l’Ande (oder Conffoulens, Canton de Carcassone; im Dép. de l’Aude?). & Frankreich & ? & A. 4. 199. & Angeblicher Meteorsteinfall (nach der France pittoresque, tome 1.). \\ \hline
        474. & - & 1817. 2. (3.) März & ? & Baltisches Meer & - & A. 4. 149. & Feuerkugel mit mutmaßlichem Steinfall. \\ \hline
        475. & - & 1818. 15. Februar & Limoges; Dép. de la Haute-Vienne. & Frankreich & 45$^\circ$ 49' N. 1$^\circ$ 12' O. & G. 60. 1818. 251. & Angeblicher, doch zweifelhafter Meteorsteinfall aus einer Feuerkugel. \\ \hline
        476. & 11. & 1818. 10. (11.) April & Zjaborzyka (Saborytz oder Zabortsch), am Slucz (Slutsch); Gouv. Volhynien. & Russland & 50$^\circ$ 15' N. 27$^\circ$ 30' (44') O. & P. 2. 1824. 153. & Meteorsteinfall; der Stein ward von Laugier analysiert. \\ \hline
        477. & 10. & 1818. - Juni & Seres in Macedonien. & Europäischen Türkei & 41$^\circ$ 3' N. 23$^\circ$ 33' O. & P. 34. 1835. 340. P. 4. 1854. 427. & 1 Stein von 15 Pfund, welcher nach Wien kam. \\ \hline
        478. & 12. & 1818. 10. August & Slobodka, Kreis Juchnow; Gouv. Smolensk. & Russland & Ungefähr 54$^\circ$ 48' N. 35$^\circ$ 10' O. & C. 310. & 1 Stein von 7 Pfund. \\ \hline
        479. & 33. & 1819. Ende April & Massa Lubrense (Massa oder Massa di Sorento), Fürstentum Salerno; Neapel. & Italien & 40$^\circ$ 38' N. 14$^\circ$ 18' O. & G. 71. 1822. 359. & Nach starken Donnerschlagen wurden in frisch entstandenen Kluften u. Gruben viele Steine mit Merkmalen des Feuers gefunden. \\ \hline
        480. & 18. & 1819. 13. Juni & Barbézieux, Dép. de la Charente; und Jonzac, Dép. de la Charente-Inférieure. & Frankreich & 45$^\circ$ 23' N. 0$^\circ$ 11' W. Und 45$^\circ$ 26' N. 0$^\circ$ 27' W. & G. 63. 1819. 24. & Nach 3 donnerähnlichen Schlagen viele Steine, deren größte von 4 u. 6 Pfund. \\ \hline
        481. & - & 1819. 24. Juli & Im Staate Ohio. & Nord-Amerika & - & P. 2. 1824. 163. & Große Feuerkugel mit starker Explosion und vermutetem Steinfall in die Urwälder. \\ \hline
        482. & - & 1819. 5. September & Studein, Herrschaft Teltsch. & Mahren & Ungefähr 49$^\circ$ 10' N. 15$^\circ$ 27' O. & G. 68. 1821. 353. & Regen von Erde und kleinen Steinchen; Letztere Quarzkörnern mit etwas Lehm und Glimmer-Flimmern ähnlich. \\ \hline
        483. & 25. & 1819. 13. Oktober & Politz, NNW. v. Kostritz bei Gera; Reuss. & Deutschland & 50$^\circ$ 57' N. 12$^\circ$ 2' O. & G. 63. 1819. 217. & 1 Stein von 7 Pfund. \\ \hline
        484. & - & 1820. 5. April & Auf ein Schiff; etwa 10 Langengrade von Antigua. & Atlantisches Meer & 20$^\circ$ 10' N. 51$^\circ$ 50' W. & P. 24. 1832. 223. & Zweifelhafter Steinfall; der nach Wien gesandte Stein war ein gewöhnlicher Kalkstein. \\ \hline
        485. & 5. & 1820. 22. Mai & Oedenburg; Gespanschaft Oedenburg. & Ungarn & 47$^\circ$ 41' N. 16$^\circ$ 36' O. & G. 68. 1821. 337. & Unter starkem Donnerschlag ein noch heißer, nach Schwefel riechender Stein von etwa 1/4 Pfund. \\ \hline
        486. & 13. & 1820. 12. Juli & Lasdany bei Lixna, N. von Dunaburg; Gouv. Witepsk. & Russland & Ungefähr 56$^\circ$ 0' N. 26$^\circ$ 25' O. & G. 68. 1821. 337. & Aus einem Feuermeteor mehrere Steine, davon einer von 40 Pfund. \\ \hline
        487. & 34. & 1820. 29. November & Cosenza; Kalabrien. & Italien & 39$^\circ$ 15' N. 16$^\circ$ 18' O. & CR. 11. 1841. 357. & Feuermeteor mit Steinfall. \\ \hline
        488. & - & 1821. 5. März & Greifswalder Kreis in Pommern. & Deutschland & Ungefähr 54$^\circ$ 4' N. 13$^\circ$ 20' O. & G. 71. 1822. 360. & Mutmaßlicher Meteorsteinfall; doch ist nicht nach Steinen gesucht worden. \\ \hline
        489. & 19. & 1821. 15. Juni & Juvinas, NNW. von Aubenas bei Privas; Dép. de l’Ardeche. & Frankreich & 44$^\circ$ 42' N. 4$^\circ$ 21' O. & G. 71. 1822. 360. & Aus einer großen Feuerkugel 1 Stein von über 220 Pfund und mehrere kleinere. \\ \hline
        490. & 18. & 1821. 21. Juni & Grafschaft Mayo. & Irland & Ungefähr 54$^\circ$ 0' N. 9$^\circ$ 30' W. & G. 72. 1822. 436. & Hagel mit Metallkernen. \\ \hline
        491. & 20. & 1822. 3. Juni & Angers; Dép. de Maine et Loire. & Frankreich & 47$^\circ$ 28' N. 0$^\circ$ 34' W. & G. 71. 1822. 361. & Aus einer Feuerkugel mehrere Steine, deren größter von 30 Unzen. \\ \hline
        492. & - & 1822. 17. Juni & Catania. & Sicilien & 37$^\circ$ 25' N. 15$^\circ$ 6' O. & P. 4. 1854. 427. & Feuerkugel, die eine Feuersbrunst verursachte. \\ \hline
        493. & 11. & 1822. 7. August & Kadonah, Distrikt von Agra; Hindostan. & Ost-Indien & Ungefähr 27$^\circ$ 12' N. 78$^\circ$ 3' O. & P. 4. 1854. 33. & Meteorsteinfall. \\ \hline
        494. & - & 1822. 10. September & Carlstad. & Schweden & 59$^\circ$ 23' N. 13$^\circ$ 32' O. & G. 75. 1823. 230. & Starke Explosion in der Luft, und man will „an verschiedenen Orten“ Meteorsteinegefunden haben. \\ \hline
        495. & 21. & 1822. 13. September & la Baffe, O. von Epinal; Vogesen. & Frankreich & 48$^\circ$ 9' N. 6$^\circ$ 35' O. & G. 75. 1823. 231. & Wahrend eines Gewitters 1 Stein in mehreren Bruchstucken, welcher nach Paris kam. \\ \hline
        496. & 12. & 1822. 30. November & Rourpour bei Futtehpoor, unweit Allahabad, Provinz Doab; Hindostan. & Ost-Indien & Ungefähr 25$^\circ$ 57' N. 80$^\circ$ 50' O. & P. 18. 1830. 179. WA. 41. 1860. 747. & Aus einer Feuerkugel unter donnerndem Getöse mehrere heiße Steine, deren größter 22 Pfund. \\ \hline
        497. & 5. & 1823. 7. August & Nobleborough, Lincoln-County; Maine. & Nord-Amerika & 44$^\circ$ 5' N. 69$^\circ$ 40' W. & P. 2. 1824. 153. & Unter Getöse wie ein Pelotonfeuer 1 Stein von 4 bis 6 Pfund in Bruchstucken. \\ \hline
        498. & 35. & 1824. 13. (15.) Januar & Renazzo (Arenazzo), N. von Cento ei Ferrara; Kirchenstaat. & Italien & 44$^\circ$ 47' N. 11$^\circ$ 18' O. & P. 2. 1824. 155. & Unter Lichterscheinung und Getöse viele Steine, deren größter 12 Pfund. \\ \hline
        499. & 2. & 1824. 18. Februar & Tounkin (Tunginsk od. Tunga), 216 Werste WSW. von Irkutsk. & Sibirien & 51$^\circ$ 50' N. 102$^\circ$ 50' O. & P. 24. 1832. 224. & Unter donnerndem Getöse 1 Stein von 5 Pfund, der nach Irkutsk gebracht ward. \\ \hline
        500. & 48. & 1824. 14. Oktober & Praskoles, OSO. von Zebrak, NO. von Horzowitz; Kreis Beraun. & Böhmen & 49$^\circ$ 52' N. 13$^\circ$ 55' O. & P. 6. 1826. 28. & Unter heftigem Getöse 1 Stein von 4 Pfund in 3 Bruchstucken, deren 2 nach Prag kamen. \\ \hline
        501. & - & 1824. 20. Oktober & Sterlitamansk am Bjajaga, 200 Werste von Orenburg. & Asiatisches Russland & 53$^\circ$ 30' N. 56$^\circ$ 5' O. & P. 6. 1826. 30. v. Humboldt Kosm. 1 136. & Bezweifelter Niederfall von Hagel mit Metallkernen. \\ \hline
        502. & 13. & 1825. 16. Januar & Oriang in Malwa, N. vom oberen Lauf des Nerbada; Hindostan. & Ost-Indien & Ungefähr 23$^\circ$ 0' N. 79$^\circ$ 0' O. & P. 6. 1826. 32. & Aus einem Feuerball mehrere noch heiße Steine, deren einer einen Mann tötete. \\ \hline
        503. & 6. & 1825. 10. Februar & Nanjemoy, Charles-County; Maryland. & Nord-Amerika & 38$^\circ$ 28' N. 77$^\circ$ 16' W. & P. 6. 1826. 33. & Unter starker Explosion 1 Stein von 16 Pfund. \\ \hline
        504. & 19. & 1825. 12. Mai & Bayden, NW. von Hungerford; Wiltshire. & England & 51$^\circ$ 30' N. 1$^\circ$ 36' W. & P. 8. 1826. 49. & Eisenmasse, die in den Besitz eines Londoner Mineralienhandlers kam. \\ \hline
        505. & - & 1825. 5. Juli & Torresilla de Carneros (Torricellas dal Camp). & Spanien & 41$^\circ$ 30' N. 5$^\circ$ 0' W. (?) & P. 6. 1826. 31. & Steinregen in Stucken von 4 bis 17 Loth; doch ungewiss, ob nicht bloßer Hagel. \\ \hline
        506. & - & 1826. [1825.] 28. Juli & Chiroky ($^\wedge$$^\wedge$$^\wedge$), unweit Cherson. & Russland & Ungefähr 46$^\circ$ 40' N. 32$^\circ$ 40' O. & P. 6. 1826. 31. & Wahrend eines Hagels einige 7 Pfund schwere Luftsteine; doch ungewiss, ob nicht bloßer Hagel. \\ \hline
        507. & 1. & 1825. 14. September & Hanaruru (Honolulu); Sandwichs-Insel Oahu (Waohoo). & Stilles Weltmeer & 21$^\circ$ 30' N. 158$^\circ$ 0' W. & P. 24. 1832. 225. & Aus einer schwarzen Wolke unter starkem Krachen 2 noch warme Steine, jeder von etwa 15 Pfund. \\ \hline
        508. & - & 1826. 15. März & Lugano; Canton Tessin. & Schweiz & 46$^\circ$ 0' N. 8$^\circ$ 56' O. & P. 18. 1830. 316. & Feuermeteor mit heftiger Explosion und mutmaßlichem Steinfall; die Steine wurden gesucht, aber nicht gefunden. \\ \hline
        509. & 14. & 1826. 19. Mai & Distrikt Paulowgrad; Gouv. Jekaterinoslaw. & Russland & Ungefähr 48$^\circ$ 32' N. 35$^\circ$ 52' O. & P. 18. 1830. 185. & 1 Stein von 80 Pfund. \\ \hline
        510. & 7. & 1826. (1827.) Sommer & Waterloo, Seneca-County; New-York. & Nord-Amerika & 42$^\circ$ 54' N. 77$^\circ$ 8' W. & P. 88. 1853. 176. & 1 etwa zweipfündiges Bruchstuck eines Steines, der in eine Mahle eingedrungen. \\ \hline
        511. & - & 1826. - August & Berg Galaplau ($^\wedge$$^\wedge$$^\wedge$); Dép. du Lot et Garonne. & Frankreich & - & G. 18. 1830. 185. & Bezweifelter Meteorsteinfall während eines Gewitters. \\ \hline
        512. & 8. & 1826. - September & Waterville, Kennebec-County; Maine. & Nord-Amerika & 44$^\circ$ 35' N. 69$^\circ$ 55' W. & P. 4. 1854. 24. & Steinbruchstucke aus einer Feuerkugel. \\ \hline
        513. & - & 1826. - - & Georgia. & Nord-Amerika & - & Athenaeum 1836. 803. (RPG.) & Meteorsteinfall, durch welchen mehrere Menschen sollen getötet worden sein. \\ \hline
        514. & 14. & 1827. 27. Februar & Mhow (Mow), Distrikt von Azim-Gesh, NNO. von Ghazeepoor; Hindostan. & Ost-Indien & 25$^\circ$ 57' N. 83$^\circ$ 36' O. & P. 24. 1832. 226. & Unter donnerndem Getöse 4-5 Stein-Bruchstucke, deren größtes von 3 Pfund, und deren eines einen Menschen tötete. \\ \hline
        515. & 9. & 1827. 9. (22.) Mai & Drake-Creek, 18 M. von Nashville, Davidson-County; Tennessee. & Nord-Amerika & ungefähr 36$^\circ$ 9' N. 87$^\circ$ 0' W. & P. 24. 1832. 226. & Unter donnerndem Getöse mehrere Steine, deren größter 11 Pfund. \\ \hline
        516. & - & 1827. 9. (22.) Mai & Sumner-County; Tennessee. & Nord-Amerika & ungefähr 36$^\circ$ 25' N. 86$^\circ$ 40' W. & B. 90. Shepard, Rep. on Am. Met. 18. & Wahrscheinlich einerlei mit dem Vorstehenden. \\ \hline
        517. & - & 1827. - August & Provinz Kuli-Schu (Kou-li-chou, Kou-tchou oder Louan-tcheou), Bezirk Young-p'ing-fou; Provinz Pe-tchi-li. & China & 39$^\circ$ 48' N. 118$^\circ$ 50' O. & P. 18. 1830. 185. EB. 85 u. 119. & Nach Zeitungsnachrichten ein Meteorstein von ungewöhnlicher Große. \\ \hline
        518. & - & 1827. (1828.) 8. August & Awatscha bei Petropawlowsk (Peter-Pauls-Hafen). & Kamtschatka & 53$^\circ$ 0' N. 158$^\circ$ 25' O. & Leonhard, Zeitschrift fur Min. 1828. 1.491. (Zeitungsnachricht.) & Aus einer Wolke über dem verloschenen Feuerberge Awatscha unter starkem Schwefeldunst ein heftiger Sandregen. \\ \hline
        519. & 15. & 1827. 5. (8.) Oktober & Kuasti-Knasti, 2 Stunden von Bialystock; Russisch-Polen. & Russland & ungefähr 53$^\circ$ 12' N. 23$^\circ$ 10' O. & P. 18. 1830. 185. & Aus einer schwarzen Wolke unter starkem Getöse mehrere Stein, deren größter 4 Pfund. \\ \hline
        520. & 11. & 1828. - Mai & Tscheroi, zwischen Widdin und Krajowa. & Europäischen Türkei & ungefähr 44$^\circ$ 25' N. 23$^\circ$ 25' O. & P. 34. 1835. 341. & Unter Orkan und Hagel 1 Stein; Anhydrit. \\ \hline
        521. & 10. & 1828. 4. Juni & 7 M. SW. von Richmond, Henrico-(nicht Chesterfield-)County; Virginia. & Nord-Amerika & 37$^\circ$ 32' N. 77$^\circ$ 35' W. & P. 17. 1829. 380. & 1 Stein von 4 Pfund. \\ \hline
        522. & 20. & 1828. - August & Allport, 5 M. NNW. von Castleton; Derbyshire. & England & 53$^\circ$ 24' N. 1$^\circ$ 48' W. & P. 4. 1854. 43. & Unter lautem explodierendem Geräusch viele Steine aus Schwefel, Kohle und Eisenoxyd bestehend. \\ \hline
        523. & 11. & 1829. 8. Mai & Forsyth, Monroe-County; Georgia. & Nord-Amerika & 33$^\circ$ 0' N. 84$^\circ$ 13' W. & P. 24. 1832. 227. & Unter starker Detonation 1 Stein von 36 Pfund. \\ \hline
        524. & - & 1829. - Juli & ? & Nord-Amerika & - & Thomson, Met. 326.* & Ein Indianer ward von 1 Meteorstein getötet. \\ \hline
        525. & 12. & 1829. 14. August & Deal bei Long-Br & Nord-Amerika & ungefähr 40$^\circ$ 17' N. 74$^\circ$ 12' O. & P. 24. 1832. 228. & Aus einem Feuermeteor unter Explosion mehrere Steine. \\ \hline
        526. & 16. & 1829. 9. September & Krasnoi-Ugol, Kreis Saposhok; Gouv. Rjasan. & Russland & ungefähr 53$^\circ$ 56' N. 40$^\circ$ 28' O. & P. 24. 1832. 228. & Unter donnerndem Getöse mehrere Steine, deren einer nach St. Petersburg kam. \\ \hline
        527. & - & 1829. 19. November & Prag. & Böhmen & 50$^\circ$ 5' N. 14$^\circ$ 25' O. & P. 24. 1832. 229. & Mikroskopisch-kristallisierte, nach Schwefel riechende Masse aus einer Feuerkugel. \\ \hline
        528. & 21. & 1830. 15. Februar & Launton, 2 M. O. von Bicester; Oxfordshire. & England & 51$^\circ$ 54' N. 1$^\circ$ 9' W. & P. 54. 1841. 291. & 1 Stein von 2 1/2 Pfund, im Besitz von D. J. Lee, Colworthhouse, Bedfordshire. \\ \hline
        529. & 22. & 1831. 18. Juli & Vouillé, WNW. von Poitiers; Dép. de la Vienne. & Frankreich & 46$^\circ$ 37' N. 0$^\circ$ 8' O. & P. 34. 1835. 341. & 1 Stein von 40 Pfund, davon Stucke nach Paris kamen. \\ \hline
        530. & 49. & 1831. 9. September & Znorow, SW. von Wessely; Kr Hradisch. & Mahren & 48$^\circ$ 54' N. 17$^\circ$ 21' O. & P. 34. 1835. 342. & Unter Donnerschlagen ein noch warmer Stein von 6 1/2 Pfund, der nach Wien kam. \\ \hline
        531. & - & 1833. 16. Juli & Nachratschinsk ($^\wedge$$^\wedge$$^\wedge$), 300 Werste von Tobolsk. & Sibirien & - & P. 34. 1835. 342. & Unter heftigem Regen und Hagel auch kleine viereckige Steine; vielleicht ebenfalls nur Hagel? \\ \hline
        532. & - & 1833. 20. November & Pressburg. & Ungarn & 48$^\circ$ 12' N. 17$^\circ$ 8' O. & P. 34. 1835. 350. & Feuerkugel mit Explosion und vermutlichem Meteorsteinfall; doch keine Steine gefunden. \\ \hline
        533. & 50. & 1833. 25. November & Blansko, N. von Brunn und SSW. von Boskowitz. & Mahren & 49$^\circ$ 20' N. 16$^\circ$ 38' O. & P. 34. 1835. 343. & Aus einem Feuermeteor unter anhaltendem Donnern 3 Stein. \\ \hline
        534. & 8. & 1833. Ende November (1834. Ende April) & Kandahar. & Afghanistan & 32$^\circ$ 40' N. 65$^\circ$ 15' O. & P. 4. 1854. 33. & Starker Meteorsteinregen, wobei ein Mann getötet ward. \\ \hline
        535. & 17. & 1833. 27. Dezember & Okniny (Okaninah) bei Kremenetz; Gouv. Volhynien. & Russland & ungefähr 50$^\circ$ 6' N. 25$^\circ$ 40' O. & W. 1860. & 1 Stein von 30 Pfund. \\ \hline
        536. & 15. & 1834. 12. Juni & Charwallas, 30 M. von Hissar, unweit Delhi; Hindostand. & Ost-Indien & ungefähr 29$^\circ$ 12' N. 75$^\circ$ 40' O. & P. 4. 1854. 33. & Mit großem Getöse 1 sehr weicher Stein von 7 bis 8 Pfund, von dem 1 Stuck nach Edinburgh kam. \\ \hline
        537. & - & 1834. 29. November & Raffaten ($^\wedge$$^\wedge$$^\wedge$), angeblich an der Grenze von Ungarn u. der Wallachei. & Ungarn & - & RPG. 37. & Angeblicher Steinregen, vielleicht einerlei mit No. 539: Szala in Ungarn.? \\ \hline
        538. & 36. & 1834.15. Dezember & Marsala, Insel Sicilien. & Italien & 37$^\circ$ 51' N. 12$^\circ$ 24' O. & P. 4. 1854. 34. & Unter Gewittersturm u. Hagel viele gelbliche Aerolithe. \\ \hline
        539. & 6. & 1834. - - & Szala; Gespanschaft Salad. & Ungarn & 46$^\circ$ 50' N. 16$^\circ$ 52' O. & P. 4. 1854. 33. & Steinfall. \\ \hline
        540. & 26. & 1835. 18. Januar & Lobau, in der Ober-Lausitz; Sachsen. & Deutschland & 51$^\circ$ 6' N. 14$^\circ$ 40' O. & P. 4. 1854. 353. & Aus einer Feuerkugel mit geringem Knalle ein stark riechender, schlackenartiger Stein in Bruchstucken. \\ \hline
        541. & 13. & 1835. 31. Juli & Charlotte, Dickson-County; Tennesse. & Nord-Amerika & 36$^\circ$ 13' N. 87$^\circ$ 36' W. & P. 73. 1848. 332. & Aus einem explodierenden Meteor eine Eisenmasse von 9-10 Pfund. \\ \hline
        542. & 22. & 1835. 4. August & Cirencester; Glocestershire. & England & 51$^\circ$ 43' N. 1$^\circ$ 58' W. & RPG. 37. & 1 Stein von 2 Pfund. \\ \hline
        543. & 23. & 1835. 13. November & Simonod (Summonod), N. von Belmont und Belley; Dép. de l'Ain. & ~ & ~ & ~ & ~ \\ \hline
        544. & 2. & 1836. 11. November & Macao am Fluss Assu (Acu oder Amargoro); Prov. Rio Grande do Norte. & ~ & ~ & ~ & ~ \\ \hline
        545. & - & 1836. 22. November & Schlesien. & Deutschland & - & P. 4. 1854. 82. & Getöse in der Luft, das als von einem Meteorsteinfall herrührend betrachtet ward. \\ \hline
        546. & - & 1836. 8. Dezember & Zug ($^\wedge$$^\wedge$$^\wedge$) (Zuz?); Ober-Engadin. & Schweiz & 46$^\circ$ 39' N. 10$^\circ$ 0' O. ? ? & Wolf. 1856. Fol. 326. (nach Stark's Met. Jahrb.)* & Angeblich ein Meteorstein von 5 Pfund, von dem aber sonst nichts bekannt ist; daher wohl zweifelhaft. \\ \hline
        547. & 7. & 1836. - - & Am Plattensee. & Ungarn & ungefähr 46$^\circ$ 50' N. 17$^\circ$ 45' O. & P. 4. 1854. 355. & 1 Meteorstein. \\ \hline
        548. & 8. & 1837. 15. Januar & Mikolowa; Gesp. Salad. & Ungarn & ? & P. 4. 1854. 356. & 1 noch glühender Meteorstein. \\ \hline
        549. & - & 1837. 28. März & Lons-le-Saulnier; Dép. du Jura. & Frankreich & 46$^\circ$ 40' N. 5$^\circ$ 32' O. & Wolf, 1856. Fol. 326. (nach Stark's Met. Jahrb.) & Angeblich ein 5' hoher und 3' breiter Meteorstein, über den aber sonst nichts bekannt geworden. \\ \hline
        550. & 14. & 1837. 5. Mai & East-Bridgewater, Plymouth-County; Massachusetts. & Nord-Amerika & 41$^\circ$ 58' N. 71$^\circ$ 8' W. & P. 4. 1854. 356. & Aus einer Feuerkugel 9 noch heiße, schlackenähnliche Steine, deren größter von 1/4 Pfund. \\ \hline
        551. & 9. & 1837. 24. Juli & Groß-Divina bei Budetin unweit Sillein; Gespanschaft Trentschin. & Ungarn & ungefähr 49$^\circ$ 15' N. 18$^\circ$ 44' O. & P. 4. 1854. 356. Partsch 79.* & 1 Stein von 19 Pfund, welcher nach Pesth kam. \\ \hline
        552. & 24. & 1837. - August & Esnandes (nicht Esnaude), N. von la Rochelle; Dép. de la Charente-Inférieure. & Frankreich & 46$^\circ$ 14' N. 1$^\circ$ 10' W. & P. 4. 1854. 357. & 1 Stein von 3 Pfund in mehreren Bruchstucken. \\ \hline
        553. & 16. & 1838. 18. April & Akburpoor, WSW. von Cawnpoor; Hindostan. & Ost-Indien & 26$^\circ$ 25' N. 79$^\circ$ 57' O. & RPG. 37. & 1 Stein von 4 Pfund. \\ \hline
        554. & 17. & 1838. 6. Juni & Chandakapoor in Berar (Haupstadt: Nagpoor); Dekan. & Ost-Indien & - & RPG. 37. & 1 Stein in 3 Bruchstucken. \\ \hline
        555. & 4. & 1838. 13. Oktober & Im Kalten Bokkeveld, 15 engl. M. N. von Tulbagh und 70 engl. M. von der Kapstadt; Cap der Guten Hoffnung. & Sud-Afrika & ungefähr 32$^\circ$ 30' S. 19$^\circ$ 30' O. & P. 4. 1854. 357. & Aus einer Feuerkugel unter heftigem Explosionen viele, Anfangs ganz weiche Steine von zusammen mehreren 100 Pfund. \\ \hline
        556. & 15. & 1839. 13. Februar & Pine-Bluff, 10 M. SW. von Little-Piney, Pulasky-County; Missouri. & Nord-Amerika & 37$^\circ$ 55' N. 92$^\circ$ 5' W. & P. 4. 1854. 359. & Aus einer Feuerkugel unter Explosionen ein Stein von wenigstens 50 Pfund in mehreren Bruchstucken. \\ \hline
        557. & - & 1839. Anf. November & Gebirge Nopalera ($^\wedge$$^\wedge$$^\wedge$), N. von Sola ($^\wedge$$^\wedge$$^\wedge$) in den Kordilleren; Mexico. & Mittel-Amerika & - & P. 4. 1854. 86 u. 360. & Starke Detonation mit mutmaßlichem Steinfall. \\ \hline
        558. & - & 1839. 29. November & Neapel. & Italien & 40$^\circ$ 53' N. 14$^\circ$ 14' O. & P. 4. 1854. 87 u. 360. & Feuerkugel mit bloß mutmaßlichem Steinfall. \\ \hline
        559. & 3. & 1840. 9. Mai & Am Fluss Karokol in der Kirgisen-Steppe. & Asiatisches Russland & - & P. 4. 1854. 360. & 1 Stein, welcher nach Moskau kam. \\ \hline
        560. & 4. & 1840. 12. Juni & Uden, O. von herzogenbusch; Nordbrabant. & Holland & 51$^\circ$ 40' N. 5$^\circ$ 35' O. & P. 59. 1843. 350. & Unter heftiger Detonation 1 noch heißer Stein von 1 Pfund 12 Loth. \\ \hline
        561. & 37. & 1840. 17. Juli & Cereseto bei Ottiglio (nicht Offiglia), SW. von Casale-Montferrat; Piemont. & Italien & 45$^\circ$ 4' N. 8$^\circ$ 20' O. & P. 50. 1840. 668. & Aus 3 Feuermeteoren unter starkem Knall 3 Steine, deren einer von 10 Pfund gefunden ward. \\ \hline
        562. & 16. & 1840. (1846.) - Oktober & Concord, Merrimac-County; New-Hampshire. & Nord-Amerika & 43$^\circ$ 12' N. 71$^\circ$ 38' W. & P. 4. 1854. 376. & Aus einer Feuerkugel unter Getöse 1 Stein von 370 Gran. \\ \hline
        563. & - & 1841. 25. Februar & les-Bois-aux-Roux ($^\wedge$$^\wedge$$^\wedge$) bei Chanteloup, S. von Coutance; Dép. de la Manche. & Frankreich & ungefähr 48$^\circ$ 54' N. 1$^\circ$ 30' O. & CR. 12. 1841. 514. & Feuerkugel, welche eine Feuersbrunst verursachte \\ \hline
        564. & 27. & 1841. 22. März & Seifersholz und Heinrichsau, beide W. von Gruneberg; Schlesien. & Deutschland & 51$^\circ$ 56' N. 15$^\circ$ 22' O. und 51$^\circ$ 54' N. 15$^\circ$ 25' O. & P. 4. 1854. 361. & Aus einer Feuerkugel unter heftiger Explosion zwei schon kalte Steinbruchstucke von 2 Pfund 9 Loth und von 11 1/2 Loth. \\ \hline
        565. & 25. & 1841. 12. Juni & Trigueres, O. von Chateau-Renard; Dép. du Loiret. & Frankreich & 47$^\circ$ 56' N. 2$^\circ$ 58' O. & P. 53. 1841. 411. & Aus einer Feuerkugel unter Explosion mehrere Steinbruchstucke von zusammen 70-80 Pfund. \\ \hline
        566. & 38. & 1841. 17. Juli & Mailand; Lombardei. & Italien & 45$^\circ$ 28' N. 9$^\circ$ 11' O. & P. 4. 1854. 364. & 1 Aerolith. \\ \hline
        567. & 26. & 1841. 5. November & Roche-Serviere, N. von Bourbon-Vendee; Dép. de la Vendee. & Frankreich & 46$^\circ$ 56' N. 1$^\circ$ 30' W. & P. 4. 1854. 366. & 1 Stein von 11 Pfund. \\ \hline
        568. & - & Vor 1841. 13. November & In den Pas-de-Calais. & Frankreich & ungefähr 50$^\circ$ 30' N. 1$^\circ$ 20' O. & SJ. 42. 1842. 203. & Eine zu Bethune im Dép. du Pas-de-Calais gesehene Feuerkugel von ungewöhnlicher Große, die mit Getöse in das Meer fiel. \\ \hline
        569. & 10. & 1842. 26. April & Pusinsko-Selo, 1 M. S. von Milena; Gesp. Warasdin. & Kroatien & 46$^\circ$ 11' N. 16$^\circ$ 4' O. & P. 4. 1854. 366. & Unter donnerähnlichem Getöse mehrere Steine von zusammen 11 Pfund. \\ \hline
        570. & 27. & 1842. 4. Juni & Aumières bei St. Georges-de-Levejac; Dép. de la Lozère. & Frankreich & ungefähr 44$^\circ$ 18' N. 3$^\circ$ 13' O. & W. 1860. & 1 im Wiener Hofkabinett befindlicher Stein. \\ \hline
        571. & 8. & 1842. 4. Juli & Logrono; Alt-Kastilien. & Spanien & 42$^\circ$ 23' N. 2$^\circ$ 30' W. & RPG. 37. & 1 Stein von 7 Pfund. \\ \hline
        572. & 23. & 1842. 5. August & Harrowgate, NW. von Sheffield; Yorkshire. & England & 53$^\circ$ 38' N. 1$^\circ$ 50' W. & P. 4. 1854. 366. & Unter heftigem Sturm und Blitzen 1 großer noch heißer Stein. \\ \hline
        573. & 18. & 1842. 30. November & Zwischen Jeetala und Mor-Monree in Myhee-Caunta, NO. von Ahmedabad; Hindostan. & Ost-Indien & ungefähr 23$^\circ$ 2' N. 72$^\circ$ 38' O. & P. 4. 1854. 366. & Steinregen; 1 Stuck davon kam nach Bombay. \\ \hline
        574. & 28. & 1842. 5. Dezember & Eaufromont, O. von Epinal; Vogesen. & Frankreich & 48$^\circ$ 10' N. 6$^\circ$ 28' O. & P. 87. 1852. 320. & Aus einer Feuerkugel eine, jedoch erst 1851 gefundene Eisenmasse v. 1 Pfund 21 Loth. \\ \hline
        575. & 17. & 1843. 25. März & Bishopville, Sumter-Distrikt; South-Carolina. & Nord-Amerika & 34$^\circ$ 12' N. 80$^\circ$ 12' W. & P. 4. 1854. 367. & Unter Explosion 1 Stein von 13 Pfund. \\ \hline
        576. & 5. & 1843. 2. Juni & Blaauw-Kapel, NNO. von Utrecht. & Holland & 52$^\circ$ 8' N. 5$^\circ$ 8' O. & P. 4. 1854. 368. & Unter starken Detonationen 2 Steine von 5 1/2 und 14 Pfund. \\ \hline
        577. & 19. & 1843. 26. Juli & Manjegaon (Mallyaum? bei Eidulabad; Khandeish. & Ost-Indien & 20$^\circ$ 32' N. 74$^\circ$ 35' O. ? ? & P. 4. 1854. 370. & Unter großem Geräusch 1 Stein in mehreren Bruchstucken. \\ \hline
        578. & - & 1843. 6. August & Rheina; Westphalen. & Deutschland & 52$^\circ$ 17' N. 7$^\circ$ 25' O. & P. 4. 1854. 371. & Feuerkugel mit mutmaßlichem Steinfall; doch hat man keine Steine gefunden. \\ \hline
        579. & 28. & 1843. 16. September & Kleinwenden bei Munchenlohra, Kreis Nordhausen; Thüringen. & Deutschland & 51$^\circ$ 24' N. 10$^\circ$ 38' O. & P. 4. 1854. 371. & Unter starkem Getöse 1 noch heißer Stein von 5 Pfund 23 Loth. \\ \hline
        580. & 18. & 1843. 30. Oktober & Werchne-Tschirskaja-Stanitza; Land der Donischen Kosaken. & Russland & 48$^\circ$ 25' N. 43$^\circ$ 10' O. & P. 72. 1848. Supl. S. 366. & Unter starker Detonation 1 Stein von 16 Pfund. \\ \hline
        581. & 3. & 1844. - Januar & Caritas-Paso am Fluss Mocorita, S. von Corrientes; la-Plata-Staaten. & Sud-Amerika & 30$^\circ$ 10' S. 58$^\circ$ 30' W. & WA. 40. 1860. 528. B. 120. & Aus einer Feuerkugel unter fürchterlichem Getöse 1 sehr beiße Eisenmasse. \\ \hline
        582. & 24. & 1844. 29. April & Killeter, WNW. von Omagh; North-Tyrone. & Irland & 54$^\circ$ 44' N. 7$^\circ$ 40' W. & RPG. 37. S. 1860. & 1 Stein. \\ \hline
        583. & 29. & 1844. 21. Oktober & Lessc, N. von Confolens; Dép. de la Charente. & Frankreich & 46$^\circ$ 4' N. 0$^\circ$ 38' O. & CR. 19. 1844. 1181. & Steinfall. \\ \hline
        584. & - & 1845. 20. Januar & Gruneberg; Schlesien. & Deutschland & 51$^\circ$ 55' N. 15$^\circ$ 30' O. & P. 4. 1854. 106. & Feuerkugel von einem Knalle begleitet, der auf einen Steinfall schließen ließ. \\ \hline
        585. & - & 1845. 1. September & Fayetteville, Cumberland-County; North-Carolina. & Nord-Amerika & 35$^\circ$ 3' N. 78$^\circ$ 50' W. & P. Supl. 2. 1848. Fol. 367. & Meteor mit starkem Licht, heftigem Knall und mutmaßlichem Steinfall. \\ \hline
        586. & - & 1846. 16. Januar & Pierre ($^\wedge$$^\wedge$$^\wedge$) bei Chàlons-sur-Saone; Dép. de Saone et Loire. & Frankreich & ungefähr 46$^\circ$ 47' N. 4$^\circ$ 50' O. & P. 4. 1854. 110. & Feuerkugel ohne Detonation, welche eine Feuersbrunst veranlasste. \\ \hline
        587. & - & 1846. 22. März & St. Paul ($^\wedge$$^\wedge$$^\wedge$) bei Bagnères-de-Luchon; Dép. de la Haute-Garonne. & Frankreich & ungefähr 42$^\circ$ 46' N. 0$^\circ$ 34' O. & P. 4. 1854. 111. & Mit Geräusch daher ziehende Feuerkugel, welche eine Scheuer in Brand steckte. \\ \hline
        588. & 39. & 1846. 8. Mai & Monte-Milone an der Potenza, SW. von Macerata, Mark Ancona, Kirchenstaat. & Italien & 43$^\circ$ 16' N. 13$^\circ$ 21' O. & P. 4. 1854. 375. & Unter heftigen Detonationen viele Steine von einigen Unzen bis zu 6 Pfund. \\ \hline
        589. & 18. & 1846. - Juli & 20 M. O. von Columbia, Richland-Distrikt; South-Carolina. & Nord-Amerika & 34$^\circ$ 0' N. 80$^\circ$ 45' W. & P. 4. 1854. 376. & Wahrend eines Gewitters ein Stein von 6 1/2 Unzen. \\ \hline
        590. & 25. & 1846. 10. August & Im Norden der Grafschaft Down. & Irland & ungefähr 54$^\circ$ 40' N. 6$^\circ$ 0' W. & SJ. 2. 11. 1851. 36. B. 118. & Beobachtetes Niederfallen einer nickelfreien Eisenmasse, welche auch keine Widmannstatten'schen Figuren zeigt. \\ \hline
        591. & 29. & 1846. 25. Dezember & Schonenberg im Mindelthal; Bayern. & Deutschland & 48$^\circ$ 9' N. 10$^\circ$ 26' O. & P. 70. 1847. 334. & Unter 4 Explosionen 1 Stein von 17 Pfund. \\ \hline
        592. & 19. & 1847. 25. Februar & Hartford, Linn-County; Iowa. & Nord-Amerika & 41$^\circ$ 58' N. 91$^\circ$ 57' W. & P. 4. 1854. 378. & Unter 3 Explosionen 3 Stein von 2 Pfund, 42 Pfund und 50 Pfund. \\ \hline
        593. & - & 1847. 2. März & Ostkuste von Aberdeenshire. & Schottland & - & Thomson 328. & Mondgrosse, mit merklichem Geräusch zerplatzende Feuerkugel mit möglichem Steinfall. \\ \hline
        594. & 51. & 1847. 14. Juli & Hauptmannsdorf, NW. von Braunau; Kreis Königgrätz. & Böhmen & 50$^\circ$ 36' N. 16$^\circ$ 19' O. & P. 72. 1847. 170. & Unter 2 heftigen Detonationen aus einer zu einer Feuerkugel erglühenden, vorher kleinen und schwarzen Wolke unter starkem Blitzen 2 Eisenmassen von 43 u. 30 1/2 Pfund. \\ \hline
        595. & 20. & 1847. 8. Dezember & Foresthill ($^\wedge$$^\wedge$$^\wedge$), Arkansas. & Nord-Amerika & - & P. 4. 1854. 380. SJ. 2. 5. 1848. Fol. 293. & Nach einer Zeitungsnachricht aus einer Wolke unter Explosion 1 noch heißer Stein.* \\ \hline
        596. & 20. & 1848. 15. Februar & Negloor (Nerulgee), am Zusammenfluss des Wurda und Tumbudra; im Collectorat von Dharwar; Dekan. & Ost-Indien & 14$^\circ$ 55' N. 75$^\circ$ 44' O. & P. 4. 1854. 380. & 1 Stein von 4 Pfund in mehreren Bruchstücken, dessen Niederfallen von glaubwürdigen Personen beobachtet worden. \\ \hline
        597. & 21. & 1848. 20. Mai & Castine, Hancock-County; Maine. & Nord-Amerika & 44$^\circ$ 29' N. 68$^\circ$ 57' W. & P. 4. 1854. 381. & Unter donnerndem Getöse 1 Stein von 1 1/2 Unzen. \\ \hline
        598. & 1. & 1848. (1854) ? 27. Dezember & Schie, Filial zu Krogstad; Amt Aggerhuus. & Norwegen & ungefähr 59$^\circ$ 56' N. 11$^\circ$ 18' O. & P. 96. 1855. 341. & Unter Lichterscheinung und lautem Geräusch 1 Stein von 1 1/2 Pfund. \\ \hline
        599. & 5. & 1849. - August & Kumadau-See (Kumatao-Bassin). & Sud-Afrika & 21$^\circ$ 25' S. 25$^\circ$ 20' O. & Livingstone 1. 85 und 2. 257. & 1 Meteorit fiel mit großem Geräusch in den See. \\ \hline
        600. & 22. & 1849. 31. Oktober & 18-20 M. von Concord, Cabarras-County; North-Carolina. & Nord-Amerika & 35$^\circ$ 15' N. 80$^\circ$ 28' W. & P. 4. 1854. 381. & Unter Explosion 1 Stein von 19 1/2 Pfund. \\ \hline
        601. & 6. & 1849. 13. November & Tripolis. & Nord-Afrika & 32$^\circ$ 50' N. 13$^\circ$ 25' O. & P. 4. 1854. 382. & Große Feuerkugel in Italien, welche bei Tripolis in einen Steinfall sich aufloste. \\ \hline
        602. & 21. & 1850. 30. November & Shalka (Shaluka oder Sulker) bei Bissempoor in West-Burdwan; Hindostand. & Ost-Indien & ungefähr 23$^\circ$ 5' N. 87$^\circ$ 22' O. & WA. 41. 1860. 253. & Unter heftiger Explosion 1 Stein, welcher nach Calcutta kam. \\ \hline
        603. & 1. & 1850. 3. Dezember & Prince-of-Wales-Strait. & Nordisches Eismeer & 73$^\circ$ 31' N. 114$^\circ$ 30' W. (nach der Karte von M. etwa 117$^\circ$ 0' W.) & Miertsching Fol. 67. u. 64. & 1 Meteor fiel nahe bei dem Schiff auf das Eis, und es wurden einige kleine eisenhaltige Steinchen aufgelesen. \\ \hline
        604. & 30. & 1851. 17. April & Gütersloh; Westphalen. & Deutschland & 51$^\circ$ 55' N. 8$^\circ$ 21' O. & P. 83. 1851. 465. & Aus einer Feuerkugel unter kanonenähnlichem Getöse 2 Steine von 1 Pfund 26 Loth und 3/4 Loth. \\ \hline
        605. & 9. & 1851. 5. November & Saragossa; Aragonien. & Spanien & 41$^\circ$ 38' N. 0$^\circ$ 45' W. & RPG. & 1 Stein. \\ \hline
        606. & 8. & 1852. Zwischen Juni und Dezember & Am Großen Tschuai (Gr. Tschui), NO. von Kuruman. & Sud-Afrika & 26$^\circ$ 30' S. 25$^\circ$ 20' O. & Livingstone 2. Fol. 257. & 1 Meteorit, den L. unter donnerndem Getöse herabfallen sah, aber nicht finden konnte. \\ \hline
        607. & 9. & 1852. Zwischen Juni und Dezember & Kuruman (Neu-Lattuku), am oberen Lauf des Kuruman-Flusses. & Sud-Afrika & 27$^\circ$ 25' N. 24$^\circ$ 10' O. & Livingstone 2. Fol. 257. & 1 Meteorit, den L. herabfallen sah, aber nicht finden konnte; es klang wie ein gewaltiger Flintenschuss und darauf wie wenn etwas von der Erde abprallte. \\ \hline
        608. & - & 1852. 8. Juli & Wedde, OSO. von Groningen, S. von Windschoten und NW. von Bourtange; Provinz Groningen. & Holland & 53$^\circ$ 5' N. 7$^\circ$ 5' O. & Gleuns Fol. 1-5.* & Unter donnernder Explosion und Feuererscheinung 1 Stein von ungefähr 1 3/4 Loth, welcher dem Museum zu Groningen übergeben ward. \\ \hline
        609. & 11. & 1852. 4. September & Fekete und Teich Istento, 1 M. W. von Mezo-Madaras, im bergigen Haidlande Mezoseg. & Siebenburgen & 46$^\circ$ 37' N. 24$^\circ$ 19' O. & P. 91. 1854. 627. WA. 11. 1853. 674. & Aus einer Feuerkugel unter starkem Donner und Getöse viele Steine, deren größter etwa 18 Pfund. \\ \hline
        610. & 12. & 1852. 13. Oktober & Borkut, 5 M. NO. von Szigeth, an der Schwarzen Theiss; Gespanschaft Marmaros. & Ungarn & 48$^\circ$ 7' N. 24$^\circ$ 17' O. & B. 101. & Unter starkem Donner 1 nach Schwefel riechender Stein von etwa 12 Pfund in 2 Bruchstucken. \\ \hline
        611. & 40. & 1853. 10. Februar & Girgenti; Sicilien. & Italien & 37$^\circ$ 17' N. 13$^\circ$ 34' O. & W. 1860. & 1 großer Stein. \\ \hline
        612. & 22. & 1853. 6. März & Segowlee (Sugouli), N. von Patna und O. von Bettiah; Hindostan. & Ost-Indien & 26$^\circ$ 45' N. 84$^\circ$ 48' O. & W. 1860. WA. 41. 1860. 754. & Etwa 30 Steine. \\ \hline
        613. & - & 1854. 4. Juli & Strehla an der Elbe; Sachsen. & Deutschland & 51$^\circ$ 22' N. 13$^\circ$ 12' O. & Wolf, Züricher Viertel-Jahr-Schr. 1856. 330. & Angeblicher Meteorsteinfall, über den aber sonst nichts bekannt geworden; daher wohl zweifelhaft. \\ \hline
        614. & 31. & 1854. 5. September & Linum, SO. von Fehrbellin; Mark Brandenburg. & Deutschland & 52$^\circ$ 46' N. 12$^\circ$ 52' O. & P. 94. 1854. 169. & Unter heftigem Getöse 1 Stein von 3 Pfund 22 Loth. \\ \hline
        615. & 19. & 1855. 11. Mai & Insel Oesel; Ostsee. & Russland & ungefähr 58$^\circ$ 20' N. 22$^\circ$ 30' O. & P. 99. 1856. 642. & Unter Donner mehrere Steine, davon im Gesamtgewicht etwa 12 Pfund gefunden wurden. \\ \hline
        616. & 32. & 1855. (nicht 1856.) 13. Mai & Bremervorde, Landdrostei Stade; Hannover. & Deutschland & 53$^\circ$ 30' N. 9$^\circ$ 8' O. & P. 96. 1855. 626. & 5 Steine, deren größter 6 Pfund, denen von Fekete ähnlich. \\ \hline
        617. & 6. & 1855. 7. Juni & St. Denis-Westrem, 1 M. WSW. von Gent. & Belgien & 51$^\circ$ 4' N. 3$^\circ$ 40' O. & P. 99. 1856. 63. & Unter Geprassel 1 Stein von 1 Pfund 12 Loth. \\ \hline
        618. & 23. & 1855. 5. August & Petersburg, Lincoln-County; Tennessee. & Nord-Amerika & 35$^\circ$ 20' N. 86$^\circ$ 50' W. & P. 103. 1858. 434. & Unter Getöse 1 noch heißer Stein von 3 Pfund. \\ \hline
        619. & - & 1856. 8. Juli & 10 M. W. von Aberdeen, Monroe-County, 142 M. NO. von Jackson; Mississippi. & Nord-Amerika & 33$^\circ$ 46' N. 88$^\circ$ 44' W. & SJ. 2. 23. 1857. 128 u. 287. SJ. 2. 24. 1857. 449. & Vermutheter, aber wieder bezweifelter Meteorsteinfall aus einem zu Marion in Alabama gesehenen Feuermeteor. \\ \hline
        620. & 41. & 1856. 17. September & Bei Civita-Vecchia ins Meer. & Italien & ungefähr 42$^\circ$ 7' N. 11$^\circ$ 46' O. & P. 99. 1856. 645. & Unter heftigem Geräusch 15 Schritte von einem Schiff beobachteter Meteorsteinfall. \\ \hline
        621. & - & 1856. 14. November & Etwa 60 geogr. M. SO. von Java. & Indisches Meer & 10$^\circ$ 38' S. 117$^\circ$ 49' O. & P. 106. 1859. 476. & Regen von schwarzen, innen hohlen, birnförmigen Eisenkügelchen. \\ \hline
        622. & 42. & 1856. 12. November & Trenzano, WSW. von Brescia; Lombardei. & Italien & 45$^\circ$ 28' N. 10$^\circ$ 2' O. & WA. 41. 1860. 569. & 3 ansehnliche Steine, deren 2 gefunden wurden; einer davon von 17 Pfund. \\ \hline
        623. & 23. & 1857. 28. Februar (?) & Parnallee bei Madras. & Ost-Indien & ungefähr 13$^\circ$ 5' N. 80$^\circ$ 20' O. & Brit. Ass. Report. (RPG.) & 2 große Steine. \\ \hline
        624. & 13. & 1857. 15. April & Kaba, SW. von Debreczin; Gespanschaft Nord-Bihar. & Ungarn & 47$^\circ$ 22' N. 21$^\circ$ 16' O. & P. 105. 1858. 329. & Aus einer Feuerkugel unter donnerndem Getöse 1 schwarzer Stein von 7 Pfund. \\ \hline
        625. & - & 1857. 17. Juni & Ottawa, am Illinois-River, 119 M. NNO. von Springfield, la-Salle-County; Illinois. & Nord-Amerika & 41$^\circ$ 20' N. 89$^\circ$ 5' W. & SJ. 2. 24. 1857. 449. & Angeblicher Niederfall einer schlackenartigen Masse, die aber einem Meteorstein unähnlich u. darum irdischen Ursprung vermuten lasst. \\ \hline
        626. & 30. & 1857. 1. Oktober & les Ormes, WSW. von Aillant-sur-Tholon; Dép. de l'Yonne. & Frankreich & 47$^\circ$ 51' N. 3$^\circ$ 15' O. & CR. 45. 1857. 687. & Aus einer Feuerkugel 1 Stein von 7 1/2 Loth. \\ \hline
        627. & 14. & 1857. 10. Oktober & Ohaba, O. von Carlsburg; Bezirk Blasendorf. & Siebenburgen & 46$^\circ$ 4' N. 23$^\circ$ 50' O. & P. 105. 1858. 334. & Unter donnerndem Getöse aus einer Feuerkugel 1 Stein von 29 Pfund. \\ \hline
        628. & 24. & 1857. 27. Dezember & Quenggouk, NNO. von Bassein in Pegu; Birma. & Ost-Indien & ungefähr 17$^\circ$ 30' N. 95$^\circ$ 0' O. & WA. 41. 1860. 750 u. 42. S. 301. & 1 Stein, von welchem sich 1 Stuck in Wien befindet. \\ \hline
        629. & 15. & 1858. 19. Mai & Kakova, NW. von Oravitza, Gesp. Krasso; Temeser Banat. & Ungarn & 45$^\circ$ 6' N. 21$^\circ$ 38' O. & WA. 34. 1859. 11. & Unter dumpfem Donnern und Sausen ein Stein von 1 Pfund 1 Loth. \\ \hline
        630. & 1. & 1858. ungefähr 1. August & Heredia (Eredia); Costa-Rica. & Mittel-Amerika & 8$^\circ$ 45' N. 83$^\circ$ 25' W. & P. 107. 1859. 162. Harris Fol. 99. & 1 Stein. \\ \hline
        631. & 31. & 1858. 9. Dezember & Clarac und Aussun, beide ONO. von Montrejeau; Dép. de la Haute-Garonne. & Frankreich & 43$^\circ$ 4' N. 0$^\circ$ 35' O. und 43$^\circ$ 5' N. 0$^\circ$ 33' O. & P. 107. 1859. 191. & Unter Explosion 1 Stein in mehreren Bruchstucken im Gesamtgewicht von 100 bis 120 Pfund; das größte 80 Pfund. \\ \hline
        632. & 24. & 1859. 26. März & Harrison-County; Kentucky. & Nord-Amerika & ungefähr 38$^\circ$ 25' N. 84$^\circ$ 30' W. & S. 1860. & Mehrere kleine Steine. \\ \hline
        633. & 25. & 1859. 11. August & Bethlehem, Albany County; New-York. & Nord-Amerika & 42$^\circ$ 27' N. 74$^\circ$ 0' W. & S. 1860. & Aus einer Feuerkugel unter 3 Explosionen mehrere Steine. \\ \hline
        634. & 26. & 1860. 1. Mai & New-Concord, Muskingum-County, u. Claysville, SO. von Cambridge, Guernsey-County; Ohio. & Nord-Amerika & ungefähr 40$^\circ$ 10' N. 81$^\circ$ 30' W. & WA. 41. 1860. 569 u. 572. & Unter mehreren Explosionen mehr als 30 Steine, darunter mehrere von 40 bis 60 Pfund, einer von 103 Pfund; im Ganzen wohl an 700 Pfund. \\ \hline
        635. & 25. & 1860. 14. Juli & Dhurmsala (Dharam-Sal) bei Kangra, ONO. von Lahore; Pendsjab. & Ost-Indien & ungefähr 31$^\circ$ 57' N. 76$^\circ$ 5' O. & WA. 42. 1860. 305. & Unter Explosion mehrere Steine, deren größter 320 Pfund A. d. p. \\ \hline
        636. & 26. & 1860. - - & Bhurlpore, W. von Agra; Hindostan. & Ost-Indien & 27$^\circ$ 14' N. 77$^\circ$ 30' O. & H. & Steinfall. \\ \hline
          &   &   &   &   &   &   &   \\ \hline
        Nachtrag &   &   &   &   &   &   &   \\ \hline
          &   & Vor Christus &   &   &   &   &   \\ \hline
        637. & - & 331. - - & Aricia in Latium, 10 Rom. M. SO. von Rom. & Italien & 41$^\circ$ 49' N. 12$^\circ$ 30' O. & Fincelius, das 1552 Jar.* & Es regnete Steine; doch ungewiss, ob nicht bloßer Hagel. \\ \hline
        638. & - & 258. - - & Albaner Gebirge (Mons Albanus); und in Rom. & Italien & 41$^\circ$ 40' N. 12$^\circ$ 40' O. und 41$^\circ$ 54' N. 12$^\circ$ 26' O. & Livius 6. Pars 1. S. 165.* (Freinsheimii suppl. lib. 7.) & Es fielen zahlreiche Steine nach Art des Hagels. \\ \hline
        639. & - & 216. (214.) - - & Praeneste in Latium, O. von Rom und NW. von Anagnia. & Italien & 41$^\circ$ 48' N. 13$^\circ$ 0' O. & Livius 7. 15. (lib. 22. c. 1.) Lycosthenes 114. & Brennende Steine (ardentes lapides, nach anderer Lesart aber brennende Fackeln, ardentes lampades) fielen vom Himmel. \\ \hline
        640. & - & 204. (202.) - - & ? & Italien & - & Livius 9. 76. (lib. 29. c. 14.) & Steinregen; doch ungewiss, ob nicht bloßer Hagel. \\ \hline
        641. & - & 188. (185.) - - & Provinz Picenum (jetzt Mark Ancona). & Italien & ungefähr 43$^\circ$ 0' N. 13$^\circ$ 30' O. & Livius 11. 402. (lib. 39. c. 22.) Lycosthenes 148. & Dreitägiger Steinregen; daher wohl nur wiederholter Hagel. \\ \hline
        642. & - & 176. (174.) - - & Crustumerium in Etrurien. & Italien & 42$^\circ$ 0' N. 12$^\circ$ 25' O. & Livius 11. 858. (lib. 41. c. 13. [17]) Lycosthenes 153. & Ein Vogel (Sangualis) ließ aus seinem Schnabel einen heiligen Stein herabfallen. \\ \hline
        643 & - & Zwischen 176 (174) und 166 (164). & Rom, und gleichzeitig zu Veji in Etrurien, 10 M. N. von Rom. & Italien & 41$^\circ$ 54' N. 12$^\circ$ 26' O. 42$^\circ$ 0' N. 12$^\circ$ 25' O. & Livius 12. 325. (lib. 44. c. 18.) & Steinregen; doch ungewiss, ob nicht bloßer Hagel. \\ \hline
          & ~ & Nach Christus &   &   &   &   & ~ \\ \hline
        644 & - & Zwischen 364 und 455 - - & Konstantinopel. & Europäischen Türkei & 41$^\circ$ 0' N. 28$^\circ$ 58' O. & Majolus 10 u. 11. (nach Modognetes). & Steinregen zur Zeit Valentinians. Vielleicht einerlei mit dem nach Chladni S. 186 i. J. 416 angeblich vom Himmel, in Wahrheit aber nur von einer Säule herabfallenden Stein? Oder mit dem nach Lycosthenes S. 285 im Jahre 407 gefallenen heftigen Hagel? \\ \hline
        645 & - & 1201. - - & ? & ? & - & P. 2. 152. (nach Cardanus). & Aus einem Cometen sollen stinkende, schwefelartige Steinchen herabfallen sein. \\ \hline
        646 & - & Vor 1556. - - & In Holstein (Holsatz). & Deutschland & - & Fincelius, das 1552 Jar. & Ein sehr großer Stein fiel aus den Wolken und ward in einer Kirche aufgehangen. \\ \hline
        647 & - & 1543. 4. Mai & Zesenhausen (Zaisenhausen), NNO. von Pforzheim; Baden. & Deutschland & 49$^\circ$ 7' N. 8$^\circ$ 53' O. & Fincelius, das 1543 Jar. Lycosthenes 580. & Aus einem Stern flog ein feuriger Drache in ein Wasser, das er austrocknete, und von da in einen Acker, in dem er auf eine Strecke von 15 Schuh die Fruchte verbrannte. \\ \hline
        ~ & ~  & ~  & Mutmaßliche oder zweifelhafte Meteorsteine, deren Fallzeit unbekannt. &  ~ & ~  & ~  & ~ \\ \hline
        648 & - & - & Troja. & Klein-Asien & 39$^\circ$ 55' N. 26$^\circ$ 15' O. & v. Dalberg Fol. 57 u. 58. & Der harte, schwere und schwarze Stern-Stein Siderites oder Ophites, welchen Apollo dem Trojaner Helenos gab. \\ \hline
        649 & - & - & Ephesus. & Klein-Asien & 38$^\circ$ 0' N. 27$^\circ$ 25' O. & C. 103. v. Hammer 4. Fol. 105.* & Angeblich vom Himmel gefallenes Bild der Diana. \\ \hline
        650 & - & - & Laodicea, O. von Ephesus. & Klein-Asien & 37$^\circ$ 50' N. 29$^\circ$ 0' O. & v. Dalberg Fol. 73. & Batylos-Stein, welcher am Eingang des Dianen-Tempels zu Laodicea stand. \\ \hline
        651 & - & - & Tyrus. & Phonizien & 33$^\circ$ 18' N. 35$^\circ$ 35' O. & v. Dalberg Fol. 57. & Der als Stern vom Himmel gefallene Stein, welchen die Göttin Astarte, nachdem sie ihn aufgehoben, der Stadt Tyrus weihte. \\ \hline
        652 & - & - & Bethel (Lus), NNO. von Jerusalem W. von Jericho. & Palästina & 31$^\circ$ 55' N. 35$^\circ$ 35' O. & 1. Mosis 28. v. 10-19. v. Dalberg Fol. 64-68. & Der von Jacob zu einem Mahlstein aufgerichtete, in späteren Zeiten verehrte und der Sage nach schwarze Jacobsstein. \\ \hline
        653 & - & - & Gileads-Hügel unfern Bethel. & Palästina & ungefähr 31$^\circ$ 55' N. 35$^\circ$ 35' O. & v. Dalberg Fol. 56 u. 65. & Von Jacob zu einem Haufen gesammelte schwarze Steine, welche, da in der ganzen Gegend gewöhnlich nur weiße Kalksteine sich vorfinden, für Meteorsteine zu halten sind. \\ \hline
        654 & - & - & Hierapolis. & Syrien & 36$^\circ$ 30' N. 37$^\circ$ 50' O. & v. Hammer 4. Fol. 105. Ersch u. Gruber 34. Fol. 199.* & Angeblich vom Himmel gefallenes Bild der Syrischen Liebesgottin Derkato. \\ \hline
        655 & - & - & ? & Arabien & - & v. Dalberg Fol. 73. & Der schwarze, von den Arabern verehrte Steingott Abadir oder Alassovid, auch Theusares oder Dusares (Deus Mars) genannt. \\ \hline
        656 & - & - & Auf verschiedenen Inseln. & Rothes Meer & - & v. Dalberg Fol. 103. & Die von den Parthischen Magiern gesuchten, angeblich dem Eisen oder dem Kupfer ähnlichen sogenannten Blitz-Steine, die an Stellen sollen gefunden worden sein, welche vom Blitz getroffen worden sind. \\ \hline
        657 & - & - & Babylon. & Babylonien & 32$^\circ$ 40' N. 44$^\circ$ 20' O. & C. 103. & Der in den Ruinen von Babylon gefundene und mit Keilschrift versehene Stein, welcher vielleicht ein Meteorstein sein durfte. \\ \hline
        658 & - & - & ? & Persien & - & v. Dalberg Fol. 58. & Der Stein Astroides, dessen Zoroaster zu seinen magischen Künsten sich bediente. \\ \hline
        659 & - & - & ? & Persien & - & v. Dalberg Fol. 167. & Der Persische Zylinder, dessen Millin in seinen Monuments inedits nouvellement expliques, Tome 1., Erwähnung tut. \\ \hline
        660 & - & - & Provinz Ghilan (Guilan oder Gkilan), an der SW. Seite des Kaspischen Meeres. & Persien & 37-38 N. 48-49 O. & S. de Sacy Chr. Arabe 3. Fol. 438.* & Die dem Eisen oder dem Kupfer ähnlichen sogenannten Blitz-Steine, welche in der Provinz Ghilan sich vorfinden. \\ \hline
        661 & - & - & Provinz Turkistan. & Tartarei & 42-45 N. 66-70 O. & S. de Sacy Chr. Arabe 3. Fol. 438. & Desgleichen in Turkistan. \\ \hline
        662 & - & - & ? & Kaschmir & ungefähr 34$^\circ$ 20' N. 74$^\circ$ 35' O. & v. Dalberg Fol. 68. & In Kaschmir verehrter, angeblich vom Himmel gefallener Stein. \\ \hline
        663 & - & - & Pagode Perwuttum (Pervatam-Berg), am Kistna-Fluss; Dekan. & Ost-Indien & 16$^\circ$ 12' N. 75$^\circ$ 5' O. & v. Dalberg Fol. 68. Ritter 6. Fol. 339.* & Als Lingam verehrter, angeblich vom Himmel gefallener Stein. \\ \hline
        664 & - & - & Paphos. & Insel Cypern & 34$^\circ$ 50' N. 32$^\circ$ 25' O. & v. Hammer, Osm. Reich; 3. Fol. 569. 4. Fol. 105. & Angeblich vom Himmel gefallenes Bild der Aphrodite. \\ \hline
        665 & - & - & Delphi. & Griechenland & 38$^\circ$ 27' N. 22$^\circ$ 33' O. & Bigot de Morogues Fol. 28. & Angeblich von Saturn auf die Erde geschlenderter schwarzer Stein, der im Apollo-Tempel war aufbewahrt worden. \\ \hline
        666 & - & - & Cyzicus in Mysien. & Klein-Asien & 40$^\circ$ 20' N. 27$^\circ$ 50' O. & P. 2. 1824. 156. & Stein, der nach Apulejus daselbst war aufbewahrt worden. \\ \hline
        667 & - & - & Campus lapideus (Plaine la Crau), zwischen Arles und Marseille. & Frankreich & ungefähr 43$^\circ$ 30' N. 5$^\circ$ 0' O. & Merula Cosm. 588. & Angeblicher Steinregen welchen Jupiter dem Herkules zur Hülfe sandte, als dieser mit den Söhnen Neptuns kämpfte. \\ \hline
        668 & - & - & Grave, ONO. von Herzogenbusch; Nordbrabant. & Holland & 51$^\circ$ 45' N. 5$^\circ$ 45' O. & C. 83 u. 223. & Angeblich vom Himmel gefallener, im Chor der Kirche eingemauerter Stein. \\ \hline
        669 & - & - & Battersea-Fields bei London. & England & 51$^\circ$ 30' N. 0$^\circ$ 5' W. & Phil. Mag. 10. 381-389.* & Ein in einem Weidenbaum gefundener mutmaßlicher Meteorstein, vielleicht um das Jahr 1838 oder um 1846 gefallen. \\ \hline
        670 & - & - & Dunsinnan. & Schottland & 56$^\circ$ 28' N. 3$^\circ$ 16' W. & C. 185. & Stein, der in den Ruinen von Macbeths Schloss gefunden worden sein soll, und welcher vielleicht ein Meteorstein sein durfte. \\ \hline
        671 & - & - & Deeresheim ($^\wedge$$^\wedge$$^\wedge$) bei Halberstadt und Osterwiek. & Deutschland & ungefähr 51$^\circ$ 55' N. 11$^\circ$ 0' O. & G. 71. 1822. 361. & Sehr zweifelhafter Meteorsteinfall. \\ \hline
        ~ & ~ & ~ & Mutmaßliche oder zweifelhafte Meteor-Eisenmassen, deren Fallzeit unbekannt. & ~ & ~ & ~ &   \\ \hline
        672 & - & - & Chotzen, NO. von Hohenmauth und ONO. von Chrudim; Kreis Chrudim. & Böhmen & 49$^\circ$ 57' N. 16$^\circ$ 10' O. & WA. 25. 1857. 545 u. 549. Geol. R. A. 2. 8. 1857. 354-357.* & Von Reuß für irdisches Eisen, von Neumann aber für Meteoreisen aus der Zeit des Planerkalkes gehalten. \\ \hline
        673 & - & - & ? & Angeblich aus Norwegen & - & C. 325. & 1 dem Pallas'schen Eisen ähnliches astiges Eisen mit Olivin im Wiener Hof-Kabinett. \\ \hline
        674 & - & - & Collina di Brianza bei Villa, NNO. von Mailand und von Monza. & Italien & 45$^\circ$ 40' N. 9$^\circ$ 17' O. & C. 349. & 200-300 Pfund; nickelfrei und zweifelhaft ob meteorischen oder irdischen Ursprungs. \\ \hline
        675 & - & - & Angeblich aus der Luft gefallener Anker, der in der Kirche zu Kloena ($^\wedge$$^\wedge$$^\wedge$) war aufbewahrt worden. & Island & ? & G. 75. 1823. 231. & Vielleicht aus Meteoreisen geschmiedet. \\ \hline
        676 & - & - & Liberia, in der Gegend, die von dem St. Johns-River begrenzt wird. Sp.-Gew.: 6,708. & West-Afrika & ungefähr 6$^\circ$ 0' N. 9$^\circ$ 30' W. & B. 113. & Von feinkörniger, kristallinischer Struktur, ähnlich wie manches Meteoreisen. \\ \hline
        677 & - & - & Kurrukpur-Hügel bei Monghir am Ganges; Bengalen. 156 Pfund Gefunden 1848. & Ost-Indien & ungefähr 25$^\circ$ 20' N. 86$^\circ$ 36' O. & WA. 41. 1860. 252. & Enthalt Nickel und Kobalt, zeigt aber keine Widmannstatten'schen Figuren. \\ \hline
        678 & - & - & Der Blitzende Stein. & Nepal & ungefähr 28$^\circ$ 0' N. 84$^\circ$ 0' O. & P. 4. 1854. 396. v. Dalberg Fol. 68. & Mutmaßliches Meteoreisen, als Bild des Mahadewa, des Indischen Gottes der Zeugung, verehrt. \\ \hline
        679 & - & - & Der Fels des Pols (Khadasu-tsilao), nicht weit von der Quelle des Gelben Flusses (Houang oder Whang); am nördlichen Ufer des Altan oder Gold-Flusses. & Ost-Asien & ungefähr 33-36 N. 95-100 O. & C. 356. AR. 1. 208. & Nach der Sage ein vom Himmel gefallener Stein, wahrscheinlich Meteoreisen. \\ \hline
        680 & - & - & Ceralvo ($^\wedge$$^\wedge$$^\wedge$), zwischen Camargo und Monterey; im Staate Nuevo-Leon. & Mexico & ungefähr 26$^\circ$ 0' N. 100$^\circ$ 0' W. & SJ. 2. 21. 1856. 216. & Eisen von wahrscheinlich meteorischem Ursprung, welches daselbst 1847, als Ambos dienend, gefunden ward. \\ \hline
        681 & - & - & An der Küste von Omoa, 10 engl. M. vom Meere, im Staate Honduras. & Mittel-Amerika & ungefähr 15$^\circ$ 25' N. 87$^\circ$ 55' W. & C. 341. & Wahrscheinlich Meteoreisen. \\ \hline
    \end{longtable}
\end{center}
\clearpage
*) Le Chou-king, recueilli par Confucius, traduit et enrichi de notes par Gaubil; Paris 1790.

*) Académie Royale de Bruxelles. Nouveau Catalogue des principales apparitions d'étoiles filantes par A. Quetelet; Bruxelles 1841.

*) E. F. F. Chladni: Über Feuer-Meteore und über die mit denselben herabgefallenen Massen; Wien 1819.

*) Conradus Lycosthenes Rubeaquensis (Conrad Wolffhart von Rufach zu Basel): Prodigiorum ac ostentorum chronicon; Basiliae 1557.

*) Chronicon Magnum Schedelii: Das buch der Chroniken und Geschichten mit Figuren und pildnussen von Anbeginn der Welt biss auf diese unsere Zeit; Augspurg durch Hannsen schönsperger 1496.

*) August Pauly: Real-Encyclopadie der klassischen Altertumswissenschaft; Stuttgart 1848.

*) Dr. J. S. C. Schweigger: Journal fur Chemie und Physik; neue Folge. Halle 1825. Band 14 (44).

*) Georg Ernestus Stahl: Joh. Joachimi Beccheri Physica subterranea. Lipsiae 1703.

*) Paulli G. F. P. N. Merulae Cosmographiae generalis libri tres: item geographiae particularis libri quatvor. Ex officinia Plantiniana Raphelengji 1605.

*) C. Suetonii Tranquilli Opera. Textu ad Codd Mss Recognito cum Jo. Aug. Ernestii Animadversionibus nova cura auctis emendatisque et Jsaaci Casauboni Commentario edidit Frid Aug. Wolfius Lipsiae 1802 (Liber 7. Ser. Sulpicius Galba).

*) Relation de l'Egypte par Abd-Allatif, medecin arabe de Bagdad, traduit et enrichi de notes par M. Silvestre de Sacy. Paris 1810.

*) Joseph Simonius Assemanus: Assemani Bibliotheca orientalis Clementino-Vaticana Romae 1721. (Caput 16. Dionysius 1. Patriarcha Jacobitarum, cognomento Telmahreusis).

*) Dieser Steinfall ist in dem geographischen Verzeichnis, Seite 67, noch nicht aufgenommen, und daher nachträglich daselbst noch einzuschalten.

*) Monumenta Germaniae Historica, edidit Georgius Hienricus Pertz. Hannoverae 1826. Tomus 1. (Einhardi Fuldensis Annales).

*) Dr. Friedrich Schnurrer: Chronik der Seuchen mit den gleichzeitigen Vorgängen in der physischen Welt und in der Geschichte der Menschen. Tübingen 1823.

*) L'Institut, Journal gènéral des sciences et travaux scientifiques, 1re Section, Tome 6, Nr. 252. Paris 1838. (Etoiles filantes signalées dans les auteurs arabes par Mr. Fraehn.)

*) M. Zacharias Rivander: Duringische Chronika 1596.

*) Gregorii Abulpharagii sive Bar-Hebraei Chronicon Syriacum, e codicibus Bodleianis descripsit maximani partem vertit notisque illustravit P. J. Bruns, edidit ex parte vertit notasque adjecit G. G. Kirsch; Lipsiae 1789.

*) Fundgruben des Orients, bearbeitet durch eine Gesellschaft von Liebhabern. Wien 1818. (Jos von Hammer: Weiterer Beitrag zur Geschichte der Luftsteine aus dem Abdschaibol-Machlukat, d. i. den Wundern der Geschopfe des Mohammed Ben Ahmed aus Tuss und des Kaswini).

*) Dieser auf der Karte von Asien noch nicht verzeichnete Steinfall ist nachträglich auch in dem geographischen Verzeichnis, Seite 67, noch einzuschalten.

*) Albertus Krantz: Saxonia. Verteütscht durch Basilium Fabrum Soranum. Leipzig 1582.

*) M. Cyriacus Spangenberg: Mansfeldische Chronien. Eisleben 1572.

*) Mattheus Dresser: Sächsisch Chronikon. Wittenberg 1596.

*) Chladni halt diese beiden Steinfalle zu Friedland in Brandenburg und zu Friedeburg an der Saale für ein und dasselbe Ereignis. Doch ist es auffallend, dass die Chroniken, welche des Steinfalles von Friedeburg an der Saale erwähnen, nur das Jahr 1304, aber nicht auch den Tag angeben, an welchem derselbe stattgefunden; wahrend Krantz für den Steinfall zu Friedland nicht nur das Jahr 1304 angibt, sondern auch ausdrücklich sagt, das Ereignis habe am St. Remigiustage (1 Okt.) stattgefunden. Auch davon, dass die Steine - wie es bei Fraedeburg der Fall war - kohlschwarz und hart wie Eisen gewesen seien, geschickt bei dem Fall von Friedland keine Erwähnung. Darum durften beide Berichte sich doch vielleicht auf zwei verschiedene Ereignisse beziehen.

*) Lebens-Beschreibung Herrn Gozens von Berlichingen; zum Druck befördert von Verono Franck von Steigerwald und Wilhelm Friedrich Pistorius. Nürnberg 1731.

*) Johann Leopold Cysat: Beschreibung dess Beruhmhten Lucerner - oder 4 Waldstatten Sees und dessen Furtrefflichen Qualiteten und sonderbaaren Eygenschafften. Lucern 1661.

*) Commentarius brevis rerum in orbe gestarum ab anno salutis 1500 usque in annum 1574 ex optimis quibusque seriptoribus congestus per F. Laurentium Surium, Carthusianum. Coloniae 1602.

*) Simonis Majoli Astensis, Episcopi Vulturariensis, Dierum Canicularium Tomi 7. Colloquiis 46. Offenbaei ad Moenum 1691 (Colloquium primum de Meteoris).

*) M. Andreas Engelius: Rerum Marchicarum Breviarium; Wittenberg 1593.

*) Johan Bangen: Thüringische Chronik oder Geschichtsbuch; Mülhausen 1599.

*) Bigot de Morogues: Mémoire historique et physique sur les chûtes des pierres; Orléans 1812.

*) Museum Wormianum, seu Historia rerum rariorum, tam Naturalium, quam artificialum, tam Domesticarum, quam Exoticarum, quae Hafniae Danorum in Aedibus Authoris servantur, adornate ab Olao Worm, Med. Doct. Lugduni Batavorum.

*) P. D. Ambrogio Soldani: Sopra una pioggetta di sassi accaduta nella sera de' 16 Giugno del 1794 in Lucignan d'Asso nel Sanese; Siena 1794.

*) J. G. Knapp: Neuere Geschichte der evangel. Missionsanstalten zur Bekehrung der Heiden in Ostindien. Halle 1771. 2tes Stuck, 1te Abt.

*) Troisieme voyage de Cook; Paris 1785.

*) M. Belli, geb. Gontard: Leben in Frankfurt a. M.; Frankfurt a. M. 1850.

*) Fr. von Dalberg: Über Meteor-Cultus der Alten, vorzüglich in Bezug auf Steine, die vom Himmel gefallen; Heidelberg 1811.

*) Diese genaueren, aus Soldani entnommenen Ortsangaben sind in dem geographischen Verzeichnis Seite 60 nachträglich zu ergänzen.

*) Domenico Tata: Memoria sulla pioggia di pietre avvenuta nella campagna Sanese il di 16 di Giugno di questo corrente anno; Napoli 1794.

*) Domenico Tata: Relazione dell' ultima eruzione del Vesuvio della sera de' 15 Giugno; Napoli 1794.

*) Siehe die ausführlichere Beschreibung Seite 15.

*) Diese genaueren Ortsangaben sind in dem geographischen Verzeichnis Seite 53 noch hinzuzufügen.

*) David Purdie Thomson: Introduction to Meteorology; Edinburgh and London 1849.

*) Dr. R. Wolf, Vierteljahrschrift der naturforschenden Gesellschaft in Zürich; Zürich 1856.

*) Paul Partsch, die Meteoriten oder vom Himmel gefallenen Steine und Eisenmassen im k. k. Hof-Mineralien-Kabinette in Wien; Wien 1843.

*) Dieser angebliche, einem von Henry Hicks, P. M., an den Herausgeber des Philadelphia Courier gerichteten und in den angegebenen Band von Sillimans Journal aufgenommenen Brief entnommene Meteorsteinfall ist zwar in dem Verzeichnis zu Karte 3 Seite 56 unter den mehr oder weniger zuverlässigen Steinfallen aufgeführt; allein da von dem Steine, der angeblich ausgegraben worden sein soll, trotz der Aufforderung in Sillimans Journal, nie auch nur ein Bruchstuck wirklich vorgelegt worden ist, so ist das ganze Ereignis wohl nur als sehr zweifelhaft, wenn nicht die ganze Erzählung als ein Amerikanischer Humbug zu betrachten.

*) Dr. W. Gleuns, Jr.: Jets over de meteoor-explosie van den 8. Julij 1852 en een' bij die gelegenheid gevonden meteoorsteen; Groningen 1852.

*) Dieser Meteorsteinfall ist in dem geographischen Verzeichnis Seite 55, so wie in dem Monats-Verzeichnis Seite 47 noch nicht aufgenommen und daher nachträglich daselbst noch einzuschalten.

*) Jobus Fincelius: Wunderzeichen. Wahrhaftige Beschreibung und gründlich Verzeichnis schrecklicher Wunderzeichen und Geschichten, die von dem Jahr 1517 an bis auf das Jahr 1556 geschehen und ergangen; Vrsel 1557.

*) T. Livii Patavini Historiarum ab urbe condita libri, qui supersunt, omnia: curante Arn. Drakenborch; Stutgardiae 1823.

*) J. von Hammer: Geschichte des Osmanischen Reiches; Pest 1828.

*) J. G. Ersch u. Gruber: Allgemeine Encyklopadie der Wissenschaften und Künste; Leipzig 1833. Band 34.

*) Silvestre de Sacy: Chrestomathie Arabe on extraits de divers écrivrains arabes, tant en prose qu-en vers; Paris 1827. tome 3. (Extraits du livre des merveilles de la nature et des singularités des choses créeés, par Mohammed Kazwini, fils de Mohammed; traduits par A. L. de Chézy).

*) Carl Ritter: Erdkunde oder allgemeine vergleichende Geographie; Berlin 1836. Bd. 6.

*) The London, Edinburgh and Dublin Philosophical Magazine and Journal of Science. Vol. 10. Fourth Series. July - December 1855.

*) Jahrbuch der k. k. Geologischen Reichsanstalt; Wien 1857.
\clearpage
\section{Verzeichnis von angeblichen Meteorsteinfallen, welche in Meteorstein-Verzeichnissen zwar hin und wieder vorkommen, aber teils als bloße Feuerkugeln, aus denen keine wirklich festen oder steinartigen Gebilde hervorgingen, zu den eigentlichen Meteorsteinfallen nicht zu zahlen, - teils, als auf irrigen Angaben beruhend, zu streichen sind.}
\begin{center}
    \footnotesize
    \begin{longtable}{| p{20mm} | p{25mm} | p{16mm} | p{50mm} |}
    \hline
        Vor Christus &   &   &   \\ \hline
        1460. - - & ? & ? & A. 4. 184. nach Lycosthenes Fol. 46. - Dieser von A. ohne alle Ortsangabe erwähnte Steinfall ist nach Lycosthenes kein anderer als der auch von A. noch besonders aufgeführte Stein- oder Hagelfall bei Gibeon zur Zeit des Josua. \\ \hline
        1082. - - & Bockbach (Aegos Potamos.) & Thrakien & Lycosthenes 49. Herold 50.* - Einerlei mit Nr. 16. 476 bis 462 v. Chr. am Ziegenfluss (Aegos Potamos); die verschiedenen alten Schriftsteller haben ein und dasselbe Ereignis oftmals in verschiedene Zeiten gesetzt. \\ \hline
        570. (520.) - - & Cybelische Berge & Insel Creta & C. 174. - Einerlei mit Nr. 4. 1478 v. Chr., welchen Steinfall Bigot de Morogues irrtümlich in das Jahr 520 (570) v. Chr. Gesetzt hat. \\ \hline
        405. (403.) - - & Am Geysshach (Aegos Potamos.) & Thrakien & Lycosthenes 82 u. 83. Herold 82. - Einerlei mit Nr. 16. 476 bis 462 v. Chr. am Ziegenfluss (Aegos Potamos); siehe vorstehend 1082 v. Chr.: Bockbach. \\ \hline
        215. (213.) - - & Sinuessa (nicht Sinuesta) & Italien & Majoli Dier. Can. S. 10. Livius 7. S. 519. (lib. 23. c. 31.) - Irrtümliche Verwechselung mit Nr. 25. dem Steinfall zu Lanuvium. \\ \hline
        Nach Christus &   &   &   \\ \hline
        412. - - & ? & ? & Lycosthenes Fol. 287. Herold 286. - Nach Herold; Hagel von Steinen; nach Lycosthenes jedoch nur gewöhnlicher Hagel, der aber zum Teil großer als handgroße Steine gewesen. \\ \hline
        416. 21. März & Konstantinopel. & Europäischen Türkei & C. 186. - War nur ein von einer Säule herabgefallener Stein. \\ \hline
        584. - Dezember & ? & ? & P. 66. 1845. 476. Quetelet 1841. 22. - Bloße Feuerkugel; von einem Steinfall ist durchaus keine Rede. \\ \hline
        649. - - & ? & Italien & C. 190. - Das Ereignis fallt nicht in das Jahr 649, sondern ist nach den von Muratori angegebenen Einzelheiten einerlei mit Nr. 62: 91 v. Chr. 7tagiger Steinfall im Lande der Vestiner. \\ \hline
        650. - - & ? & Italien (?) & P. 4. 1854. 8. Lycosthenes 322. - Der ganzen Beschreibung nach offenbar nur eine Verwechselung mit Nr. 168: dem 950 (951, 952 oder 953) zu Augsburg gefallenen Stein.* \\ \hline
        820. - - & ? & Deutschland (?) & P. 6. 1826. 22. (nach Schnurrer) P. 4. 1854. 450. Unrichtige Jahreszahl für Nr. 138: 823. Hagel mit angeblichen Steinen. \\ \hline
        823. (822.) (824.) (825.) Vor dem 24. Juni. & Autun (Augustudinum in Burgund.) & Frankreich & C. 191. Ann. Fuld. (Pertz 1. 358.) - War kein Stein, sondern ein ungeheures, wahrend eines Sturmes vom Himmel gefallenes Stuck Eis von 15' Lange, 6' Breite und 2' Dicke (oder nach anderer Angabe von 12' Lange, 7' Breite und 4' Dicke). \\ \hline
        893. - - & ? & Asien & P. 24. 1832. 221. K. 3. 265. C. 192. Abulfaradsch (Bar-Hebraeus) Chr. Syr. 181. - Einerlei mit Nr. 158: 893 oder 897, Ahmed-Abad bei Kufah, und wohl nur aus Versehen nochmals und ohne Angabe des Ortes als ein hiervon verschiedener Steinfall aufgeführt. \\ \hline
        963. - - & ? & Italien & P. 4. 1854. 8. A. 4. 187. Lycosthenes 363. Herold. 351. - Nach Vergleichung der ursprünglichen Quellen offenbar einerlei mit Nr. 170: 956, Italien.* \\ \hline
        Zwischen 964 u. 972. & ~ & Italien & C. 193. A. 4. 187. - Desgleichen.* \\ \hline
        1002. 14. September & Arabien & ~ & P. 66. 1845. 476. l'Institut 4. 350. - Es fiel ein Stern, der nach Verlauf einer Stunde, wahrend welcher er mit abnehmendem Glanze sich am Himmel bewegte, zerplatzte. Von einem Steinfall ist nicht die Rede. \\ \hline
        Vor 1009. - - & Joigny. & Frankreich & Michaud: hist. d. Crois. 1. 32.* Michaud: Bibl. d. Cr. 1. 201 u. 202.* - Angeblich 2 Jahre lang andauernder Steinregen, der jedoch zu rätselhaft, um nicht fur eine Fabel gehalten zu werden. \\ \hline
        Um 1009. (852.) - - & Cordova oder Lurgea. (Lorges?) & Spanien & C. 195. von Ende Fol. 29.* - Einerlei mit Nr. 183: dem Eisenfall von Tschurdschan, welcher von Avicenna irrtümlich an diese Orte versetzt ward. \\ \hline
        1104. - - & ? & ? & Lycosthenes. 391. Herold. 373. - Nach Herold Hagel mit großen Kisslingen (Steinen); nach Lycosthenes jedoch nur gewöhnlicher Hagel. \\ \hline
        1151. - - & Zwischen Abdaha und Tarschena; am Euphrat. & Mesopotamien & P. 24. 1832. 222. K. 3. 266. Abulfaradsch (Bar-Hebraeus) Chr. Syr. 348. - Kein Steinfall. War nach Bar-Hebraeus S. 348 nur ein heftiger Regen, der Felsen wegführte und eine Überschwemmung des Euphrat verursachte. \\ \hline
        1186. 30. Juni & Bergen. & Belgien & P. 66. 1845. 476. - Einerlei mit Nr. 205: Mons, das im Verzeichnis P. 66. 476. fehlt. \\ \hline
        1189. - - & ? & ? & P. 6. 1826. 23. Schnurrer 257 u. 258. Sind sämtlich, nach dem ganzen Wortlaut in den alten Chroniken, einerlei mit Nr. 206: 1190 (1191, 1194), Clermont und Compiegne bei Beauvais. Auch die Sächsischen und Thüringischen Chroniken sagen durchaus nicht, dass das Ereignis in Sachsen stattgefunden habe, sondern erwähnen desselben ohne Beifügung irgend einer weiteren Ortsangabe. \\ \hline
        1191. - - & In Sachsen. & Deutschland & C. 198. P. 6. 1826. 23. G. 53. 1816. 308 und 310. G. 29. 1808. 375. Sind sämtlich, nach dem ganzen Wortlaut in den alten Chroniken, einerlei mit Nr. 206: 1190 (1191, 1194), Clermont und Compiegne bei Beauvais. Auch die Sächsischen und Thüringischen Chroniken sagen durchaus nicht, dass das Ereignis in Sachsen stattgefunden habe, sondern erwähnen desselben ohne Beifügung irgend einer weiteren Ortsangabe. \\ \hline
        1194. - - & ? & ? & A. 4. 188. Sind sämtlich, nach dem ganzen Wortlaut in den alten Chroniken, einerlei mit Nr. 206: 1190 (1191, 1194), Clermont und Compiegne bei Beauvais. Auch die Sächsischen und Thüringischen Chroniken sagen durchaus nicht, dass das Ereignis in Sachsen stattgefunden habe, sondern erwähnen desselben ohne Beifügung irgend einer weiteren Ortsangabe. \\ \hline
        1198. 24. Juni & ? & Frankreich & A. 4. 188. nach Lycosthenes Fol. 427. - Verwechselung mit einem nach Lycosthenes um Johanni in Frankreich gefallenen Honigtau und dem von demselben unmittelbar darauf erwähnten Stein- oder Hagelfall bei Chelles und Tremblai vom 8. Juni (Juli) 1198 (Nr. 208). \\ \hline
        1198. - Juli & ? & Frankreich & A. 4. 188. nach den Rec. des Hist. des Gaules. - Ohne Zweifel - jedoch hier ohne nähere Ortsangabe - dasselbe Ereignis wie Nr. 208: der am 8 Juni (Juli) desselben Jahres zwischen Chelles und Tremblai stattgehabte Stein- oder Hagelfall. \\ \hline
        1240. (zwischen 1215 und 1250.) - - & Kloster des heiligen Gabriel bei Cremona. & Italien & C. 199. - Mythe; nach Chladni ein „frommer Betrug“ und überdies nur Hagel. \\ \hline
        12.. - - & Würzburg. & Deutschland & C. 199. - Stein, im Schottenkloster aufbewahrt, aber ohne alle Ähnlichkeit mit einem Meteorstein. \\ \hline
        1305. - - & Vandals. & Österreich & RPG. 33. - Wohl nur eine Verwechselung mit Nr. 240: 1304. 1. Oktober Friedland in Brandenburg, das sich auch als Vredeland in Vandalia aufgezeichnet findet. \\ \hline
        1388. 8. März & Mosul. & Asiatische Türkei & C. 78. - Druckfehler; soll heißen 1130. (nicht 1138) 8. März (Nr. 198). \\ \hline
        1438. - - & Luzern. & Schweiz & A. 4. 189. P. 4. 1854. 40. - Aus einer Feuerkugel eine Flüssigkeit wie geronnenes Blut mit gleichzeitigem bloßem Staubfall. \\ \hline
        1448. - September & Augsburg. & Deutschland & Lycosthenes 481. Herold. 447. Fincelius das 1528 Jar. - Die mit dem Hagel gefallenen angeblichen Steine sind nach Lycosthenes offenbar ebenfalls nur große Schlossen. \\ \hline
        1470. Anf. Juni & Rom. & Italien & Lycosthenes 487. Herold. 450. - Nach Herold Hagel mit 1/2 Pfund schweren Steinen; nach Lycosthenes aber nur große Schlossen. \\ \hline
        1471. - - & Brescia (Brixia) & Italien & Lycosthenes 488. Herold. 451. - Nach Herold Hagel mit Steinen wie Straußeneier, welche aber nach Lycosthenes ebenfalls nur sehr große Schlossen waren. \\ \hline
        1497. 25. Juli & ? & Deutschland & C. 209. - Nur Hagel. \\ \hline
        1502. 22. Juni & Bern, Solothurn u. Biel. & Schweiz & Lycosthenes 511. Herold. 464. - Angeblicher Hagel mit Steinen; jedoch augenscheinlich nur ungewöhnlich starkes Hagelwetter. \\ \hline
        1510. (1520.) - - & Abdun. & Italien & C. 211. G. 50. 1815. 237. - Verwechselung mit Nr. 275: 1511. 4. September unweit der Adda bei Crema; der Ausdruck „prope Abduam“ ist falsch verstanden worden (Chladni). \\ \hline
        1538. - - & Tripergola bei Neapel. & Italien & Thomson. 314. Erdbeben mit Feuerausbruch und regenartigem Sand- und Steinauswurf, in dessen Folge der Lucriner See vertrocknete u. ein neuer Berg sich emportürmte. \\ \hline
        1539. - - & Zurich. & Schweiz & Lycosthenes 567. Herold. 498. - Augenscheinlich nur großer Hagel. \\ \hline
        1544. - - & Neisse. (Nissa) & Schlesien & Fincelius, das 1544 Jar. Lycosthenes 585. Herold. 509. - Hagel mit angeblichen Steinen, welche nach Lycosthenes und Herold jedoch augenscheinlich nur große Schlossen gewesen. \\ \hline
        1548. 6. November & Mansfeld. & Deutschland & C. 364. - Feuerkugel mit rothlicher Flüssigkeit und einer schwärzlichen Masse wie geronnenes Blut. \\ \hline
        1552. 19. Mai & Wittenberg. & Deutschland & Lycosthenes 622. Herold. 531. Fincelius - Steinregen; doch offenbar nur Hagel. \\ \hline
        1552. 24. August & Dordrecht. & Holland & Lycosthenes 619. Herold. 531. Fincelius. - Hagel mit Pfund schweren Steinen, die nach dem Zerschmelzen einen stinkenden Dampf gaben; also sicherlich ebenfalls nur große Schlossen. \\ \hline
        1557. 25. Januar (25. November) & ? & Italien & P. 4. 1854. 441. K. 3. 267. - Nur Feuermeteor mit Getose. \\ \hline
        1586. 3. Dezember & Verden. & Deutschland & C. 366. - Feuermeteor mit einer teils blutroten, teils schwärzlichen Masse. \\ \hline
        1589. 16. August & Oderberg. & Deutschland & Angelus Ann. M. Brand. 405.* - Unwetter mit Hühnereigroßen eckigen Hagelsteinen; dem gesamten Wortlaute nach augenscheinlich nur große Schlossen. \\ \hline
        1618. - - & ? & Ungarn & P. 4. 1854. 451. - Nur an dieser Stelle ohne weitere nähere Angabe vorkommend und daher ohne Zweifel nur eine Verwechselung mit Nr. 326: dem auch in P. 4. 1854. Fol. 33 ohne Tag und Monat aufgeführten Steinfall von Murakoz, End August 1618. \\ \hline
        1652. - Mai & ? & Italien & P. 4. 1854. 424. - Bloße Sternschnuppenmaterie. \\ \hline
        1678. 6. (oder 16., nicht 26.) Februar & Frankfurt a. M. & Deutschland & C. 104. P. 66. 1845. 476. v. Lersner: Nachtrag Fol. 762.* - Angeblich vom Himmel gefallenes, nach Aussage der Wache aber natürliches Feuer, das noch eine Viertelstunde lang geglimmt und gedampft haben soll. \\ \hline
        1680. 18. Mai & London. & England & C. 239. - Nur Hagel. \\ \hline
        1683. 12. Januar & Castrovillari. & Italien & RPG. 34. - Druckfehler; einerlei mit Nr. 305: 1583. 9. Januar. \\ \hline
        1683. 3. März & Piemont. & Italien & RPG. 34. - Druckfehler; einerlei mit Nr. 306: 1583. 2. März. \\ \hline
        1686. 31. Januar & Rauden. & Kurland & G. 68. 1821. 347. - Schwarze, membranförmige Masse (Meteorpapier). \\ \hline
        1690. 2. Januar & Jena. & Deutschland & P. 18. 1830. 177. - 1 Klumpen Feuer; doch hat man nichts Bleibendes gefunden. \\ \hline
        1692. 9. April & Temesvar. & Ungarn & C. 105. P. 4. 1854. 33. (nach den Rep. of Brit. Ass. 1850.* - Nur „Feuerkugel mit erschrecklichem Knall.“ \\ \hline
        1717. - - & An der Donau. & ? & C. 107. P. 4. 1854. 33. (nach den Rep. of Br. Ass. 1850. Fol. 127.) - Wohl nur Verwechselung mit der am 10. August 1717 in Schlesien, Polen, Preußen, Ungarn und der Lausitz gesehenen Feuerkugel. \\ \hline
        1718. 24. März & Insel Lethy. & ? & C. 369. - Feuerkugel mit gallertartiger Substanz. \\ \hline
        1727. 22. Juli & Liboschitz. & Böhmen & A. 4. 193. - Wohl nur Verwechselung mit Nr. 369: 1723. 22. Juni, Pleskowitz und Liboschitz bei Reichstadt. \\ \hline
        1731. - - & Lessay bei Coutance (Normandie). & Frankreich & C. 241. - Angeblich geschmolzene Metallmasse; nach Chladni aber wahrscheinlich nur in Folge eines Gewitters. \\ \hline
        1740. - - & An der Donau. & - & P. 4. 1854. 33. (nach den Rep. of Br. Ass. 1850.) - Wohl nur Verwechselung mit Nr. 378: 1740. 25. Oktober Hazargrad. \\ \hline
        1743. - - & Lowositz (Liboschitz). & Böhmen & C. 243. - Nach Chladni wohl nur irrtümliche Jahreszahl fur Nr. 369: 1723. 22. Juni Pleskowitz und Liboschitz bei Reichstadt. \\ \hline
        1751. - - & Constanz. & Deutschland & C. 243. - Verwechselung mit Nr. 381: 1750. 1. (11.) Oktober Nicorps bei Coutance in der Normandie. \\ \hline
        1768. - - & Provinz Cotentin. & Frankreich & C. 252. - Der um diese Zeit nach Paris gesandte Stein rührte ohne Zweifel von dem Steinfall Nr. 381: 1750. 1. (11.) Oktober zu Nicorps bei Coutance, Provinz Cotentin, her. \\ \hline
        1779. - - & Segovia. & Spanien & C. 254. - Irrtümlich für Nr. 397: 1773. 17. November Sena bei Sigena. \\ \hline
        1785. 10. Januar & ? & Frankreich & C. 131. - Nur Feuerkugel mit Knall. \\ \hline
        1789. 20. (24.) August & Bordeaux (auch Roquefort oder Landes.) & Frankreich & G. 18. 1804. 264. Bigot de Morogues Fol. 121. - Verwechselung mit Nr. 413: 1790, 24. Juli. Barbotan. \\ \hline
        1791. 20. Oktober & Menabilly in Cornwallis. & England & C. 261. - Nur Hagel. \\ \hline
        1792. 27. (29.) August & La Paz. & Peru & P. 6. 1826. 27. - Meteorstaub. \\ \hline
        1796. 8. März & Ober-Lausitz. & Deutschland & C. 374. - Feuerkugel mit schaumiger und klebriger Masse. \\ \hline
        1798. 12. März & Genf. & Schweiz & P. 66. 1845. 476. C. 136. - Feuerkugel, aus welcher der Steinfall von Sales hervorging. \\ \hline
        1798. 13. Dezember & Krakau. & Polen & P. 66. 1845. 476. - In keinem anderen Meteorsteinverzeichnis zu finden, und wohl nur Feuerkugel, wie viele andere angebliche Steinfalle in jenem Verzeichnis. \\ \hline
        1803. 21. Januar & Bojanow. & Schlesien & P. 4. 1854. 42. - Nur Sternschnuppen-Materie. \\ \hline
        1806. 23. September & Weimar. & Deutschland & C. 147. - Nur Feuerkugel. \\ \hline
        1808. - - & ? & Ungarn & C. 147. P. 4. 1854. 33. (nach den Rep. of Br. Ass. 1850.) - Wohl nur Verwechselung mit der zu Wien und in der umliegenden Gegend gesehenen Feuerkugel vom 15. August 1808. \\ \hline
        1811. - Juli & Heidelberg. & Deutschland & P. 4. 1854. 43. - Feuerkugel mit schleimiger Masse. \\ \hline
        1812. - - & ? & Ungarn & C. 155. P. 4. 1854. 33. (nach den Rep. of Br. Ass. 1850.) - Wohl nur Verwechselung mit der zu Carlsruhe, Nurnberg, Salzberg, Wien und in Böhmen gesehenen Feuerkugel vom 15. November 1812. \\ \hline
        1813. 27. Januar oder 8. März & Brunn. & Mahren & C. 155. - Mit Geräusch berstende Feuerkugel. \\ \hline
        1813. 15. Dezember & Geißenheim im Rheingau. & Deutschland & C. 309. - Irrtümliche und unbestätigte Zeitungsnachricht. \\ \hline
        1814. M. März & ? & Finnland & P. 66. 1845. 476. - In keinem anderen Meteorstein-Verzeichnis zu finden, und daher wohl nur eine irrtümliche Angabe fur Nr. 465: 1813. 13. Dezember Lontalax in Finnland, das in jenem Verzeichnis ebenfalls aufgeführt ist. \\ \hline
        1814. - - & Gespanschaft Sarosch. & Ungarn & P. 4. 1854. 33. (nach den Rep. of Br. Ass. 1850.) - Angeblich 1 Stein von 133 Pfund; doch ohne Zweifel nur eine Verwechselung mit der 1815 bei Lenarto in der Gesp. Saroseh gefundenen Eisenmasse von 194 Pfund. \\ \hline
        1816. 19. Juli & Sternenberg (angeblich bei Bonn.) & Deutschland & C. 309. - Irrtümliche Zeitungsnachricht. \\ \hline
        1816. - - & Pesth und Nagybanya. & Ungarn & C. 160. P. 4. 1854. 33. (nach den Rep. of Br. Ass. 1850.) - Wohl nur Verwechselungen mit der am 8. (9.) Januar 1816 zu Pesth beobachteten Feuerkugel und derjenigen, welche am 7. August 1816 mit Knall und donnerndem Nachhall zersprang, wobei jedoch von keinem Steinfall die Rede ist. \\ \hline
        1818. 17. Juli & Juilly. & Frankreich & C. 309. - Irrtum. \\ \hline
        1818. 31. Oktober & Mehadiah. & Österreich & C. 167. Report of Brit. Ass. 1850.* - Bloße Feuerkugel. \\ \hline
        1818. 23. (nicht 6.) September & Kilkel (nach A. 4. 199. angeblich in Preußen mit Bezugnahme auf K. 3. 287, wo aber nur einfach „im Kirchspiel Kilkel“ angegeben ist.) & ? & K. 3. 287. P. 4. 1854. 436. A. 4. 199. - Nur in dem Verzeichnis von K. ohne Quellenangabe als „Steinfall“ aufgeführt; nach P. 4. 1854. 436. aber bloße Feuerkugel. \\ \hline
        1818. 13. und 17. November & Gosport. & England & K. 3. 287. Quetelet. 1839. 35.* und 1841. 39 u. 48.* - Die von Q. nach K. 3. 287. angeführten Aerolithen sind nach diesem Letzteren nur Feuerkugeln; von Steinen geschieht bei K. keine Erwähnung. \\ \hline
        1819. 6. August & ? & Mahren & K. 3. 287. Quetelet. 1839. 35. 1841. 40 und 48. - Desgleichen; auch ward das Ereignis nicht, wie von Q. irrtümlich angegeben, auf dem Meere (en mer), sondern nach K. und G. 68. 361. in Mahren beobachtet. \\ \hline
        1819. 13. August & Amherst in Massachusetts. & Nordamerika & G. 71. 1822. 354. - Feuerkugel-Materie. \\ \hline
        1820. 6. August & Ovelgönne. & Deutschland & G. 68. 1821. 371. G. 75. 1823. 114. - Feuerkugel, welche in einem Heuschober, jedoch nur durch natürliche Verbrennung, eine Bimsstein-artige Masse erzeugte. \\ \hline
        1820. 12. November & Chotimschk (im Gouv. Kursk.) & Russland & P. 66. 1845. 476. K. 3. 289. - Feuerkugel, die mit einem Knall zerplatzte. \\ \hline
        1821. 24. Dezember & ? & Deutschland & P. 66. 1845. 476. K. 3. 290. - Bloße Feuerkugel. \\ \hline
        1822. 13. Juni & Christiania & Norwegen & P. 4. 1854. 427. K. 3. 291. - Feuerkugel mit harziger Masse. \\ \hline
        1822. 19. Juni (Juli) & Hamburg & Deutschland & P. 4. 1854. 427. K. 3. 291. - Bloße Feuerkugel. \\ \hline
        1822. 12. November & Potsdam und Taucha (bei Leipzig.) & Deutschland & K. 3. 292. Quetelet. 1839. 36. 1841. 40 und 48. - Die von Q. nach K. 3. 292. angeführten Aerolithen sind nach diesem Letzteren nur Feuerkugeln; von Steinen geschieht bei K. keine Erwähnung. \\ \hline
        1823. 9. August & Giengen in Württemberg (nicht Gingen oder Singen.) & Deutschland & K. 3. 292. Quetelet. 1839. 37. 1841. 40. - Desgleichen. \\ \hline
        1823. 12. August & Tübingen. & Deutschland & Desgleichen. - \\ \hline
        1824. 3. Februar (Ende Januar) & Boulogne. & Frankreich & P. 66. 1845. 476. P. 4. 1854. 418. K. 3. 293. - Bloße Feuerkugel. \\ \hline
        1824. 14. Mai & Irkutsk (30 oder 80 Werste davon.) & Sibirien & P. 2. 1824. 155. P. 66. 1845. 476. P. 4. 1854. 425. - Durch ungenaue Zeitungsnachrichten veranlasste Verwechselung mit Nr. 499: 1824. 18. Februar Tounkin bei Irkutsk. \\ \hline
        1824. 23. August & Buenos-Ayres. & Sud-Amerika & P. 4. 1854. 433. P. 6. 1826. 28. - Meteorstaub. \\ \hline
        1824. 17. Dezember & Neuhaus. & Böhmen & P. 66. 1845. 476. P. 6. 1826. 31. P. 4. 1854. 447. K. 3. 293. - Feuerkugel mit wahrscheinlich harziger Masse. \\ \hline
        1826. 1. April (oder August) & Saarbrücken. & Deutschland & P. 4. 1854. 423. K. 3. 295. - Bloße Feuerkugel. \\ \hline
        1828. - - & Puerto Santa Maria. & Spanien & P. 38. 1830. 187. - Angeblich eine entsetzliche Menge von Aerolithen, so dass die Steine 4 Fuß hoch in der Straße gelegen haben sollen; daher unglaublich. \\ \hline
        1829. 18. September & Bohumilitz. & Böhmen & P. 66. 1845. 476. - Nicht Falltag, sondern nur Fundtag des Steines. \\ \hline
        1829. 26. September & Düsseldorf. & Deutschland & P. 66. 1845. 476. K. 3. 297. - Bloße Feuerkugel. \\ \hline
        1831. - Dezember & ? & Mahren & K. 3. 299. nach Plieninger.* - Dieser der Wiener Zeitung 1832. Nr. 11. entnommene Steinfall ist kein anderer als Nr. 530: 1831. 9. September Znorow bei Wessely; obgleich dieser Letztere von Plieninger in Band 20. 1831. Fol. 348. ebenfalls aufgeführt wird. Der Zeitungsartikel sagt irrtümlich „am 9. Dezember“ anstatt am 9. September. \\ \hline
        1832. 19. Dezember & ? & England & P. 66. 1845. 476. - In keinem anderen Meteorstein-Verzeichnis vorkommend; daher wahrscheinlich bloß Feuerkugel, wie viele andere angebliche Steinfalle jenes Verzeichnisses. \\ \hline
        1833. 12. November & ? & Nordamerika & P. 4. 1854. 443. - Sternschnuppen-Materie. \\ \hline
        1834. 1. Januar & Zetiz. & Deutschland & P. 34. 1835. 344. P. 66. 1845. 476. - Irrtümliche Nachricht. \\ \hline
        1835. 6. September & Gotha. & Deutschland & P. 4. 1854. 80 u. 436. - Fettige, nach Schwefel riechende Feuerkugel-Materie, die nachher verdunstete. \\ \hline
        1836. 8. Februar & Rivoli. & Italien & P. 66. 1845. 418. P. 4. 1854. 81 u. 418. - Bloß Feuerkugel, die mit Geräusch zerplatzte. \\ \hline
        1836. 12. Februar & Orval bei Coutance. & Frankreich & A. 4. 267. - Eine bei einem Sumpfe in der Nahe von Orval mit Explosionen niedergefallene, auch zu Cherbourg gesehene Feuerkugel; von Steinen ist aber keine Rede. \\ \hline
        1836. 18. September & ? & Italien & P. 4. 1854. 436. - Feuerkugel-Materie. \\ \hline
        1841. 10. August & Iwan, SO. von Oedenburg. & Ungarn & P. 66. 1845. 476. P. 4. 1854. 364. P. 54. 1841. 279. - Art Bohnerz von nicht-meteorischem Ursprung. \\ \hline
        1841. - September & ? & Ungarn & Thomson. 327. - Tausende von mehr als hagelgroßen Meteorsteinen; sicher nur eine Verwechselung mit dem Vorigen. \\ \hline
        1841. 29. September & Bayonne. & Frankreich & P. 66. 1845. 476. P. 4. 1854. 92 u. 437. - Bloß Feuerkugel. \\ \hline
        1842. 5. Dezember & Langres. (Dép. de la Haute-Marne.) & Frankreich & A. 4. 203. AR. 12. 1842. 1118. - Einerlei mit Nr. 574. 1842. 5. Dezember Eaufromont. \\ \hline
        1843. 10. (12.) November & An der Donau. & ? & P. 4. 1854. 375. Rep. of Br. Ass. 1848. - Lauter Knall aus einer Feuerkugel; doch schien nichts herabzufallen. \\ \hline
        1844. 2. Oktober & St. Andrews (auf der Insel Cuba.) & West-Indien & RPG. 37. - Bloße Feuerkugel. (RPG.) \\ \hline
        1844. 21. Oktober & Favars, Canton Layssac. & Angeblich in der Schweiz & P. 4. 1854. 375 u. 105. - Verwechselung mit Nr. 583: dem Steinfall vom 21. Oktober 1844. zu Lessac im Dép. de la Charente oder vielleicht auch mit der Feuerkugel vom 19. (20.) November 1844. zu Layssac in Sudfrankreich. \\ \hline
        1846. 7. Juni & Darmstadt. & Deutschland & P. 4. 1854. 428. - Nicht der Tag des Falles, sondern nur des Fundes einer irrtümlich fur meteorisch gehaltenen Eisenmasse. \\ \hline
        1846. (1847.) 11. November & Lowell in Massachusetts & Nord-Amerika & P. 4. 1854. 117, 377 u. 444. RPG. 37. - Bloß Feuerkugel. \\ \hline
        1849. 19. März & Poonah. & Ost-Indien & RPG. 38. Rep. of Br. Ass. 1849 u. 1850. - Nach Br. Ass. Rep. 1849. (publ. 1850) Fol. 18, 34 u. 38, und 1850. (publ. 1851) Fol. 127. bloß zerplatzte Feuerkugel. \\ \hline
        1850. 25. Januar & Tripolis. & Nord-Afrika & P. 4. 1854. 382. - Ist nach neuerer Angabe kein Steinfall, sondern nur der Tag, an welchem Richardson an Lord Palmerston den Steinfall Nr. 601, welcher am 13. November 1849 zu Tripolis stattgefunden, brieflich mitteilte.* \\ \hline
        1850. 22. Juni & Oviedo. & Spanien & RPG. 38. - Soll nach einer neueren Mitteilung nur eine mit Explosion zerplatzte Feuerkugel sein.* \\ \hline
        1851. - - & Barcelona. & Spanien & RPG. 58. - Verwechselung mit Nr. 605: 1851. 5. November Saragossa. \\ \hline
        1853. - April & Mannheim. & Deutschland & Neue Preuss. Zeitung 1853. Nr. 118. - Müßige Erfindung und Zeitungsente. \\ \hline
    \end{longtable}
\end{center}
\clearpage
\section{Angebliche Meteorsteine, deren Fallzeit unbekannt, welche aber als irrig sich erwiesen.}
\begin{table}[!ht]
    \centering
    \footnotesize
    \begin{tabular}{|l|l|p{45mm}|}
    \hline
        Halberstadt & Deutschland & C. 83. - Der angebliche Donnerkeil in der Kirche ist eine alte Streitaxt, und der Stein auf dem Domplatz ein Konglomerat mit Versteinerungen. \\ \hline
        Coln & Deutschland & C. 187. - Nur ein in Folge eines Sturmes vom Thurm des Doms herabgefallener Stein. \\ \hline
        London & England & C. 185. A. 4. 185. - Der stein in dem Kronungsstuhl der Könige ist kein Meteorstein. \\ \hline
        Persepolis & Persien & C. 185. - Der Stein mit Keilschrift ist kein Meteorstein, sondern nur ein schwarzer Basalt. \\ \hline
    \end{tabular}
\end{table}
\clearpage
\section{Angebliche Meteor-Eisenmassen, deren Fallzeit unbekannt, welche aber fur irrig oder nicht meteorisch zu halten.}
\begin{table}[!ht]
    \centering
    \footnotesize
    \begin{tabular}{|p{40mm}|p{20mm}|p{60mm}|}
    \hline
        Olvenstadt bei Magdeburg & Deutschland & P. 34. 1835. 346. P. 4. 1854. 390. B. 115. - Hutten-Erzeugnis. \\ \hline
        Aken an der Elbe & Deutschland & G. 18. 1804. 308. B. 52. - Verwechselung mit dem Eisen von Aachen (dem Folgenden). \\ \hline
        Aachen & Deutschland & C. 346. - Kunst-Erzeugnis. \\ \hline
        Groß-Kamsdorf bei Saalfeld (Grube Eiserner Johannes.) & Deutschland & C. 351. B. 111. WA. 25. 1857. 542. - Nach Klaproth und Reuß irdisches Eisen. \\ \hline
        Wolfsmuhl bei Thorn & Deutschland & P. 4. 1854. 452. P. 94. 1854. 169. B. 114. - 20,000 Ztr; von Karsten für meteorisch gehalten; von Rose dagegen für Eisenschlacke erklärt. \\ \hline
        Kyrburger Grube (im Hachenburg’schen.) & Deutschland & B. 113. - Nach Karsten ein Kunst-Erzeugniss. \\ \hline
        Mühlhausen in Thüringen & Deutschland & B. 113. WA. 25. 1857. 542. - Irdisches, nickelfreies Eisen im Keuperkalk. \\ \hline
        Darmstadt & Deutschland & P. 4. 1854. 428. B. 113. Oberhessische Gesellsch. 1860. Fol. 83 und 84.* - Nach neuester Untersuchung kein Meteor-Eisen. \\ \hline
        ? & Schweiz & Schweigger 14. (44.) 1825. Fol. 357. Becher (Stahl) Ph. Subt. 602. Etterlyn Bl. 6. S. 2.* Lycosthenes 344. - Das große Stuck Eisen (ysen), von welchem Becher sagt, dass es nach Peterman Etterlyn in der Schweiz gefallen sei, war kein Eisen, sondern das in verschiedenen alten Chroniken erwähnte große Stuck Eis (yss), welches in Jahr 823 bei Autun in Burgund gefallen ist. \\ \hline
        Cilly & Steiermark & C. 353. - Von v. Widmannstatten nicht für meteorisch gehalten. \\ \hline
        Auval bei Prag & Böhmen & WA. 25. 1857. 563. - Irdisches Eisen. \\ \hline
        Leadhills & Schottland & C. 356. - Nickelfrei und mit Blende verbunden; daher nach Chladni wohl irdischen Ursprunges. \\ \hline
        Oulle, bei Allemont in der Dauphiné & Frankreich & WA. 25. 1857. 542. B. 113. - Irdischen Ursprungs. \\ \hline
        Florac, Dép. de la Lozère & Frankreich & C. 355. - Von Chladni fur Hutten-Erzeugnis gehalten. \\ \hline
        Auvergne (Angeblich von den Bergen der Auvergne; nach anderer Angabe: aus den Ardennen oder von den Seven-Mountains, dem Siebengebirge?) & Frankreich & SJ. 33. 1838. 257 und 258. P. 4. 1854. 384. - Nach allen angeführten Einzelheiten eine Verwechselung mit dem Eisen von Bitburg (Bittburg) in der Eifel. \\ \hline
        Olahpian & Ungarn & B. 112. WA. 9. 462 – Im Sande mit Gold und Platin zusammenhangend; daher wohl irdischen Ursprunges. \\ \hline
        ? & Makedonien & P. 18. 1830. 190. - Diese nach von Hoff auf Seite 65 als Meteoreisen aufgeführte Meteormasse ist kein Eisen, sondern ein gewöhnlicher, sehr eisenhaltiger Meteorstein; welches daher auf S. 65. nachträglich zu verbessern ist. \\ \hline
        Canaan in Connecticut & Nord-Amerika & P. 24. 1832. 232. B. 112. WA. 25. 1825. 542. - Kunst-Erzeugnis. \\ \hline
        Sergipe & Brasilien (nicht Ost-Indien) & P. 4. 1854. 396. WA. 41. 1860. 252. - Verwechselung mit dem Bemdego- oder Bahia-Eisen. \\ \hline
    \end{tabular}
\end{table}
\normalsize
*) Johann Herold: Wunderwerck oder Gottes vnergrundthches vorbilden. Auss Herrn. Conrad Lycosthenes Latinisch zusammen getragener Beschreibung in vier Bücher gezogen und Verteütscht. Basel 1557.

*) Diese irrtümlichen Steinfalle finden sich in dem geographischen Verzeichnis Seite 59 und 67 noch aufgeführt, und sind daher an beiden Orten nachträglich zu streichen.

*) M. Michaud: Histoire des Croisades; Bruxelles et Leipzig 1841.

*) M. Michaud: Bibliotheque des Croisades; Paris 1829 (Cinq Livres de l'histoire de son temps, ecrite par Raoul Glaber, moine de Cluni).

*) Von Ende: Über Massen und Steine, die vom Monde auf die Erde gefallen sind. Braunschweig 1804.

*) M. Andreas Angelus Struthiomontanus (Andreas Engel von Straussberg): Annales Marchiae Brandenburgicae.

*) Achill. Augusti von Lersuer: Nachgehohlte, vermehrte, und kontinuierter Chronica der Weitberuhmten freien Reichs- Wahl- und Handels-Stadt Frankfurt am Main; aus des Seel. Auetoris hinterlassenem Manuscripto zusammengetragen, und durch eigenen Verlag zum Druck befördert durch Georg. Augustum von Lersner. Frankfurt am Main, 1734. Buch 1. Cap. 37.

*) Reports of British Association of 1850.

*) In Bezug auf alle diese, den Reports of British Association for the Advancement of Science, 1849 (1850), entnommenen angeblichen Meteorsteinfalle in Ungarn und an der Donau heißt es in dem Aufsatz: „A Catalogue of observations of luminous Meteors by the Rev. Baden Powell, M. A., F. R. S. etc. Savilian Professor of Geometry, Oxford“ wörtlich: „For the following list of Meteorites, which have fallen in Hungary, I am indebted to W. W. Smyth Esq. M. A. Geologist to the Geological Survey.“ Und nun werden die einzelnen Falle, nämlich deren Jahreszahl und Ort, ohne alle und jede weitere nähere Angabe - wie oben in den betreffenden Fallen bemerkt - aufgeführt. Da jedoch durchaus keine Quelle aus irgend einer Deutschen Zeitschrift mitgeteilt wird, diese Letzteren im Gegenteil - wie es scheint - von den meisten dieser angeblichen Meteorsteinfalle durchaus keine Erwähnung tun, sondern meist nur Feuerkugeln in den betreffenden Jahren und an den betreffenden Orten auffuhren: so darf diese Angabe in den British Association Reports wohl gewiss nur als sehr unzuverlässig betrachtet werden. Waren aus den betreffenden Feuerkugeln wirklich Meteorsteine hervorgegangen: wir wurden wohl sicher eher zuverlässige Nachrichten darüber aus Ungarn selbst oder über Wien erhalten haben, als in einer dazu noch so wenig zuverlässigen Weise erst auf dem weiten Umweg über England.

*) Academie Royale de Bruxelles. Catalogue des principales apparitions d'etoiles filantes par A. Quetelet; Bruxelles 1839.

*) Academie Roylae de Bruxelles. Nouveau Catalogue des principales apparitions d'etoiles filantes par A. Quetelet; Bruxelles 1841.

*) Korrespondenzblatt des Königl. Würtemb. Landwirtschaftlichen Vereins. Neue Folge. Band 1 (der ganzen Reihenfolge Band 21). Stuttgart und Tübingen 1832. Darinnen Seite 348: Meteorologische Chronik vom Jahr 1832 von Prof. Plieninger; Nachtrag von 1831.

*) Diese beiden irrtümlichen Meteorsteinfalle sind daher in dem Monats-Verzeichnis Seite 47. und in dem Verzeichnis zu Karte 2, Seite 65, nachträglich zu streichen.

*) Achter Bericht der Oberhessischen Gesellschaft für Natur- und Heilkunde; Gießen, Mai 1860.

*) Peterman Etterlyn, gerichtschriber zu Lutzern, Kronica von der loblichen Eydtgnoschaft jr harkommen vnd sust seltzam strittenn vnd geschichten. Basel 1507.
\clearpage
\section{Schluss-Zusammenstellung.}
\begin{center}
Von bekannter Fallzeit.
\end{center}
\begin{itemize}
    \item 287 mehr oder minder zuverlässige Steinfalle. (Seite 350 bis 394)
    \item 17 mehr oder minder zuverlässige Eisenfalle. (und 396 bis 440.)
    \item 337 mehr oder minder zweifelhafte Steinfalle. (Seite 396 bis 441.)
    \item 6 mehr oder minder zweifelhafte Eisenfalle. (Seite 396 bis 441.)
\end{itemize}
\begin{center}
zusammen: 647.
\end{center}
\begin{center}
Von unbekannter Fallzeit.
\end{center}
\begin{itemize}
    \item 17 mehr oder minder zuverlässige Steinfalle. (Seite 350 bis 394.)
    \item 97 mehr oder minder zuverlässige Eisenfalle. (Seite 350 bis 394.)
    \item 24 mehr oder minder zweifelhafte Steinfalle. (Seite 441 bis 443.)
    \item 10 mehr oder minder zweifelhafte Eisenfalle. (Seite 443.)
\end{itemize}
\begin{center}
zusammen: 148.
\end{center}

\paragraph{}
In Allem: 795 Falle.
\end{document}
